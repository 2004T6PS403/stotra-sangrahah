% !TeX program = XeLaTeX
% !TeX root = nityaparayanam-vedam.tex
%\part{वेदमन्त्राः}
\fontsize{19pt}{23pt}\selectfont
\newcounter{anuvakam}
\newcommand{\anuvakamend}{\refstepcounter{anuvakam}\\[-0.8ex]
\rule[0.5ex]{0.93\textwidth}{1.5pt}\bfseries{[\devanumber{\arabic{anuvakam}}]}%\hrulefill
}
\setlength{\emergencystretch}{3em}
% !TeX program = XeLaTeX
% !TeX root = ../../shloka.tex

\sect{गुरुपरम्परा-स्तुतिः}

\centerline{॥ॐ श्री गणेशाय नमः॥}
\centerline{॥ॐ श्री गुरुभ्यो नमः॥}
\centerline{॥हरिः ॐ॥}


\setlength{\shlokaspaceskip}{0pt}
\fourlineindentedshloka*
{नारायणं पद्मभुवं वसिष्ठं शक्तिं च तत्पुत्रपराशरं च}%
{व्यासं शुकं गौडपदं महान्तं गोविन्दयोगीन्द्रमथास्य शिष्यम्}% 
{श्री शङ्कराचार्यमथास्य पद्मपादं च हस्तामलकं च शिष्यम्}%
{तं तोटकं वार्तिककारमन्यानस्मद्गुरून् सन्ततमानतोऽस्मि}% 
\setlength{\shlokaspaceskip}{24pt}
\twolineshloka*
{श्रुतिस्मृतिपुराणानाम् आलयं करुणालयम्}
{नमामि भगवत्पादशङ्करं लोकशङ्करम्}

\twolineshloka*
{अपारकरुणासिन्धुं ज्ञानदं शान्तरूपिणम्}
{श्रीचन्द्रशेखरगुरुं प्रणमामि मुदाऽन्वहम्}

\fourlineindentedshloka*
{परित्यज्य मौनं वटाधःस्थितिं च}
{व्रजन् भारतस्य प्रदेशात्प्रदेशम्}
{मधुस्यन्दिवाचा जनान्धर्ममार्गे}
{नयन् श्रीजयेन्द्रो गुरुर्भाति चित्ते}

\twolineshloka*
{नमामः शङ्करान्वाख्य विजयेन्द्रसरस्वतीम्}
{श्रीगुरुं शिष्टमार्गानुनेतारं सन्मतिप्रदम्}

\twolineshloka*
{सदाशिवसमारम्भां शङ्कराचार्यमध्यमाम्}
{अस्मदाचार्यपर्यन्तां वन्दे गुरुपरम्पराम्}

\fourlineindentedshloka*
{ऐङ्कार-ह्रीङ्कार-रहस्ययुक्त-}
{श्रीङ्कार-गूढार्थ-महाविभूत्या}
{ओङ्कार-मर्म-प्रतिपादिनीभ्याम्}
{नमो नमः श्रीगुरुपादुकाभ्याम्}

\twolineshloka*
{अचिन्त्याव्यक्तरूपाय निर्गुणाय गुणात्मने}
{समस्तजगदाधारमूर्तये ब्रह्मणे नमः}

\closesection
% !TeX program = XeLaTeX
% !TeX root = ../../shloka.tex
\begin{center}
\sect{स्वस्ति-वाचनम्/गुरुवन्दनम्}
{॥ॐ श्री गुरुभ्यो नमः॥}\\
{॥श्री महात्रिपुरसुन्दरी समेत श्री चन्द्रमौलीश्वराय नमः॥}\\
{॥श्री काञ्ची कामकोटि पीठाधिपति जगद्गुरु\\ श्री शङ्कराचार्य श्री चरणयोः प्रणामाः॥}
\end{center}

\noindent स्वस्ति श्रीमदखिल भूमण्डलालङ्कार त्रयस्त्रिंशत्कोटि देवता सेवित\\
श्री कामाक्षीदेवीसनाथ श्रीमदेकाम्रनाथ श्री महादेवीसनाथ\\
 श्री हस्तिगिरिनाथ साक्षात्कार परमाधिष्ठान सत्यव्रत नामाङ्कित\\
 काञ्ची दिव्यक्षेत्रे शारदामठसुस्थितानाम् अतुलित सुधारस माधुर्य\\
 कमलासन कामिनी धम्मिल सम्फुल्ल मल्लिका मालिका\\
 निष्यन्दमकरन्दझरी सौवस्तिक वाङ्निगुम्भ विजृम्भणानन्द \\
तुन्दिलित मनीषिमण्डलानाम् अनवरत अद्वैत \\
विद्याविनोदरसिकानां निरन्तरालङ्कृतीकृत शान्ति\\
 दान्तिभूम्नां सकल भुवनचक्रप्रतिष्ठापक  श्रीचक्रप्रतिष्ठा विख्यात\\
 यशोऽलङ्कृतानां निखिल पाषाण्ड षण्ड\\
 कण्टकोद्घाटनेन विशदीकृत वेदवेदान्तमार्ग\\
 षण्मतप्रतिष्ठापकाचार्याणां  श्रीमत्परमहंस परिव्राजकाचार्यवर्य\\
श्रीजगद्गुरु श्रीमच्छङ्कर भगवत्पादाचार्याणाम्  अधिष्ठाने \\
सिंहासनाभिषिक्त श्रीमत् चन्द्रशेखरेन्द्र सरस्वती संयमीन्द्राणाम् \\
अन्तेवासिवर्य श्रीमत् जयेन्द्र सरस्वती श्रीपादानां \\
तदन्तेवासिवर्य श्रीमत् शङ्करविजयेन्द्र सरस्वती श्रीपादानां च\\
 चरणनिलयोः सप्रश्रयं साञ्जलिबन्धं च नमस्कुर्मः॥
\clearpage
\input{"../../../vedamantra-book/mantras/MrtyunjayaHomaMantrah.tex"}
\clearpage
\input{"../../../vedamantra-book/mantras/Ganapatyatharvashirsham.tex"}
\clearpage
\input{"../../../vedamantra-book/mantras/DurgaSuktam.tex"} 
\clearpage
\input{"../../../vedamantra-book/mantras/MedhaSuktam.tex"} 
\closesection

\clearpage
\input{"../../../vedamantra-book/mantras/NavagrahaSuktam.tex"} 
\clearpage
\input{"../../../vedamantra-book/mantras/AyushyaSuktam.tex"} 
 \vspace{-1ex}
\chapt{मेधा-विलास-सिद्धर्थं जपमन्त्रम्}
सद॑स॒स्पति॒मद्भु॑तं प्रि॒यमिन्द्र॑स्य॒ काम्यम्। सनिं॑ मे॒धाम॑यासिषम्।

\chapt{वेदविस्मरणाय जपमन्त्राः}
\vspace{-1.5ex}
\centerline{\normalsize (तैत्तिरीयारण्यकम् – ४/प्रपाठकः – १०/अनुवाकः – ८–९)}
यश्छन्द॑सामृष॒भो वि॒श्वरू॑प॒श्छन्दोभ्य॒श्छन्दास्यावि॒वेश॑। सता शिक्यः पुरोवाचो॑पनि॒षदिन्द्रो ज्ये॒ष्ठ इ॑न्द्रि॒याय॒ ऋषि॑भ्यो॒ नमो॑ दे॒वेभ्य॑ स्व॒धा पि॒तृभ्यो॒ भूर्भुव॒ सुव॒श्छन्द॒ ओम्॥२२॥ नमो॒ ब्रह्म॑णे धा॒रणं॑ मे अ॒स्त्वनि॑राकरणं धा॒रयि॑ता भूयासं॒ कर्ण॑योः श्रु॒तं मा च्योढ्वं॒ ममा॒मुष्य॒ ओम्॥२३॥
\clearpage
\input{"../../../vedamantra-book/mantras/VaishvadevaMantrah.tex"} 
\clearpage
\input{"../../../vedamantra-book/mantras/BhagyaSuktam.tex"} 
\clearpage
\input{"../../../vedamantra-book/mantras/PavamanaSuktam.tex"} 
\clearpage
\input{"../../../vedamantra-book/mantras/BhuSuktam.tex"} 
\clearpage
\input{"../../../vedamantra-book/mantras/ShriSuktam.tex"} 
\clearpage
\input{"../../../vedamantra-book/mantras/VishnuSuktam.tex"} 
\clearpage

\chapt{त्रिसुपर्णमन्त्राः}
\vspace{-1ex}
\centerline{\normalsize (तैत्तिरीयारण्यकम् – ४/प्रपाठकः – १०/अनुवाकः – ३८–४०)}
ब्रह्म॑मेतु॒ माम्। मधु॑मेतु॒ माम्। ब्रह्म॑मे॒व मधु॑मेतु॒ माम्। यास्ते॑ सोम प्र॒जाव॒त्सोभि॒ सो अ॒हम्। दुःस्व॑प्न॒हन्दु॑रुष्षह। यास्ते॑ सोम प्रा॒णास्तां जु॑होमि। त्रिसु॑पर्ण॒मया॑चितं ब्राह्म॒णाय॑ दद्यात्। ब्र॒ह्म॒ह॒त्यां वा ए॒ते घ्न॑न्ति। ये ब्राह्म॒णास्त्रिसु॑पर्णं॒ पठ॑न्ति। ते सोमं॒ प्राप्नु॑वन्ति। आ॒स॒ह॒स्रात्प॒ङ्क्तिं पुन॑न्ति। ओम्॥५५॥

ब्रह्म॑ मे॒धया। मधु॑ मे॒धया। ब्रह्म॑मे॒व मधु॑ मे॒धया। अ॒द्या नो॑ देव सवितः प्र॒जाव॑त्सावी॒ सौभ॑गम्। परा॑ दु॒ष्वप्नि॑य सुव। विश्वा॑नि देव सवितर्दुरि॒तानि॒ परा॑ सुव। यद्भ॒द्रं तन्म॒ आ सु॑व। मधु॒ वाता॑ ऋताय॒ते मधु॑ क्षरन्ति॒ सिन्ध॑वः। माध्वीर्नः स॒न्त्वोष॑धीः। मधु॒ नक्त॑मु॒तोषसि॒ मधु॑म॒त्पार्थि॑व॒ रज॑। मधु॒ द्यौर॑स्तु नः पि॒ता। मधु॑मान्नो॒ वन॒स्पति॒र्मधु॑मा अस्तु॒ सूर्य॑। माध्वी॒र्गावो॑ भवन्तु नः। य इ॒मं त्रिसु॑पर्ण॒मया॑चितं ब्राह्म॒णाय॑ दद्यात्। भ्रू॒ण॒ह॒त्यां वा ए॒ते घ्न॑न्ति। ये ब्राह्म॒णास्त्रिसु॑पर्णं॒ पठ॑न्ति। ते सोमं॒ प्राप्नु॑वन्ति। आ॒स॒ह॒स्रात्प॒ङ्क्तिं पुन॑न्ति। ओम्॥५६॥

ब्रह्म॑ मे॒धवा। मधु॑ मे॒धवा। ब्रह्म॑मे॒व मधु॑ मे॒धवा। ब्र॒ह्मा दे॒वानां पद॒वीः क॑वी॒नामृषि॒र्विप्रा॑णां महि॒षो मृ॒गाणाम्। श्ये॒नो गृध्रा॑णा॒ स्वधि॑ति॒र्वना॑ना॒ सोम॑ प॒वित्र॒मत्ये॑ति॒ रेभ\sn{}। ह॒सः शु॑चि॒षद्वसु॑रन्तरिक्ष॒सद्धोता॑ वेदि॒षदति॑थिर्दुरोण॒सत्। नृ॒षद्व॑र॒सदृ॑त॒सद्व्यो॑म॒सद॒ब्जा गो॒जा ऋ॑त॒जा अ॑द्रि॒जा ऋ॒तं बृ॒हत्। ऋ॒चे त्वा॑ रु॒चे त्वा॒ समित्स्र॑वन्ति स॒रितो॒ न धेना। अ॒न्तर्\mbox{}हृ॒दा मन॑सा पू॒यमा॑नाः। घृ॒तस्य॒ धारा॑ अ॒भिचा॑कशीमि। हि॒र॒ण्ययो॑ वेत॒सो मध्य॑ आसाम्। \mbox{तस्मि\hspace{1ex}\hspace{-1ex}न्त्सु}\-प॒र्णो म॑धु॒कृत्कु॑ला॒यी भज॑न्नास्ते॒ मधु॑\-दे॒वताभ्यः। तस्या॑सते॒ हर॑यः स॒प्ततीरे स्व॒धां दुहा॑ना अ॒मृत॑स्य॒ धाराम्। य इ॒दं त्रिसु॑पर्ण॒मया॑चितं ब्राह्म॒णाय॑ दद्यात्। वी॒र॒ह॒त्यां वा ए॒ते घ्न॑न्ति। ये ब्राह्म॒णास्त्रिसु॑पर्णं॒ पठ॑न्ति। ते सोमं॒ प्राप्नु॑वन्ति। आ॒स॒ह॒स्रात्प॒ङ्क्तिं पुन॑न्ति। ओम्॥५७॥
\closesection

हिर॑ण्यवर्णा॒ शुच॑यः पाव॒का यासु॑ जा॒तः क॒श्यपो॒ यास्विन्द्र॑।
अ॒ग्निं या गर्भं॑ दधि॒रे विरू॑पा॒स्ता न॒ आप॒ श स्यो॒ना भ॑वन्तु॥ 
यासा॒ राजा॒ वरु॑णो॒ याति॒ मध्ये॑ सत्यानृ॒ते अ॑व॒पश्यं॒ जना॑नाम्।
म॒धु॒श्चुत॒ शुच॑यो॒ याः पा॑व॒कास्ता न॒ आप॒ श स्यो॒ना भ॑वन्तु॥ 
यासां दे॒वा दि॒वि कृ॒ण्वन्ति॑ भ॒क्षं या अ॒न्तरि॑क्षे बहु॒धा भव॑न्ति।
याः पृ॑थि॒वीं पय॑सो॒न्दन्ति॑ शु॒क्रास्ता न॒ आप॒ श स्यो॒ना भ॑वन्तु॥ 
शि॒वेन॑ मा॒ चक्षु॑षा पश्यताऽऽपः शि॒वया॑ त॒नुवोप॑ स्पृशत॒ त्वचं॑ मे।
सर्वा अ॒ग्नी र॑प्सु॒षदो॑ हुवे वो॒ मयि॒ वर्चो॒ बल॒मोजो॒ नि ध॑त्त॥
\clearpage

\input{"../../../vedamantra-book/mantras/AghamarshanaSuktam.tex"} 
\input{"../../../vedamantra-book/mantras/RudraPrashnah.tex"} 
\clearpage
\input{"../../../vedamantra-book/mantras/ChamakaPrashnah.tex"} 
\input{"../../../vedamantra-book/mantras/PurushaSuktam.tex"} 
\clearpage
\input{"../../../vedamantra-book/mantras/NarayanaSuktam.tex"} 
\clearpage
\setmainfont[Script=Devanagari,Mapping=tex-text]{Sanskrit 2003}
%\fontsize{16pt}{20pt}\selectfont