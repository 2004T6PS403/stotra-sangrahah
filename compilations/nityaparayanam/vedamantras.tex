%! TeX program = XeLaTeX
%!TeX root = nityaparayanam-vedam.tex
%\part{वेदमन्त्राः}
\fontsize{19pt}{23pt}\selectfont
\newcounter{anuvakam}
\newcommand{\anuvakamend}{\refstepcounter{anuvakam}\\[-0.8ex]
\rule[0.5ex]{0.93\textwidth}{1.5pt}\bfseries{[\devanumber{\arabic{anuvakam}}]}%\hrulefill
}
\setlength{\emergencystretch}{3em}
\input{"../../../vedamantra-book/mantras/KriminashanaMantrah.tex"}
\clearpage
\input{"../../../vedamantra-book/mantras/MrtyunjayaHomaMantrah.tex"}
\clearpage
\input{"../../../vedamantra-book/mantras/Ganapatyatharvashirsham.tex"}
\clearpage
\input{"../../../vedamantra-book/mantras/DurgaSuktam.tex"} 
\clearpage
\input{"../../../vedamantra-book/mantras/MedhaSuktam.tex"} 
\closesection

\clearpage
\input{"../../../vedamantra-book/mantras/NavagrahaSuktam.tex"} 
\clearpage
\input{"../../../vedamantra-book/mantras/AyushyaSuktam.tex"} 
\clearpage
\chapt{वेदविस्मरणाय जपमन्त्राः}
\centerline{\normalsize (तैत्तिरीयारण्यकम्/प्रपाठकः – १०/अनुवाकः – ८–९)}
यश्छन्द॑सामृष॒भो वि॒श्वरू॑प॒श्छन्दोभ्य॒श्छन्दास्यावि॒वेश॑। सता शिक्यः पुरोवाचो॑पनि॒षदिन्द्रो ज्ये॒ष्ठ इ॑न्द्रि॒याय॒ ऋषि॑भ्यो॒ नमो॑ दे॒वेभ्य॑ स्व॒धा पि॒तृभ्यो॒ भूर्भुव॒ सुव॒श्छन्द॒ ओम्॥२२॥ नमो॒ ब्रह्म॑णे धा॒रणं॑ मे अ॒स्त्वनि॑राकरणं धा॒रयि॑ता भूयासं॒ कर्ण॑योः श्रु॒तं मा च्योढ्वं॒ ममा॒मुष्य॒ ओम्॥२३॥

\chapt{मेधा-विलास-सिद्ध्यर्थं जपमन्त्रः}
सद॑स॒स्पति॒मद्भु॑तं प्रि॒यमिन्द्र॑स्य॒ काम्यम्। सनिं॑ मे॒धाम॑यासिषम्।
\closesection

\chapt{ब्राह्मण-वेदविद्भ्यो नमस्काराः}
ये अ॒र्वाङु॒त वा॑ पुरा॒णे वे॒दं वि॒द्वास॑म॒भितो॑ वदन्त्यादि॒त्यमे॒व ते परि॑वदन्ति॒ सर्वे॑ अ॒ग्निं द्वि॒तीयं॑ तृ॒तीयं॑ च ह॒समिति॒ याव॑ती॒र्वै दे॒वता॒स्ताः सर्वा॑ वेद॒विदि॑ ब्राह्म॒णे व॑सन्ति॒ तस्माद्ब्राह्म॒णेभ्यो॑ वेद॒विद्भ्यो॑ दि॒वे दि॑वे॒ नम॑स्कुर्या॒न्नाश्ली॒लं कीर्तयेदे॒ता ए॒व दे॒वता प्रीणाति॥

\clearpage
\input{"../../../vedamantra-book/mantras/BhagyaSuktam.tex"} 
\clearpage
\input{"../../../vedamantra-book/mantras/PavamanaSuktam.tex"} 
\clearpage
\input{"../../../vedamantra-book/mantras/BhuSuktam.tex"} 
\clearpage
\input{"../../../vedamantra-book/mantras/ShriSuktam.tex"} 
\clearpage
\input{"../../../vedamantra-book/mantras/VishnuSuktam.tex"} 
\clearpage

\chapt{त्रिसुपर्णमन्त्राः}
\vspace{-1ex}
\centerline{\normalsize (तैत्तिरीयारण्यकम्/प्रपाठकः – १०/अनुवाकः – ३८–४०)}
ब्रह्म॑मेतु॒ माम्। मधु॑मेतु॒ माम्। ब्रह्म॑मे॒व मधु॑मेतु॒ माम्। यास्ते॑ सोम प्र॒जाव॒थ्सोभि॒ सो अ॒हम्। दुःस्व॑प्न॒हन्दु॑रुष्षह। यास्ते॑ सोम प्रा॒णाꣴस्तां जु॑होमि। त्रिसु॑पर्ण॒मया॑चितं ब्राह्म॒णाय॑ दद्यात्। ब्र॒ह्म॒ह॒त्यां वा ए॒ते घ्न॑न्ति। ये ब्रा᳚ह्म॒णास्त्रिसु॑पर्णं॒ पठ॑न्ति। ते सोमं॒ प्राप्नु॑वन्ति। आ॒स॒ह॒स्रात्प॒ङ्क्तिं पुन॑न्ति। ओम्॥५५॥

ब्रह्म॑ मे॒धया᳚। मधु॑ मे॒धया᳚। ब्रह्म॑मे॒व मधु॑ मे॒धया᳚। अ॒द्या नो॑ देव सवितः प्र॒जाव॑थ्सावीः॒ सौभ॑गम्। परा॑ दु॒ष्वप्नि॑यꣳ सुव। विश्वा॑नि देव सवितर्दुरि॒तानि॒ परा॑ सुव। यद्भ॒द्रं तन्म॒ आ सु॑व। मधु॒ वाता॑ ऋताय॒ते मधु॑ क्षरन्ति॒ सिन्ध॑वः। माध्वी᳚र्नः स॒न्त्वोष॑धीः। मधु॒ नक्त॑मु॒तोषसि॒ मधु॑म॒त्पार्थि॑व॒ꣳ॒ रजः॑। मधु॒ द्यौर॑स्तु नः पि॒ता। मधु॑मान्नो॒ वन॒स्पति॒र्मधु॑माꣳ अस्तु॒ सूर्यः॑। माध्वी॒र्गावो॑ भवन्तु नः। य इ॒मं त्रिसु॑पर्ण॒मया॑चितं ब्राह्म॒णाय॑ दद्यात्। भ्रू॒ण॒ह॒त्यां वा ए॒ते घ्न॑न्ति। ये ब्रा᳚ह्म॒णास्त्रिसु॑पर्णं॒ पठ॑न्ति। ते सोमं॒ प्राप्नु॑वन्ति। आ॒स॒ह॒स्रात्प॒ङ्क्तिं पुन॑न्ति। ओम्॥५६॥

ब्रह्म॑ मे॒धवा᳚। मधु॑ मे॒धवा᳚। ब्रह्म॑मे॒व मधु॑ मे॒धवा᳚। ब्र॒ह्मा दे॒वानां᳚ पद॒वीः क॑वी॒नामृषि॒र्विप्रा॑णां महि॒षो मृ॒गाणा᳚म्। श्ये॒नो गृध्रा॑णा॒ꣴ॒ स्वधि॑ति॒र्वना॑ना॒ꣳ॒ सोमः॑ प॒वित्र॒मत्ये॑ति॒ रेभ\sn{}। ह॒ꣳ॒सः शु॑चि॒षद्वसु॑रन्तरिक्ष॒सद्धोता॑ वेदि॒षदति॑थिर्दुरोण॒सत्। नृ॒षद्व॑र॒सदृ॑त॒सद्व्यो॑म॒सद॒ब्जा गो॒जा ऋ॑त॒जा अ॑द्रि॒जा ऋ॒तं बृ॒हत्। ऋ॒चे त्वा॑ रु॒चे त्वा॒ समिथ्स्र॑वन्ति स॒रितो॒ न धेनाः᳚। अ॒न्तर्\mbox{}हृ॒दा मन॑सा पू॒यमा॑नाः। घृ॒तस्य॒ धारा॑ अ॒भिचा॑कशीमि। हि॒र॒ण्ययो॑ वेत॒सो मध्य॑ आसाम्। \mbox{तस्मि ᳚\hspace{-1.25ex}न्थ्सु}\-प॒र्णो म॑धु॒कृत्कु॑ला॒यी भज॑न्नास्ते॒ मधु॑\-दे॒वता᳚भ्यः। तस्या॑सते॒ हर॑यः स॒प्ततीरे᳚ स्व॒धां दुहा॑ना अ॒मृत॑स्य॒ धारा᳚म्। य इ॒दं त्रिसु॑पर्ण॒मया॑चितं ब्राह्म॒णाय॑ दद्यात्। वी॒र॒ह॒त्यां वा ए॒ते घ्न॑न्ति। ये ब्रा᳚ह्म॒णास्त्रिसु॑पर्णं॒ पठ॑न्ति। ते सोमं॒ प्राप्नु॑वन्ति। आ॒स॒ह॒स्रात्प॒ङ्क्तिं पुन॑न्ति। ओम्॥५७॥

\closesection
\clearpage

\input{"../../../vedamantra-book/mantras/AghamarshanaSuktam.tex"} 
\closesection

हिर॑ण्यवर्णा॒ शुच॑यः पाव॒का यासु॑ जा॒तः क॒श्यपो॒ यास्विन्द्र॑।
अ॒ग्निं या गर्भं॑ दधि॒रे विरू॑पा॒स्ता न॒ आप॒ श स्यो॒ना भ॑वन्तु॥ 
यासा॒ राजा॒ वरु॑णो॒ याति॒ मध्ये॑ सत्यानृ॒ते अ॑व॒पश्यं॒ जना॑नाम्।
म॒धु॒श्चुत॒ शुच॑यो॒ याः पा॑व॒कास्ता न॒ आप॒ श स्यो॒ना भ॑वन्तु॥ 
यासां दे॒वा दि॒वि कृ॒ण्वन्ति॑ भ॒क्षं या अ॒न्तरि॑क्षे बहु॒धा भव॑न्ति।
याः पृ॑थि॒वीं पय॑सो॒न्दन्ति॑ शु॒क्रास्ता न॒ आप॒ श स्यो॒ना भ॑वन्तु॥ 
शि॒वेन॑ मा॒ चक्षु॑षा पश्यताऽऽपः शि॒वया॑ त॒नुवोप॑ स्पृशत॒ त्वचं॑ मे।
सर्वा अ॒ग्नी र॑फ्सु॒षदो॑ हुवे वो॒ मयि॒ वर्चो॒ बल॒मोजो॒ नि ध॑त्त॥
\clearpage
\input{"../../../vedamantra-book/mantras/VaishvadevaMantrah.tex"} 
\clearpage
\input{"../../../vedamantra-book/mantras/RudraPrashnah.tex"} 
\clearpage
\input{"../../../vedamantra-book/mantras/ChamakaPrashnah.tex"} 
\input{"../../../vedamantra-book/mantras/PurushaSuktam.tex"} 
\clearpage
\input{"../../../vedamantra-book/mantras/NarayanaSuktam.tex"} 
\clearpage
\setmainfont[Script=Devanagari,Mapping=tex-text]{Sanskrit 2003}
%\fontsize{16pt}{20pt}\selectfont