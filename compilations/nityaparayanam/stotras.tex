% !TeX program = XeLaTeX
% !TeX root = nityaparayanam-shloka.tex
\Large
%\part{स्तोत्राणि}
\sectionmark{\mbox{}}
\begin{center}

\sect{भागवतसमरणम्}

\uvacha{पाण्डवा ऊचुः}
\fourlineindentedshloka*
{प्रह्लाद-नारद-पराशर-पुण्डरीक}
{व्यासाम्बरीष-शुक-शौनक-भीष्म-दाल्भ्यान्}
{रुक्माङ्गदार्जुन-वसिष्ठ-विभीषणादीन्}
{पुण्यानिमान् परमभागवतान् समरामि}

\sect{प्रातः स्मरणम्}

\twolineshloka*
{कर्कोटकस्य नागस्य दमयन्त्या नलस्य च}
{ऋतुपर्णस्य राजर्षेः कीर्तनं कलिनाशनम्}

\sect{चिरञ्जीविस्तोत्रम्}
\fourlineindentedshloka*
{अश्वत्थामा बलिर्व्यासो हनुमांश्च विभीषणः}
{कृपः परशुरामश्च सप्तैते चिरञ्जीविनः}
{सप्तैतान् संस्मरेन्नित्यं मार्कण्डेयमथाष्टमम्}
{जीवेद्वर्षशतं प्राज्ञ अपमृत्युविवर्जितः}

\sect{पञ्चकन्यास्मरणम्}
\twolineshloka*
{अहल्या द्रौपदी सीता तारा मन्दोदरी तथा}
{पञ्चकन्याः स्मरेन्नित्यं महापातकनाशनम्}

\sect{अर्जुन-नामानि   (विराटपर्वान्तर्गतम्)}

\twolineshloka*
{अर्जुनः फाल्गुनो जिष्णुः किरीटी श्वेतवाहनः}
{बीभत्सुर्विजयः कृष्णः सव्यसाची धनञ्जयः}
% !TeX program = XeLaTeX
% !TeX root = ../../shloka.tex

\sect{सुन्दरहनुमान् महामन्त्रनामस्तोत्रम्}

\twolineshloka
{हनुमान् अञ्जनासूनुर्वायुपुत्रो महाबलः}
{कपीन्द्रः पिङ्गळाक्षश्च लङ्काद्वीपभयङ्करः}

\twolineshloka
{प्रभञ्जनसुतो वीरः सीताशोकविनाशकः}
{अक्षहन्ता रामसखो रामकार्यदुरन्धरः}

\twolineshloka
{महौषधगिरेर्धारी वानरप्राणदायकः}
{वारीशतारकश्चैव मैनाकगिरिभञ्जनः}

\twolineshloka
{निरञ्जनो जितक्रोधो कदळीवनसंवृतः}
{ऊर्ध्वरेता महासत्त्वः सर्वमन्त्रप्रवर्तकः}

\twolineshloka
{महालिङ्गप्रतिष्ठाता बाष्पकृज्जपतान्तरः}
{शिवध्यानपरो नित्यं शिवपूजापरायणः}


॥इति श्री पराशरसंहितान्तर्गतं श्री सुन्दरहनुमान् महामन्त्रनामस्तोत्रं सम्पूर्णम्॥

\sect{गायत्री-ध्यानम्}
\fourlineindentedshloka*
{मुक्ता-विद्रुम-हेम-नील-धवळच्छायैर्मुखैस्त्र्यक्षणैः}
{युक्तामिन्दु-निबद्ध-रत्न-मकुटां तत्त्वार्थ-वर्णात्मिकाम्}
{गायत्रीं वरदाभयाङ्कुशकशाः शुभ्रं कपालं गदाम्}
{शङ्खं चक्रमथारविन्दयुगलं हस्तैर्वहन्तीं भजे}


% !TeX program = XeLaTeX
% !TeX root = ../../shloka.tex

\sect{आदित्यहृदयम्}
\twolineshloka
{ततो युद्धपरिश्रान्तं समरे चिन्तया स्थितम्}
{रावणं चाग्रतो दृष्ट्वा युद्धाय समुपस्थितम्}

\twolineshloka
{दैवतैश्च समागम्य द्रष्टुमभ्यागतो रणम्}
{उपागम्याब्रवीद्रामम् अगस्त्यो भगवान् ऋषिः}

\twolineshloka
{राम राम महाबाहो शृणु गुह्यं सनातनम्}
{येन सर्वानरीन् वत्स समरे विजयिष्यसि}

\twolineshloka
{आदित्यहृदयं पुण्यं सर्वशत्रुविनाशनम्}
{जयावहं जपेन्नित्यम् अक्षय्यं परमं शिवम्}

\twolineshloka
{सर्वमङ्गलमाङ्गल्यं सर्वपापप्रणाशनम्}
{चिन्ताशोकप्रशमनम् आयुर्वर्धनमुत्तमम्}

\twolineshloka
{रश्मिमन्तं समुद्यन्तं देवासुरनमस्कृतम्}
{पूजयस्व विवस्वन्तं भास्करं भुवनेश्वरम्}

\twolineshloka
{सर्वदेवात्मको ह्येष तेजस्वी रश्मिभावनः}
{एष देवासुरगणान् लोकान् पाति गभस्तिभिः}

\twolineshloka
{एष ब्रह्मा च विष्णुश्च शिवः स्कन्दः प्रजापतिः}
{महेन्द्रो धनदः कालो यमः सोमो ह्यपां पतिः}

\twolineshloka
{पितरो वसवः साध्या ह्यश्विनौ मरुतो मनुः}
{वायुर्वह्निः प्रजाप्राण ऋतुकर्ता प्रभाकरः}

\twolineshloka
{आदित्यः सविता सूर्यः खगः पूषा गभस्तिमान्}
{सुवर्णसदृशो भानुर्हिरण्यरेता दिवाकरः}

\twolineshloka
{हरिदश्वः सहस्रार्चिः सप्तसप्तिर्मरीचिमान्}
{तिमिरोन्मथनः शम्भुस्त्वष्टा मार्तण्ड अंशुमान्}

\twolineshloka
{हिरण्यगर्भः शिशिरस्तपनो भास्करो रविः}
{अग्निगर्भोऽदितेः पुत्रः शङ्खः शिशिरनाशनः}

\twolineshloka
{व्योमनाथस्तमोभेदी ऋग्यजुस्सामपारगः}
{घनवृष्टिरपां मित्रो विन्ध्यवीथीप्लवङ्गमः}

\twolineshloka
{आतपी मण्डली मृत्युः पिङ्गलः सर्वतापनः}
{कविर्विश्वो महातेजा रक्तः सर्वभवोद्भवः}

\twolineshloka
{नक्षत्रग्रहताराणाम् अधिपो विश्वभावनः}
{तेजसामपि तेजस्वी द्वादशात्मन् नमोऽस्तु ते}

\twolineshloka
{नमः पूर्वाय गिरये पश्चिमायाद्रये नमः}
{ज्योतिर्गणानां पतये दिनाधिपतये नमः}

\twolineshloka
{जयाय जयभद्राय हर्यश्वाय नमो नमः}
{नमो नमः सहस्रांशो आदित्याय नमो नमः}

\twolineshloka
{नम उग्राय वीराय सारङ्गाय नमो नमः}
{नमः पद्मप्रबोधाय मार्तण्डाय नमो नमः}

\twolineshloka
{ब्रह्मेशानाच्युतेशाय सूर्यायादित्यवर्चसे}
{भास्वते सर्वभक्षाय रौद्राय वपुषे नमः}

\twolineshloka
{तमोघ्नाय हिमघ्नाय शत्रुघ्नायामितात्मने}
{कृतघ्नघ्नाय देवाय ज्योतिषां पतये नमः}

\twolineshloka
{तप्तचामीकराभाय वह्नये विश्वकर्मणे}
{नमस्तमोऽभिनिघ्नाय रुचये लोकसाक्षिणे}

\twolineshloka
{नाशयत्येष वै भूतं तदेव सृजति प्रभुः}
{पायत्येष तपत्येष वर्षत्येष गभस्तिभिः}

\twolineshloka
{एष सुप्तेषु जागर्ति भूतेषु परिनिष्ठितः}
{एष एवाग्निहोत्रं च फलं चैवाग्निहोत्रिणाम्}

\twolineshloka
{वेदाश्च क्रतवश्चैव क्रतूनां फलमेव च}
{यानि कृत्यानि लोकेषु सर्व एष रविः प्रभुः}

%\dnsub{फलश्रुतिः}
\twolineshloka
{एनमापत्सु कृच्छ्रेषु कान्तारेषु भयेषु च}
{कीर्तयन् पुरुषः कश्चिन्नावसीदति राघव}

\twolineshloka
{पूजयस्वैनमेकाग्रो देवदेवं जगत्पतिम्}
{एतत् त्रिगुणितं जप्त्वा युद्धेषु विजयिष्यसि}

\twolineshloka
{अस्मिन् क्षणे महाबाहो रावणं त्वं वधिष्यसि}
{एवमुक्त्वा तदाऽगस्त्यो जगाम च यथाऽऽगतम्}

\twolineshloka
{एतच्छ्रुत्वा महातेजा नष्टशोकोऽभवत्तदा}
{धारयामास सुप्रीतो राघवः प्रयतात्मवान्}

\twolineshloka
{आदित्यं प्रेक्ष्य जप्त्वा तु परं हर्षमवाप्तवान्}
{त्रिराचम्य शुचिर्भूत्वा धनुरादाय वीर्यवान्}

\twolineshloka
{रावणं प्रेक्ष्य हृष्टात्मा युद्धाय समुपागमत्}
{सर्वयत्नेन महता वधे तस्य धृतोऽभवत्}

\twolineshloka
{अथ रविरवदन्निरीक्ष्य रामं मुदितमनाः परमं प्रहृष्यमाणः}
{निशिचरपतिसङ्क्षयं विदित्वा सुरगणमध्यगतो वचस्त्वरेति}
॥इत्यार्षे श्रीमद्रामायणे वाल्मीकीये आदिकाव्ये युद्धकाण्डे आदित्यहृदयं नाम सप्तोत्तरशततमः सर्गः॥
% !TeX program = XeLaTeX
% !TeX root = ../../shloka.tex

\sect{षष्ठीदेवी स्तोत्रम्}
\centerline{प्रियव्रत उवाच}
\twolineshloka
{नमो देव्यै महादेव्यै सिद्‍ध्यै शान्त्यै नमो नमः}
{शुभायै देवसेनायै षष्ठीदेव्यै नमो नमः}%१॥

\twolineshloka
{वरदायै पुत्रदायै धनदायै नमो नमः}
{सुखदायै मोक्षदायै षष्ठीदेव्यै नमो नमः}%२॥

\twolineshloka
{शक्तेः षष्ठांशरूपायै सिद्धायै च नमो नमः}
{मायायै सिद्धयोगिन्यै षष्ठीदेव्यै नमो नमः}%३॥

\twolineshloka
{पारायै पारदायै च षष्ठीदेव्यै नमो नमः}
{सारायै सारदायै च पारायै सर्वकर्मणाम्}%४॥

\threelineshloka
{बालाधिष्ठातृदेव्यै च षष्ठीदेव्यै नमो नमः}
{कल्याणदायै कल्याण्यै फलदायै च कर्मणाम्}
{प्रत्यक्षायै च भक्तानां षष्ठीदेव्यै नमो नमः}%५॥

\twolineshloka
{पूज्यायै स्कन्दकान्तायै सर्वेषां सर्वकर्मसु}
{देवरक्षणकारिण्यै षष्ठीदेव्यै नमो नमः}%६॥

\twolineshloka
{शुद्धसत्त्वस्वरूपायै वन्दितायै नृणां सदा}
{हिंसाक्रोधैर्वर्जितायै षष्ठीदेव्यै नमो नमः}%७॥

\twolineshloka
{धनं देहि प्रियां देहि पुत्रं देहि सुरेश्वरि}
{धर्मं देहि यशो देहि षष्ठीदेव्यै नमो नमः}%८॥

\twolineshloka
{भूमिं देहि प्रजां देहि देहि विद्यां सुपूजिते}
{कल्याणं च जयं देहि षष्ठीदेव्यै नमो नमः}%९॥

\twolineshloka
{इति देवीं च संस्तूय लेभे पुत्रं प्रियव्रतः}
{यशस्विनं च राजेन्द्रं षष्ठीदेवीप्रसादतः}%

\twolineshloka
{षष्ठीस्तोत्रमिदं ब्रह्मण् यः शृणोति च वत्सरम्}
{अपुत्रो लभते पुत्रं वरं सुचिरजीवनम्}

\twolineshloka
{वर्षमेकं च या भक्त्या संयतेदं शृणोति च}
{सर्वपापाद्विनिर्मुक्ता महावन्ध्या प्रसूयते}

\twolineshloka
{वीरपुत्रं च गुणिनं विद्यावन्तं यशस्विनम्}
{सुचिरायुष्मन्तमेव षष्ठीमातृप्रसादतः}

\twolineshloka
{काकवन्ध्या च या नारी मृतापत्या च या भवेत्}
{वर्षं श्रुत्वा लभेत्पुत्रं षष्ठीदेवीप्रसादतः}

\twolineshloka
{रोगयुक्ते च बाले च पिता माता शृणोति च}
{मासं  च मुच्यते बालः षष्ठीदेवीप्रसादतः}

{॥इति~श्रीब्रह्मवैवर्तमहापुराणे~प्रकृतिखण्डे श्री~नारद-नारायण-संवादे षष्ठ्युपाख्याने श्री~प्रियव्रतविरचितं श्री~षष्ठीदेवीस्तोत्रं सम्पूर्णम्॥}
% !TeX program = XeLaTeX
% !TeX root = ../../shloka.tex

\sect{बालरक्षा}

\twolineshloka*
{नमोऽस्तु ते व्यास विशालबुद्धे फुल्लारविन्दायतपत्रनेत्र}
{येन त्वया भारततैलपूर्णः प्रज्वालितो ज्ञानमयः प्रदीपः}

\fourlineindentedshloka*
{कस्तूरीतिलकं ललाटपटले वक्षःस्थले कौस्तुभम्}
{नासाग्रे वरमौक्तिकं करतले वेणुः करे कङ्कणम्}
{सर्वाङ्गे हरिचन्दनं सुललितं कण्ठे च मुक्तावली}
{गोपस्त्रीपरिवेष्टितो विजयते गोपालचूडामणिः}

\fourlineindentedshloka*
{अस्ति स्वस्तरुणीकराग्रविगलत् कल्पप्रसूनाप्लुतम्}
{वस्तुप्रस्तुतवेणुनादलहरी निर्वाणनिर्व्याकुलम्}
{स्रस्तस्रस्तनिबद्धनीविविलसत् गोपीसहस्रावृतम्}
{हस्तन्यस्तनतापवर्गमखिलोदारं किशोराकृति}

गोप्य ऊचुः

\fourlineindentedshloka
{अव्यादजोऽङ्घ्रि मणिमांस्तव जान्वथोरू}
{यज्ञोऽच्युतः कटितटं जठरं हयास्यः}
{हृत्केशवस्त्वदुर ईश इनस्तु कण्ठम्}
{विष्णुर्भुजं मुखमुरुक्रम ईश्वरः कम्}

\fourlineindentedshloka
{चक्र्यग्रतः सहगदो हरिरस्तु पश्चात्}
{त्वत्पार्श्वयोर्धनुरसी मधुहाऽजनश्च}
{कोणेषु शङ्ख उरुगाय उपर्युपेन्द्रः}
{तार्क्ष्यः क्षितौ हलधरः पुरुषः समन्तात्}

\twolineshloka
{इन्द्रियाणि हृषीकेशः प्राणान्नारायणोऽवतु}
{श्वेतद्वीपपतिश्चित्तं मनो योगेश्वरोऽवतु}

\twolineshloka
{पृश्निगर्भस्तु ते बुद्धिमात्मानं भगवान्परः}
{क्रीडन्तं पातु गोविन्दः शयानं पातु माधवः}

\twolineshloka
{व्रजन्तमव्याद्वैकुण्ठ आसीनं त्वां श्रियः पतिः}
{भुञ्जानं यज्ञभुक्पातु सर्वग्रहभयङ्करः}

\twolineshloka
{डाकिन्यो यातुधान्यश्च कुष्माण्डा येऽर्भकग्रहाः}
{भूतप्रेतपिशाचाश्च यक्षरक्षोविनायकाः}

\twolineshloka
{कोटरा रेवती ज्येष्ठा पूतना मातृकादयः}
{उन्मादा ये ह्यपस्मारा देहप्राणेन्द्रियद्रुहः}

\twolineshloka
{स्वप्नदृष्टा महोत्पाता वृद्धा बालग्रहाश्च ये}
{सर्वे नश्यन्तु ते विष्णोर्नामग्रहणभीरवः}

{॥इति श्रीमद्भागवते महापुराणे पारमहंस्यां संहितायां दशमस्कन्धे षष्ठमेऽध्याये गोपीकृतबालरक्षा सम्पूर्णा॥}
% !TeX program = XeLaTeX
% !TeX root = ../../shloka.tex

\sect{प्रज्ञाविवर्धन-कार्त्तिकेय-स्तोत्रम्}

\uvacha{स्कन्द उवाच}
\twolineshloka
{योगीश्वरो महासेनः कार्त्तिकेयोऽग्निनन्दनः}
{स्कन्दः कुमारः सेनानीः स्वामी शङ्करसम्भवः}

\twolineshloka
{गाङ्गेयस्ताम्रचूडश्च ब्रह्मचारी शिखिध्वजः}
{तारकारिरुमापुत्रः क्रौञ्चारिश्च षडाननः}

\twolineshloka
{शब्दब्रह्मसमुद्रश्च सिद्धः सारस्वतो गुहः}
{सनत्कुमारो भगवान् भोगमोक्षफलप्रदः}

\twolineshloka
{शरजन्मा गणाधीशपूर्वजो मुक्तिमार्गकृत्}
{सर्वागमप्रणेता च वाञ्छितार्थप्रदर्शनः}

\twolineshloka
{अष्टाविंशतिनामानि मदीयानीति यः पठेत्}
{प्रत्यूषं श्रद्धया युक्तो मूको वाचस्पतिर्भवेत्}

\twolineshloka
{महामन्त्रमयानीति मम नामानुकीर्तनम्}
{महाप्रज्ञामवाप्नोति नात्र कार्या विचारणा}
॥इति~श्री~रुद्रयामले~प्रज्ञाविवर्धनाख्यं श्रीमत्कार्तिकेयस्तोत्रं~सम्पूर्णम्॥
\sect{श्रीमद्रामायण-ध्यानम्}

\twolineshloka*
{शुक्लाम्बरधरं विष्णुं शशिवर्णं चतुर्भुजम्}
{प्रसन्नवदनं ध्यायेत् सर्वविघ्नोपशान्तये}

\twolineshloka*
{वागीशाद्याः सुमनसः सर्वार्थानामुपक्रमे}
{यं नत्वा कृतकृत्याः स्युस्तं नमामि गजाननम्}

\dnsub{श्री~गुरु प्रार्थना}

\twolineshloka*
{गुरुर्ब्रह्मा गुरुर्विष्णुर्गुरुर्देवो महेश्वरः}
{गुरुः साक्षात् परं ब्रह्म तस्मै श्री गुरवे नमः}

\twolineshloka*
{सदाशिवसमारम्भां शङ्कराचार्यमध्यमाम्}
{अस्मदाचार्यपर्यन्तां वन्दे गुरुपरम्पराम्}

\twolineshloka*
{अखण्डमण्डलाकारं व्याप्तं येन चराचरम्}
{तत्पदं दर्शितं येन तस्मै श्री गुरवे नमः}

\dnsub{श्री~सरस्वती प्रार्थना}
\fourlineindentedshloka*
{दोर्भिर्युक्ता चतुर्भिः स्फटिकमणिनिभैरक्षमालां दधाना}
{हस्तेनैकेन पद्मं सितमपि च शुकं पुस्तकं चापरेण}
{भासा कुन्देन्दुशङ्खस्फटिकमणिनिभा भासमानाऽसमाना}
{सा मे वाग्देवतेयं निवसतु वदने सर्वदा सुप्रसन्ना}

\dnsub{श्री~वाल्मीकि नमस्क्रिया}
\twolineshloka
{कूजन्तं राम रामेति मधुरं मधुराक्षरम्}
{आरुह्य कविताशाखां वन्दे वाल्मीकिकोकिलम्}

\twolineshloka
{वाल्मीकेर्मुनिसिंहस्य कवितावनचारिणः}
{शृण्वन् रामकथानादं को न याति परां गतिम्}

\twolineshloka
{यः पिबन् सततं रामचरितामृतसागरम्}
{अतृप्तस्तं मुनिं वन्दे प्राचेतसमकल्मषम्}

\resetShloka
\dnsub{श्री~हनुमन्नमस्क्रिया}

\twolineshloka
{गोष्पदीकृत-वाराशिं मशकीकृत-राक्षसम्}
{रामायण-महामाला-रत्नं वन्देऽनिलात्मजम्}

\twolineshloka
{अञ्जनानन्दनं वीरं जानकीशोकनाशनम्}
{कपीशमक्षहन्तारं वन्दे लङ्काभयङ्करम्}

\twolineshloka
{उल्लङ्घ्य सिन्धोः सलिलं सलीलं यः शोकवह्निं जनकात्मजायाः}
{आदाय तेनैव ददाह लङ्कां नमामि तं प्राञ्जलिराञ्जनेयम्}

\twolineshloka
{आञ्जनेयमतिपाटलाननं काञ्चनाद्रि-कमनीय-विग्रहम्}
{पारिजात-तरुमूल-वासिनं भावयामि पवमान-नन्दनम्}

\twolineshloka
{यत्र यत्र रघुनाथकीर्तनं तत्र तत्र कृतमस्तकाञ्जलिम्}
{बाष्पवारिपरिपूर्णलोचनं मारुतिं नमत राक्षसान्तकम्}

\twolineshloka
{मनोजवं मारुततुल्यवेगं जितेन्द्रियं बुद्धिमतां वरिष्ठम्}
{वातात्मजं वानरयूथमुख्यं श्रीरामदूतं शिरसा नमामि}

\mbox{}\\
\resetShloka
\dnsub{श्री~रामायणप्रार्थना}

\fourlineindentedshloka
{यः कर्णाञ्जलिसम्पुटैरहरहः सम्यक् पिबत्यादरात्}
{वाल्मीकेर्वदनारविन्दगलितं रामायणाख्यं मधु}
{जन्म-व्याधि-जरा-विपत्ति-मरणैरत्यन्त-सोपद्रवम्}
{संसारं स विहाय गच्छति पुमान् विष्णोः पदं शाश्वतम्}

\twolineshloka
{तदुपगत-समास-सन्धियोगं सममधुरोपनतार्थ-वाक्यबद्धम्}
{रघुवरचरितं मुनिप्रणीतं दशशिरसश्च वधं निशामयध्वम्}

\twolineshloka
{वाल्मीकि-गिरिसम्भूता रामसागरगामिनी}
{पुनातु भुवनं पुण्या रामायणमहानदी}

\twolineshloka
{श्लोकसारजलाकीर्णं सर्गकल्लोलसङ्कुलम्}
{काण्डग्राहमहामीनं वन्दे रामायणार्णवम्}

\twolineshloka
{वेदवेद्ये परे पुंसि जाते दशरथात्मजे}
{वेदः प्राचेतसादासीत् साक्षाद्रामायणात्मना}

\resetShloka
\dnsub{श्री~रामध्यानम्}

\fourlineindentedshloka
{वैदेहीसहितं सुरद्रुमतले हैमे महामण्डपे}
{मध्ये पुष्पकमासने मणिमये वीरासने सुस्थितम्}
{अग्रे वाचयति प्रभञ्जनसुते तत्त्वं मुनिभ्यः परम्}
{व्याख्यान्तं भरतादिभिः परिवृतं रामं भजे श्यामलम्}

\fourlineindentedshloka
{वामे भूमिसुता पुरश्च हनुमान् पश्चात् सुमित्रासुतः}
{शत्रुघ्नो भरतश्च पार्श्वदलयोर्वाय्वादिकोणेषु च}
{सुग्रीवश्च विभीषणश्च युवराट् तारासुतो जाम्बवान्}
{मध्ये नीलसरोजकोमलरुचिं रामं भजे श्यामलम्}

\twolineshloka
{रामं रामानुजं सीतां भरतं भरतानुजम्}
{सुग्रीवं वायुसूनुं च प्रणमामि पुनः पुनः}

\twolineshloka
{नमोऽस्तु रामाय सलक्ष्मणाय देव्यै च तस्यै जनकात्मजायै}
{नमोऽस्तु रुद्रेन्द्रयमानिलेभ्यो नमोऽस्तु चन्द्रार्कमरुद्गणेभ्यः}

\centerline{\textbf{ॐ श्री गुरुभ्यो नमः।}}

\resetShloka
\dnsub{मङ्गलश्लोकाः}
\fourlineindentedshloka
{स्वस्ति प्रजाभ्यः परिपालयन्ताम्}
{न्यायेन मार्गेण महीं महीशाः}
{गोब्राह्मणेभ्यः शुभमस्तु नित्यम्}
{लोकाः समस्ताः सुखिनो भवन्तु}

\twolineshloka
{काले वर्षतु पर्जन्यः पृथिवी सस्यशालिनी}
{देशोऽयं क्षोभरहितो ब्राह्मणाः सन्तु निर्भयाः}

\twolineshloka
{अपुत्राः पुत्रिणः सन्तु पुत्रिणः सन्तु पौत्रिणः}
{अधनाः सधनाः सन्तु जीवन्तु शरदां शतम्}

\twolineshloka
{चरितं रघुनाथस्य शतकोटि-प्रविस्तरम्}
{एकैकमक्षरं पुंसां महापातकनाशनम्}

\twolineshloka
{शृण्वन् रामायणं भक्त्या यः पादं पदमेव वा}
{स याति ब्रह्मणः स्थानं ब्रह्मणा पूज्यते सदा}

\twolineshloka
{रामाय रामभद्राय रामचन्द्राय वेधसे}
{रघुनाथाय नाथाय सीतायाः पतये नमः}

\twolineshloka
{यन्मङ्गलं सहस्राक्षे सर्वदेवनमस्कृते}
{वृत्रनाशे समभवत् तत्ते भवतु मङ्गलम्}

\twolineshloka
{यन्मङ्गलं सुपर्णस्य विनताऽकल्पयत् पुरा}
{अमृतं प्रार्थयानस्य तत्ते भवतु मङ्गलम्}

\twolineshloka
{अमृतोत्पादने दैत्यान् घ्नतो वज्रधरस्य यत्}
{अदितिर्मङ्गलं प्रादात् तत्ते भवतु मङ्गलम्}

\twolineshloka
{त्रीन् विक्रमान् प्रक्रमतो विष्णोरमिततेजसः}
{यदासीन्मङ्गलं राम तत्ते भवतु मङ्गलम्}

\twolineshloka
{ऋषयः सागरा द्वीपा वेदा लोका दिशश्च ते}
{मङ्गलानि महाबाहो दिशन्तु तव सर्वदा}

\twolineshloka
{मङ्गलं कोसलेन्द्राय महनीयगुणाब्धये}
{चक्रवर्तितनूजाय सार्वभौमाय मङ्गलम्}

\fourlineindentedshloka*
{कायेन वाचा मनसेन्द्रियैर्वा}
{बुद्‌ध्याऽऽत्मना वा प्रकृतेः स्वभावात्}
{करोमि यद्यत् सकलं परस्मै}
{नारायणायेति समर्पयामि}

% !TeX program = XeLaTeX
% !TeX root = ../../shloka.tex

%\chapter[विष्णुसहस्रनामस्तोत्रम्]{॥विष्णुसहस्रनामस्तोत्रम्॥}
\sect{विष्णुसहस्रनामस्तोत्रम्}
\twolineshloka
{शुक्लाम्बरधरं विष्णुं शशिवर्णं चतुर्भुजम्}
{प्रसन्नवदनं ध्यायेत् सर्वविघ्नोपशान्तये}

\twolineshloka
{यस्य द्विरदवक्त्राद्याः पारिषद्याः परः शतम्}
{विघ्नं निघ्नन्ति सततं विष्वक्सेनं तमाश्रये}

\twolineshloka
{नारायणं नमस्कृत्य नरं चैव नरोत्तमम्}
{देवीं सरस्वतीं व्यासं ततो जयमुदीरयेत्}

\twolineshloka
{व्यासं वसिष्ठनप्तारं शक्तेः पौत्रमकल्मषम्}
{पराशरात्मजं वन्दे शुकतातं तपोनिधिम्}

\twolineshloka
{व्यासाय विष्णुरूपाय व्यासरूपाय विष्णवे}
{नमो वै ब्रह्मनिधये वासिष्ठाय नमो नमः}

\twolineshloka
{अविकाराय शुद्धाय नित्याय परमात्मने}
{सदैकरूपरूपाय विष्णवे सर्वजिष्णवे}

\twolineshloka
{यस्य स्मरणमात्रेण जन्मसंसारबन्धनात्}
{विमुच्यते नमस्तस्मै विष्णवे प्रभविष्णवे}

\centerline{ॐ नमो विष्णवे प्रभविष्णवे}

\uvacha{श्री वैशम्पायन उवाच}
\twolineshloka
{श्रुत्वा धर्मानशेषेण पावनानि च सर्वशः}
{युधिष्ठिरः शान्तनवं पुनरेवाभ्यभाषत}

\uvacha{श्री युधिष्ठिर उवाच}
\twolineshloka
{किमेकं दैवतं लोके किं वाऽप्येकं परायणम्}
{स्तुवन्तः कं कमर्चन्तः प्राप्नुयुर्मानवाः शुभम्}

\twolineshloka
{को धर्मः सर्वधर्माणां भवतः परमो मतः}
{किं जपन् मुच्यते जन्तुर्जन्मसंसारबन्धनात्}

\uvacha{श्री भीष्म उवाच}
\twolineshloka
{जगत्प्रभुं देवदेवमनन्तं पुरुषोत्तमम्}
{स्तुवन् नामसहस्रेण पुरुषः सततोत्थितः}

\twolineshloka
{तमेव चार्चयन्नित्यं भक्त्या पुरुषमव्ययम्}
{ध्यायन् स्तुवन् नमस्यंश्च यजमानस्तमेव च}

\twolineshloka
{अनादिनिधनं विष्णुं सर्वलोकमहेश्वरम्}
{लोकाध्यक्षं स्तुवन्नित्यं सर्वदुःखातिगो भवेत्}

\twolineshloka
{ब्रह्मण्यं सर्वधर्मज्ञं लोकानां कीर्तिवर्धनम्}
{लोकनाथं महद्भूतं सर्वभूतभवोद्भवम्}

\twolineshloka
{एष मे सर्वधर्माणां धर्मोऽधिकतमो मतः}
{यद्भक्त्या पुण्डरीकाक्षं स्तवैरर्चेन्नरः सदा}

\twolineshloka
{परमं यो महत्तेजः परमं यो महत्तपः}
{परमं यो महद्ब्रह्म परमं यः परायणम्}

\twolineshloka
{पवित्राणां पवित्रं यो मङ्गलानां च मङ्गलम्}
{दैवतं दैवतानां च भूतानां योऽव्ययः पिता}

\twolineshloka
{यतः सर्वाणि भूतानि भवन्त्यादियुगागमे}
{यस्मिंश्च प्रलयं यान्ति पुनरेव युगक्षये}

\twolineshloka
{तस्य लोकप्रधानस्य जगन्नाथस्य भूपते}
{विष्णोर्नामसहस्रं मे शृणु पापभयापहम्}

\twolineshloka
{यानि नामानि गौणानि विख्यातानि महात्मनः}
{ऋषिभिः परिगीतानि तानि वक्ष्यामि भूतये}

\twolineshloka
{ऋषिर्नाम्नां सहस्रस्य वेदव्यासो महामुनिः}
{छन्दोऽनुष्टुप् तथा देवो भगवान् देवकीसुतः}

\twolineshloka
{अमृतांशूद्भवो बीजं शक्तिर्देवकिनन्दनः}
{त्रिसामा हृदयं तस्य शान्त्यर्थे विनियुज्यते}

\twolineshloka
{विष्णुं जिष्णुं महाविष्णुं प्रभविष्णुं महेश्वरम्}
{अनेकरूपदैत्यान्तं नमामि पुरुषोत्तमम्}

\dnsub{पूर्वन्यासः}
अस्य श्रीविष्णोर्दिव्यसहस्रनामस्तोत्रमहामन्त्रस्य।\\
श्री वेदव्यासो भगवान् ऋषिः। अनुष्टुप् छन्दः।\\
श्रीमहाविष्णुः परमात्मा श्रीमन्नारायणो देवता।\\
अमृतांशूद्भवो भानुरिति बीजम्। देवकीनन्दनः स्रष्टेति शक्तिः।\\
उद्भवः क्षोभणो देव इति परमो मन्त्रः।\\
शङ्खभृन्नन्दकी चक्रीति कीलकम्।\\
शार्ङ्गधन्वा गदाधर इत्यस्त्रम्। \\
रथाङ्गपाणिरक्षोभ्य इति नेत्रम्। \\
त्रिसामा सामगः सामेति कवचम्।\\
आनन्दं परब्रह्मेति योनिः।\\
ऋतुः सुदर्शनः काल इति दिग्बन्धः। \\
श्रीविश्वरूप इति ध्यानम्।\\
श्रीमहाविष्णुप्रीत्यर्थे सहस्रनामजपे विनियोगः॥\\

\resetShloka
\setlength{\shlokaspaceskip}{12pt}
\dnsub{ध्यानम्}
\fourlineindentedshloka
{क्षीरोदन्वत्प्रदेशे शुचिमणिविलसत्सैकतेर्मौक्तिकानाम्}
{मालाकॢप्तासनस्थः स्फटिकमणिनिभैर्मौक्तिकैर्मण्डिताङ्गः}
{शुभ्रैरभ्रैरदभ्रैरुपरिविरचितैर्मुक्तपीयूषवर्षैः}
{आनन्दी नः पुनीयादरिनलिनगदाशङ्खपाणिर्मुकुन्दः}

\fourlineindentedshloka
{भूः पादौ यस्य नाभिर्वियदसुरनिलश्चन्द्रसूर्यौ च नेत्रे}
{कर्णावाशाः शिरो द्यौर्मुखमपि दहनो यस्य वास्तेयमब्धिः}
{अन्तःस्थं यस्य विश्वं सुरनरखगगोभोगिगन्धर्वदैत्यैः}
{चित्रं रंरम्यते तं त्रिभुवनवपुषं विष्णुमीशं नमामि}

\centerline{ॐ नमो भगवते वासुदेवाय}
\fourlineindentedshloka
{शान्ताकारं भुजगशयनं पद्मनाभं सुरेशम्}
{विश्वाधारं गगनसदृशं मेघवर्णं शुभाङ्गम्}
{लक्ष्मीकान्तं कमलनयनं योगिहृद्‌ध्यानगम्यम्}
{वन्दे विष्णुं भवभयहरं सर्वलोकैकनाथम्}

\fourlineindentedshloka
{मेघश्यामं पीतकौशेयवासम्}
{श्रीवत्साङ्कं कौस्तुभोद्भासिताङ्गम्}
{पुण्योपेतं पुण्डरीकायताक्षम्}
{विष्णुं वन्दे सर्वलोकैकनाथम्}

\twolineshloka
{नमः समस्तभूतानामादिभूताय भूभृते}
{अनेकरूपरूपाय विष्णवे प्रभविष्णवे}

\fourlineindentedshloka
{सशङ्खचक्रं सकिरीटकुण्डलम्}
{सपीतवस्त्रं सरसीरुहेक्षणम्}
{सहारवक्षःस्थलशोभिकौस्तुभम्}
{नमामि विष्णुं शिरसा चतुर्भुजम्}

\fourlineindentedshloka
{छायायां पारिजातस्य हेमसिंहासनोपरि}
{आसीनमम्बुदश्याममायताक्षमलङ्कृतम्}
{चन्द्राननं चतुर्बाहुं श्रीवत्साङ्कितवक्षसम्}
{रुक्मिणीसत्यभामाभ्यां सहितं कृष्णमाश्रये}

\dnsub{हरिः ॐ}
\centerline{॥विश्वस्मै नमः॥}\nopagebreak[4]
\resetShloka
\twolineshloka
{विश्वं विष्णुर्वषट्कारो भूतभव्यभवत्प्रभुः}
{भूतकृद्भूतभृद्भावो भूतात्मा भूतभावनः}

\twolineshloka
{पूतात्मा परमात्मा च मुक्तानां परमा गतिः}
{अव्ययः पुरुषः साक्षी क्षेत्रज्ञोऽक्षर एव च}

\twolineshloka
{योगो योगविदां नेता प्रधानपुरुषेश्वरः}
{नारसिंहवपुः श्रीमान् केशवः पुरुषोत्तमः}

\twolineshloka
{सर्वः शर्वः शिवः स्थाणुर्भूतादिर्निधिरव्ययः}
{सम्भवो भावनो भर्ता प्रभवः प्रभुरीश्वरः}

\twolineshloka
{स्वयम्भूः शम्भुरादित्यः पुष्कराक्षो महास्वनः}
{अनादिनिधनो धाता विधाता धातुरुत्तमः}

\twolineshloka
{अप्रमेयो हृषीकेशः पद्मनाभोऽमरप्रभुः}
{विश्वकर्मा मनुस्त्वष्टा स्थविष्ठः स्थविरो ध्रुवः}

\twolineshloka
{अग्राह्यः शाश्वतः कृष्णो लोहिताक्षः प्रतर्दनः}
{प्रभूतस्त्रिककुब्धाम पवित्रं मङ्गलं परम्}

\twolineshloka
{ईशानः प्राणदः प्राणो ज्येष्ठः श्रेष्ठः प्रजापतिः}
{हिरण्यगर्भो भूगर्भो माधवो मधुसूदनः}

\twolineshloka
{ईश्वरो विक्रमी धन्वी मेधावी विक्रमः क्रमः}
{अनुत्तमो दुराधर्षः कृतज्ञः कृतिरात्मवान्}

\twolineshloka
{सुरेशः शरणं शर्म विश्वरेता प्रजाभवः}
{अहः संवत्सरो व्यालः प्रत्ययः सर्वदर्शनः}

\twolineshloka
{अजः सर्वेश्वरः सिद्धः सिद्धिः सर्वादिरच्युतः}
{वृषाकपिरमेयात्मा सर्वयोगविनिःसृतः}

\twolineshloka
{वसुर्वसुमनाः सत्यः समात्माऽसम्मितः समः}
{अमोघः पुण्डरीकाक्षो वृषकर्मा वृषाकृतिः}

\twolineshloka
{रुद्रो बहुशिरा बभ्रुर्विश्वयोनिः शुचिश्रवाः}
{अमृतः शाश्वतः स्थाणुर्वरारोहो महातपाः}

\twolineshloka
{सर्वगः सर्वविद्भानुर्विष्वक्सेनो जनार्दनः}
{वेदो वेदविदव्यङ्गो वेदाङ्गो वेदवित् कविः}

\twolineshloka
{लोकाध्यक्षः सुराध्यक्षो धर्माध्यक्षः कृताकृतः}
{चतुरात्मा चतुर्व्यूहश्चतुर्दंष्ट्रश्चतुर्भुजः}

\twolineshloka
{भ्राजिष्णुर्भोजनं भोक्ता सहिष्णुर्जगदादिजः}
{अनघो विजयो जेता विश्वयोनिः पुनर्वसुः}

\twolineshloka
{उपेन्द्रो वामनः प्रांशुरमोघः शुचिरूर्जितः}
{अतीन्द्रः सङ्ग्रहः सर्गो धृतात्मा नियमो यमः}

\twolineshloka
{वेद्यो वैद्यः सदायोगी वीरहा माधवो मधुः}
{अतीन्द्रियो महामायो महोत्साहो महाबलः}

\twolineshloka
{महाबुद्धिर्महावीर्यो महाशक्तिर्महाद्युतिः}
{अनिर्देश्यवपुः श्रीमानमेयात्मा महाद्रिधृक्}

\twolineshloka
{महेष्वासो महीभर्ता श्रीनिवासः सतां गतिः}
{अनिरुद्धः सुरानन्दो गोविन्दो गोविदां पतिः}

\twolineshloka
{मरीचिर्दमनो हंसः सुपर्णो भुजगोत्तमः}
{हिरण्यनाभः सुतपा पद्मनाभः प्रजापतिः}

\twolineshloka
{अमृत्युः सर्वदृक् सिंहः सन्धाता सन्धिमान् स्थिरः}
{अजो दुर्मर्षणः शास्ता विश्रुतात्मा सुरारिहा}

\twolineshloka
{गुरुर्गुरुतमो धाम सत्यः सत्यपराक्रमः}
{निमिषोऽनिमिषः स्रग्वी वाचस्पतिरुदारधीः}

\twolineshloka
{अग्रणीर्ग्रामणीः श्रीमान् न्यायो नेता समीरणः}
{सहस्रमूर्धा विश्वात्मा सहस्राक्षः सहस्रपात्}

\twolineshloka
{आवर्तनो निवृत्तात्मा संवृतः सम्प्रमर्दनः}
{अहः संवर्तको वह्निरनिलो धरणीधरः}

\twolineshloka
{सुप्रसादः प्रसन्नात्मा विश्वधृग्विश्वभुग्विभुः}
{सत्कर्ता सत्कृतः साधुर्जह्नुर्नारायणो नरः}

\twolineshloka
{असङ्ख्येयोऽप्रमेयात्मा विशिष्टः शिष्टकृच्छुचिः}
{सिद्धार्थः सिद्धसङ्कल्पः सिद्धिदः सिद्धिसाधनः}

\twolineshloka
{वृषाही वृषभो विष्णुर्वृषपर्वा वृषोदरः}
{वर्धनो वर्धमानश्च विविक्तः श्रुतिसागरः}

\twolineshloka
{सुभुजो दुर्धरो वाग्मी महेन्द्रो वसुदो वसुः}
{नैकरूपो बृहद्रूपः शिपिविष्टः प्रकाशनः}

\twolineshloka
{ओजस्तेजोद्युतिधरः प्रकाशात्मा प्रतापनः}
{ऋद्धः स्पष्टाक्षरो मन्त्रश्चन्द्रांशुर्भास्करद्युतिः}

\twolineshloka
{अमृतांशूद्भवो भानुः शशबिन्दुः सुरेश्वरः}
{औषधं जगतः सेतुः सत्यधर्मपराक्रमः}

\twolineshloka
{भूतभव्यभवन्नाथः पवनः पावनोऽनलः}
{कामहा कामकृत्कान्तः कामः कामप्रदः प्रभुः}

\twolineshloka
{युगादिकृद्युगावर्तो नैकमायो महाशनः}
{अदृश्यो व्यक्तरूपश्च सहस्रजिदनन्तजित्}

\twolineshloka
{इष्टोऽविशिष्टः शिष्टेष्टः शिखण्डी नहुषो वृषः}
{क्रोधहा क्रोधकृत्कर्ता विश्वबाहुर्महीधरः}

\twolineshloka
{अच्युतः प्रथितः प्राणः प्राणदो वासवानुजः}
{अपान्निधिरधिष्ठानमप्रमत्तः प्रतिष्ठितः}

\twolineshloka
{स्कन्दः स्कन्दधरो धुर्यो वरदो वायुवाहनः}
{वासुदेवो बृहद्भानुरादिदेवः पुरन्दरः}

\twolineshloka
{अशोकस्तारणस्तारः शूरः शौरिर्जनेश्वरः}
{अनुकूलः शतावर्तः पद्मी पद्मनिभेक्षणः}

\twolineshloka
{पद्मनाभोऽरविन्दाक्षः पद्मगर्भः शरीरभृत्}
{महर्द्धिरृद्धो वृद्धात्मा महाक्षो गरुडध्वजः}

\twolineshloka
{अतुलः शरभो भीमः समयज्ञो हविर्हरिः}
{सर्वलक्षणलक्षण्यो लक्ष्मीवान् समितिञ्जयः}

\twolineshloka
{विक्षरो रोहितो मार्गो हेतुर्दामोदरः सहः}
{महीधरो महाभागो वेगवानमिताशनः}

\twolineshloka
{उद्भवः क्षोभणो देवः श्रीगर्भः परमेश्वरः}
{करणं कारणं कर्ता विकर्ता गहनो गुहः}

\twolineshloka
{व्यवसायो व्यवस्थानः संस्थानः स्थानदो ध्रुवः}
{परर्द्धिः परमस्पष्टस्तुष्टः पुष्टः शुभेक्षणः}

\twolineshloka
{रामो विरामो विरतो मार्गो नेयो नयोऽनयः}
{वीरः शक्तिमतां श्रेष्ठो धर्मो धर्मविदुत्तमः}

\twolineshloka
{वैकुण्ठः पुरुषः प्राणः प्राणदः प्रणवः पृथुः}
{हिरण्यगर्भः शत्रुघ्नो व्याप्तो वायुरधोक्षजः}

\twolineshloka
{ऋतुः सुदर्शनः कालः परमेष्ठी परिग्रहः}
{उग्रः संवत्सरो दक्षो विश्रामो विश्वदक्षिणः}

\twolineshloka
{विस्तारः स्थावरः स्थाणुः प्रमाणं बीजमव्ययम्}
{अर्थोऽनर्थो महाकोशो महाभोगो महाधनः}

\twolineshloka
{अनिर्विण्णः स्थविष्ठोऽभूर्धर्मयूपो महामखः}
{नक्षत्रनेमिर्नक्षत्री क्षमः क्षामः समीहनः}

\twolineshloka
{यज्ञ इज्यो महेज्यश्च क्रतुः सत्रं सतां गतिः}
{सर्वदर्शी विमुक्तात्मा सर्वज्ञो ज्ञानमुत्तमम्}

\twolineshloka
{सुव्रतः सुमुखः सूक्ष्मः सुघोषः सुखदः सुहृत्}
{मनोहरो जितक्रोधो वीरबाहुर्विदारणः}

\twolineshloka
{स्वापनः स्ववशो व्यापी नैकात्मा नैककर्मकृत्}
{वत्सरो वत्सलो वत्सी रत्नगर्भो धनेश्वरः}

\twolineshloka
{धर्मगुब्धर्मकृद्धर्मी सदसत्क्षरमक्षरम्}
{अविज्ञाता सहस्रांशुर्विधाता कृतलक्षणः}

\twolineshloka
{गभस्तिनेमिः सत्त्वस्थः सिंहो भूतमहेश्वरः}
{आदिदेवो महादेवो देवेशो देवभृद्गुरुः}

\twolineshloka
{उत्तरो गोपतिर्गोप्ता ज्ञानगम्यः पुरातनः}
{शरीरभूतभृद्भोक्ता कपीन्द्रो भूरिदक्षिणः}

\twolineshloka
{सोमपोऽमृतपः सोमः पुरुजित् पुरुसत्तमः}
{विनयो जयः सत्यसन्धो दाशार्हः सात्त्वतां पतिः}

\twolineshloka
{जीवो विनयितासाक्षी मुकुन्दोऽमितविक्रमः}
{अम्भोनिधिरनन्तात्मा महोदधिशयोऽन्तकः}

\twolineshloka
{अजो महार्हः स्वाभाव्यो जितामित्रः प्रमोदनः}
{आनन्दो नन्दनो नन्दः सत्यधर्मा त्रिविक्रमः}

\twolineshloka
{महर्षिः कपिलाचार्यः कृतज्ञो मेदिनीपतिः}
{त्रिपदस्त्रिदशाध्यक्षो महाशृङ्गः कृतान्तकृत्}

\twolineshloka
{महावराहो गोविन्दः सुषेणः कनकाङ्गदी}
{गुह्यो गभीरो गहनो गुप्तश्चक्रगदाधरः}

\twolineshloka
{वेधाः स्वाङ्गोऽजितः कृष्णो दृढः सङ्कर्षणोऽच्युतः}
{वरुणो वारुणो वृक्षः पुष्कराक्षो महामनाः}

\twolineshloka
{भगवान् भगहाऽऽनन्दी वनमाली हलायुधः}
{आदित्यो ज्योतिरादित्यः सहिष्णुर्गतिसत्तमः}

\twolineshloka
{सुधन्वा खण्डपरशुर्दारुणो द्रविणप्रदः}
{दिवःस्पृक् सर्वदृग्व्यासो वाचस्पतिरयोनिजः}

\twolineshloka
{त्रिसामा सामगः साम निर्वाणं भेषजं भिषक्}
{सन्न्यासकृच्छमः शान्तो निष्ठा शान्तिः परायणम्}

\twolineshloka
{शुभाङ्गः शान्तिदः स्रष्टा कुमुदः कुवलेशयः}
{गोहितो गोपतिर्गोप्ता वृषभाक्षो वृषप्रियः}

\twolineshloka
{अनिवर्ती निवृत्तात्मा सङ्क्षेप्ता क्षेमकृच्छिवः}
{श्रीवत्सवक्षाः श्रीवासः श्रीपतिः श्रीमतां वरः}

\twolineshloka
{श्रीदः श्रीशः श्रीनिवासः श्रीनिधिः श्रीविभावनः}
{श्रीधरः श्रीकरः श्रेयः श्रीमाँल्लोकत्रयाश्रयः}

\twolineshloka
{स्वक्षः स्वङ्गः शतानन्दो नन्दिर्ज्योतिर्गणेश्वरः}
{विजितात्माऽविधेयात्मा सत्कीर्तिश्छिन्नसंशयः}

\twolineshloka
{उदीर्णः सर्वतश्चक्षुरनीशः शाश्वतः स्थिरः}
{भूशयो भूषणो भूतिर्विशोकः शोकनाशनः}

\twolineshloka
{अर्चिष्मानर्चितः कुम्भो विशुद्धात्मा विशोधनः}
{अनिरुद्धोऽप्रतिरथः प्रद्युम्नोऽमितविक्रमः}

\twolineshloka
{कालनेमिनिहा वीरः शौरिः शूरजनेश्वरः}
{त्रिलोकात्मा त्रिलोकेशः केशवः केशिहा हरिः}

\twolineshloka
{कामदेवः कामपालः कामी कान्तः कृतागमः}
{अनिर्देश्यवपुर्विष्णुर्वीरोऽनन्तो धनञ्जयः}

\twolineshloka
{ब्रह्मण्यो ब्रह्मकृद्-ब्रह्मा ब्रह्म ब्रह्मविवर्धनः}
{ब्रह्मविद्-ब्राह्मणो ब्रह्मी ब्रह्मज्ञो ब्राह्मणप्रियः}

\twolineshloka
{महाक्रमो महाकर्मा महातेजा महोरगः}
{महाक्रतुर्महायज्वा महायज्ञो महाहविः}

\twolineshloka
{स्तव्यः स्तवप्रियः स्तोत्रं स्तुतिः स्तोता रणप्रियः}
{पूर्णः पूरयिता पुण्यः पुण्यकीर्तिरनामयः}

\twolineshloka
{मनोजवस्तीर्थकरो वसुरेता वसुप्रदः}
{वसुप्रदो वासुदेवो वसुर्वसुमना हविः}

\twolineshloka
{सद्गतिः सत्कृतिः सत्ता सद्भूतिः सत्परायणः}
{शूरसेनो यदुश्रेष्ठः सन्निवासः सुयामुनः}

\twolineshloka
{भूतावासो वासुदेवः सर्वासुनिलयोऽनलः}
{दर्पहा दर्पदो दृप्तो दुर्धरोऽथापराजितः}

\twolineshloka
{विश्वमूर्तिर्महामूर्तिर्दीप्तमूर्तिरमूर्तिमान्}
{अनेकमूर्तिरव्यक्तः शतमूर्तिः शताननः}

\twolineshloka
{एको नैकः सवः कः किं यत् तत्पदमनुत्तमम्}
{लोकबन्धुर्लोकनाथो माधवो भक्तवत्सलः}

\twolineshloka
{सुवर्णवर्णो हेमाङ्गो वराङ्गश्चन्दनाङ्गदी}
{वीरहा विषमः शून्यो घृताशीरचलश्चलः}

\twolineshloka
{अमानी मानदो मान्यो लोकस्वामी त्रिलोकधृक्}
{सुमेधा मेधजो धन्यः सत्यमेधा धराधरः}

\twolineshloka
{तेजोवृषो द्युतिधरः सर्वशस्त्रभृतां वरः}
{प्रग्रहो निग्रहो व्यग्रो नैकशृङ्गो गदाग्रजः}

\twolineshloka
{चतुर्मूर्तिश्चतुर्बाहुश्चतुर्व्यूहश्चतुर्गतिः}
{चतुरात्मा चतुर्भावश्चतुर्वेदविदेकपात्}

\twolineshloka
{समावर्तोऽनिवृत्तात्मा दुर्जयो दुरतिक्रमः}
{दुर्लभो दुर्गमो दुर्गो दुरावासो दुरारिहा}

\twolineshloka
{शुभाङ्गो लोकसारङ्गः सुतन्तुस्तन्तुवर्धनः}
{इन्द्रकर्मा महाकर्मा कृतकर्मा कृतागमः}

\twolineshloka
{उद्भवः सुन्दरः सुन्दो रत्ननाभः सुलोचनः}
{अर्को वाजसनः शृङ्गी जयन्तः सर्वविज्जयी}

\twolineshloka
{सुवर्णबिन्दुरक्षोभ्यः सर्ववागीश्वरेश्वरः}
{महाह्रदो महागर्तो महाभूतो महानिधिः}

\twolineshloka
{कुमुदः कुन्दरः कुन्दः पर्जन्यः पावनोऽनिलः}
{अमृताशोऽमृतवपुः सर्वज्ञः सर्वतोमुखः}

\twolineshloka
{सुलभः सुव्रतः सिद्धः शत्रुजिच्छत्रुतापनः}
{न्यग्रोधोऽदुम्बरोऽश्वत्थश्चाणूरान्ध्रनिषूदनः}

\twolineshloka
{सहस्रार्चिः सप्तजिह्वः सप्तैधाः सप्तवाहनः}
{अमूर्तिरनघोऽचिन्त्यो भयकृद्भयनाशनः}

\twolineshloka
{अणुर्बृहत् कृशः स्थूलो गुणभृन्निर्गुणो महान्}
{अधृतः स्वधृतः स्वास्यः प्राग्वंशो वंशवर्धनः}

\twolineshloka
{भारभृत् कथितो योगी योगीशः सर्वकामदः}
{आश्रमः श्रमणः क्षामः सुपर्णो वायुवाहनः}

\twolineshloka
{धनुर्धरो धनुर्वेदो दण्डो दमयिता दमः}
{अपराजितः सर्वसहो नियन्ताऽनियमोऽयमः}

\twolineshloka
{सत्त्ववान् सात्त्विकः सत्यः सत्यधर्मपरायणः}
{अभिप्रायः प्रियार्होऽर्हः प्रियकृत् प्रीतिवर्धनः}

\twolineshloka
{विहायसगतिर्ज्योतिः सुरुचिर्हुतभुग्विभुः}
{रविर्विरोचनः सूर्यः सविता रविलोचनः}

\twolineshloka
{अनन्तो हुतभुग्भोक्ता सुखदो नैकजोऽग्रजः}
{अनिर्विण्णः सदामर्षी लोकाधिष्ठानमद्भुतः}

\twolineshloka
{सनात् सनातनतमः कपिलः कपिरव्ययः}
{स्वस्तिदः स्वस्तिकृत् स्वस्ति स्वस्तिभुक् स्वस्तिदक्षिणः}

\twolineshloka
{अरौद्रः कुण्डली चक्री विक्रम्यूर्जितशासनः}
{शब्दातिगः शब्दसहः शिशिरः शर्वरीकरः}

\twolineshloka
{अक्रूरः पेशलो दक्षो दक्षिणः क्षमिणां वरः}
{विद्वत्तमो वीतभयः पुण्यश्रवणकीर्तनः}

\twolineshloka
{उत्तारणो दुष्कृतिहा पुण्यो दुःस्वप्ननाशनः}
{वीरहा रक्षणः सन्तो जीवनः पर्यवस्थितः}

\twolineshloka
{अनन्तरूपोऽनन्तश्रीर्जितमन्युर्भयापहः}
{चतुरश्रो गभीरात्मा विदिशो व्यादिशो दिशः}

\twolineshloka
{अनादिर्भूर्भुवो लक्ष्मीः सुवीरो रुचिराङ्गदः}
{जननो जनजन्मादिर्भीमो भीमपराक्रमः}

\twolineshloka
{आधारनिलयोऽधाता पुष्पहासः प्रजागरः}
{ऊर्ध्वगः सत्पथाचारः प्राणदः प्रणवः पणः}

\twolineshloka
{प्रमाणं प्राणनिलयः प्राणभृत् प्राणजीवनः}
{तत्त्वं तत्त्वविदेकात्मा जन्ममृत्युजरातिगः}

\twolineshloka
{भूर्भुवःस्वस्तरुस्तारः सविता प्रपितामहः}
{यज्ञो यज्ञपतिर्यज्वा यज्ञाङ्गो यज्ञवाहनः}

\twolineshloka
{यज्ञभृद्-यज्ञकृद्-यज्ञी यज्ञभुग्-यज्ञसाधनः}
{यज्ञान्तकृद्-यज्ञगुह्यमन्नमन्नाद एव च}

\twolineshloka
{आत्मयोनिः स्वयञ्जातो वैखानः सामगायनः}
{देवकीनन्दनः स्रष्टा क्षितीशः पापनाशनः}

\twolineshloka
{शङ्खभृन्नन्दकी चक्री शार्ङ्गधन्वा गदाधरः}
{रथाङ्गपाणिरक्षोभ्यः सर्वप्रहरणायुधः}
सर्वप्रहरणायुध ॐ नम इति।

\twolineshloka
{वनमाली गदी शार्ङ्गी शङ्खी चक्री च नन्दकी}
{श्रीमान् नारायणो विष्णुर्वासुदेवोऽभिरक्षतु}%{(एवं त्रिः)}
श्री वासुदेवोऽभिरक्षतु ॐ नम इति।

\dnsub{फलश्रुति श्लोकाः}
\resetShloka
\twolineshloka
{इतीदं कीर्तनीयस्य केशवस्य महात्मनः}
{नाम्नां सहस्रं दिव्यानामशेषेण प्रकीर्तितम्}

\twolineshloka
{य इदं शृणुयान्नित्यं यश्चापि परिकीर्तयेत्}
{नाशुभं प्राप्नुयात् किञ्चित् सोऽमुत्रेह च मानवः}

\twolineshloka
{वेदान्तगो ब्राह्मणः स्यात् क्षत्रियो विजयी भवेत्}
{वैश्यो धनसमृद्धः स्याच्छूद्रः सुखमवाप्नुयात्}

\twolineshloka
{धर्मार्थी प्राप्नुयाद्धर्ममर्थार्थी चार्थमाप्नुयात्}
{कामानवाप्नुयात् कामी प्रजार्थी चाऽऽप्नुयात्प्रजाम्}

\twolineshloka
{भक्तिमान् यः सदोत्थाय शुचिस्तद्गतमानसः}
{सहस्रं वासुदेवस्य नाम्नामेतत् प्रकीर्तयेत्}

\twolineshloka
{यशः प्राप्नोति विपुलं याति प्राधान्यमेव च}
{अचलां श्रियमाप्नोति श्रेयः प्राप्नोत्यनुत्तमम्}

\twolineshloka
{न भयं क्वचिदाप्नोति वीर्यं तेजश्च विन्दति}
{भवत्यरोगो द्युतिमान् बलरूपगुणान्वितः}

\twolineshloka
{रोगार्तो मुच्यते रोगाद्बद्धो मुच्येत बन्धनात्}
{भयान्मुच्येत भीतस्तु मुच्येताऽऽपन्न आपदः}

\twolineshloka
{दुर्गाण्यतितरत्याशु पुरुषः पुरुषोत्तमम्}
{स्तुवन्नामसहस्रेण नित्यं भक्तिसमन्वितः}

\twolineshloka
{वासुदेवाश्रयो मर्त्यो वासुदेवपरायणः}
{सर्वपापविशुद्धात्मा याति ब्रह्म सनातनम्}

\twolineshloka
{न वासुदेवभक्तानामशुभं विद्यते क्वचित्}
{जन्ममृत्युजराव्याधिभयं नैवोपजायते}

\twolineshloka
{इमं स्तवमधीयानः श्रद्धाभक्तिसमन्वितः}
{युज्येताऽऽत्मसुखक्षान्तिश्रीधृतिस्मृतिकीर्तिभिः}

\twolineshloka
{न क्रोधो न च मात्सर्यं न लोभो नाशुभा मतिः}
{भवन्ति कृतपुण्यानां भक्तानां पुरुषोत्तमे}

\twolineshloka
{द्यौः सचन्द्रार्कनक्षत्रा खं दिशो भूर्महोदधिः}
{वासुदेवस्य वीर्येण विधृतानि महात्मनः}

\twolineshloka
{ससुरासुरगन्धर्वं सयक्षोरगराक्षसम्}
{जगद्वशे वर्ततेदं कृष्णस्य सचराचरम्}

\twolineshloka
{इन्द्रियाणि मनो बुद्धिः सत्त्वं तेजो बलं धृतिः}
{वासुदेवात्मकान्याहुः क्षेत्रं क्षेत्रज्ञ एव च}

\twolineshloka
{सर्वागमानामाचारः प्रथमं परिकल्पते}
{आचारप्रभवो धर्मो धर्मस्य प्रभुरच्युतः}

\twolineshloka
{ऋषयः पितरो देवा महाभूतानि धातवः}
{जङ्गमाजङ्गमं चेदं जगन्नारायणोद्भवम्}

\twolineshloka
{योगो ज्ञानं तथा साङ्ख्यं विद्याः शिल्पादि कर्म च}
{वेदाः शास्त्राणि विज्ञानमेतत्सर्वं जनार्दनात्}

\twolineshloka
{एको विष्णुर्महद्भूतं पृथग्भूतान्यनेकशः}
{त्रीँल्लोकान् व्याप्य भूतात्मा भुङ्क्ते विश्वभुगव्ययः}

\twolineshloka
{इमं स्तवं भगवतो विष्णोर्व्यासेन कीर्तितम्}
{पठेद्य इच्छेत् पुरुषः श्रेयः प्राप्तुं सुखानि च}

\twolineshloka
{विश्वेश्वरमजं देवं जगतः प्रभुमव्ययम्}
{भजन्ति ये पुष्कराक्षं न ते यान्ति पराभवम्}
न ते यान्ति पराभवम् ॐ नम इति।

\uvacha{अर्जुन उवाच}
\twolineshloka
{पद्मपत्रविशालाक्ष पद्मनाभ सुरोत्तम}
{भक्तानामनुरक्तानां त्राता भव जनार्दन}

\uvacha{श्रीभगवानुवाच}
\twolineshloka
{यो मां नामसहस्रेण स्तोतुमिच्छति पाण्डव}
{सोऽहमेकेन श्लोकेन स्तुत एव न संशयः}
स्तुत एव न संशय ॐ नम इति।

\uvacha{व्यास उवाच}
\twolineshloka
{वासनाद्वासुदेवस्य वासितं भुवनत्रयम्}
{सर्वभूतनिवासोऽसि वासुदेव नमोऽस्तु ते}
श्री वासुदेव नमोऽस्तुत ॐ नम इति।

\uvacha{पार्वत्युवाच}
\twolineshloka
{केनोपायेन लघुना विष्णोर्नामसहस्रकम्}
{पठ्यते पण्डितैर्नित्यं श्रोतुमिच्छाम्यहं प्रभो}

\uvacha{श्री ईश्वर उवाच}
\twolineshloka
{श्रीराम राम रामेति रमे रामे मनोरमे}
{सहस्रनाम तत्तुल्यं राम नाम वरानने}%{(एवं त्रिः)}
श्रीरामनाम वरानन ॐ नम इति।

\uvacha{ब्रह्मोवाच}
\twolineshloka
{नमोऽस्त्वनन्ताय सहस्रमूर्तये सहस्रपादाक्षिशिरोरुबाहवे}
{सहस्रनाम्ने पुरुषाय शाश्वते सहस्रकोटियुगधारिणे नमः}
सहस्रकोटियुगधारिणे नम ॐ नम इति।

\uvacha{सञ्जय उवाच}
\twolineshloka
{यत्र योगेश्वरः कृष्णो यत्र पार्थो धनुर्धरः}
{तत्र श्रीर्विजयो भूतिर्ध्रुवा नीतिर्मतिर्मम}

\uvacha{श्रीभगवानुवाच}
\twolineshloka
{अनन्याश्चिन्तयन्तो मां ये जनाः पर्युपासते}
{तेषां नित्याभियुक्तानां योगक्षेमं वहाम्यहम्}

\twolineshloka
{परित्राणाय साधूनां विनाशाय च दुष्कृताम्}
{धर्मसंस्थापनार्थाय सम्भवामि युगे युगे}

\fourlineindentedshloka
{आर्ता विषण्णाः शिथिलाश्च भीताः}
{घोरेषु च व्याधिषु वर्तमानाः}
{सङ्कीर्त्य नारायणशब्दमात्रम्}
{विमुक्तदुःखाः सुखिनो भवन्ति}

\twolineshloka*
{कायेन वाचा मनसेन्द्रियैर्वा बुद्‌ध्याऽऽत्मना वा प्रकृतेः स्वभावात्}
{करोमि यद्यत् सकलं परस्मै नारायणायेति समर्पयामि}
॥ॐ तत्सदिति श्रीमन्महाभारते शतसाहस्र्यां संहितायां वैयासिक्याम् आनुशासनिकपर्वणि श्री भीष्मयुधिष्ठिरसंवादे श्री विष्णोर्दिव्यसहस्रनामस्तोत्रं सम्पूर्णम्॥
\setlength{\shlokaspaceskip}{24pt}

\end{center}
