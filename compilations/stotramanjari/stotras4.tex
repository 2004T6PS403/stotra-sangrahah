% !TeX program = XeLaTeX
% !TeX root = stotramanjari.tex
\Large
\part{स्तोत्राणि}
\sectionmark{\mbox{}}
\begin{center}
\phantomsection\addcontentsline{toc}{chapter}{गणेशस्तोत्राणि}
\clearpage
% !TeX program = XeLaTeX
% !TeX root = ../../shloka.tex

\sect{महागणेशपञ्चरत्नम्}
\fourlineindentedshloka
{मुदाकरात्तमोदकं सदाविमुक्तिसाधकम्}
{कलाधरावतंसकं विलासिलोकरक्षकम्}
{अनायकैकनायकं विनाशितेभदैत्यकम्}
{नताशुभाशुनाशकं नमामि तं विनायकम्}

\fourlineindentedshloka
{नतेतरातिभीकरं नवोदितार्कभास्वरम्}
{नमत्सुरारिनिर्जरं नताधिकापदुद्धरम्}
{सुरेश्वरं निधीश्वरं गजेश्वरं गणेश्वरम्}
{महेश्वरं तमाश्रये परात्परं निरन्तरम्}

\fourlineindentedshloka
{समस्तलोकशङ्करं निरस्तदैत्यकुञ्जरम्}
{दरेतरोदरं वरं वरेभवक्त्रमक्षरम्}
{कृपाकरं क्षमाकरं मुदाकरं यशस्करम्}
{मनस्करं नमस्कृतां नमस्करोमि भास्वरम्}

\fourlineindentedshloka
{अकिञ्चनार्तिमार्जनं चिरन्तनोक्तिभाजनम्}
{पुरारिपूर्वनन्दनं सुरारिगर्वचर्वणम्}
{प्रपञ्चनाशभीषणं धनञ्जयादिभूषणम्}
{कपोलदानवारणं भजे पुराणवारणम्}

\fourlineindentedshloka
{नितान्तकान्तदन्तकान्तिमन्तकान्तकात्मजम्}
{अचिन्त्यरूपमन्तहीनमन्तरायकृन्तनम्}
{हृदन्तरे निरन्तरं वसन्तमेव योगिनाम्}
{तमेकदन्तमेव तं विचिन्तयामि सन्ततम्}

\fourlineindentedshloka*
{महागणेशपञ्चरत्नमादरेण योऽन्वहम्}
{प्रजल्पति प्रभातके हृदि स्मरन् गणेश्वरम्}
{अरोगतामदोषतां सुसाहितीं सुपुत्रताम्}
{समाहितायुरष्टभूतिमभ्युपैति सोऽचिरात्}
॥इति श्रीमच्छङ्कराचार्यविरचितं श्री~महागणेशपञ्चरत्नं सम्पूर्णम्॥
% !TeX program = XeLaTeX
% !TeX root = ../../shloka.tex
\sect{गणपत्यष्टोत्तरशतनामस्तोत्रम्}

\dnsub{ध्यानम्}
\fourlineindentedshloka*
{ॐकारसन्निभमिभाननमिन्दुभालम्}
{मुक्ताग्रबिन्दुममलद्युतिमेकदन्तम्}
{लम्बोदरं कलचतुर्भुजमादिदेवम्}
{ध्यायेन्महागणपतिं मतिसिद्धिकान्तम्}

\dnsub{स्तोत्रम्}
\twolineshloka
{गणेश्वरो गणक्रीडो महागणपतिस्तथा}
{विश्वकर्ता विश्वमुखो दुर्जयो धूर्जयो जयः}

\twolineshloka
{सुरूपः सर्वनेत्राधिवासो वीरासनाश्रयः}
{योगाधिपस्तारकस्थः पुरुषो गजकर्णकः}

\twolineshloka
{चित्राङ्गः श्यामदशनो भालचन्द्रश्चतुर्भुजः}
{शम्भुतेजा यज्ञकायः सर्वात्मा सामबृंहितः}

\twolineshloka
{कुलाचलांसो व्योमनाभिः कल्पद्रुमवनालयः}
{निम्ननाभिः स्थूलकुक्षिः पीनवक्षा बृहद्भुजः}

\twolineshloka
{पीनस्कन्धः कम्बुकण्ठो लम्बोष्ठो लम्बनासिकः}
{सर्वायवसम्पूर्णः सर्वलक्षणलक्षितः}

\twolineshloka
{इक्षुचापधरः शूली कान्तिकन्दलिताश्रयः}
{अक्षमालाधरो ज्ञानमुद्रावान् विजयावहः}

\twolineshloka
{कामिनीकामनाकाममालिनीकेलिलालितः}
{अमोघसिद्धिराधार आधाराधेयवर्जितः}

\twolineshloka
{इन्दीवरदलश्याम इन्दुमण्डलनिर्मलः}
{कर्मसाक्षी कर्मकर्ता कर्माकर्मफलप्रदः}

\twolineshloka
{कमण्डलुधरः कल्पः कपर्दी कटिसूत्रभृत्}
{कारुण्यदेहः कपिलो गुह्यागमनिरूपितः}

\twolineshloka
{गुहाशयो गुहाब्धिस्थो घटकुम्भो घटोदरः}
{पूर्णानन्दः परानन्दो धनदो धरणीधरः}

\twolineshloka
{बृहत्तमो ब्रह्मपरो ब्रह्मण्यो ब्रह्मवित्प्रियः}
{भव्यो भूतालयो भोगदाता चैव महामनाः}

\twolineshloka
{वरेण्यो वामदेवश्च वन्द्यो वज्रनिवारणः}
{विश्वकर्ता विश्वचक्षुर्हवनं हव्यकव्यभुक्}

\twolineshloka
{स्वतन्त्रः सत्यसङ्कल्पस्तथा सौभाग्यवर्धनः}
{कीर्तिदः शोकहारी च त्रिवर्गफलदायकः}

\twolineshloka
{चतुर्बाहुश्चतुर्दन्तश्चतुर्थातिथिसम्भवः}
{सहस्रशीर्षा पुरुषः सहस्राक्षः सहस्रपात्}

\twolineshloka
{कामरूपः कामगतिर्द्विरदो द्वीपरक्षकः}
{क्षेत्राधिपः क्षमाभर्ता लयस्थो लड्डुकप्रियः}

\twolineshloka
{प्रतिवादिमुखस्तम्भो दुष्टचित्तप्रसादनः}
{भगवान् भक्तिसुलभो याज्ञिको याजकप्रियः}

\twolineshloka
{इत्येवं देवदेवस्य गणराजस्य धीमतः}
{शतमष्टोत्तरं नाम्नां सारभूतं प्रकीर्तितम्}

\threelineshloka
{सहस्रनाम्नामाकृष्य मया प्रोक्तं मनोहरम्}
{ब्राह्मे मुहूर्ते चोत्थाय स्मृत्वा देवं गणेश्वरम्}
{पठेत्स्तोत्रमिदं भक्त्या गणराजः प्रसीदति}

{॥ इति श्रीगणेशपुराणे उपासनाखण्डे श्रीगणपत्यष्टोत्तरशतनामस्तोत्रं सम्पूर्णम् ॥}

\closesection
\clearpage
\phantomsection\addcontentsline{toc}{chapter}{शिवस्तोत्राणि}
% !TeX program = XeLaTeX
% !TeX root = ../../shloka.tex

\sect{शिवमानसपूजा}
\fourlineindentedshloka
{रत्नैः कल्पितमासनं हिमजलैः स्नानं च दिव्याम्बरम्}
{नानारत्नविभूषितं मृगमदामोदाङ्कितं चन्दनम्}
{जातीचम्पकबिल्वपत्ररचितं पुष्पं च धूपं तथा}
{दीपं देव दयानिधे पशुपते हृत्कल्पितं गृह्यताम्}

\fourlineindentedshloka
{सौवर्णे नवरत्नखण्डरचिते पात्रे घृतं पायसम्}
{भक्ष्यं पञ्चविधं पयोदधियुतं रम्भाफलं पानकम्}
{शाकानामयुतं जलं रुचिकरं कर्पूरखण्डोज्ज्वलम्}
{ताम्बूलं मनसा मया विरचितं भक्त्या प्रभो स्वीकुरु}

\fourlineindentedshloka
{छत्रं चामरयोर्युगं व्यजनकं चादर्शकं निर्मलम्}
{वीणाभेरिमृदङ्गकाहलकला गीतं च नृत्यं तथा}
{साष्टाङ्गं प्रणतिः स्तुतिर्बहुविधा ह्येतत् समस्तं मया}
{सङ्कल्पेन समर्पितं तव विभो पूजां गृहाण प्रभो}

\fourlineindentedshloka
{आत्मा त्वं गिरिजा मतिः सहचराः प्राणाः शरीरं गृहम्}
{पूजा ते विषयोपभोगरचना निद्रा समाधिस्थितिः}
{सञ्चारः पदयोः प्रदक्षिणविधिः स्तोत्राणि सर्वा गिरो-}
{यद्यत्कर्म करोमि तत्तदखिलं शम्भो तवाऽऽराधनम्}

\fourlineindentedshloka
{करचरणकृतं वाक्कायजं कर्मजं वा}
{श्रवणनयनजं वा मानसं वाऽपराधम्}
{विहितमविहितं वा सर्वमेतत् क्षमस्व}
{जय जय करुणाब्धे श्रीमहादेव शम्भो}
॥इति श्रीमच्छङ्कराचार्यविरचिता श्रीशिवमानसपूजा सम्पूर्णा॥
% !TeX program = XeLaTeX
% !TeX root = ../../shloka.tex

\sect{वैद्यनाथाष्टकम्}
\twolineshloka
{श्रीराम-सौमित्रि-जटायु-वेद-षडाननादित्य-कुजार्चिताय}
{श्रीनीलकण्ठाय दयामयाय श्रीवैद्यनाथाय नमः शिवाय}

%\twolineshloka*
%{महादेव महादेव महादेव महादेव महादेव महादेव महादेव महादेव}
%{महादेव महादेव महादेव महादेव महादेव महादेव महादेव महादेव}
%
\twolineshloka
{गङ्गाप्रवाहेन्दुजटाधराय त्रिलोचनाय स्मरकालहन्त्रे}
{समस्तदेवैरभिपूजिताय श्रीवैद्यनाथाय नमः शिवाय}
%\hfill महादेव महादेव …

\twolineshloka
{भक्तःप्रियाय त्रिपुरान्तकाय पिनाकिने दुष्टहराय नित्यम्}
{प्रत्यक्षलीलाय मनुष्यलोके श्रीवैद्यनाथाय नमः शिवाय}
%\hfill महादेव महादेव …

\twolineshloka
{प्रभूतवातादि-समस्तरोगप्रनाशकर्त्रे मुनिवन्दिताय}
{प्रभाकरेन्द्वग्निविलोचनाय श्रीवैद्यनाथाय नमः शिवाय}
%\hfill महादेव महादेव …

\twolineshloka
{वाक्श्रोत्रनेत्राङ्घ्रि-विहीनजन्तोर्वाक्श्रोत्रनेत्राङ्घ्रि-सुखप्रदाय}
{कुष्ठादिसर्वोन्नतरोगहन्त्रे श्रीवैद्यनाथाय नमः शिवाय}
%\hfill महादेव महादेव …

\twolineshloka
{वेदान्तवेद्याय जगन्मयाय योगीश्वरध्येयपदाम्बुजाय}
{त्रिमूर्तिरूपाय सहस्रनाम्ने श्रीवैद्यनाथाय नमः शिवाय}
%\hfill महादेव महादेव …

\twolineshloka
{स्वतीर्थमृद्भस्मभृताङ्गभाजां पिशाचदुःखार्तिभयापहाय}
{आत्मस्वरूपाय शरीरभाजां श्रीवैद्यनाथाय नमः शिवाय}
%\hfill महादेव महादेव …

\twolineshloka
{श्रीनीलकण्ठाय वृषध्वजाय स्रग्ग्न्धभस्माद्यभिशोभिताय}
{सुपुत्रदारादिसुभाग्यदाय श्रीवैद्यनाथाय नमः शिवाय}
%\hfill महादेव महादेव …

\twolineshloka*
{बालाम्बिकेश वैद्येश भवरोगहरेति च}
{जपेन्नामत्रयन्नित्यं महारोगनिवारणम्}
%\hfill महादेव महादेव …
\twolineshloka*
{महादेव महादेव महादेव महादेव महादेव महादेव महादेव महादेव}
{महादेव महादेव महादेव महादेव महादेव महादेव महादेव महादेव}

% !TeX program = XeLaTeX
% !TeX root = ../../shloka.tex

\sect{मार्गबन्धुस्तोत्रम्}

\threelineshloka* 
{शम्भो महादेव देव}
{शिव शम्भो महादेव देवेश शम्भो}
{शम्भो महादेव देव}

\twolineshloka
{फालावनम्रत्किरीटं फालनेत्रार्चिषा-दग्ध-पञ्चेषुकीटम्}
{शूलाहतारातिकूटं शुद्धमर्धेन्दुचूडं भजे मार्गबन्धुम्}

\twolineshloka
{अङ्गे विराजद्भुजङ्गम् अभ्र-गङ्गा-तरङ्गाभि-रामोत्तमाङ्गम्}
{ओङ्कारवाटी-कुरङ्गं सिद्धसंसेविताङ्घ्रिं भजे मार्गबन्धुम्}

\twolineshloka
{नित्यं चिदानन्दरूपं निह्नुताशेष-लोकेश-वैरिप्रतापम्}
{कार्तस्वरागेन्द्र-चापं कृत्तिवासं भजे दिव्य सन्मार्गबन्धुम्}

\twolineshloka
{कन्दर्प-दर्पघ्नमीशं कालकण्ठं महेशं महाव्योमकेशम्}
{कुन्दाभदन्तं सुरेशं कोटिसूर्यप्रकाशं भजे मार्गबन्धुम्}

\twolineshloka
{मन्दारभूतेरुदारं मन्दरागेन्द्रसारं महागौर्यदूरम्}
{सिन्धूर-दूर-प्रचारं सिन्धुराजातिधीरं भजे मार्गबन्धुम्}

\twolineshloka*
{अप्पय्ययज्वेन्द्रगीतं स्तोत्रराजं पठेद्यस्तु भक्त्या प्रयाणे}
{तस्यार्थसिद्धिं विधत्ते मार्गमध्येऽभयं चाशुतोषो महेशः}
% !TeX program = XeLaTeX
% !TeX root = ../../shloka.tex

\sect{उमामहेश्वरस्तोत्रम्}

\fourlineindentedshloka
{नमः शिवाभ्यां नवयौवनाभ्याम्‌}
{परस्पराश्लिष्टवपुर्धराभ्याम्‌}
{नगेन्द्रकन्यावृषकेतनाभ्याम्‌}
{नमो नमः शङ्करपार्वतीभ्याम्‌}%१

\fourlineindentedshloka
{नमः शिवाभ्यां सरसोत्सवाभ्याम्‌}
{नमस्कृताभीष्टवरप्रदाभ्याम्‌}
{नारायणेनार्चितपादुकाभ्याम्}
{नमो नमः शङ्करपार्वतीभ्याम्‌}%२

\fourlineindentedshloka
{नमः शिवाभ्यां वृषवाहनाभ्याम्‌}
{विरिञ्चिविष्ण्विन्द्रसुपूजिताभ्याम्‌}
{विभूतिपाटीरविलेपनाभ्याम्‌}
{नमो नमः शङ्करपार्वतीभ्याम्‌}%३

\fourlineindentedshloka
{नमः शिवाभ्यां जगदीश्वराभ्याम्}
{जगत्पतिभ्यां जयविग्रहाभ्याम्‌}
{जम्भारिमुख्यैरभिवन्दिताभ्याम्‌}
{नमो नमः शङ्करपार्वतीभ्याम्‌}%४

\fourlineindentedshloka
{नमः शिवाभ्यां परमौषधाभ्याम्‌}
{पञ्चाक्षरी-पञ्जररञ्जिताभ्याम्‌}
{प्रपञ्च-सृष्टि-स्थिति-संहृताभ्याम्‌}
{नमो नमः शङ्करपार्वतीभ्याम्‌}%५

\fourlineindentedshloka
{नमः शिवाभ्यामतिसुन्दराभ्याम्‌}
{अत्यन्तमासक्तहृदम्बुजाभ्याम्‌}
{अशेषलोकैकहितङ्कराभ्याम्‌}
{नमो नमः शङ्करपार्वतीभ्याम्‌}%६

\fourlineindentedshloka
{नमः शिवाभ्यां कलिनाशनाभ्याम्‌}
{कङ्कालकल्याणवपुर्धराभ्याम्‌}
{कैलासशैलस्थितदेवताभ्याम्‌}
{नमो नमः शङ्करपार्वतीभ्याम्‌}%७

\fourlineindentedshloka
{नमः शिवाभ्यामशुभापहाभ्याम्‌}
{अशेषलोकैकविशेषिताभ्याम्‌}
{अकुण्ठिताभ्यां स्मृतिसम्भृताभ्याम्‌}
{नमो नमः शङ्करपार्वतीभ्याम्‌}%८

\fourlineindentedshloka
{नमः शिवाभ्यां रथवाहनाभ्याम्‌}
{रवीन्दुवैश्वानरलोचनाभ्याम्‌}
{राका-शशाङ्काभ-मुखाम्बुजाभ्याम्‌}
{नमो नमः शङ्करपार्वतीभ्याम्‌}%९

\fourlineindentedshloka
{नमः शिवाभ्यां जटिलन्धराभ्याम्‌}
{जरामृतिभ्यां च विवर्जिताभ्याम्‌}
{जनार्दनाब्जोद्भवपूजिताभ्याम्‌}
{नमो नमः शङ्करपार्वतीभ्याम्‌}%१०

\fourlineindentedshloka
{नमः शिवाभ्यां विषमेक्षणाभ्याम्‌}
{बिल्वच्छदामल्लिकदामभृद्‌भ्याम्‌}
{शोभावती-शान्तवतीश्वराभ्याम्‌}
{नमो नमः शङ्करपार्वतीभ्याम्‌}%११

\fourlineindentedshloka
{नमः शिवाभ्यां पशुपालकाभ्याम्‌}
{जगत्रयीरक्षण-बद्धहृद्‌भ्याम्‌}
{समस्तदेवासुरपूजिताभ्याम्‌}
{नमो नमः शङ्करपार्वतीभ्याम्‌}%१२

\fourlineindentedshloka
{स्तोत्रं त्रिसन्ध्यं शिवपार्वतीभ्याम्‌}
{भक्त्या पठेद्-द्वादशकं नरो यः}
{स सर्वसौभाग्य-फलानि भुङ्क्ते}
{शतायुरन्ते शिवलोकमेति}%१३

{॥इति श्रीमच्छङ्कराचार्यविरचितं श्री उमामहेश्वरस्तोत्रं सम्पूर्णम्‌॥}
% !TeX program = XeLaTeX
% !TeX root = ../../shloka.tex

\sect{शिवाष्टोत्तरशतनामस्तोत्रम्}

\dnsub{ध्यानम्}
\fourlineindentedshloka*
{ध्यायेन्नित्यं महेशं रजतगिरिनिभं चारुचन्द्रावतंसम्}
{रत्नाकल्पोज्ज्वलाङ्गं परशुमृगवराभीतिहस्तं प्रसन्नम्}
{पद्मासीनं समन्तात् स्तुतममरगणैर्व्याघ्रकृत्तिं वसानम्}
{विश्वाद्यं विश्वबीजं निखिलभयहरं पञ्चवक्त्रं त्रिनेत्रम्}

\dnsub{स्तोत्रम्}
\twolineshloka
{शिवो महेश्वरः शम्भुः पिनाकी शशिशेखरः}
{वामदेवो विरूपाक्षः कपर्दी नीललोहितः}

\twolineshloka
{शङ्करः शूलपाणिश्च खट्वाङ्गी विष्णुवल्लभः}
{शिपिविष्टोऽम्बिकानाथः श्रीकण्ठो भक्तवत्सलः}

\twolineshloka
{भवः शर्वस्त्रिलोकेशः शितिकण्ठः शिवाप्रियः}
{उग्रः कपालिः कामारिरन्धकासुरसूदनः}

\twolineshloka
{गङ्गाधरो ललाटाक्षः कालकालः कृपानिधिः}
{भीमः परशुहस्तश्च मृगपाणिर्जटाधरः}

\twolineshloka
{कैलासवासी कवची कठोरस्त्रिपुरान्तकः}
{वृषाङ्को वृषभारूढो भस्मोद्धूलितविग्रहः}

\twolineshloka
{सामप्रियः स्वरमयस्त्रयीमूर्तिरनीश्वरः}
{सर्वज्ञः परमात्मा च सोमसूर्याग्निलोचनः}

\twolineshloka
{हविर्यज्ञमयः सोमः पञ्चवक्त्रः सदाशिवः}
{विश्वेश्वरो वीरभद्रो गणनाथः प्रजापतिः}

\twolineshloka
{हिरण्यरेता दुर्धर्षो गिरीशो गिरिशोऽनघः}
{भुजङ्गभूषणो भर्गो गिरिधन्वा गिरिप्रियः}

\twolineshloka
{कृत्तिवासाः पुरारातिर्भगवान् प्रमथाधिपः}
{मृत्युञ्जयः सूक्ष्मतनुर्जगद्व्यापी जगद्गुरुः}

\twolineshloka
{व्योमकेशो महासेनजनकश्चारुविक्रमः}
{रुद्रो भूतपतिः स्थाणुरहिर्बुध्न्यो दिगम्बरः}

\twolineshloka
{अष्टमूर्तिरनेकात्मा सात्त्विकः शुद्धविग्रहः}
{शाश्वतः खण्डपरशुरजपाशविमोचकः}

\twolineshloka
{मृडः पशुपतिर्देवो महादेवोऽव्ययः प्रभुः}
{पूषदन्तभिदव्यग्रो दक्षाध्वरहरो हरः}

\twolineshloka
{भगनेत्रभिदव्यक्तः सहस्राक्षः सहस्रपात्}
{अपवर्गप्रदोऽनन्तस्तारकः परमेश्वरः}

\dnsub{फलश्रुतिः}
\twolineshloka
{इमानि दिव्यनामानि जप्यन्ते सर्वदा मया}
{नामकल्पलतेयं मे सर्वाभीष्टप्रदायिनि}

\twolineshloka
{नामान्येतानि सुभगे शिवदानि न संशयः}
{वेदसर्वस्वभूतानि नामान्येतानि वस्तुतः}

\twolineshloka
{एतानि यानि नामानि तानि सर्वार्थदान्यतः}
{जप्यन्ते सादरं नित्यं मया नियमपूर्वकम्}

\twolineshloka
{वेदेषु शिवनामानि श्रेष्ठान्यघहराणि च}
{सन्त्यनन्तानि सुभगे वेदेषु विविधेष्वपि}

\twolineshloka
{तेभ्यो नामानि सङ्गृह्य कुमाराय महेश्वरः}
{अष्टोत्तरसहस्रं तु नाम्नामुपदिशत् पुरा}

{॥इति शाक्तप्रमोदे श्रीशिवाष्टोत्तरशतनामस्तोत्रं सम्पूर्णम्॥}
\closesection
\clearpage
\phantomsection\addcontentsline{toc}{chapter}{रामस्तोत्राणि}
%% !TeX program = XeLaTeX
% !TeX root = ../../shloka.tex
\sect{अहल्याकृत-रामस्तोत्रम्}
\uvacha{अहल्योवाच}

\fourlineindentedshloka
{अहो कृतार्थाऽस्मि जगन्निवास ते}
{पादाब्जसंलग्नरजः कणादहम्}
{स्पृशामि यत्पद्मजशङ्करादिभिः}
{विमृग्यते रन्धितमानसैः सदा}% १

\fourlineindentedshloka
{अहो विचित्रं तव राम चेष्टितम्}
{मनुष्यभावेन विमोहितं जगत्}
{चलस्यजस्रं चरणादिवर्जितः}
{सम्पूर्ण आनन्दमयोऽतिमायिकः}% २

\fourlineindentedshloka
{यत्पादपङ्कजपरागपवित्रगात्रा}
{भागीरथी भवविरिञ्चिमुखान्  पुनाति}
{साक्षात्स एव मम दृग्विषयो यदाऽऽस्ते}
{किं वर्ण्यते मम पुराकृतभागधेयम्}% ३

\fourlineindentedshloka
{मर्त्यावतारे मनुजाकृतिं हरिम्}
{रामाभिधेयं रमणीयदेहिनम्}
{धनुर्धरं पद्मविशाललोचनम्}
{भजामि नित्यं न परान्  भजिष्ये}% ४

\fourlineindentedshloka
{यत्पादपङ्कजरजः श्रुतिभिर्विमृग्यम्}
{यन्नाभिपङ्कजभवः कमलासनश्च}
{यन्नामसाररसिको भगवान्पुरारिः}
{तं  रामचन्द्रमनिशं हृदि भावयामि}% ५

\fourlineindentedshloka
{यस्यावतारचरितानि विरिञ्चिलोके}
{गायन्ति नारदमुखा भवपद्मजाद्याः}
{आनन्दजाश्रुपरिषिक्तकुचाग्रसीमा}
{वागीश्वरी च तमहं शरणं प्रपद्ये}% ६

\fourlineindentedshloka
{सोऽयं परात्मा पुरुषः पुराणः}
{एकः स्वयं ज्योतिरनन्त आद्यः}
{मायातनुं लोकविमोहनीयाम्}
{धत्ते परानुग्रह एष रामः}% ७

\fourlineindentedshloka
{अयं हि विश्वोद्भवसंयमानाम्}
{एकः  स्वमायागुणबिम्बितो यः}
{विरिञ्चिविष्ण्वीश्वरनामभेदान्}
{धत्ते स्वतन्त्रः परिपूर्ण आत्मा}% ८

\fourlineindentedshloka
{नमोऽस्तु ते राम तवाङ्घ्रिपङ्कजम्}
{श्रिया धृतं वक्षसि लालितं प्रियात्}
{आक्रान्तमेकेन जगत्त्रयं पुरा}
{ध्येयं मुनीन्द्रैरभिमानवर्जितैः}% ९

\twolineshloka
{जगतामादिभूतस्त्वं जगत्त्वं जगदाश्रयः}
{सर्वभूतेष्वसंयुक्त एको भाति भवान् परः}% १०

\twolineshloka
{ओङ्कारवाच्यस्त्वं राम वाचामविषयः पुमान्}
{वाच्यवाचकभेदेन भवानेव जगन्मयः}% ११

\twolineshloka
{कार्यकारणकर्तृत्वफलसाधनभेदतः}
{एको विभासि राम त्वं मायया बहुरूपया}% १२

\twolineshloka
{त्वन्मायामोहितधियस्त्वां न जानन्ति तत्त्वतः}
{मानुषं त्वाऽभिमन्यन्ते मायिनं परमेश्वरम्}% १३

\twolineshloka
{आकाशवत्त्वं सर्वत्र बहिरन्तर्गतोऽमलः}
{असङ्गो ह्यचलो नित्यः शुद्धो बुद्धः सदव्ययः}% १४

\twolineshloka
{योषिन्मूढाऽहमज्ञा ते तत्त्वं जाने कथं विभो}
{तस्मात्ते शतशो राम नमस्कुर्यामनन्यधीः}% १५

\twolineshloka
{देव मे यत्रकुत्रापि स्थिताया अपि सर्वदा}
{त्वत्पादकमले सक्ता भक्तिरेव  सदाऽस्तु मे}% १६

\twolineshloka
{नमस्ते पुरुषाध्यक्ष नमस्ते भक्तवत्सल}
{नमस्तेऽस्तु हृषीकेश नारायण नमोऽस्तु ते}% १७

\fourlineindentedshloka
{भवभयहरमेकं भानुकोटिप्रकाशम्}
{करधृतशरचापं कालमेघावभासम्}
{कनकरुचिरवस्त्रं रत्नवत्कुण्डलाढ्यम्}
{कमलविशदनेत्रं सानुजं राममीडे}% १८

\twolineshloka
{स्तुत्वैवं पुरुषं साक्षाद्राघवं पुरतः स्थितम्}
{परिक्रम्य प्रणम्याऽऽशु सानुज्ञाता ययौ पतिम्}% १९

\twolineshloka
{अहल्यया कृतं स्तोत्रं यः पठेद्भक्तिसंयुतः}
{स मुच्यतेऽखिलैः पापैः परं ब्रह्माधिगच्छति}% २०

\twolineshloka
{पुत्राद्यर्थे पठेद्भक्त्या रामं हृदि निधाय च}
{संवत्सरेण लभते वन्ध्या अपि सुपुत्रकम्}% २१

\onelineshloka
{सर्वान् कामानवाप्नोति रामचन्द्रप्रसादतः}% २२

\fourlineindentedshloka
{ब्रह्मघ्नो गुरुतल्पगोऽपि पुरुषः स्तेयी सुरापोऽपि वा}
{मातृभ्रातृविहिंसकोऽपि सततं भोगैकबद्धातुरः}
{नित्यं स्तोत्रमिदं जपन्  रघुपतिं भक्त्या हृदिस्थं स्मरन्}
{ध्यायन् मुक्तिमुपैति किं पुनरसौ स्वाचारयुक्तो नरः}% २३

॥इति श्रीमदध्यात्मरामायणे श्री अहल्याविरचितं श्री~रामचन्द्रस्तोत्रं सम्पूर्णम्॥

% !TeX program = XeLaTeX
% !TeX root = ../../shloka.tex

\sect{आपदुद्धारण स्तोत्रम्}
\twolineshloka
{ॐ आपदामपहर्तारं दातारं सर्वसम्पदाम्}
{लोकाभिरामं श्रीरामं भूयो भूयो नमाम्यहम्}

\twolineshloka
{आर्तानामार्तिहन्तारं भीतानां भीतिनाशनम्}
{द्विषतां कालदण्डं तं रामचन्द्रं नमाम्यहम्}

\twolineshloka
{नमः कोदण्डहस्ताय सन्धीकृतशराय च}
{खण्डिताखिलदैत्याय रामायऽऽपन्निवारिणे}

\twolineshloka
{रामाय रामभद्राय रामचन्द्राय वेधसे}
{रघुनाथाय नाथाय सीतायाः पतये नमः}

\twolineshloka
{अग्रतः पृष्ठतश्चैव पार्श्वतश्च महाबलौ}
{आकर्णपूर्णधन्वानौ रक्षेतां रामलक्ष्मणौ}

\twolineshloka
{सन्नद्धः कवची खड्गी चापबाणधरो युवा}
{गच्छन् ममाग्रतो नित्यं रामः पातु सलक्ष्मणः}

\twolineshloka
{अच्युतानन्तगोविन्द नामोच्चारणभेषजात्}
{नश्यन्ति सकला रोगाः सत्यं सत्यं वदाम्यहम्}

\twolineshloka
{सत्यं सत्यं पुनः सत्यमुद्धृत्य भुजमुच्यते}
{वेदाच्छास्त्रं परं नास्ति न देवं केशवात्परम्}

\twolineshloka
{शरीरे जर्झरीभूते व्याधिग्रस्ते कलेवरे}
{औषधं जाह्नवीतोयं वैद्यो नारायणो हरिः}

\twolineshloka
{आलोड्य सर्वशास्त्राणि विचार्य च पुनः पुनः}
{इदमेकं सुनिष्पन्नं ध्येयो नारायणो हरिः}

% !TeX program = XeLaTeX
% !TeX root = ../../shloka.tex

\sect{गायत्री रामयाणम्}

\twolineshloka*
{शुक्लाम्बरधरं विष्णुं शशिवर्णं चतुर्भुजम्}
{प्रसन्नवदनं ध्यायेत् सर्वविघ्नोपशान्तये}

\twolineshloka*
{वागीशाद्याः सुमनसः सर्वार्थानामुपक्रमे}
{यं नत्वा कृतकृत्याः स्युस्तं नमामि गजाननम्}

\mbox{}\\
\dnsub{श्री~सरस्वती प्रार्थना}
\fourlineindentedshloka*
{दोर्भिर्युक्ता चतुर्भिं स्फटिकमणिनिभैरक्षमालां दधाना}
{हस्तेनैकेन पद्मं सितमपि च शुकं पुस्तकं चापरेण}
{भासा कुन्देन्दुशङ्खस्फटिकमणिनिभा भासमानाऽसमाना}
{सा मे वाग्देवतेयं निवसतु वदने सर्वदा सुप्रसन्ना}

\mbox{}\\
\dnsub{श्री~वाल्मीकि नमस्क्रिया}

\twolineshloka
{कूजन्तं राम रामेति मधुरं मधुराक्षरम्}
{आरुह्य कविताशाखां वन्दे वाल्मीकिकोकिलम्}

\twolineshloka
{वाल्मीकेर्मुनिसिंहस्य कवितावनचारिणः}
{शृण्वन् रामकथानादं को न याति परां गतिम्}

\twolineshloka
{यः पिबन् सततं रामचरितामृतसागरम्}
{अतृप्तस्तं मुनिं वन्दे प्राचेतसमकल्मषम्}

\begin{minipage}{\linewidth}
\centering
\resetShloka
\dnsub{श्री~हनुमन्नमस्क्रिया}

\twolineshloka
{गोष्पदीकृत-वाराशिं मशकीकृत-राक्षसम्}
{रामायण-महामाला-रत्नं वन्देऽनिलात्मजम्}
\end{minipage}

\twolineshloka
{अञ्जनानन्दनं वीरं जानकीशोकनाशनम्}
{कपीशमक्षहन्तारं वन्दे लङ्काभयङ्करम्}

\twolineshloka
{उल्लङ्घ्य सिन्धोः सलिलं सलीलं यः शोकवह्निं जनकात्मजायाः}
{आदाय तेनैव ददाह लङ्कां नमामि तं प्राञ्जलिराञ्जनेयम्}

\twolineshloka
{आञ्जनेयमतिपाटलाननं काञ्चनाद्रि-कमनीय-विग्रहम्}
{पारिजात-तरुमूल-वासिनं भावयामि पवमान-नन्दनम्}

\twolineshloka
{यत्र यत्र रघुनाथकीर्तनं तत्र तत्र कृतमस्तकाञ्जलिम्}
{बाष्पवारिपरिपूर्णलोचनं मारुतिं नमत राक्षसान्तकम्}

\twolineshloka
{मनोजवं मारुततुल्यवेगं जितेन्द्रियं बुद्धिमतां वरिष्ठम्}
{वातात्मजं वानरयूथमुख्यं श्रीरामदूतं शरणं प्रपद्ये}

\mbox{}\\
\resetShloka
\dnsub{श्री~रामायणप्रार्थना}

\fourlineindentedshloka
{यः कर्णाञ्जलिसम्पुटैरहरहः सम्यक् पिबत्यादरात्}
{वाल्मीकेर्वदनारविन्दगलितं रामायणाख्यं मधु}
{जन्म-व्याधि-जरा-विपत्ति-मरणैरत्यन्त-सोपद्रवम्}
{संसारं स विहाय गच्छति पुमान् विष्णोः पदं शाश्वतम्}

\twolineshloka
{तदुपगत-समास-सन्धियोगं सममधुरोपनतार्थ-वाक्यबद्धम्}
{रघुवरचरितं मुनिप्रणीतं दशशिरसश्च वधं निशामयध्वम्}

\twolineshloka
{वाल्मीकि-गिरिसम्भूता रामसागरगामिनी}
{पुनातु भुवनं पुण्या रामायणमहानदी}

\twolineshloka
{श्लोकसारजलाकीर्णं सर्गकल्लोलसङ्कुलम्}
{काण्डग्राहमहामीनं वन्दे रामायणार्णवम्}

\twolineshloka
{वेदवेद्ये परे पुंसि जाते दशरथात्मजे}
{वेदः प्राचेतसादासीत् साक्षाद्रामायणात्मना}

\mbox{}\\
\resetShloka
\dnsub{श्री~रामध्यानम्}

\fourlineindentedshloka
{वैदेहीसहितं सुरद्रुमतले हैमे महामण्डपे}
{मध्ये पुष्पकमासने मणिमये वीरासने सुस्थितम्}
{अग्रे वाचयति प्रभञ्जनसुते तत्त्वं मुनिभ्यः परम्}
{व्याख्यान्तं भरतादिभिः परिवृतं रामं भजे श्यामलम्}

\fourlineindentedshloka
{वामे भूमिसुता पुरश्च हनुमान् पश्चात् सुमित्रासुतः}
{शत्रुघ्नो भरतश्च पार्श्वदलयोर्वाय्वादिकोणेषु च}
{सुग्रीवश्च विभीषणश्च युवराट् तारासुतो जाम्बवान्}
{मध्ये नीलसरोजकोमलरुचिं रामं भजे श्यामलम्}

\twolineshloka
{नमोऽस्तु रामाय सलक्ष्मणाय देव्यै च तस्यै जनकात्मजायै}
{नमोऽस्तु रुद्रेन्द्रयमानिलेभ्यो नमोऽस्तु चन्द्रार्कमरुद्गणेभ्यः}

\resetShloka
\dnsub{गायत्री रामयाणम्}
\twolineshloka
{तपः स्वाध्यायनिरतं तपस्वी वाग्विदां वरम्}
{नारदं परिपप्रच्छ वाल्मीकिर्मुनिपुङ्गवम्}%(१.१.१)

\twolineshloka
{स हत्वा राक्षसान् सर्वान् यज्ञघ्नान् रघुनन्दनः}
{ऋषिभिः पूजितः सम्यक् यथेन्द्रो विजयी पुरा}%(१.३०.२३)

\twolineshloka
{विश्वामित्रस्तु धर्मात्मा श्रुत्वा जनकभाषितम्}
{वत्स राम धनुः पश्य इति राघवमब्रवीत्}%(१.६७.१२)

\twolineshloka
{तुष्टावास्य तदा वंशं  प्रविश्य च विशाम्पतेः}
{शयनीयं नरेन्द्रस्य तदासाद्य व्यतिष्ठत}%(२.१५.२०)

\twolineshloka
{वनवासं हि सङ्ख्याय वासांस्याभरणानि च}
{भर्तारमनुगच्छन्त्यै सीतायै श्वशुरो ददौ}%(२.४०.१५)

\twolineshloka
{राजा सत्यं च धर्मं च  राजा कुलवतां कुलम्}
{राजा माता पिता चैव राजा हितकरो नृणाम्}%(२.६७.३४)

\twolineshloka
{निरीक्ष्य स मुहूर्तं तु ददर्श भरतो गुरुम्}
{उटजे राममासीनं जटामण्डलधारिणम्}%(२.९९.२५)

\twolineshloka
{यदि बुद्धिः कृता द्रष्टुम् अगस्त्यं तं महामुनिम्}
{अद्यैव गमने बुद्धिं रोचयस्व महायशाः}%(३.११.४४)

\twolineshloka
{भरतस्यार्यपुत्रस्य श्वश्रूणां मम च प्रभो}
{मृगरूपमिदं व्यक्तं विस्मयं जनयिष्यति}%(३.४३.१७)

\twolineshloka
{गच्छ शीघ्रमितो राम सुग्रीवं तं महाबलम्}
{वयस्यं तं कुरु क्षिप्रमितो गत्वाऽद्य राघव}% (३.७२.१७)

\twolineshloka
{देशकालौ प्रतीक्षस्व क्षममाणः प्रियाप्रिये}
{सुखदुःखसहः काले  सुग्रीववशगो भव}% (४.२२.२०)

\twolineshloka
{वन्द्यास्ते तु तपः सिद्धास्तपसा वीतकल्मषाः}
{प्रष्टव्याश्चापि सीतायाः प्रवृत्तिं विनयान्वितैः}%(४.४३.३४)

\twolineshloka
{स निर्जित्य पुरीं श्रेष्ठां लङ्कां तां कामरूपिणीम्}
{विक्रमेण महातेजा हनूमान्मारुतात्मजः}%(५.४.१)

\twolineshloka
{धन्या देवाः सगन्धर्वाः सिद्धाश्च परमर्षयः}
{मम पश्यन्ति ये नाथं रामं राजीवलोचनम्}%(५.२६.४१)

\twolineshloka
{मङ्गलाभिमुखी तस्य सा तदासीन्महाकपेः}
{उपतस्थे विशालाक्षी प्रयता हव्यवाहनम्}% (५.५३.२६)  

\fourlineindentedshloka
{हितं महार्थं मृदु हेतुसंहितम्}
{व्यतीतकालायतिसम्प्रतिक्षमम्}
{निशम्य तद्वाक्यमुपस्थितज्वरः}
{प्रसङ्गवानुत्तरमेतदब्रवीत्}%(६.१०.२७)

\twolineshloka
{धर्मात्मा रक्षसां श्रेष्ठः सम्प्राप्तोऽयं विभीषणः}
{लङ्कैश्वर्यं ध्रुवं श्रीमानयं प्राप्नोत्यकण्टकम्}%(६.४१.६८)

\fourlineindentedshloka
{यो वज्रपाताशनिसन्निपातान्}{न चुक्षुभे नापि चचाल राजा}
{स रामबाणाभिहतो भृशार्तः}{चचाल चापं च मुमोच वीरः}%(६.५९.१४०)

\twolineshloka
{यस्य विक्रममासाद्य राक्षसा निधनं गताः}
{तं मन्ये राघवं वीरं नारायणमनामयम्}%(६.७२.११)

\twolineshloka
{न ते ददर्शिरे रामं दहन्तमरिवाहिनीम्}
{मोहिताः परमास्त्रेण गान्धर्वेण महात्मना}%६.९४.२६

\twolineshloka
{प्रणम्य देवताभ्यश्च ब्राह्मणेभ्यश्च मैथिली}
{बद्धाञ्जलिपुटा चेदमुवाचाग्निसमीपतः}%(६.११९.२३)

\twolineshloka
{चलनात्पर्वतेन्द्रस्य गणा देवाश्च कम्पिताः}
{चचाल पार्वती चापि तदाऽऽश्लिष्टा महेश्वरम्} % (७.१६.२६)

\twolineshloka
{दाराः पुत्राः पुरं राष्ट्रं भोगाच्छादनभोजनम्}
{सर्वमेवाविभक्तं नौ भविष्यति हरीश्वर}%(७.३४.४१)

\twolineshloka
{यामेव रात्रिं शत्रुघ्नः पर्णशालां समाविशत्}
{तामेव रात्रिं सीताऽपि प्रसूता दारकद्वयम्}%(७.६६.१)

\twolineshloka
{इदं रामायणं कृत्स्नं गायत्रीबीजसंयुतम्}
{त्रिसन्ध्यं यः पठेन्नित्यं सर्वपापैः प्रमुच्यते}

॥इति श्री गायत्री रामायणं सम्पूर्णम्॥

\mbox{}\\
\resetShloka
\dnsub{मङ्गलश्लोकाः}
\fourlineindentedshloka
{स्वस्ति प्रजाभ्यः परिपालयन्ताम्}
{न्यायेन मार्गेण महीं महीशाः}
{गोब्राह्मणेभ्यः शुभमस्तु नित्यम्}
{लोकाः समस्ताः सुखिनो भवन्तु}

\twolineshloka
{काले वर्षतु पर्जन्यः पृथिवी सस्यशालिनी}
{देशोऽयं क्षोभरहितो ब्राह्मणाः सन्तु निर्भयाः}

\twolineshloka
{अपुत्राः पुत्रिणः सन्तु पुत्रिणः सन्तु पौत्रिणः}
{अधनाः सधनाः सन्तु जीवन्तु शरदां शतम्}

\twolineshloka
{चरितं रघुनाथस्य शतकोटि-प्रविस्तरम्}
{एकैकमक्षरं पुंसां महापातकनाशनम्}

\twolineshloka
{शृण्वन् रामायणं भक्त्या यः पादं पदमेव वा}
{स याति ब्रह्मणः स्थानं ब्रह्मणा पूज्यते सदा}

\twolineshloka
{रामाय रामभद्राय रामचन्द्राय वेधसे}
{रघुनाथाय नाथाय सीतायाः पतये नमः}

\twolineshloka
{यन्मङ्गलं सहस्राक्षे सर्वदेवनमस्कृते}
{वृत्रनाशे समभवत् तत्ते भवतु मङ्गलम्}

\twolineshloka
{यन्मङ्गलं सुपर्णस्य विनताऽकल्पयत् पुरा}
{अदितिर्मङ्गलं प्रादात् तत्ते भवतु मङ्गलम्}

\twolineshloka
{त्रीन् विक्रमान् प्रक्रमतो विष्णोरमिततेजसः}
{यदासीन्मङ्गलं राम तत्ते भवतु मङ्गलम्}

\twolineshloka
{ऋतवः सागरा द्वीपा वेदा लोका दिशश्च ये}
{मङ्गलानि महाबाहो दिशन्तु तव सर्वदा}

\twolineshloka*
{कायेन वाचा मनसेन्द्रियैर्वा बुद्‌ध्याऽऽत्मना वा प्रकृतेः स्वभावात्}
{करोमि यद्यत् सकलं परस्मै नारायणायेति समर्पयामि}

\closesection
\clearpage

\phantomsection\addcontentsline{toc}{chapter}{शक्तिस्तोत्राणि}
% !TeX program = XeLaTeX
% !TeX root = ../../shloka.tex

\sect{कामाक्षी सुप्रभातम्}

\fourlineindentedshloka*
{जगदवन विधौ त्वं जागरूका भवानि}
{तव तु जननि निद्रामात्मवत्कल्पयित्वा}
{प्रतिदिवसमहं त्वां बोधयामि प्रभाते}
{त्वयि कृतमपराधं सर्वमेतं क्षमस्व}

\fourlineindentedshloka*
{यदि प्रभातं तव सुप्रभातम्}
{तदा प्रभातं मम सुप्रभातम्}
{तस्मात् प्रभाते तव सुप्रभातम्}
{वक्ष्यामि मातः कुरु सुप्रभातम्}

\dnsub{गुरु ध्यानम्}
\fourlineindentedshloka*
{यस्याङ्घ्रिपद्म-मकरन्दनिषेवणात् त्वम्}
{जिह्वां गताऽसि वरदे मम मन्दबद्धः}
{यस्याम्ब नित्यमनघे हृदये विभासि}
{तं चन्द्रशेखरगुरुं प्रणमामि नित्यम्}

\fourlineindentedshloka*
{जये जयेन्द्रो गुरुणा ग्रहीतो}
{मठाधिपत्ये शशिशेखरेण}
{यथा गुरुः सर्वगुणोपपन्नो}
{जयत्यसौ मङ्गलमातनोतु}

\fourlineindentedshloka*
{शुभं दिशतु नो देवी कामाक्षी सर्वमङ्गला}
{शुभं दिशतु नो देवी कामकोटी-मठेशः}
{शुभं दिशतु तच्छिष्य-सद्गुरुर्नो जयेन्द्रो}
{सर्वंमङ्गलमेवास्तु मङ्गलानि भवन्तु नः}

\dnsub{सुप्रभातम्}
\fourlineindentedshloka
{कामाक्षि देव्यम्ब तवाऽऽर्द्रदृष्ट्या}
{मूकः स्वयं मूककविर्यथाऽसीत्}
{तथा कुरु त्वं परमेशजाये}
{त्वत्पादमूले प्रणतं दयार्द्रे}

\twolineshloka
{उत्तिष्ठोत्तिष्ठ वरदे उत्तिष्ठ जगदीश्वरि}
{उत्तिष्ठ जगदाधारे त्रैलोक्यं मङ्गलं कुरु}

\fourlineindentedshloka
{शृणोषि कच्चिद्-ध्वनिरुत्थितोऽयम्}
{मृदङ्गभेरीपटहानकानाम्}
{वेदध्वनिं शिक्षितभूसुराणाम्}
{शृणोषि भद्रे कुरु सुप्रभातम्}

\fourlineindentedshloka
{शृणोषि भद्रे ननु शङ्खघोषम्}
{वैतालिकानां मधुरं च गानम्}
{शृणोषि मातः पिककुक्कुटानाम्}
{ध्वनिं प्रभाते कुरु सुप्रभातम्}

\fourlineindentedshloka
{मातर्निरीक्ष्य वदनं भगवान् शशाङ्को -}
{लज्जान्वितः स्वयमहो निलयं प्रविष्टः}
{द्रष्टुं त्वदीय वदनं भगवान् दिनेशो -}
{ह्यायाति देवि सदनं कुरु सुप्रभातम्}

\fourlineindentedshloka
{पश्याम्ब केचिद्-धृतपूर्णकुम्भाः}
{केचिद्-दयार्द्रे धृतपुष्पमालाः}
{काचित् शुभाङ्ग्यो ननुवाद्यहस्ताः}
{तिष्ठन्ति तेषां कुरु सुप्रभातम्}

\fourlineindentedshloka
{भेरीमृदङ्गपणवानकवाद्यहस्ताः}
{स्तोतुं महेशदयिते स्तुतिपाठकास्त्वाम्}
{तिष्ठन्ति देवि समयं तव काङ्क्षमाणाः}
{ह्युत्तिष्ठ दिव्यशयनात् कुरु सुप्रभातम्}

\fourlineindentedshloka
{मातर्निरीक्ष्य वदनं भगवान् त्वदीयम्}
{नैवोत्थितः शशिधिया शयितस्तवाङ्के}
{सम्बोधयाऽऽशु गिरिजे विमलं प्रभातम्}
{जातं महेशदयिते कुरु सुप्रभातम्}

\fourlineindentedshloka
{अन्तश्चरन्त्यास्तव भूषणानाम्}
{झल्झल्ध्वनिं नूपुरकङ्कणानाम्}
{श्रुत्वा प्रभाते तव दर्शनार्थी}
{द्वारि स्थितोऽहं कुरु सुप्रभातम्}

\fourlineindentedshloka
{वाणी पुस्तकमम्बिके गिरिसुते पद्मानि पद्मासना}
{रम्भा त्वम्बरडम्बरं गिरिसुता गङ्गा च गङ्गाजलम्}
{काली तालयुगं मृदङ्गयुगलं बृन्दा च नन्दा तथा}
{नीला निर्मलदर्पण-धृतवती तासां प्रभातं शुभम्}

\fourlineindentedshloka
{उत्थाय देवि शयनाद्भगवान् पुरारिः}
{स्नातुं प्रयाति गिरिजे सुरलोकनद्याम्}
{नैको हि गन्तुमनघे रमते दयार्द्रे}
{ह्युत्तिष्ठ देवि शयनात् कुरु सुप्रभातम्}

\fourlineindentedshloka
{पश्याम्ब केचित्फलपुष्पहस्ताः}
{केचित् पुराणानि पठन्ति मातः}
{पठन्ति वेदान् बहवस्तवाऽऽरे}
{तेषां जनानां कुरु सुप्रभातम्}

\fourlineindentedshloka
{लावण्यशेवधिमवेक्ष्य चिरं त्वदीयम्}
{कन्दर्पदर्पदलनोऽपि वशं गतस्ते}
{कामारि-चुम्बित-कपोलयुगं त्वदीयम्}
{द्रष्टुं स्थिता वयमये कुरु सुप्रभातम्}

\fourlineindentedshloka
{गाङ्गेयतोयममवाह्य मुनीश्वरास्त्वाम्}
{गङ्गाजलैः स्नपयितुं बहवो घटांश्च}
{धृत्वा शिरःसु भवतीमभिकाङ्क्षमाणाः}
{द्वारि स्थिता हि वरदे कुरु सुप्रभातम्}

\fourlineindentedshloka
{मन्दार-कुन्द-कुसुमैरपि जातिपुष्पैः}
{मालाकृता विरचितानि मनोहराणि}
{माल्यानि दिव्यपदयोरपि दातुमम्ब}
{तिष्ठन्ति देवि मुनयः कुरु सुप्रभातम्}

\fourlineindentedshloka
{काञ्ची-कलाप-परिरम्भनितम्बबिम्बम्}
{काश्मीर-चन्दन-विलेपित-कण्ठदेशम्}
{कामेश-चुम्बित-कपोलमुदारनासाम्}
{द्रष्टुं स्थिता वयमये कुरु सुप्रभातम्}

\fourlineindentedshloka
{मन्दस्मितं विमलचारुविशालनेत्रम्}
{कण्ठस्थलं कमलकोमलगर्भगौरम्}
{चक्राङ्कितं च युगलं पदयोर्मृगाक्षि}
{द्रष्टुं स्थिता वयमये कुरु सुप्रभातम्}

\fourlineindentedshloka
{मन्दस्मितं त्रिपुरनाशकरं पुरारेः}
{कामेश्वरप्रणयकोपहरं स्मितं ते}
{मन्दस्मितं विपुलहासमवेक्षितुं ते}
{मातः स्थिता वयमये कुरु सुप्रभातम्}

\fourlineindentedshloka
{माता शिशूनां परिरक्षणार्थम्}
{न चैव निद्रावशमेति लोके}
{माता त्रयाणां जगतां गतिस्त्वम्}
{सदा विनिद्रा कुरु सुप्रभातम्}

\fourlineindentedshloka
{मातर्मुरारिकमलासनवन्दिताङ्घ्र्याः}
{हृद्यानि दिव्यमधुराणि मनोहराणि}
{श्रोतुं तवाम्ब वचनानि शुभप्रदानि}
{द्वारि स्थिता वयमये कुरु सुप्रभातम्}

\fourlineindentedshloka
{दिगम्बरो ब्रह्मकपालपाणिः}
{विकीर्णकेशः फणिवेष्टिताङ्गः}
{तथाऽपि मातस्तव देविसङ्गात्}
{महेश्वरोऽभूत् कुरु सुप्रभातम्}

\fourlineindentedshloka
{अयि तु जननि दत्तस्तन्यपानेन देवि}
{द्रविडशिशुरभूद्वै ज्ञानसम्पन्नमूर्तिः}
{द्रविडतनयभुक्तक्षीरशेषं भवानि}
{वितरसि यदि मातः सुप्रभातं भवेन्मे}

\fourlineindentedshloka
{जननि तव कुमारः स्तन्यपानप्रभावात्}
{शिशुरपि तव भर्तुः कर्णमूले भवानि}
{प्रणवपदविशेषं बोधयामास देवि}
{यदि मयि च कृपा ते सुप्रभातं भवेन्मे}

\fourlineindentedshloka
{त्वं विश्वनाथस्य विशालनेत्रा}
{हालास्यनाथस्य नु मीननेत्रा}
{एकाम्रनाथस्य नु कामनेत्रा}
{कामेशजाये कुरु सुप्रभातम्}

\fourlineindentedshloka
{श्रीचन्द्रशेखरगुरुर्भगवान् शरण्ये}
{त्वत्पादभक्तिभरितः फलपुष्पपाणिः}
{एकाम्रनाथदयिते तव दर्शनार्थी}
{तिष्ठत्ययं यतिवरो मम सुप्रभातम्}

\fourlineindentedshloka
{एकाम्रनाथदयिते ननु कामपीठे}
{सम्पूजिताऽसि वरदे गुरुशङ्करेण}
{श्रीशङ्करादिगुरुवर्य-समर्चिताङ्घ्रिम्}
{द्रष्टुं स्थिता वयमये कुरु सुप्रभातम्}

\fourlineindentedshloka
{दुरितशमनदक्षौ मृत्युसन्तासदक्षौ}
{चरणमुपगतानां मुक्तिदौ ज्ञानदौ तौ}
{अभयवरदहस्तौ द्रष्टुमम्ब स्थितोऽहम्}
{त्रिपुरदलनजाये सुप्रभातं ममाऽऽर्ये}

\fourlineindentedshloka
{मातस्त्वदीयचरणं हरिपद्मजाद्यैः}
{वन्द्यं रथाङ्ग-सरसीरुह-शङ्खचिह्नम्}
{द्रष्टुं च योगिजनमानसराजहंसम्}
{द्वारि स्थितोऽस्मि वरदे कुरु सुप्रभातम्}

\fourlineindentedshloka
{पश्यन्तु केचिद्वदनं त्वदीयम्}
{स्तुवन्तु कल्याणगुणांस्तवान्ये}
{नमन्तु पादाब्जयुगं त्वदीयाः}
{द्वारि स्थितानां कुरु सुप्रभातम्}

\fourlineindentedshloka
{केचित् सुमेरोः शिखरेऽतितुङ्गे}
{केचिन्मणिद्वीपवरे विशाले}
{पश्यन्तु केचित् त्वमृताब्धिमध्ये}
{पश्याम्यहं त्वामिह सुप्रभातम्}
\setlength{\shlokaspaceskip}{16pt}
\fourlineindentedshloka
{शम्भोर्वामाङ्कसंस्थां शशिनिभवदनां नीलपद्मायताक्षीम्}
{श्यामाङ्गां चारुहासां निबिडतरकुचां पक्वबिम्बाधरोष्ठीम्}
{कामाक्षीं कामदात्रीं कुटिलकचभरां भूषणैर्भूषिताङ्गीम्}
{पश्यामः सुप्रभाते प्रणतजनिमतामद्य नः सुप्रभातम्}
\setlength{\shlokaspaceskip}{24pt}
\fourlineindentedshloka
{कामप्रदाकल्पतरुर्विभासि}
{नान्या गतिर्मे ननु चातकोऽहम्}
{वर्षस्यमोघः कनकाम्बुधाराः}
{काश्चित्तु धारा मयि कल्पयाऽऽशु}

\twolineshloka
{त्रिलोचनप्रियां वन्दे वन्दे त्रिपुरसुन्दरीम्}
{त्रिलोकनायिकां वन्दे सुप्रभातं ममाम्बिके}

\fourlineindentedshloka
{कामाक्षि देव्यम्ब तवाऽऽर्द्रदृष्ट्या}
{कृतं मयेदं खलु सुप्रभातम्}
{सद्यः फलं मे सुखमम्ब लब्धम्}
{तथा च मे दुःखदशा गता हि}

\fourlineindentedshloka
{ये वा प्रभाते पुरतस्तवाऽऽर्ये}
{पठन्ति भक्त्या ननु सुप्रभातम्}
{शृण्वन्ति ये वा त्वयि बद्धचित्ताः}
{तेषां प्रभातं कुरु सुप्रभातम्}

॥इति~श्री~लक्ष्मीकान्त-शर्मा-विरचितं श्री~कामाक्षीसुप्रभातं सम्पूर्णम्॥
% !TeX program = XeLaTeX
% !TeX root = ../../shloka.tex

\sect{कामाक्षी माहात्म्यम्}

\twolineshloka
{स्वामिपुष्करिणीतीर्थं पूर्वसिन्धुः पिनाकिनी}
{शिलाह्रदश्चतुर्मध्यं यावत् तुण्डीरमण्डलम्}

\twolineshloka
{मध्ये तुण्डीरभूवृत्तं कम्पा-वेगवती-द्वयोः}
{तयोर्मध्यं कामकोष्ठं कामाक्षी तत्र वर्तते}

\twolineshloka
{स एव विग्रहो देव्या मूलभूतोऽद्रिराड्भुवः}
{नान्योऽस्ति विग्रहो देव्याः काञ्च्यां तन्मूलविग्रहः}

\twolineshloka
{जगत्कामकलाकारं नाभिस्थानं भुवः परम्}
{पदपद्मस्य कामाक्ष्याः महापीठमुपास्महे}

\threelineshloka
{कामकोटिः स्मृतः सोऽयं कारणादेव चिन्नभः}
{यत्र कामकृतो धर्मो जन्तुना येन केन वा}
{सकृद्वाऽपि सुधर्माणां फलं फलति कोटिशः}

\twolineshloka
{यो जपेत् कामकोष्ठेऽस्मिन् मन्त्रमिष्टार्थदैवतम्}
{कोटिवर्णफलेनैव मुक्तिलोकं स गच्छति}

\twolineshloka
{यो वसेत् कमकोष्ठेऽस्मिन् क्षणार्धं वा तदर्धकम्}
{मुच्यते सर्वपापेभ्यः साक्षाद्देवी नराकृतिः}

\twolineshloka
{गायत्रीमण्डपाधारं भूनाभिस्थानमुत्तमम्}
{पुरुषार्थप्रदं शम्भोर्बिलाभ्रं तं नमाम्यहम्}

\twolineshloka
{यः कुर्यात् कामकोष्ठस्य बिलाभ्रस्य प्रदक्षिणम्}
{पदसङ्ख्याक्रमेणैव गोगर्भजननं लभेत्}

\twolineshloka
{विश्वकारणनेत्राढ्यां श्रीमत्त्रिपुरसुन्दरीम्}
{बन्धकासुरसंहन्त्रीं कामाक्षीं तामहं भजे}

\threelineshloka
{पराजन्मदिने काञ्च्यां महाभ्यन्तरमार्गतः}
{योऽर्चयेत् तत्र कामाक्षीं कोटिपूजाफलं लभेत्}
{तत्फलोत्पन्नकैवल्यं सकृत् कामाक्षिसेवया}

\twolineshloka
{त्रिस्थाननिलयं देवं त्रिविधाकारमच्युतम्}
{प्रतिलिङ्गाग्रसंयुक्तं भूतबन्धं तमाश्रये}

\twolineshloka*
{य इदं प्रातरुत्थाय स्नानकाले पठेन्नरः}
{द्वादशश्लोकमात्रेण श्लोकोक्तफलमाप्नुयात्}

॥इति श्री कामाक्षी-विलासे त्रयोविंशे अध्याये श्री कामाक्षी माहात्म्यं सम्पूर्णम्॥
% !TeX program = XeLaTeX
% !TeX root = ../../shloka.tex

\sect{मीनाक्षीपञ्चरत्नम्}

\fourlineindentedshloka
{उद्यद्भानु-सहस्रकोटिसदृशां केयूरहारोज्ज्वलाम्}
{बिम्बोष्ठीं स्मितदन्तपङ्क्तिरुचिरां पीताम्बरालङ्कृताम्‌}
{विष्णुब्रह्मसुरेन्द्रसेवितपदां तत्त्वस्वरूपां शिवाम्}
{मीनाक्षीं प्रणतोऽस्मि सन्ततमहं कारुण्यवारान्निधिम्‌}%॥ १॥

\fourlineindentedshloka
{मुक्ताहारलसत्किरीटरुचिरां पूर्णेन्दुवक्त्रप्रभाम्}
{शिञ्जन्नूपुरकिङ्किणीमणिधरां पद्मप्रभाभासुराम्‌}
{सर्वाभीष्टफलप्रदां गिरिसुतां वाणीरमासेविताम्}
{मीनाक्षीं प्रणतोऽस्मि सन्ततमहं कारुण्यवारान्निधिम्‌}%॥ २॥

\fourlineindentedshloka
{श्रीविद्यां शिववामभागनिलयां ह्रीङ्कारमन्त्रोज्ज्वलाम्}
{श्रीचक्राङ्कित-बिन्दुमध्यवसतीं श्रीमत्सभानायकीम्}
{श्रीमट्षण्मुखविघ्नराजजननीं श्रीमज्जगन्मोहिनीम्}
{मीनाक्षीं प्रणतोऽस्मि सन्ततमहं कारुण्यवारान्निधिम्}%॥ ३॥

\fourlineindentedshloka
{श्रीमत्सुन्दरनायकीं भयहरां ज्ञानप्रदां निर्मलाम्‌}
{श्यामाभां कमलासनार्चितपदां नारायणस्यानुजाम्‌}
{वीणावेणुमृदङ्गवाद्यरसिकां नानाविधाडाम्बिकाम्}
{मीनाक्षीं प्रणतोऽस्मि सन्ततमहं कारुण्यवारान्निधिम्‌}%॥ ४॥

\fourlineindentedshloka
{नानायोगिमुनीन्द्रहृन्निवसतीं नानार्थसिद्धिप्रदाम्‌}
{नानापुष्पविराजिताङ्घ्रियुगलां नारायणेनार्चिताम्‌}
{नादब्रह्ममयीं परात्परतरां नानार्थतत्त्वात्मिकाम्}
{मीनाक्षीं प्रणतोऽस्मि सन्ततमहं कारुण्यवारान्निधिम्‌}%॥ ५॥

॥इति श्रीमच्छङ्कराचार्यविरचितं श्री~मीनाक्षीपञ्चरत्नं सम्पूर्णम्॥
% !TeX program = XeLaTeX
% !TeX root = ../../shloka.tex

\sect{अन्नपूर्णास्तोत्रम्}
\fourlineindentedshloka
{नित्यानन्दकरी वराभयकरी सौन्दर्यरत्नाकरी}
{निर्धूताखिलघोरपावनकरी प्रत्यक्षमाहेश्वरी}
{प्रालेयाचलवंशपावनकरी काशीपुराधीश्वरी}
{भिक्षां देहि कृपावलम्बनकरी माताऽन्नपूर्णेश्वरी}

\fourlineindentedshloka
{नानारत्नविचित्रभूषणकरी हेमाम्बराडम्बरी}
{मुक्ताहारविलम्बमानविलसद्-वक्षोजकुम्भान्तरी}
{काश्मीरागरुवासिता रुचिकरी काशीपुराधीश्वरी}
{भिक्षां देहि कृपावलम्बनकरी माताऽन्नपूर्णेश्वरी}

\fourlineindentedshloka
{योगानन्दकरी रिपुक्षयकरी धर्मैकनिष्ठाकरी}
{चन्द्रार्कानलभासमानलहरी त्रैलोक्यरक्षाकरी}
{सर्वैश्वर्यकरी तपःफलकरी काशीपुराधीश्वरी}
{भिक्षां देहि कृपावलम्बनकरी माताऽन्नपूर्णेश्वरी}

\fourlineindentedshloka
{कैलासाचलकन्दरालयकरी गौरी उमा शङ्करी}
{कौमारी निगमार्थगोचरकरी ओङ्कारबीजाक्षरी}
{मोक्षद्वारकवाटपाटनकरी काशीपुराधीश्वरी}
{भिक्षां देहि कृपावलम्बनकरी माताऽन्नपूर्णेश्वरी}

\fourlineindentedshloka
{दृश्यादृश्यविभूतिवाहनकरी ब्रह्माण्डभाण्डोदरी}
{लीलानाटकसूत्रखेलनकरी विज्ञानदीपाङ्कुरी}
{श्रीविश्वेशमनःप्रसादनकरी काशीपुराधीश्वरी}
{भिक्षां देहि कृपावलम्बनकरी माताऽन्नपूर्णेश्वरी}

\fourlineindentedshloka
{आदिक्षान्तसमस्तवर्णनकरी शम्भोस्त्रिभावाकरी}
{काश्मीरा त्रिपुरेश्वरी त्रिनयनी विश्वेश्वरी शर्वरी}
{स्वर्गद्वारकवाटपाटनकरी काशीपुराधीश्वरी}
{भिक्षां देहि कृपावलम्बनकरी माताऽन्नपूर्णेश्वरी}

\fourlineindentedshloka
{उर्वी सर्वजनेश्वरी जयकरी माता कृपासागरी}
{वेणीनीलसमानकुन्तलधरी नित्यान्नदानेश्वरी}
{साक्षान्मोक्षकरी सदा शुभकरी काशीपुराधीश्वरी}
{भिक्षां देहि कृपावलम्बनकरी माताऽन्नपूर्णेश्वरी}

\fourlineindentedshloka
{देवी सर्वविचित्ररत्नरचिता दाक्षायणी सुन्दरी}
{वामे स्वादुपयोधरा प्रियकरी सौभाग्यमाहेश्वरी}
{भक्ताभीष्टकरी सदा शुभकरी काशीपुराधीश्वरी}
{भिक्षां देहि कृपावलम्बनकरी माताऽन्नपूर्णेश्वरी}

\fourlineindentedshloka
{चन्द्रार्कानलकोटिकोटिसदृशी चन्द्रांशुबिम्बाधरी}
{चन्द्रार्काग्निसमानकुण्डलधरी चन्द्रार्कवर्णेश्वरी}
{मालापुस्तकपाशसाङ्कुशधरी काशीपुराधीश्वरी}
{भिक्षां देहि कृपावलम्बनकरी माताऽन्नपूर्णेश्वरी}

\fourlineindentedshloka
{क्षत्रत्राणकरी महाऽभयकरी माता कृपासागरी}
{सर्वानन्दकरी सदा शिवकरी विश्वेश्वरी श्रीधरी}
{दक्षाक्रन्दकरी निरामयकरी काशीपुराधीश्वरी}
{भिक्षां देहि कृपावलम्बनकरी माताऽन्नपूर्णेश्वरी}

\twolineshloka*
{अन्नपूर्णे सदापूर्णे शङ्करप्राणवल्लभे}
{ज्ञानवैराग्यसिद्‌ध्यर्थं भिक्षां देहि च पार्वति}

\twolineshloka*
{माता च पार्वती देवी पिता देवो महेश्वरः}
{बान्धवाः शिवभक्ताश्च स्वदेशो भुवनत्रयम्}
॥इति  श्रीमच्छङ्कराचार्यविरचितं श्री~अन्नपूर्णास्तोत्रं सम्पूर्णम्॥
% !TeX program = XeLaTeX
% !TeX root = ../../shloka.tex

\sect{गौर्यष्टोत्तरशतनामस्तोत्रम्}
\twolineshloka
{गौरी गणेशजननी गिरिराजतनूद्भवा}
{गुहाम्बिका जगन्माता गङ्गाधरकुटुम्बिनी}
\twolineshloka
{वीरभद्रप्रसूर्विश्वव्यापिनी विश्वरूपिणी}
{अष्टमूर्त्यात्मिका कष्टदारिद्र्यशमनी शिवा}
\twolineshloka
{शाम्भवी शङ्करी बाला भवानी भद्रदायिनी}
{माङ्गल्यदायिनी सर्वमङ्गला मञ्जुभाषिणी}
\twolineshloka
{महेश्वरी महामाया मन्त्राराध्या महाबला}
{हेमाद्रिजा हैमवती पार्वती पापनाशिनी}
\twolineshloka
{नारायणांशजा नित्या निरीशा निर्मलाऽम्बिका}
{मृडानी मुनिसंसेव्या मानिनी मेनकात्मजा}
\twolineshloka
{कुमारी कन्यका दुर्गा कलिदोषनिषूदिनी}
{कात्यायनी कृपापूर्णा कल्याणी कमलार्चिता}
\twolineshloka
{सती सर्वमयी चैव सौभाग्यदा सरस्वती}
{अमलाऽमरसंसेव्या अन्नपूर्णाऽमृतेश्वरी}
\twolineshloka
{अखिलागमसंसेव्या सुखसच्चित्सुधारसा}
{बाल्याराधितभूतेशा भानुकोटिसमद्युतिः}
\twolineshloka
{हिरण्मयी परा सूक्ष्मा शीतांशुकृतशेखरा}
{हरिद्राकुङ्कुमाराध्या सर्वकालसुमङ्गली}
\twolineshloka
{सर्वबोधप्रदा सामशिखा वेदान्तलक्षणा}
{कर्मब्रह्ममयी कामकलना काङ्क्षितार्थदा}
\twolineshloka
{चन्द्रार्कायितताटङ्का चिदम्बरशरीरिणी}
{श्रीचक्रवासिनी देवी कला कामेश्वरप्रिया}
\twolineshloka
{मारारातिप्रियार्धाङ्गी मार्कण्डेयवरप्रदा}
{पुत्रपौत्रप्रदा पुण्या पुरुषार्थप्रदायिनी}
\twolineshloka
{सत्यधर्मरता सर्वसाक्षिणी सर्वरूपिणी}
{श्यामला बगला चण्डी मातृका भगमालिनी}
\twolineshloka
{शूलिनी विरजा स्वाहा स्वधा प्रत्यङ्गिराम्बिका}
{आर्या दाक्षायणी दीक्षा सर्ववस्तूत्तमोत्तमा}
\threelineshloka
{शिवाभिधाना श्रीविद्या प्रणवार्थस्वरूपिणी}
{ह्रीङ्कारी नादरूपा च त्रिपुरा त्रिगुणेश्वरी}
{सुन्दरी स्वर्णगौरी च षोडशाक्षरदेवता}

{॥इति श्री गौर्यष्टोत्तरशतनामस्तोत्रं सम्पूर्णम्॥}
% !TeX program = XeLaTeX
% !TeX root = ../../shloka.tex

\sect{दुर्गापञ्चरत्नम्}

\fourlineindentedshloka
{ते ध्यान-योगानुगता अपश्यन्}
{त्वामेव देवीं स्वगुणैर्निगूढाम्}
{त्वमेव शक्तिः परमेश्वरस्य}
{मां पाहि सर्वेश्वरि मोक्षदात्रि}%॥ १॥

\fourlineindentedshloka
{देवात्मशक्तिः श्रुतिवाक्यगीता}
{महर्षि लोकस्य पुरः प्रसन्ना}
{गुहा परं व्योम सतः प्रतिष्ठा}
{मां पाहि सर्वेश्वरि मोक्षदात्रि}%॥ २॥

\fourlineindentedshloka
{परास्य शक्तिर्विविधैव श्रूयसे}
{श्वेताश्व-वाक्योदित-देवि दुर्गे}
{स्वाभाविकी ज्ञानबलक्रिया ते}
{मां पाहि सर्वेश्वरि मोक्षदात्रि}%॥ ३॥

\fourlineindentedshloka
{देवात्मशब्देन शिवात्मभूता}
{यत्कूर्मवायव्यवचो विवृत्या}
{त्वं पाशविच्छेदकरी प्रसिद्धा}
{मां पाहि सर्वेश्वरि मोक्षदात्रि}%॥ ४॥

\fourlineindentedshloka
{त्वं ब्रह्मपुच्छा विविधा मयूरी}
{ब्रह्म-प्रतिष्ठाऽस्युपदिष्ट-गीता}
{ज्ञानस्वरूपात्मतयाऽखिलानाम्}
{मां पाहि सर्वेश्वरि मोक्षदात्रि}%॥ ५॥

{॥इति श्री काञ्चीपुरजगद्गुरुणा~श्रीमच्चन्द्रशेखरेन्द्र-\\सरस्वती-स्वामिना विरचितं श्री दुर्गापञ्चरत्नं सम्पूर्णम्॥}

\clearpage
% !TeX program = XeLaTeX
% !TeX root = ../../shloka.tex

\sect{दुर्गाष्टोत्तरशतनामस्तोत्रम्}
\dnsub{न्यासः}
अस्य श्रीदुर्गाष्टोत्तरशतनामास्तोत्रमालामन्त्रस्य\\
महाविष्णुमहेश्वराः ऋषयः। अनुष्टुप् छन्दः।\\
श्रीदुर्गापरमेश्वरी देवता।\\
ह्रां बीजम्। ह्रीं शक्तिः। ह्रूं कीलकम्।\\
सर्वाभीष्टसिद्ध्यर्थे जपहोमार्चने विनियोगः।\\

\dnsub{स्तोत्रम्}
\twolineshloka
{सत्या साध्या भवप्रीता भवानी भवमोचनी}
{आर्या दुर्गा जया चाऽऽद्या त्रिनेत्रा शूलधारिणी}

\twolineshloka
{पिनाकधारिणी चित्रा चण्डघण्टा महातपाः}
{मनो बुद्धिरहङ्कारा चिद्रूपा च चिदाकृतिः}

\twolineshloka
{अनन्ता भाविनी भव्या ह्यभव्या च सदागतिः}
{शाम्भवी देवमाता च चिन्ता रत्नप्रिया तथा}

\twolineshloka
{सर्वविद्या दक्षकन्या दक्षयज्ञविनाशिनी}
{अपर्णाऽनेकवर्णा च पाटला पाटलावती}

\twolineshloka
{पट्टाम्बरपरीधाना कलमञ्जीररञ्जिनी}
{ईशानी च महाराज्ञी ह्यप्रमेयपराक्रमा}

\twolineshloka
{रुद्राणी क्रूररूपा च सुन्दरी सुरसुन्दरी}
{वनदुर्गा च मातङ्गी मतङ्गमुनिकन्यका}

\twolineshloka
{ब्राह्मी माहेश्वरी चैन्द्री कौमारी वैष्णवी तथा}
{चामुण्डा चैव वाराही लक्ष्मीश्च पुरुषाकृतिः}

\twolineshloka
{विमला ज्ञानरूपा च क्रिया नित्या च बुद्धिदा}
{बहुला बहुलप्रेमा महिषासुरमर्दिनी}

\twolineshloka
{मधुकैटभहन्त्री च चण्डमुण्डविनाशिनी}
{सर्वशास्त्रमयी चैव सर्वदानवघातिनी}

\twolineshloka
{अनेकशस्त्रहस्ता च सर्वशस्त्रास्त्रधारिणी}
{भद्रकाली सदाकन्या कैशोरी युवतिर्यतिः}

\twolineshloka
{प्रौढाऽप्रौढा वृद्धमाता घोररूपा महोदरी}
{बलप्रदा घोररूपा महोत्साहा महाबला}

\twolineshloka
{अग्निज्वाला रौद्रमुखी कालरात्री तपस्विनी}
{नारायणी महादेवी विष्णुमाया शिवात्मिका}

\twolineshloka
{शिवदूती कराली च ह्यनन्ता परमेश्वरी}
{कात्यायनी महाविद्या महामेधास्वरूपिणी}

\twolineshloka
{गौरी सरस्वती चैव सावित्री ब्रह्मवादिनी}
{सर्वतत्त्वैकनिलया वेदमन्त्रस्वरूपिणी}

\dnsub{फलश्रुतिः}
\twolineshloka
{इदं स्तोत्रं महादेव्या नाम्नाम् अष्टोत्तरं शतम्}
{यः पठेत् प्रयतो नित्यं भक्तिभावेन चेतसा}

\twolineshloka
{शत्रुभ्यो न भयं तस्य तस्य शत्रुक्षयं भवेत्}
{सर्वदुःखदरिद्राच्च सुसुखं मुच्यते ध्रुवम्}

\twolineshloka
{विद्यार्थी लभते विद्यां धनार्थी लभते धनम्}
{कन्यार्थी लभते कन्यां कन्या च लभते वरम्}

\twolineshloka
{ऋणी ऋणाद्विमुच्येत ह्यपुत्रो लभते सुतम्}
{रोगाद्विमुच्यते रोगी सुखमत्यन्तमश्नुते}

\twolineshloka
{भूमिलाभो भवेत् तस्य सर्वत्र विजयी भवेत्}
{सर्वान् कामानवाप्नोति महादेवीप्रसादतः}

\twolineshloka
{कुङ्कुमैर्बिल्वपत्रैश्च सुगन्धै रक्तपुष्पकैः}
{रक्तपत्रैर्विशेषेण पूजयन् भद्रमश्नुते}
॥इति श्री~दुर्गाष्टोत्तरशतनामस्तोत्रं सम्पूर्णम्॥

% !TeX program = XeLaTeX
% !TeX root = ../../shloka.tex

%\clearpage
\sect{महिषासुरमर्दिनि स्तोत्रम्}
\setlength{\shlokaspaceskip}{10pt}
\fourlineindentedshloka
{अयि गिरिनन्दिनि नन्दितमेदिनि विश्वविनोदिनि नन्दिनुते}
{गिरिवर-विन्ध्य-शिरोधिनिवासिनि विष्णुविलासिनि जिष्णुनुते}
{भगवति हे शितिकण्ठकुटुम्बिनि भूरिकुटुम्बिनि भूरिकृते}
{जय जय हे महिषासुरमर्दिनि रम्यकपर्दिनि शैलसुते}

\fourlineindentedshloka
{सुरवरवर्षिणि दुर्धरधर्षिणि दुर्मुखमर्षिणि हर्षरते}
{त्रिभुवनपोषिणि शङ्करतोषिणि किल्बिषमोषिणि घोषरते}
{दनुज-निरोषिणि दितिसुत-रोषिणि दुर्मद-शोषिणि सिन्धुसुते}
{जय जय हे महिषासुरमर्दिनि रम्यकपर्दिनि शैलसुते}

\fourlineindentedshloka
{अयि जगदम्ब-मदम्ब-कदम्ब-वनप्रिय-वासिनि हासरते}
{शिखरि शिरोमणि तुङ्ग-हिमालय-शृङ्ग-निजालय-मध्यगते}
{मधु-मधुरे मधु-कैटभ-गञ्जिनि कैटभ-भञ्जिनि रासरते}
{जय जय हे महिषासुरमर्दिनि रम्यकपर्दिनि शैलसुते}

\fourlineindentedshloka
{अयि शतखण्ड-विखण्डित-रुण्ड-वितुण्डित-शुण्ड-गजाधिपते}
{रिपु-गज-गण्ड-विदारण-चण्ड-पराक्रम-शुण्ड-मृगाधिपते}
{निज-भुज-दण्ड-निपातित-खण्ड-विपातित-मुण्ड-भटाधिपते}
{जय जय हे महिषासुरमर्दिनि रम्यकपर्दिनि शैलसुते}

\fourlineindentedshloka
{अयि रण-दुर्मद-शत्रु-वधोदित-दुर्धर-निर्जर-शक्तिभृते}
{चतुर-विचार-धुरीण-महाशिव-दूतकृत-प्रमथाधिपते}
{दुरित-दुरीह-दुराशय-दुर्मति-दानवदूत-कृतान्तमते}
{जय जय हे महिषासुरमर्दिनि रम्यकपर्दिनि शैलसुते}

\fourlineindentedshloka
{अयि शरणागत-वैरि-वधूवर-वीर-वराभय-दायकरे}
{त्रिभुवन-मस्तक-शूल-विरोधि शिरोधि कृतामल-शूलकरे}
{दुमिदुमि-तामर-दुन्दुभिनाद-महो-मुखरीकृत-तिग्मकरे}
{जय जय हे महिषासुरमर्दिनि रम्यकपर्दिनि शैलसुते}

\fourlineindentedshloka
{अयि निज-हुङ्कृति मात्र-निराकृत-धूम्रविलोचन-धूम्रशते}
{समर-विशोषित-शोणित-बीज-समुद्भव-शोणित-बीजलते}
{शिव-शिव-शुम्भ-निशुम्भ-महाहव-तर्पित-भूत-पिशाचरते}
{जय जय हे महिषासुरमर्दिनि रम्यकपर्दिनि शैलसुते}

\fourlineindentedshloka
{धनुरनु-सङ्ग-रणक्षणसङ्ग-परिस्फुर-दङ्ग-नटत्कटके}
{कनक-पिशङ्ग-पृषत्क-निषङ्ग-रसद्भट-शृङ्ग-हतावटुके}
{कृत-चतुरङ्ग-बलक्षिति-रङ्ग-घटद्बहुरङ्ग-रटद्बटुके}
{जय जय हे महिषासुरमर्दिनि रम्यकपर्दिनि शैलसुते}

\fourlineindentedshloka
{जय जय जप्य-जयेजय-शब्द-परस्तुति-तत्पर-विश्वनुते}
{भण-भण-भिञ्जिमि-भिङ्कृत-नूपुर-सिञ्जित-मोहित-भूतपते}
{नटित-नटार्ध-नटीनट-नायक-नाटित-नाट्य-सुगानरते}
{जय जय हे महिषासुरमर्दिनि रम्यकपर्दिनि शैलसुते}

\fourlineindentedshloka
{अयि सुमनः सुमनः सुमनः सुमनः सुमनोहर-कान्तियुते}
{श्रित-रजनी-रजनी-रजनी-रजनी-रजनीकर-वक्त्रवृते}
{सुनयन-विभ्रमर-भ्रमर-भ्रमर-भ्रमर-भ्रमराधिपते}
{जय जय हे महिषासुरमर्दिनि रम्यकपर्दिनि शैलसुते}

\fourlineindentedshloka
{सहित-महाहव-मल्लम-तल्लिक-मल्लित-रल्लक-मल्लरते}
{विरचित-वल्लिक-पल्लिक-मल्लिक-भिल्लिक-भिल्लिक-वर्गवृते}
{सितकृत-फुल्लसमुल्ल-सितारुण-तल्लज-पल्लव-सल्ललिते}
{जय जय हे महिषासुरमर्दिनि रम्यकपर्दिनि शैलसुते}

\fourlineindentedshloka
{अविरल-गण्ड-गलन्मद-मेदुर-मत्त-मतङ्गज-राजपते}
{त्रिभुवन-भूषण-भूत-कलानिधि रूप-पयोनिधि राजसुते}
{अयि सुद-तीजन-लालसमानस-मोहन-मन्मथ-राजसुते}
{जय जय हे महिषासुरमर्दिनि रम्यकपर्दिनि शैलसुते}

\fourlineindentedshloka
{कमल-दलामल-कोमल-कान्ति कलाकलितामल-भाललते}
{सकल-विलास-कलानिलयक्रम-केलि-चलत्कल-हंसकुले}
{अलिकुल-सङ्कुल-कुवलय-मण्डल-मौलिमिलद्भकुलालि-कुले}
{जय जय हे महिषासुरमर्दिनि रम्यकपर्दिनि शैलसुते}

\fourlineindentedshloka
{करमुरली-रव-वीजित-कूजित-लज्जित-कोकिल-मञ्जुमते}
{मिलित-पुलिन्द-मनोहर-गुञ्जित-रञ्जितशैल-निकुञ्जगते}
{निजगुणभूत-महाशबरीगण-सद्गुण-सम्भृत-केलितले}
{जय जय हे महिषासुरमर्दिनि रम्यकपर्दिनि शैलसुते}

\fourlineindentedshloka
{कटितट-पीत-दुकूल-विचित्र-मयूख-तिरस्कृत-चन्द्ररुचे}
{प्रणत-सुरासुर-मौलिमणिस्फुर-दंशुल-सन्नख-चन्द्ररुचे}
{जित-कनकाचल-मौलिपदोर्जित-निर्भर-कुञ्जर-कुम्भकुचे}
{जय जय हे महिषासुरमर्दिनि रम्यकपर्दिनि शैलसुते}

%\begin{flushright}
\fourlineindentedshloka
{विजित-सहस्रकरैक-सहस्रकरैक-सहस्रकरैकनुते}
{कृतसुरतारक-सङ्गरतारक-सङ्गरतारक-सूनुसुते}
{सुरथ-समाधि समानसमाधि समाधिसमाधि सुजातरते}
{जय जय हे महिषासुरमर्दिनि रम्यकपर्दिनि शैलसुते}

\fourlineindentedshloka
{पदकमलं करुणानिलये वरिवस्यति योऽनुदिनं स शिवे}
{अयि कमले कमलानिलये कमलानिलयः स कथं न भवेत्}
{तव पदमेव परम्पदमित्यनुशीलयतो मम किं न शिवे}
{जय जय हे महिषासुरमर्दिनि रम्यकपर्दिनि शैलसुते}

\fourlineindentedshloka
{कनकलसत्कल-सिन्धुजलैरनुसिञ्चिनुते गुण-रङ्गभुवम्}
{भजति स किं न शचीकुच-कुम्भ-तटी-परिरम्भ-सुखानुभवम्}
{तव चरणं शरणं करवाणि नतामरवाणि निवासि शिवम्}
{जय जय हे महिषासुरमर्दिनि रम्यकपर्दिनि शैलसुते}

\fourlineindentedshloka
{तव विमलेन्दुकुलं वदनेन्दुमलं सकलं ननु कूलयते}
{किमु पुरुहूत-पुरीन्दुमुखी-सुमुखीभिरसौ विमुखी क्रियते}
{मम तु मतं शिवनामधने भवती कृपया किमुत क्रियते}
{जय जय हे महिषासुरमर्दिनि रम्यकपर्दिनि शैलसुते}

\fourlineindentedshloka
{अयि मयि दीनदयालुतया कृपयैव त्वया भवितव्यमुमे}
{अयि जगतो जननी कृपयाऽसि यथाऽसि तथाऽनुमितासिरते}
{यदुचितमत्र भवत्युररि कुरुतादुरुतापमपाकुरुते}
{जय जय हे महिषासुरमर्दिनि रम्यकपर्दिनि शैलसुते}
%\end{flushright}
॥इति~श्रीमच्छङ्कराचार्यविरचितं
श्री~महिषासुरमर्दिनि-स्तोत्रं~सम्पूर्णम्॥
\setlength{\shlokaspaceskip}{24pt}
\closesection
\clearpage

\phantomsection\addcontentsline{toc}{chapter}{लक्ष्मीस्तोत्राणि}
% !TeX program = XeLaTeX
% !TeX root = ../../shloka.tex

\sect{लक्ष्म्यष्टोत्तरशतनामस्तोत्रम्}
%
%देव्युवाच
%\twolineshloka
%{देवदेव महादेव त्रिकालज्ञ महेश्वर}
%{करुणाकर देवेश भक्तानुग्रहकारक}
%{अष्टोत्तरशतं लक्ष्म्याः श्रोतुमिच्छामि तत्त्वतः॥}
%
%ईश्वर उवाच
%\twolineshloka
%{देवि साधु महाभागे महाभाग्यप्रदायकम्}
%{सर्वैश्वर्यकरं पुण्यं सर्वपापप्रणाशनम्}
%
%\twolineshloka
%{सर्वदारिद्र्यशमनं श्रवणाद्भुक्तिमुक्तिदम्}
%{राजवश्यकरं दिव्यं गुह्याद्गुह्यतमं परम्}
%
%\twolineshloka
%{दुर्लभं सर्वदेवानां चतुःषष्टिकलास्पदम्}
%{पद्मादीनां वरान्तानां विधीनां नित्यदायकम्}
%\twolineshloka
%{समस्तदेवसंसेव्यमणिमाद्यष्टसिद्धिदम्}
%{किमत्र बहुनोक्तेन देवी प्रत्यक्षदायकम्}
%\twolineshloka
%{तव प्रीत्याऽद्य वक्ष्यामि समाहितमनाः शृणुम्}
%{अष्टोत्तरशतस्यास्य महालक्ष्मीस्तु देवता}
%\twolineshloka
%{क्लीम्बीजपदमित्युक्तं शक्तिस्तु भुवनेश्वरी}
%{अङ्गन्यासः करन्यासः स इत्यादिः प्रकीर्तितः}

\dnsub{ध्यानम्}
\fourlineindentedshloka*
{वन्दे पद्मकरां प्रसन्नवदनां सौभाग्यदां भाग्यदाम्}
{हस्ताभ्यामभयप्रदां मणिगणैर्नानाविधैर्भूषिताम्}
{भक्ताभीष्टफलप्रदां हरिहरब्रह्मादिभिः सेविताम्}
{पार्श्वे पङ्कजशङ्खपद्मनिधिभिर्युक्तां सदा शक्तिभिः}

\twolineshloka*
{सरसिजनिलये सरोजहस्ते धवलतमांशुकगन्धमाल्यशोभे}
{भगवति हरिवल्लभे मनोज्ञे त्रिभुवनभूतिकरि प्रसीद मह्यम्}

\dnsub{स्तोत्रम्}
\resetShloka
\twolineshloka
{प्रकृतिं विकृतिं विद्यां सर्वभूतहितप्रदाम्}
{श्रद्धां विभूतिं सुरभिं नमामि परमात्मिकाम्}

\twolineshloka
{वाचं पद्मालयां पद्मां शुचिं स्वाहां स्वधां सुधाम्}
{धन्यां हिरण्मयीं लक्ष्मीं नित्यपुष्टां विभावरीम्}

\twolineshloka
{अदितिं च दितिं दीप्तां वसुधां वसुधारिणीम्}
{नमामि कमलां कान्तां कामाक्षीं क्रोधसम्भवाम्}

\twolineshloka
{अनुग्रहपदां बुद्धिमनघां हरिवल्लभाम्}
{अशोकाममृतां दीप्तां लोकशोकविनाशिनीम्}

\twolineshloka
{नमामि धर्मनिलयां करुणां लोकमातरम्}
{पद्मप्रियां पद्महस्तां पद्माक्षीं पद्मसुन्दरीम्}

\twolineshloka
{पद्मोद्भवां पद्ममुखीं पद्मनाभप्रियां रमाम्}
{पद्ममालाधरां देवीं पद्मिनीं पद्मगन्धिनीम्}

\twolineshloka
{पुण्यगन्धां सुप्रसन्नां प्रसादाभिमुखीं प्रभाम्}
{नमामि चन्द्रवदनां चन्द्रां चन्द्रसहोदरीम्}

\twolineshloka
{चतुर्भुजां चन्द्ररूपामिन्दिरामिन्दुशीतलाम्}
{आह्लादजननीं पुष्टिं शिवां शिवकरीं सतीम्}

\twolineshloka
{विमलां विश्वजननीं तुष्टिं दारिद्र्यनाशिनीम्}
{प्रीतिपुष्करिणीं शान्तां शुक्लमाल्याम्बरां श्रियम्}

\twolineshloka
{भास्करीं बिल्वनिलयां वरारोहां यशस्विनीम्}
{वसुन्धरामुदाराङ्गां हरिणीं हेममालिनीम्}

\twolineshloka
{धनधान्यकरीं सिद्धिं स्त्रैणसौम्यां शुभप्रदाम्}
{नृपवेश्मगतानन्दां वरलक्ष्मीं वसुप्रदाम्}

\twolineshloka
{शुभां हिरण्यप्राकारां समुद्रतनयां जयाम्}
{नमामि मङ्गलां देवीं विष्णुवक्षःस्थलस्थिताम्}

\twolineshloka
{विष्णुपत्नीं प्रसन्नाक्षीं नारायणसमाश्रिताम्}
{दारिद्र्यध्वंसिनीं देवीं सर्वोपद्रवहारिणीम्}

\twolineshloka
{नवदुर्गां महाकालीं ब्रह्मविष्णुशिवात्मिकाम्}
{त्रिकालज्ञानसम्पन्नां नमामि भुवनेश्वरीम्}

\fourlineindentedshloka
{लक्ष्मीं क्षीरसमुद्रराजतनयां श्रीरङ्गधामेश्वरीम्}
{दासीभूतसमस्तदेववनितां लोकैकदीपाङ्कुराम्}
{श्रीमन्मन्दकटाक्षलब्धविभवब्रह्मेन्द्रगङ्गाधराम्}
{त्वां त्रैलोक्यकुटुम्बिनीं सरसिजां वन्दे मुकुन्दप्रियाम्}

\fourlineindentedshloka
{मातर्नमामि कमले कमलायताक्षि}
{श्रीविष्णुहृत्कमलवासिनि विश्वमातः}
{क्षीरोदजे कमलकोमलगर्भगौरि}
{लक्ष्मि प्रसीद सततं नमतां शरण्ये}

\dnsub{फलश्रुतिः}
\twolineshloka
{त्रिकालं यो जपेद्विद्वान् षण्मासं विजितेन्द्रियः}
{दारिद्र्यध्वंसनं कृत्वा सर्वमाप्नोत्ययत्नतः}

\twolineshloka
{देवीनामसहस्रेषु पुण्यमष्टोत्तरं शतम्}
{येन श्रियमवाप्नोति कोटिजन्मदरिद्रितः}

\twolineshloka
{भृगुवारे शतं धीमान् पठेद्वत्सरमात्रकम्}
{अष्टैश्वर्यमवाप्नोति कुबेर इव भूतले}

\twolineshloka
{दारिद्र्यमोचनं नाम स्तोत्रमम्बापरं शतम्}
{येन श्रियमवाप्नोति कोटिजन्मदरिद्रितः}

\threelineshloka
{भुक्त्वा तु विपुलान् भोगानस्याः सायुज्यमाप्नुयात्}
{प्रातःकाले पठेन्नित्यं सर्वदुःखोपशान्तये}
{पठंस्तु चिन्तयेद्देवीं सर्वाभरणभूषिताम्}
॥इति श्री लक्ष्म्यष्टोत्तरशतनामस्तोत्रं सम्पूर्णम्॥
% !TeX program = XeLaTeX
% !TeX root = ../../shloka.tex

\sect{कनकधारास्तवम्}
\fourlineindentedshloka
{अङ्गं हरेः पुलकभूषणमाश्रयन्ती}
{भृङ्गाङ्गनेव मुकुलाभरणं तमालम्}
{अङ्गीकृताखिलविभूतिरपाङ्गलीला}
{माङ्गल्यदास्तु मम मङ्गलदेवतायाः}

\fourlineindentedshloka
{मुग्धा मुहुर्विदधती वदने मुरारेः}
{प्रेमत्रपाप्रणिहितानि गतागतानि}
{माला दृशोर्मधुकरीव महोत्पले या}
{सा मे श्रियं दिशतु सागरसम्भवायाः}

\fourlineindentedshloka
{आमीलिताक्षमधिगम्य मुदा मुकुन्दम्}
{आनन्दकन्दमनिमेषमनङ्गतन्त्रम्}
{आकेकरस्थितकनीनिकपक्ष्मनेत्रम्}
{भूत्यै भवेन्मम भुजङ्गशयाङ्गनायाः}

\fourlineindentedshloka
{बाह्वन्तरे मधुजितः श्रितकौस्तुभे या}
{हारावलीव हरिनीलमयी विभाति}
{कामप्रदा भगवतोऽपि कटाक्षमाला}
{कल्याणमावहतु मे कमलालयायाः}

\fourlineindentedshloka
{कालाम्बुदालिललितोरसि कैटभारेः}
{धाराधरे स्फुरति या तडिदङ्गनेव}
{मातुः समस्तजगतां महनीयमूर्तिः}
{भद्राणि मे दिशतु भार्गवनन्दनायाः}

\fourlineindentedshloka
{प्राप्तं पदं प्रथमतः खलु यत्प्रभावात्}
{माङ्गल्यभाजि मधुमाथिनि मन्मथेन}
{मय्यापतेत्तदिह मन्थरमीक्षणार्धम्}
{मन्दालसं च मकरालयकन्यकायाः}

\fourlineindentedshloka
{विश्वामरेन्द्रपदवीभ्रमदानदक्षम्}
{आनन्दहेतुरधिकं मुरविद्विषोऽपि}
{ईषन्निषीदतु मयि क्षणमीक्षणार्द्धम्}
{इन्दीवरोदरसहोदरमिन्दिरायाः}

\fourlineindentedshloka
{इष्टा विशिष्टमतयोऽपि यया दयार्द्र-}
{दृष्ट्या त्रिविष्टपपदं सुलभं लभन्ते}
{दृष्टिः प्रहृष्टकमलोदरदीप्तिरिष्टाम्}
{पुष्टिं कृषीष्ट मम पुष्करविष्टरायाः}

\fourlineindentedshloka
{दद्याद्दयानुपवनो द्रविणाम्बुधाराम्}
{अस्मिन्नकिञ्चनविहङ्गशिशौ विषण्णे}
{दुष्कर्मघर्ममपनीय चिराय दूरम्}
{नारायणप्रणयिनीनयनाम्बुवाहः}

\fourlineindentedshloka
{गीर्देवतेति गरुडध्वजसुन्दरीति}
{शाकम्बरीति शशिशेखरवल्लभेति}
{सृष्टिस्थितिप्रलयकेलिषु संस्थितायै}
{तस्यै नमस्त्रिभुवनैकगुरोस्तरुण्यै}

\fourlineindentedshloka
{श्रुत्यै नमोऽस्तु शुभकर्मफलप्रसूत्यै}
{रत्यै नमोऽस्तु रमणीयगुणार्णवायै}
{शक्त्यै नमोऽस्तु शतपत्रनिकेतनायै}
{पुष्ट्यै नमोऽस्तु पुरुषोत्तमवल्लभायै}

\fourlineindentedshloka
{नमोऽस्तु नालीकनिभाननायै}
{नमोऽस्तु दुग्धोदधिजन्मभूम्यै}
{नमोऽस्तु सोमामृतसोदरायै}
{नमोऽस्तु नारायणवल्लभायै}

\fourlineindentedshloka
{नमोऽस्तु हेमाम्बुजपीठिकायै}
{नमोऽस्तु भूमण्डलनायिकायै}
{नमोऽस्तु देवादिदयापरायै}
{नमोऽस्तु शार्ङ्गायुधवल्लभायै}

\fourlineindentedshloka
{नमोऽस्तु देव्यै भृगुनन्दनायै}
{नमोऽस्तु विष्णोरुरसि स्थितायै}
{नमोऽस्तु लक्ष्म्यै कमलालयायै}
{नमोऽस्तु दामोदरवल्लभायै}

\fourlineindentedshloka
{नमोऽस्तु कान्त्यै कमलेक्षणायै}
{नमोऽस्तु भूत्यै भुवनप्रसूत्यै}
{नमोऽस्तु देवादिभिरर्चितायै}
{नमोऽस्तु नन्दात्मजवल्लभायै}

\fourlineindentedshloka
{सम्पत्कराणि सकलेन्द्रियनन्दनानि}
{साम्राज्यदानविभवानि सरोरुहाक्षि}
{त्वद्वन्दनानि दुरिताहरणोद्यतानि}
{मामेव मातरनिशं कलयन्तु मान्ये}

\fourlineindentedshloka
{यत्कटाक्षसमुपासनाविधिः}
{सेवकस्य सकलार्थसम्पदः}
{सन्तनोति वचनाङ्गमानसैः}
{त्वां मुरारिहृदयेश्वरीं भजे}

\fourlineindentedshloka
{सरसिजनिलये सरोजहस्ते}
{धवळतमांशुकगन्धमाल्यशोभे}
{भगवति हरिवल्लभे मनोज्ञे}
{त्रिभुवनभूतिकरि प्रसीद मह्यम्}

\fourlineindentedshloka
{दिग्घस्तिभिः कनककुम्भमुखावसृष्ट-}
{स्वर्वाहिनी विमलचारुजलाप्लुताङ्गीम्}
{प्रातर्नमामि जगतां जननीमशेष-}
{लोकाधिनाथगृहिणीम् अमृताब्धिपुत्रीम्}

\fourlineindentedshloka
{कमले कमलाक्षवल्लभे त्वं}
{करुणापूरतरङ्गितैरपाङ्गैः}
{अवलोकय मामकिञ्चनानां}
{प्रथमं पात्रमकृत्रिमं दयायाः}

\fourlineindentedshloka
{स्तुवन्ति ये स्तुतिभिरमीभिरन्वहम्}
{त्रयीमयीं त्रिभुवनमातरं रमाम्}
{गुणाधिका गुरुतरभाग्यभागिनो}
{भवन्ति ते भुवि बुधभाविताशयाः}

\fourlineindentedshloka*
{देवि प्रसीद जगदीश्वरि लोकमातः}
{कल्याणगात्रि कमलेक्षणजीवनाथे}
{दारिद्र्यभीतिहृदयं शरणागतं माम्}
{आलोकय प्रतिदिनं सदयैरपाङ्गैः}
॥इति  श्रीमच्छङ्कराचार्यविरचितं श्री कनकधारास्तवं सम्पूर्णम्॥
% !TeX program = XeLaTeX
% !TeX root = ../../shloka.tex

\sect{महालक्ष्म्यष्टकम्}
\centerline{इन्द्र उवाच}
\twolineshloka
{नमस्तेऽस्तु महामाये श्रीपीठे सुरपूजिते}
{शङ्खचक्रगदाहस्ते महालक्ष्मि नमोऽस्तु ते}

\twolineshloka
{नमस्ते गरुडारूढे कोलासुरभयङ्करि}
{सर्वपापहरे देवि महालक्ष्मि नमोऽस्तु ते}

\twolineshloka
{सर्वज्ञे सर्ववरदे सर्वदुष्टभयङ्करि}
{सर्वदुःखहरे देवि महालक्ष्मि नमोऽस्तु ते}

\twolineshloka
{सिद्धिबुद्धिप्रदे देवि भुक्तिमुक्तिप्रदायिनि}
{मन्त्रमूर्ते सदा देवि महालक्ष्मि नमोऽस्तु ते}

\twolineshloka
{आद्यन्तरहिते देवि आद्यशक्तिमहेश्वरि}
{योगजे योगसम्भूते महालक्ष्मि नमोऽस्तु ते}

\twolineshloka
{स्थूलसूक्ष्ममहारौद्रे महाशक्ति महोदरे}
{महापापहरे देवि महालक्ष्मि नमोऽस्तु ते}

\twolineshloka
{पद्मासनस्थिते देवि परब्रह्मस्वरूपिणि}
{परमेशि जगन्मातर्महालक्ष्मि नमोऽस्तु ते}

\twolineshloka
{श्वेताम्बरधरे देवि नानालङ्कारभूषिते}
{जगत्स्थिते जगन्मातर्महालक्ष्मि नमोऽस्तु ते}

%\dnsub{फलश्रुतिः}
\twolineshloka*
{महालक्ष्म्यष्टकं स्तोत्रं यः पठेद्भक्तिमान्नरः}
{सर्वसिद्धिमवाप्नोति राज्यं प्राप्नोति सर्वदा}

\twolineshloka*
{एककाले पठेन्नित्यं महापापविनाशनम्}
{द्विकालं यः पठेन्नित्यं धनधान्यसमन्वितः}

\twolineshloka*
{त्रिकालं यः पठेन्नित्यं महाशत्रुविनाशनम्}
{महालक्ष्मीर्भवेन्नित्यं प्रसन्ना वरदा शुभा}

॥इति श्रीमद्पद्मपुराणे श्री~महालक्ष्म्यष्टकं सम्पूर्णम्॥
\closesection
\clearpage

\phantomsection\addcontentsline{toc}{chapter}{सरस्वतीस्तोत्राणि}
% !TeX program = XeLaTeX
% !TeX root = ../../shloka.tex

\sect{सरस्वत्यष्टोत्तरशतनामस्तोत्रम्}

\dnsub{ध्यानम्}\nopagebreak[4]
\fourlineindentedshloka*
{या कुन्देन्दुतुषारहारधवला या शुभ्रवस्त्रावृता}
{या वीणावरदण्डमण्डितकरा या श्वेतपद्मासना}
{या ब्रह्माच्युतशङ्करप्रभृतिभिर्देवैः सदा पूजिता}
{सा मां पातु सरस्वती भगवती निःशेषजाड्यापहा}

\dnsub{स्तोत्रम्}
\twolineshloka
{सरस्वती महाभद्रा महामाया वरप्रदा}
{श्रीप्रदा पद्मनिलया पद्माक्षी पद्मवक्त्रका}

\twolineshloka
{शिवानुजा पुस्तकभृत् ज्ञानमुद्रा रमा परा}
{कामरूपा महाविद्या महापातकनाशिनी}

\twolineshloka
{महाश्रया मालिनी च महाभोगा महाभुजा}
{महाभागा महोत्साहा दिव्याङ्गा सुरवन्दिता}

\twolineshloka
{महाकाली महापाशा महाकारा महाङ्कुशा}
{पीता च विमला विश्वा विद्युन्माला च वैष्णवी}

\twolineshloka
{चन्द्रिका चन्द्रवदना चन्द्रलेखविभूषिता}
{सावित्री सुरसा देवी दिव्यालङ्कारभूषिता}

\twolineshloka
{वाग्देवी वसुदा तीव्रा महाभद्रा महाबला}
{भोगदा भारती भामा गोविन्दा गोमती शिवा}

\twolineshloka
{जटिला विन्ध्यवासा च विन्ध्याचलविराजिता}
{चण्डिका वैष्णवी ब्राह्मी ब्रह्मज्ञानैकसाधना}

\twolineshloka
{सौदामिनी सुधामूर्तिः सुभद्रा सुरपूजिता}
{सुवासिनी सुनासा च विनिद्रा पद्मलोचना}

\twolineshloka
{विद्यारूपा विशालाक्षी ब्रह्मजाया महाफला}
{त्रयीमूर्ती त्रिकालज्ञा त्रिगुणा शास्त्ररूपिणी}

\twolineshloka
{शुम्भासुरप्रमथिनी शुभदा च स्वरात्मिका}
{रक्तबीजनिहन्त्री च चामुण्डा चाम्बिका तथा}

\twolineshloka
{मुण्डकायप्रहरणा धूम्रलोचनमर्दना}
{सर्वदेवस्तुता सौम्या सुरासुरनमस्कृता}

\twolineshloka
{कालरात्रिः कलाधारा रूपसौभाग्यदायिनी}
{वाग्देवी च वरारोहा वाराही वारिजासना}

\twolineshloka
{चित्राम्बरा चित्रगन्धा चित्रमाल्यविभूषिता}
{कान्ता कामप्रदा वन्द्या विद्याधरसुपूजिता}

\twolineshloka
{श्वेतानना नीलभुजा चतुर्वर्गफलप्रदा}
{चतुराननसाम्राज्या रक्तमध्या निरञ्जना}

\twolineshloka
{हंसासना नीलजङ्घा ब्रह्मविष्णुशिवात्मिका}
{एवं सरस्वतीदेव्या नाम्नामष्टोत्तरं शतम्}

॥इति श्री सरस्वत्यष्टोत्तरशतनामस्तोत्रं सम्पूर्णम्॥
% !TeX program = XeLaTeX
% !TeX root = ../../shloka.tex

\sect{शारदाभुजङ्गप्रयाताष्टकम्}

\fourlineindentedshloka
{सुवक्षोजकुम्भां सुधापूर्णकुम्भाम्}
{प्रसादावलम्बां प्रपुण्यावलम्बाम्}
{सदास्येन्दुबिम्बां सदानोष्ठबिम्बाम्}
{भजे शारदाम्बामजस्रं मदम्बाम्}

\fourlineindentedshloka
{कटाक्षे दयार्द्रो करे ज्ञानमुद्राम्}
{कलाभिर्विनिद्रां कलापैः सुभद्राम्}
{पुरस्त्रीं विनिद्रां पुरस्तुङ्गभद्राम्}
{भजे शारदाम्बामजस्रं मदम्बाम्}

\fourlineindentedshloka
{ललामाङ्कफालां लसद्गानलोलाम्}
{स्वभक्तैकपालां यशःश्रीकपोलाम्}
{करे त्वक्षमालां कनत्प्रत्नलोलाम्}
{भजे शारदाम्बामजस्रं मदम्बाम्}

\fourlineindentedshloka
{सुसीमन्तवेणीं दृशा निर्जितैणीम्}
{रमत्कीरवाणीं नमद्वज्रपाणीम्}
{सुधामन्थरास्यां मुदा चिन्त्यवेणीम्}
{भजे शारदाम्बामजस्रं मदम्बाम्}

\fourlineindentedshloka
{सुशान्तां सुदेहां दृगन्ते कचान्ताम्}
{लसत्सल्लताङ्गीमनन्तामचिन्त्याम्}
{स्मरेत्तापसैः सङ्गपूर्वस्थितां ताम्}
{भजे शारदाम्बामजस्रं मदम्बाम्}

\fourlineindentedshloka
{कुरङ्गे तुरङ्गे मृगेन्द्रे खगेन्द्रे}
{मराले मदेभे महोक्षेऽधिरूढाम्}
{महत्यां नवम्यां सदा सामरूपाम्}
{भजे शारदाम्बामजस्रं मदम्बाम्}

\fourlineindentedshloka
{ज्वलत्कान्तिवह्निं जगन्मोहनाङ्गीम्}
{भजे मानसाम्भोजसुभ्रान्तभृङ्गीम्}
{निजस्तोत्रसङ्गीतनृत्यप्रभाङ्गीम्}
{भजे शारदाम्बामजस्रं मदम्बाम्}

\fourlineindentedshloka
{भवाम्भोजनेत्राजसम्पूज्यमानाम्}
{लसन्मन्दहासप्रभावक्त्रचिह्नाम्}
{चलच्चञ्चलाचारुताटङ्ककर्णो}
{भजे शारदाम्बामजस्रं मदम्बाम्}
॥इति~श्रीमच्छङ्कराचार्यविरचितं श्री~शारदाभुजङ्गप्रयाताष्टकं सम्पूर्णम्॥
\closesection
\clearpage

\phantomsection\addcontentsline{toc}{chapter}{सुब्रह्मण्यस्तोत्राणि}
% !TeX program = XeLaTeX
% !TeX root = ../../shloka.tex

\sect{सुब्रह्मण्यभुजङ्गम्}
\fourlineindentedshloka
{सदा बालरूपाऽपि विघ्नाद्रिहन्त्री}
{महादन्तिवक्त्राऽपि पञ्चास्यमान्या}
{विधीन्द्रादिमृग्या गणेशाभिधा मे}
{विधत्तां श्रियं काऽपि कल्याणमूर्तिः}

\fourlineindentedshloka
{न जानामि शब्दं न जानामि चार्थम्}
{न जानामि पद्यं न जानामि गद्यम्}
{चिदेका षडास्य हृदि द्योतते मे}
{मुखान्निःसरन्ते गिरश्चापि चित्रम्}

\fourlineindentedshloka
{मयूराधिरूढं महावाक्यगूढम्}
{मनोहारिदेहं महच्चित्तगेहम्}
{महीदेवदेवं महावेदभावम्}
{महादेवबालं भजे लोकपालम्}

\fourlineindentedshloka
{यदा सन्निधानं गता मानवा मे}
{भवाम्भोधिपारं गतास्ते तदैव}
{इति व्यञ्जयन् सिन्धुतीरे य आस्ते}
{तमीडे पवित्रं पराशक्तिपुत्रम्}

\fourlineindentedshloka
{यथाब्धेस्तरङ्गा लयं यान्ति तुङ्गाः}
{तथैवापदः सन्निधौ सेवतां मे}
{इतीवोर्मिपङ्क्तिर्नृणां दर्शयन्तम्}
{सदा भावये हृत्सरोजे गुहं तम्}

\fourlineindentedshloka
{गिरौ मन्निवासे नरा येऽधिरूढाः}
{तदा पर्वते राजते तेऽधिरूढाः}
{इतीव ब्रुवन् गन्धशैलाधिरूढः}
{स देवो मुदे मे सदा षण्मुखोऽस्तु}

\fourlineindentedshloka
{महाम्भोधितीरे महापापचोरे}
{मुनीन्द्रानुकूले सुगन्धाख्यशैले}
{गुहायां वसन्तं स्वभासा लसन्तम्}
{जनार्तिं हरन्तं श्रयामो गुहं तम्}

\fourlineindentedshloka
{लसत् स्वर्णगेहे नृणां कामदोहे}
{सुमस्तोमसञ्छन्नमाणिक्यमञ्चे}
{समुद्यत् सहस्रार्कतुल्यप्रकाशम्}
{सदा भावये कार्त्तिकेयं सुरेशम्}

\fourlineindentedshloka
{रणद्धंसके मञ्जुलेऽत्यन्तशोणे}
{मनोहारिलावण्यपीयूषपूर्णे}
{मनःषट्पदो मे भवक्लेशतप्तः}
{सदा मोदतां स्कन्द ते पादपद्मे}

\fourlineindentedshloka
{सुवर्णाभदिव्याम्बरैर्भासमानाम्}
{क्वणत्किङ्किणीमेखलाशोभमानाम्}
{लसद्धेमपट्टेन विद्योतमानाम्}
{कटिं भावये स्कन्द ते दीप्यमानाम्}

\fourlineindentedshloka
{पुलिन्देशकन्याघनाभोगतुङ्ग-}
{स्तनालिङ्गनासक्तकाश्मीररागम्}
{नमस्याम्यहं तारकारे तवोरः}
{स्वभक्तावने सर्वदा सानुरागम्}

\fourlineindentedshloka
{विधौ कॢप्तदण्डान् स्वलीलाधृताण्डान्}
{निरस्तेभशुण्डान् द्विषत् कालदण्डान्}
{हतेन्द्रारिषण्डान् जगत्त्राणशौण्डान्}
{सदा ते प्रचण्डान् श्रये बाहुदण्डान्}

\fourlineindentedshloka
{सदा शारदाः षण्मृगाङ्का यदि स्युः}
{समुद्यन्त एव स्थिताश्चेत् समन्तात्}
{सदा पूर्णबिम्बाः कलङ्कैश्च हीनाः}
{तदा त्वन्मुखानां ब्रुवे स्कन्द साम्यम्}

\fourlineindentedshloka
{स्फुरन् मन्दहासैः सहंसानि चञ्चत्}
{कटाक्षावलीभृङ्गसङ्घोज्ज्वलानि}
{सुधास्यन्दिबिम्बाधराणीशसूनो}
{तवाऽऽलोकये षण्मुखाम्भोरुहाणि}

\fourlineindentedshloka
{विशालेषु कर्णान्तदीर्घेष्वजस्रम्}
{दयास्यन्दिषु द्वादशस्वीक्षणेषु}
{मयीषत्कटाक्षः सकृत् पातितश्चेत्}
{भवेत्ते दयाशील का नाम हानिः}

\fourlineindentedshloka
{सुताङ्गोद्भवो मेऽसि जीवेति षड्धा}
{जपन् मन्त्रमीशो मुदा जिघ्रते यान्}
{जगद्भारभृद्भ्यो जगन्नाथ तेभ्यः}
{किरीटोज्ज्वलेभ्यो नमो मस्तकेभ्यः}

\fourlineindentedshloka
{स्फुरद्रत्नकेयूरहाराभिरामः}
{चलत् कुण्डलश्रीलसद्गण्डभागः}
{कटौ पीतवासाः करे चारुशक्तिः}
{पुरस्तान्ममास्तां पुरारेस्तनूजः}

\fourlineindentedshloka
{इहाऽऽयाहि वत्सेति हस्तान् प्रसार्या-}
{ऽऽह्वयत्यादराच्छङ्करे मातुरङ्कात्}
{समुत्पत्य तातं श्रयन्तं कुमारम्}
{हराश्लिष्टगात्रं भजे बालमूर्तिम्}

\fourlineindentedshloka
{कुमारेशसूनो गुह स्कन्द सेना-}
{पते शक्तिपाणे मयूराधिरूढ}
{पुलिन्दात्मजाकान्त भक्तार्तिहारिन्}
{प्रभो तारकारे सदा रक्ष मां त्वम्}

\fourlineindentedshloka
{प्रशान्तेन्द्रिये नष्टसंज्ञे विचेष्टे}
{कफोद्गारिवक्त्रे भयोत्कम्पिगात्रे}
{प्रयाणोन्मुखे मय्यनाथे तदानीम्}
{द्रुतं मे दयालो भवाग्रे गुह त्वम्}

\fourlineindentedshloka
{कृतान्तस्य दूतेषु चण्डेषु कोपात्}
{दहच्छिन्द्धि भिन्द्धीति मां तर्जयत्सु}
{मयूरं समारुह्य मा भैरिति त्वम्}
{पुरः शक्तिपाणिर्ममाऽऽयाहि शीघ्रम्}

\fourlineindentedshloka
{प्रणम्यासकृत्पादयोस्ते पतित्वा}
{प्रसाद्य प्रभो प्रार्थयेऽनेकवारम्}
{न वक्तुं क्षमोऽहं तदानीं कृपाब्धे}
{न कार्यान्तकाले मनागप्युपेक्षा}

\fourlineindentedshloka
{सहस्राण्डभोक्ता त्वया शूरनामा}
{हतस्तारकः सिंहवक्त्रश्च दैत्यः}
{ममान्तर्हृदिस्थं मनःक्लेशमेकम्}
{न हंसि प्रभो किं करोमि क्व यामि}

\fourlineindentedshloka
{अहं सर्वदा दुःखभारावसन्नो -}
{भवान् दीनबन्धुस्त्वदन्यं न याचे}
{भवद्भक्तिरोधं सदा कॢप्तबाधम्}
{ममाधिं द्रुतं नाशयोमासुत त्वम्}

\fourlineindentedshloka
{अपस्मारकुष्ठक्षयार्शः प्रमेह-}
{ज्वरोन्मादगुल्मादिरोगा महान्तः}
{पिशाचाश्च सर्वे भवत् पत्रभूतिम्}
{विलोक्य क्षणात् तारकारे द्रवन्ते}

\fourlineindentedshloka
{दृशि स्कन्दमूर्तिः श्रुतौ स्कन्दकीर्तिः}
{मुखे मे पवित्रं सदा तच्चरित्रम्}
{करे तस्य कृत्यं वपुस्तस्य भृत्यम्}
{गुहे सन्तु लीना ममाशेषभावाः}

\fourlineindentedshloka
{मुनीनामुताहो नृणां भक्तिभाजाम्}
{अभीष्टप्रदाः सन्ति सर्वत्र देवाः}
{नृणामन्त्यजानामपि स्वार्थदाने}
{गुहाद्देवमन्यं न जाने न जाने}

\fourlineindentedshloka
{कलत्रं सुता बन्धुवर्गः पशुर्वा}
{नरो वाऽथ नारि गृहे ये मदीयाः}
{यजन्तो नमन्तः स्तुवन्तो भवन्तम्}
{स्मरन्तश्च ते सन्तु सर्वे कुमार}

\fourlineindentedshloka
{मृगाः पक्षिणो दंशका ये च दुष्टाः}
{तथा व्याधयो बाधका ये मदङ्गे}
{भवच्छक्तितीक्ष्णाग्रभिन्नाः सुदूरे}
{विनश्यन्तु ते चूर्णितक्रौञ्चशैल}

\fourlineindentedshloka
{जनित्री पिता च स्वपुत्रापराधम्}
{सहेते न किं देवसेनाधिनाथ}
{अहं चातिबालो भवान् लोकतातः}
{क्षमस्वापराधं समस्तं महेश}

\fourlineindentedshloka
{नमः केकिने शक्तये चापि तुभ्यम्}
{नमश्छाग तुभ्यं नमः कुक्कुटाय}
{नमः सिन्धवे सिन्धुदेशाय तुभ्यम्}
{पुनः स्कन्दमूर्ते नमस्ते नमोऽस्तु}

\fourlineindentedshloka
{जयाऽऽनन्दभूमन् जयापारधामन्}
{जयामोघकीर्ते जयाऽऽनन्दमूर्ते}
{जयाऽऽनन्दसिन्धो जयाशेषबन्धो}
{जय त्वं सदा मुक्तिदानेशसूनो}

\fourlineindentedshloka
{भुजङ्गाख्यवृत्तेन कॢप्तं स्तवं यः}
{पठेद्भक्तियुक्तो गुहं सम्प्रणम्य}
{स पुत्रान् कलत्रं धनं दीर्घमायुः}
{लभेत् स्कन्दसायुज्यमन्ते नरः सः}
॥इति  श्रीमच्छङ्कराचार्यविरचितं श्री सुब्रह्मण्यभुजङ्गं सम्पूर्णम्॥

\closesection
\clearpage

\phantomsection\addcontentsline{toc}{chapter}{कृष्णस्तोत्राणि}
% !TeX program = XeLaTeX
% !TeX root = ../../shloka.tex

\sect{कृष्णाष्टकम् ३}
\twolineshloka
{वसुदेवसुतं देवं कंसचाणूरमर्दनम्}
{देवकीपरमानन्दं कृष्णं वंदे जगद्गुरुम्}

\twolineshloka
{अतसीपुष्पसङ्काशं हारनूपुरशोभितम्}
{रत्नकङ्कणकेयूरं कृष्णं वन्दे जगद्गुरुम्}

\twolineshloka
{कुटिलालकसंयुक्तं पूर्णचन्द्रनिभाननम्}
{विलसत् कुण्डलधरं कृष्णं वन्दे जगद्गुरुम्}

\twolineshloka
{मन्दारगन्धसंयुक्तं चारुहासं चतुर्भुजम्}
{बर्हिपिञ्छावचूडाङ्गं कृष्णं वन्दे जगद्गुरुम्}

\twolineshloka
{उत्फुल्लपद्मपत्राक्षं नीलजीमूतसन्निभम्}
{यादवानां शिरोरत्नं कृष्णं वन्दे जगद्गुरुम्}

\twolineshloka
{रुक्मिणीकेळिसंयुक्तं पीताम्बरसुशोभितम्}
{अवाप्ततुलसीगन्धं कृष्णं वन्दे जगद्गुरुम्}

\twolineshloka
{गोपिकानां कुचद्वन्द्वं कुङ्कुमाङ्कितवक्षसम्}
{श्रीनिकेतं महेष्वासं कृष्णं वन्दे जगद्गुरुम्}

\twolineshloka
{श्रीवत्साङ्कं महोरस्कं वनमालाविराजितम्}
{शङ्खचक्रधरं देवं कृष्णं वन्दे जगद्गुरुम्}

\twolineshloka*
{कृष्णाष्टकमिदं पुण्यं प्रातरुत्थाय यः पठेत्}
{कोटिजन्मकृतं पापं स्मरणेन विनश्यति}
॥इति श्री~कृष्णाष्टकं सम्पूर्णम्॥
% !TeX program = XeLaTeX
% !TeX root = ../../shloka.tex

\sect{श्री कृष्ण-जननम्}
\addtocounter{shlokacount}{7}
\threelineshloka
{निशीथे तम उद्भूते जायमाने जनार्दने}
{देवक्यां देवरूपिण्यां विष्णुः सर्वगुहाशयः}
{आविरासीद्यथा प्राच्यां दिशीन्दुरिव पुष्कलः}

\fourlineindentedshloka
{तमद्भुतं बालकमम्बुजेक्षणम्}{चतुर्भुजं शङ्खगदाद्युदायुधम्}
{श्रीवत्सलक्ष्मं गलशोभिकौस्तुभम्}{पीताम्बरं सान्द्रपयोदसौभगम्}


\fourlineindentedshloka
{महार्ह-वैदूर्य-किरीट-कुण्डल-}{त्विषा परिष्वक्तसहस्रकुन्तलम्}
{उद्दाम-काञ्च्यङ्गद-कङ्कणादिभिर्-}{विरोचमानं वसुदेव ऐक्षत}

॥इति श्रीमद्भागवते महापुराणे पारमहंस्यां संहितायां दशमस्कन्धे पूर्वार्थे तृतीयेऽध्याये श्री कृष्ण-जन्मानुवर्णनम्॥

% !TeX program = XeLaTeX
% !TeX root = ../../shloka.tex

\sect{गोविन्दाष्टकम्}
\fourlineindentedshloka
{सत्यं ज्ञानमनन्तं नित्यमनाकाशं परमाकाशम्}
{गोष्ठप्राङ्गणरिङ्खणलोलमनायासं परमायासम्}
{मायाकल्पितनानाकारमनाकारं भुवनाकारम्}
{क्ष्मामा नाथमनाथं प्रणमत गोविन्दं परमानन्दम्}

\fourlineindentedshloka
{मृत्स्नामत्सीहेति यशोदाताडनशैशव-सन्त्रासम्}
{व्यादितवक्त्रालोकितलोकालोकचतुर्दशलोकालिम्}
{लोकत्रयपुरमूलस्तम्भं लोकालोकमनालोकम्}
{लोकेशं परमेशं प्रणमत गोविन्दं परमानन्दम्}

\fourlineindentedshloka
{त्रैविष्टपरिपुवीरघ्नं क्षितिभारघ्नं भवरोगघ्नम्}
{कैवल्यं नवनीताहारमनाहारं भुवनाहारम्}
{वैमल्यस्फुटचेतोवृत्तिविशेषाभासमनाभासम्}
{शैवं केवलशान्तं प्रणमत गोविन्दं परमानन्दम्}

\fourlineindentedshloka
{गोपालं प्रभुलीलाविग्रहगोपालं कुलगोपालम्}
{गोपीखेलनगोवर्धनधृतलीलालालितगोपालम्}
{गोभिर्निगदित-गोविन्दस्फुटनामानं बहुनामानम्}
{गोधीगोचरदूरं प्रणमत गोविन्दं परमानन्दम्}

\fourlineindentedshloka
{गोपीमण्डलगोष्ठीभेदं भेदावस्थमभेदाभम्}
{शश्वद्गोखुरनिर्धूतोद्गतधूलीधूसरसौभाग्यम्}
{श्रद्धाभक्तिगृहीतानन्दमचिन्त्यं चिन्तितसद्भावम्}
{चिन्तामणिमहिमानं प्रणमत गोविन्दं परमानन्दम्}

\fourlineindentedshloka
{स्नानव्याकुलयोषिद्वस्त्रमुपादायागमुपारूढम्}
{व्यादित्सन्तीरथ दिग्वस्त्रा दातुमुपाकर्षन्तं ताः}
{निर्धूतद्वयशोकविमोहं बुद्धं बुद्धेरन्तःस्थम्}
{सत्तामात्रशरीरं प्रणमत गोविन्दं परमानन्दम्}

\fourlineindentedshloka
{कान्तं कारणकारणमादिमनादिं कालघनाभासम्}
{कालिन्दीगतकालियशिरसि सुनृत्यन्तं मुहुरत्यन्तम्}
{कालं कालकलातीतं कलिताशेषं कलिदोषघ्नम्}
{कालत्रयगतिहेतुं प्रणमत गोविन्दं परमानन्दम्}

\fourlineindentedshloka
{बृन्दावनभुवि बृन्दारकगण बृन्दाराधित वन्द्येऽहम्}
{कुन्दाभामलमन्दस्मेरसुधानन्दं सुहृदानन्दम्}
{वन्द्याशेषमहामुनिमानसवन्द्यानन्दपदद्वन्द्वम्}
{वन्द्याशेषगुणाब्धिं प्रणमत गोविन्दं परमानन्दम्}

\fourlineindentedshloka*
{गोविन्दाष्टकमेतदधीते गोविन्दार्पितचेता यः}
{गोविन्द अच्युत माधव विष्णो गोकुलनायक कृष्णेति}
{गोविन्दाङ्घ्रि-सरोजध्यान-सुधाजलधौत-समस्ताघः}
{गोविन्दं परमानन्दामृतम् अन्तःस्थं स तमभ्येति}
{॥इति श्रीमच्छङ्कराचार्यविरचितं श्री~गोविन्दाष्टकं सम्पूर्णम् ॥}
% !TeX program = XeLaTeX
% !TeX root = ../../shloka.tex

\sect{मधुराष्टकम्}
\twolineshloka
{अधरं मधुरं वदनं मधुरं नयनं मधुरं हसितं मधुरम्}
{हृदयं मधुरं गमनं मधुरं मधुराधिपतेरखिलं मधुरम्}

\twolineshloka
{वचनं मधुरं चरितं मधुरं वसनं मधुरं वलितं मधुरम्}
{चलितं मधुरं भ्रमितं मधुरं मधुराधिपतेरखिलं मधुरम्}

\twolineshloka
{वेणुर्मधुरो रेणुर्मधुरः पाणिर्मधुरः पादौ मधुरौ}
{नृत्यं मधुरं सख्यं मधुरं मधुराधिपतेरखिलं मधुरम्}

\twolineshloka
{गीतं मधुरं पीतं मधुरं भुक्तं मधुरं सुप्तं मधुरम्}
{रूपं मधुरं तिलकं मधुरं मधुराधिपतेरखिलं मधुरम्}

\twolineshloka
{करणं मधुरं तरणं मधुरं हरणं मधुरं रमणं मधुरम्}
{वमितं मधुरं शमितं मधुरं मधुराधिपतेरखिलं मधुरम्}

\twolineshloka
{गुञ्जा मधुरा माला मधुरा यमुना मधुरा वीची मधुरा}
{सलिलं मधुरं कमलं मधुरं मधुराधिपतेरखिलं मधुरम्}

\twolineshloka
{गोपी मधुरा लीला मधुरा युक्तं मधुरं मुक्तं मधुरम्}
{दृष्टं मधुरं शिष्टं मधुरं मधुराधिपतेरखिलं मधुरम्}

\twolineshloka
{गोपा मधुरा गावो मधुरा यष्टिर्मधुरा सृष्टिर्मधुरा}
{दलितं मधुरं फलितं मधुरं मधुराधिपतेरखिलं मधुरम्}
{॥इति श्रीमद्वल्लभाचार्यविरचितं मधुराष्टकं सम्पूर्णम्॥}

% !TeX program = XeLaTeX
% !TeX root = ../../shloka.tex

\sect{अच्युताष्टकम्}

\fourlineindentedshloka
{अच्युतं केशवं राम-नारायणम्}
{कृष्ण-दामोदरं वासुदेवं हरिम्}
{श्रीधरं माधवं गोपिकावल्लभम्}
{जानकीनायकं रामचन्द्रं भजे}

\fourlineindentedshloka
{अच्युतं केशवं सत्यभामाधवम्}
{माधवं श्रीधरं राधिकाराधितम्}
{इन्दिरा मन्दिरं चेतसा सुन्दरम्}
{देवकीनन्दनं नन्दनं सन्दधे}

\fourlineindentedshloka
{विष्णवे जिष्णवे शङ्खिने चक्रिणे}
{रुक्मिणी-रागिने जानकी-जानये}
{वल्लवी-वल्लभायाऽर्चितायात्मने}
{कंस-विध्वंसिने वंशिने ते नमः}

\fourlineindentedshloka
{कृष्ण गोविन्द हे राम नारायण}
{श्रीपते वासुदेवार्जित-श्रीनिधे}
{अच्युतानन्त हे माधवाधोक्षज}
{द्वारका-नायक द्रौपदी-रक्षक}

\fourlineindentedshloka
{राक्षसक्षोभितः सीतया शोभितो}
{दण्डकारण्य-भू-पुण्यता-कारणः}
{लक्ष्मणेनान्वितो वानरैः सेवितो-}
{ऽगस्त्सम्पूजितो राघवः पातु माम्}

\fourlineindentedshloka
{धेनुकारिष्टकोऽनिष्टकृद्-द्वेषिणाम्}
{केशिहा कंसहृद्-वंशिकावादकः}
{पूतनाकोपकः सूरजा-खेलनो}
{बाल-गोपालकः पातु मां सर्वदा}

\fourlineindentedshloka
{विद्युदाद्योतवान् प्रस्फुरद्वाससम्}
{प्रावृडम्भोदवत् प्रोल्लसद्विग्रहम्}
{वन्यया मालया शोभितोरस्थलम्}
{लोहिताङ्घ्रिद्वयं वारिजाक्षं भजे}

\fourlineindentedshloka
{कुञ्चितैः कुन्तलैर्भ्राजिमानाननम्}
{रत्नमौलिं लसत् कुण्डलं गण्डयोः}
{हारकेयूरकं कङ्कण-प्रोज्ज्वलम्}
{किङ्किणी-मञ्जुलं श्यामलं तं भजे}

\fourlineindentedshloka*
{अच्युतस्याष्टकं यः पठेदिष्टदम्}
{प्रेमतः प्रत्यहं पूरुषः सस्पृहम्}
{वृत्ततः सुन्दरं वेद्यविश्वम्बरम्}
{तस्य वश्यो हरिर्जायते सत्वरम्}

॥इति श्रीमच्छङ्कराचार्यविरचितं श्री~अच्युताष्टकं सम्पूर्णम्॥
% !TeX program = XeLaTeX
% !TeX root = ../../shloka.tex

\sect{रङ्गनाथ गद्यम्}
\begin{flushleft}
स्वाधीन-त्रिविध-चेतनाचेतन-स्वरूप-स्थिति-प्रवृत्ति-भेदम्'
क्लेश-कर्माद्यशेष-दोषासंस्पृष्टं' स्वाभाविकानवधिकातिशय-
ज्ञान'-बलैश्वर्य'-वीर्य'-शक्ति-तेजः सौशील्य'-वात्सल्य-
मार्दवार्जव'-सौहार्द'-साम्य'-कारुण्य-माधुर्य-गाम्भीर्यौदार्य'-
चातुर्य'-स्थैर्य'-धैर्य'-शौर्य-पराक्रम'-सत्यकाम'-सत्यसङ्कल्प'-
कृतित्व'-कृतज्ञताद्यसङ्ख्येय-कल्याण-गुणगणौघ-महार्णवम्'
\mbox{परब्रह्मभूतं' पुरुषोत्तमं' श्रीरङ्गशायिनम्' अस्मत्स्वामिनं' प्रबुद्ध'}\\
नित्य-नियाम्य' नित्य-दास्यैकरसात्मस्वभावोऽहम्'
तदेकानुभवः' तदेकप्रियः' परिपूर्णं भगवन्तं'
विशदतमानुभवेन निरन्तरमनुभूय' तदनुभव-जनितानवधिकातिशय-%
प्रीतिकारिता-ऽशेषावस्थोचित-
अशेषशेषतैकरतिरूप' नित्य-किङ्करो भवानि॥१॥


स्वात्म-नित्य-नियाम्य'-नित्यदास्यैकरसात्म-स्वभावानुसन्धान-पूर्वक'
भगवदनवधिकातिशय-स्वाम्याद्यखिल-गुणगणानुभवजनित-
अनवधिकातिशय-प्रीतिकारिता-ऽशेषावस्थोचिता-ऽशेषशेषतैकरतिरूप'-%
नित्य-कैङ्कर्य-प्राप्त्युपाय-भूतभक्ति'\\
तदुपाय-सम्यग्ज्ञान' तदुपाय-समीचीनक्रिया'
तदनुगुण-सात्त्विकतास्तिक्यादि समस्तात्म-गुणविहीनः'
दुरुत्तरानन्त' तद्विपर्यय-ज्ञानक्रियानुगुण-%
अनादिपाप- वासना-महार्णवान्तर्निमग्नः'
तिलतैलवत्' दारुवह्निवत्' दुर्विवेच-त्रिगुणक्षणक्षरण-%
स्वभावाचेतन-प्रकृति-व्याप्तिरूप'-दुरत्यय'-भगवन्माया-तिरोहित-स्वप्रकाशः'
अनाद्यविद्या-सञ्चिता-ऽनन्ता-ऽशक्य-विस्रंसन'-कर्मपाश-प्रग्रथितः'
अनागता-ऽनन्तकाल-समीक्षयाऽपि' अदृष्ट-सन्तारोपायः'
निखिल-जन्तु-जात-शरण्य! श्रीमन्! नारायण!
तव चरणारविन्दयुगलं शरणमहं प्रपद्ये॥२॥

एवमवस्थितस्याऽपि' अर्थित्वमात्रेण' परमकारुणिको भगवान्'\\
स्वानुभव-प्रीत्या' उपनीतैकान्तिका-ऽत्यन्तिक-\\
नित्य-कैङ्कर्यैकरतिरूप-नित्य-दास्यं' दास्यतीति' \\
विश्वासपूर्वकं भगवन्तं नित्य-किङ्करतां प्रार्थये॥३॥

तवानुभूति-सम्भूत-प्रीतिकारित-दासताम्।\\
देहि मे कृपया नाथ! न जाने गतिमन्यथा॥४॥

सर्वावस्थोचिता-ऽशेषशेषतैकरतिस्तव।\\
भवेयं पुण्डरीकाक्ष! त्वमेवैवं कुरुष्व माम्॥५॥

एवम्भूत-तत्त्वयाथात्म्यावबोध-तदिच्छारहितस्याऽपि'\\
एतदुच्चारण-मात्रावलम्बनेन उच्यमानार्थ-परमार्थ-निष्ठम्'\\
मे मनः त्वमेव अद्यैव कारय॥६॥

अपार-करुणाम्बुधे! अनालोचित-विशेषाशेष-लोक-शरण्य!
प्रणतार्तिहर! आश्रित-वात्सल्यैक-महोदधे! 
अनवरत-विदित-निखिल-भूत-जात-याथात्म्य!
सत्यकाम! सत्यसङ्कल्प! आपत्सख! काकुत्स्थ! श्रीमन्!
नारायण! पुरुषोत्तम! श्रीरङ्गनाथ! मम नाथ! नमोऽस्तु ते॥
\end{flushleft}
\centerline{॥इति श्रीमद्रामानुजविरचितं श्री~रङ्गनाथ गद्यं सम्पूर्णम्॥}

% !TeX program = XeLaTeX
% !TeX root = ../../shloka.tex

\sect{दामोदराष्टकम्}
\fourlineindentedshloka
{नमामीश्वरं सच्चिदानन्दरूपम्}{लसत्कुण्डलं गोकुले भ्राजमानम्}
{यशोदाभियोलूखलाद्-धावमानम्}{परामृष्टमत्यन्ततो द्रुत्य गोप्या}

\fourlineindentedshloka
{रुदन्तं मुहुर्नेत्रयुग्मं मृजन्तम्}{कराम्भोजयुग्मेन सातङ्कनेत्रम्}
{मुहुः श्वासकम्पत्रिरेखाङ्ककण्ठ-}{स्थितग्रैव-दामोदरं भक्तिबद्धम्}

\fourlineindentedshloka
{इतीदृक् स्वलीलाभिरानन्दकुण्डे}{स्वघोषं निमज्जन्तमाख्यापयन्तम्}
{तदीयेषिताज्ञेषु भक्तैर्जितत्वम्}{पुनः प्रेमतस्तं शतावृत्ति वन्दे}

\fourlineindentedshloka
{वरं देव मोक्षं न मोक्षावधिं वा}{न चान्यं वृणेऽहं वरेषादपीह}
{इदं ते वपुर्नाथ गोपालबालम्}{सदा मे मनस्याविरास्तां किमन्यैः}

\fourlineindentedshloka
{इदं ते मुखाम्भोजमत्यन्तनीलैर्-}{वृतं कुन्तलैः स्निग्ध-रक्तैश्च गोप्या}
{मुहुश्चुम्बितं बिम्बरक्ताधरं मे}{मनस्याविरास्ताम् अलं लक्षलाभैः}

\fourlineindentedshloka
{नमो देव दामोदरानन्त विष्णो}{प्रसीद प्रभो दुःखजालाब्धिमग्नम्}
{कृपादृष्टिवृष्ट्यातिदीनं बतानु}{गृहाणेश माम् अज्ञमेध्यक्षिदृश्यः}

\fourlineindentedshloka
{कुवेरात्मजौ बद्धमूर्त्यैव यद्वत्}{त्वया मोचितौ भक्तिभाजौ कृतौ च}
{तथा प्रेमभक्तिं स्वकं मे प्रयच्छ}{न मोक्षे ग्रहो मेऽस्ति दामोदरेह}

\fourlineindentedshloka
{नमस्तेऽस्तु दाम्ने स्फुरद्दीप्तिधाम्ने}{त्वदीयोदरायाथ विश्वस्य धाम्ने}
{नमो राधिकायै त्वदीयप्रियायै}{नमोऽनन्तलीलाय देवाय तुभ्यम्}

{॥इति श्रीमद्पद्मपुराणे श्री~दामोदराष्टकं सम्पूर्णम् ॥}
% !TeX program = XeLaTeX
% !TeX root = ../../shloka.tex

\sect{नारायण केशादिपादवर्णनम्}

\fourlineindentedshloka
{अग्रे पश्यामि तेजो निबिडतरकलायावलीलोभनीयम्}
{पीयूषाप्लावितोऽहं तदनु तदुदरे दिव्यकैशोरवेषम्}
{तारुण्यारम्भरम्यं परमसुखरसास्वादरोमाञ्चिताङ्गै-}
{रावीतं नारदाद्यैर्विलसदुपनिषत्सुन्दरीमण्डलैश्च}

\fourlineindentedshloka
{नीलाभं कुञ्चिताग्रं घनममलतरं संयतं चारुभङ्ग्या}
{रत्नोत्तंसाभिरामं वलयितमुदयच्चन्द्रकैः पिञ्छजालैः}
{मन्दारस्रङ्निवीतं तव पृथुकबरीभारमालोकयेऽहम्}
{स्निग्धश्चेतोर्ध्वपुण्ड्रामपि च सुललितां फालबालेन्दुवीथीम्}

\fourlineindentedshloka
{हृद्यं पूर्णानुकम्पार्णवमृदुलहरीचञ्चलभ्रूविलासै-}
{रानीलस्निग्धपक्ष्मावलिपरिलसितं नेत्रयुग्मं विभो ते}
{सान्द्रच्छायं विशालारुणकमलदलाकारमामुग्धतारम्}
{कारुण्यालोकलीलाशिशिरितभुवनं क्षिप्यतां मय्यनाथे}

\fourlineindentedshloka
{उत्तुङ्गोल्लासिनासं हरिमणिमुकुरप्रोल्लसद्गण्डपाली-}
{व्यालोलत्कर्णपाशाञ्चितमकरमणीकुण्डलद्वन्द्वदीप्रम्}
{उन्मीलद्दन्तपङ्क्तिस्फुरदरुणतरच्छायबिम्बाधरान्तः}
{प्रीतिप्रस्यन्दिमन्दस्मितमधुरतरं वक्त्रमुद्भासतां मे}

\fourlineindentedshloka
{बाहुद्वन्द्वेन रत्नोज्वलवलयभृता शोणपाणिप्रवाळे-}
{नोपात्तां वेणुनाळीं प्रसृतनखमयूखाङ्गुलीसङ्गशाराम्}
{कृत्वा वक्त्रारविन्दे सुमधुरविकसद्रागमुद्भाव्यमानैः}
{शब्दब्रह्मामृतैस्त्वं शिशिरितभुवनैस्सिञ्च मे कर्णवीथीम्}

\fourlineindentedshloka
{उत्सर्पत्कौस्तुभश्रीततिभिररुणितं कोमळं कण्ठदेशम्}
{वक्षः श्रीवत्सरम्यं तरळतरसमुद्दीप्रहारप्रतानम्}
{नानावर्णप्रसूनावलिकिसलयिनीं वन्यमालां विलोल-}
{ल्लोलम्बां लम्बमानामुरसि तव तथा भावये रत्नमालाम्}

\fourlineindentedshloka
{अङ्गे पञ्चाङ्गरागैरतिशयविकसत्सौरभाकृष्टलोकम्}
{लीनानेकत्रिलोकीविततिमपि कृशां बिभ्रतं मध्यवल्लीम्}
{शक्राश्मन्यस्ततप्तोज्वलकनकनिभं पीतचेलं दधानम्}
{ध्यायामो दीप्तरश्मिस्फुटमणिरशनाकिङ्गिणीमण्डितं त्वाम्}

\fourlineindentedshloka
{ऊरू चारू तवोरू घनमसृणरुचौ चित्तचोरौ रमायाः}
{विश्वक्षोभं विशङ्क्य ध्रुवमनिशमुभौ पीतचेलावृताङ्गौ}
{आनम्राणां पुरस्तान्न्यसनधृतसमस्तार्थपाळीसमुद्ग-}
{च्छायां जानुद्वयं च क्रमपृथुलमनोज्ञे च जङ्घे निषेवे}

\fourlineindentedshloka
{मञ्जीरं मञ्जुनादैरिव पदभजनं श्रेय इत्यालपन्तम्}
{पादाग्रं भ्रान्तिमज्जत्प्रणतजनमनोमन्दरोद्धारकूर्मम्}
{उत्तुङ्गाताम्रराजन्नखरहिमकरज्योत्स्नया चाऽश्रितानाम्}
{सन्तापध्वान्तहन्त्त्रीं ततिमनुकलये मङ्गलामङ्गुलीनाम्}

\fourlineindentedshloka
{योगीन्द्राणां त्वदङ्गेष्वधिकसुमधुरं मुक्तिभाजां निवासो}
{भक्तानां कामवर्षद्युतरुकिसलयं नाथ ते पादमूलम्}
{नित्यं चित्तस्थितं मे पवनपुरपते कृष्ण कारुण्यसिन्धो}
{हृत्वा निःशेषतापान्प्रदिशतु परमानन्दसन्दोहलक्ष्मीम्}

\fourlineindentedshloka
{अज्ञात्वा ते महत्त्वं यदिह निगदितं विश्वनाथ क्षमेथाः}
{स्तोत्रं चैतत्सहस्रोत्तरमधिकतरं त्वत्प्रसादाय भूयात्}
{द्वेधा नारायणीयं श्रुतिषु च जनुषा स्तुत्यतावर्णनेन}
{स्फीतं लीलावतारैरिदमिह कुरुतामायुरारोग्यसौख्यम्}

{॥इति श्रीमन्नारायणीये शततम-दशकं सम्पूर्णम्॥}

% !TeX program = XeLaTeX
% !TeX root = ../../shloka.tex

\sect{विष्णुभुजङ्गप्रयातस्तोत्रम्‌}
\fourlineindentedshloka
{चिदंशं विभुं निर्मलं निर्विकल्पम्‌}
{निरीहं निराकारमोङ्कारगम्यम्‌}
{गुणातीतमव्यक्तमेकं तुरीयम्‌}
{परं ब्रह्म यं वेद तस्मै नमस्ते}% १}%

\fourlineindentedshloka
{विशुद्धं शिवं शान्तमाद्यन्तशून्यम्‌}
{जगज्जीवनं ज्योतिरानन्दरूपम्‌}
{अदिग्देशकालव्यवच्छेदनीयम्‌} 
{त्रयी वक्ति यं वेद तस्मै नमस्ते}% २}%

\fourlineindentedshloka
{महायोगपीठे परिभ्राजमाने} 
{धरण्यादितत्त्वात्मके शक्तियुक्ते}
{गुणाहस्करे वह्निबिम्बार्धमध्ये} 
{समासीनमोङ्कर्णिकेऽष्टाक्षराब्जे}% ३}%

\fourlineindentedshloka
{समानोदितानेकसूर्येन्दुकोटि-}
{प्रभापूरतुल्यद्युतिं दुर्निरीक्षम्‌}
{न शीतं न चोष्णं सुवर्णावदात-}
{प्रसन्नं सदानन्दसंवित्स्वरूपम्‌}% ४}%

\fourlineindentedshloka
{सुनासापुटं सुन्दरभ्रूललाटम्‌}
{किरीटोचिताकुञ्चितस्निग्धकेशम्‌}
{स्फुरत् पुण्डरीकाभिरामायताक्षम्‌}
{समुत्फुल्लरत्नप्रसूनावतंसम्‌}% ५}%

\fourlineindentedshloka
{लसत् कुण्डलामृष्टगण्डस्थलान्तम्‌}
{जपारागचोराधरं चारुहासम्‌}
{अलिव्याकुलामोलिमन्दारमालम्‌}
{महोरस्फुरत्कौस्तुभोदारहारम्‌}% ६}%

\fourlineindentedshloka
{सुरत्नाङ्गदैरन्वितं बाहुदण्डैः}
{चतुर्भिश्चलत्कङ्कणालङ्कृताग्रैः}
{उदारोदरालङ्कृतं पीतवस्त्रम्‌}
{पदद्वन्द्वनिर्धूतपद्माभिरामम्‌}% ७}%

\fourlineindentedshloka
{स्वभक्तेषु सन्दर्शिताकारमेवम्‌}
{सदा भावयन् सन्निरुद्धेन्द्रियाश्वः}
{दुरापं नरो याति संसारपारम्‌}
{परस्मै परेभ्योऽपि तस्मै नमस्ते}% ८}%

\fourlineindentedshloka
{श्रिया शातकुम्भद्युतिस्निग्धकान्त्या}
{धरण्या च दूर्वादलश्यामलाङ्ग्या}
{कलत्रद्वयेनामुना तोषिताय} 
{त्रिलोकीगृहस्थाय विष्णो नमस्ते}% ९}%

\fourlineindentedshloka
{शरीरं कलत्रं सुतं बन्धुवर्गम्‌}
{वयस्यं धनं सद्म भृत्यं भुवं च}
{समस्तं परित्यज्य हा कष्टमेको} 
{गमिष्यामि दुःखेन दूरं किलाहम्‌}% १०}%

\fourlineindentedshloka
{जरेयं पिशाचीव हा जीवतो मे}
{वसामक्ति रक्तं च मांसं बलं च}
{अहो देव सीदामि दीनानुकम्पिन्}
{किमद्यापि हन्त त्वयोदासितव्यम्‌}% ११}%

\fourlineindentedshloka
{कफव्याहतोष्णोल्बणश्वासवेग-}
{व्यथाविस्फुरत्सर्वमर्मास्थिबन्धाम्‌}
{विचिन्त्याहमन्त्यामसङ्ख्यामवस्थाम्}
{बिभेमि प्रभो किं करोमि प्रसीद}% १२}%

\fourlineindentedshloka
{लपन्नच्युतानन्त गोविन्द विष्णो}
{मुरारे हरे नाथ नारायणेति}
{यथाऽनुस्मरिष्यामि भक्त्या भवन्तम्‌}
{तथा मे दयाशील देव प्रसीद}% १३}%

\fourlineindentedshloka
{भुजङ्गप्रयातं पठेद्यस्तु भक्त्या}
{समाधाय चित्ते भवन्तं मुरारे}
{स मोहं विहायाऽऽशु युष्मत्प्रसादात्}
{समाश्रित्य योगं व्रजत्यच्युतं त्वाम्‌}% १४}%

॥इति~श्रीमच्छङ्कराचर्यविरचितं श्री विष्णुभुजङ्गप्रयातस्तोत्रं सम्पूर्णम्‌॥ 

\clearpage
%!TeX program = Xelatex
%!TeX root = ../../shloka.tex
\sect{कृष्णाष्टोत्तरशतनामस्तोत्रम्}
ॐ अस्य श्रीकृष्णाष्टोत्तरशतनामस्तोत्रस्य श्रीशेष ऋषिः।\\
अनुष्टुप्-छन्दः। श्रीकृष्णो देवता।\\
 श्रीकृष्णप्रीत्यर्थे श्री कृष्णाष्टोत्तरशतनामजपे विनियोगः।

\dnsub{ध्यानम्}
\fourlineindentedshloka*
{शिखिमुकुटविशेषं नीलपद्माङ्गदेशम्}
{विधुमुखकृतकेशं कौस्तुभापीतवेशम्}
{मधुररवकलेशं शं भजे भ्रातृशेषम्}
{व्रजजनवनितेशं माधवं राधिकेशम्}

{श्रीशेष उवाच}

\twolineshloka*
{वसुन्धरे वरारोहे जनानामस्ति मुक्तिदम्}
{सर्वमङ्गलमूर्धन्यमणिमाद्यष्टसिद्धिदम्}

\twolineshloka*
{महापातककोटिघ्नं सर्वतीर्थफलप्रदम्}
{समस्तजपयज्ञानां फलदं पापनाशनम्}

\twolineshloka*
{शृणु देवि प्रवक्ष्यामि नाम्नामष्टोत्तर शतम्}
{सहस्रनाम्नां पुण्यानां त्रिरावृत्या तु यत्फलम्}

\twolineshloka*
{एकावृत्या तु कृष्णस्य नामैकं तत्प्रयच्छति}
{तस्मात्पुण्यतरं चैतत्स्तोत्रं पातकनाशनम्}

\twolineshloka*
{नाम्नामष्टोत्तरशतस्याहमेव ऋषिः प्रिये}
{छन्दोऽनुष्टुब्देवता तु योगः कृष्णप्रियावहः}

\dnsub{स्तोत्रम्}
\twolineshloka
{श्रीकृष्णः कमलानाथो वासुदेवः सनातनः}
{वसुदेवात्मजः पुण्यो लीलामानुषविग्रहः}

\twolineshloka
{श्रीवत्सकौस्तुभधरो यशोदावत्सलो हरिः}
{चतुर्भुजात्तचक्रासिगदाशङ्खाम्बुजायुधः}

\twolineshloka
{देवकीनन्दनः श्रीशो नन्दगोपप्रियात्मजः}
{यमुनावेगसंहारी बलभद्रप्रियानुजः}

\twolineshloka
{पूतनाजीवितहरः शकटासुरभञ्जनः}
{नन्दव्रजजनानन्दी सच्चिदानन्दविग्रहः}

\twolineshloka
{नवनीतविलिप्ताङ्गो नवनीतनटोऽनघः}
{नवनीतनवाहारो मुचुकुन्दप्रसादकः}

\twolineshloka
{षोडशस्त्रीसहस्रेशस्त्रिभङ्गी मधुराकृतिः}
{शुकवागमृताब्धीन्दुर्गोविन्दो योगिनां पतिः}

\twolineshloka
{वत्सवाटचरोऽनन्तो धेनुकासुरभञ्जनः}
{तृणीकृततृणावर्तो यमलार्जुनभञ्जनः}

\twolineshloka
{उत्तालतालभेत्ता च तमालश्यामलाकृतिः}
{गोपगोपीश्वरो योगी कोटिसूर्यसमप्रभः}

\twolineshloka
{इलापतिः परञ्ज्योतिर्यादवेन्द्रो यदूद्वहः}
{वनमाली पीतवासाः पारिजातापहारकः}

\twolineshloka
{गोवर्धनाचलोद्धर्ता गोपालः सर्वपालकः}
{अजो निरञ्जनः कामजनकः कञ्जलोचनः}

\twolineshloka
{मधुहा मथुरानाथो द्वारकानायको बली}
{वृन्दावनान्तसञ्चारी तुलसीदामभूषणः}

\twolineshloka
{स्यमन्तकमणेर्हर्ता नरनारायणात्मकः}
{कुब्जाकृष्णाम्बरधरो मायी परमपूरुषः}

\twolineshloka
{मुष्टिकासुरचाणूरमल्लयुद्धविशारदः}
{संसारवैरी कंसारिर्मुरारिर्नरकान्तकः}

\twolineshloka
{अनादिब्रह्मचारी च कृष्णाव्यसनकर्षकः}
{शिशुपालशिरश्छेत्ता दुर्योधनकुलान्तकः}

\twolineshloka
{विदुराक्रूरवरदो विश्वरूपप्रदर्शकः}
{सत्यवाक् सत्यसङ्कल्पः सत्यभामारतो जयी}

\twolineshloka
{सुभद्रापूर्वजो विष्णुर्भीष्ममुक्तिप्रदायकः}
{जगद्गुरुर्जगन्नाथो वेणुनादविशारदः}

\twolineshloka
{वृषभासुरविध्वंसी बाणासुरकरान्तकः}
{युधिष्ठिरप्रतिष्ठाता बर्हिबर्हावतंसकः}

\twolineshloka
{पार्थसारथिरव्यक्तो गीतामृतमहोदधिः}
{कालीयफणिमाणिक्यरञ्जितश्रीपदाम्बुजः}

\twolineshloka
{दामोदरो यज्ञभोक्ता दानवेन्द्रविनाशकः}
{नारायणः परब्रह्म पन्नगाशनवाहनः}

\twolineshloka
{जलक्रीडासमासक्तगोपीवस्त्रापहारकः}
{पुण्यश्लोकस्तीर्थपादो वेदवेद्यो दयानिधिः}

\twolineshloka
{सर्वतीर्थात्मकः सर्वग्रहरूपी परात्परः}
{इत्येवं कृष्णदेवस्य नाम्नामष्टोत्तरं शतम्}

\twolineshloka
{कृष्णेन कृष्णभक्तेन श्रुत्वा गीतामृतं पुरा}
{स्तोत्रं कृष्णप्रियकरं कृतं तस्मान्मया श्रुतम्}

\twolineshloka
{कृष्णप्रेमामृतं नाम परमानन्ददायकम्}
{अत्युपद्रवदुःखघ्नं परमायुष्यवर्धनम्}

\twolineshloka
{दानं व्रतं तपस्तीर्थं यत्कृतं त्विह जन्मनि}
{पठतां शृण्वतां चैव कोटिकोटिगुणं भवेत्}

\twolineshloka
{पुत्त्रप्रदमपुत्त्राणामगतीनां गतिप्रदम्}
{धनावहं दरिद्राणां जयेच्छूनां जयावहम्}

\twolineshloka
{शिशूनां गोकुलानां च पुष्टिदं पुण्यवर्धनम्}
{बालरोगग्रहादीनां शमनं शान्तिकारकम्}

\twolineshloka
{अन्ते कृष्णस्मरणदं भवतापत्रयापहम्}
{असिद्धसाधकं भद्रे जपादिकरमात्मनाम्}

\twolineshloka
{कृष्णाय यादवेन्द्राय ज्ञानमुद्राय योगिने}
{नाथाय रुक्मिणीशाय नमो वेदान्तवेदिने}

\twolineshloka
{इमं मन्त्रं महादेवि जपन्नेव दिवानिशम्}
{सर्वग्रहानुग्रहभाक् सर्वप्रियतमो भवेत्}

\twolineshloka
{पुत्रपौत्रैः परिवृतः सर्वसिद्धिसमृद्धिमान्}
{निषेव्यभोगानन्तेऽपि कृष्णसायुज्यमाप्युनात्}

{॥ इति श्रीब्रह्माण्डे महापुराणे वायुप्रोक्ते मध्यभागे तृतीय उपोद्घातपादे भार्गवचरिते षट्त्रिंशत्तमोऽध्यायान्तर्गत श्रीकृष्णाष्टोत्तरशतनामस्तोत्रं सम्पूर्णम् ॥}

\sect{गजेन्द्र-मोक्षः नारायणीयतः}

\twolineshloka
{इन्द्रद्युम्नः पाण्ड्यखण्डाधिराजस्त्वद्भक्तात्मा चन्दनाद्रौ कदाचित्}
{त्वत्सेवायां मग्नधीरालुलोके नैवागस्त्यं प्राप्तमातिथ्यकामम्}%॥१॥

\twolineshloka
{कुम्भोद्भूतिस्संभृतक्रोधभारः स्तब्धात्मा त्वं हस्तिभूयं भजेति}
{शप्{}त्वाथैनं प्रत्यगात्सोऽपि लेभे हस्तीन्द्रत्वं त्वत्स्मृतिव्यक्तिधन्यम्}%॥२॥

\twolineshloka
{दुग्धाम्भोधेर्मध्यभाजि त्रिकूटे क्रोडञ्छैले यूथपोऽयं वशाभिः}
{सर्वान्जन्तूनत्यवर्तिष्ट शक्त्या त्वद्भक्तानां कुत्र नोत्कर्षलाभः}%॥३॥

\twolineshloka
{स्वेन स्थेम्ना दिव्यदेहत्वशक्त्या सोऽयं खेदानप्रजानन् कदाचित्}
{शैलप्रान्ते घर्मतान्तः सरस्यां यूथैः सार्धं त्वत्प्रणुन्नोऽभिरेमे}%॥४॥

\twolineshloka
{हूहूस्तावद्देवलस्यापि शापत् ग्राहीभूतस्तज्जले वर्तमानः}
{जग्राहैनं हस्तिनं पाददेशे शान्त्यर्थं हि श्रान्तिदोऽसि स्वकानाम्}%॥५॥

\twolineshloka
{त्वत्सेवाया वैभवाद्दुर्निरोधं युध्यन्तं तं वत्सराणां सहस्रम्}
{प्राप्ते काले त्वत्पदैकाग्र्यसिद्ध्यै नक्राक्रान्तं हस्तिवीरं व्यधास्त्वम्}%॥६॥

\twolineshloka
{आर्तिव्यक्तप्राक्तनज्ञानभक्तिः शुण्डोत्क्षिप्तैः पुण्डरीकैस्समर्चन्}
{पूर्वाभ्यस्तं निर्विशेषात्मनिष्ठं स्तोत्रश्रेष्ठं सोऽन्दगादीत्परात्मन्}%॥७॥

\twolineshloka
{श्रुत्वा स्तोत्रं निर्गुणस्थं समस्तं ब्रह्मेशाद्यैर्नाहमित्यप्रयाते}
{सर्वात्मा त्वं भूरिकारुण्यवेगात् तार्क्ष्यारूढः प्रेक्षितोऽभूः पुरस्तात्}%॥८॥

\twolineshloka
{हस्तीन्द्रं तं हस्तपद्मेन धृत्वा चक्रेण त्वं नक्रवर्यं व्यदारीः}
{गन्धर्वेऽस्मिन्मुक्तशापे स हस्ती त्वत्सारूप्यं प्राप्य देदीप्यते स्म}%॥९॥

\twolineshloka
{एतद्{}वृत्तं त्वां च मां च प्रगे यो गायेत्सोऽयं भूयसे श्रेयसे स्यात्}
{इत्युक्त्वैनं तेन सार्धं गतस्त्वं धिष्ण्यं विष्णो पाहि वातालयेश}%१०॥

॥इति श्रीमन्नारायणीये षड्विंश-दशकं सम्पूर्णम्॥

\closesection
\clearpage

\phantomsection\addcontentsline{toc}{chapter}{गुरुस्तोत्राणि}
% !TeX program = XeLaTeX
% !TeX root = ../../shloka.tex

\sect{दक्षिणामूर्त्यष्टकम्}

\fourlineindentedshloka
{विश्वं दर्पणदृश्यमाननगरीतुल्यं निजान्तर्गतम्}
{पश्यन्नात्मनि मायया बहिरिवोद्भूतं यदा निद्रया}
{यः साक्षात्कुरुते प्रबोधसमये स्वात्मानमेवाद्वयम्}
{तस्मै श्रीगुरुमूर्तये नम इदं श्रीदक्षिणामूर्तये}

\fourlineindentedshloka
{बीजस्यान्तरिवाङ्कुरो जगदिदं प्राङ्\mbox{}निर्विकल्पं पुनः}
{मायाकल्पितदेशकालकलनावैचित्र्यचित्रीकृतम्}
{मायावीव विजृम्भयत्यपि महायोगीव यः स्वेच्छया}
{तस्मै श्रीगुरुमूर्तये नम इदं श्रीदक्षिणामूर्तये}

\fourlineindentedshloka
{यस्यैव स्फुरणं सदाऽऽत्मकमसत्कल्पार्थगं भासते}
{साक्षात् तत्त्वमसीति वेदवचसा यो बोधयत्याश्रितान्}
{यत्साक्षात्करणाद्भवेन्न पुनरावृत्तिर्भवाम्भोनिधौ}
{तस्मै श्रीगुरुमूर्तये नम इदं श्रीदक्षिणामूर्तये}

\fourlineindentedshloka
{नानाच्छिद्रघटोदरस्थितमहादीपप्रभाभास्वरम्}
{ज्ञानं यस्य तु चक्षुरादिकरणद्वारा बहिः स्पन्दते}
{जानामीति तमेव भान्तमनुभात्येतत्समस्तं जगत्}
{तस्मै श्रीगुरुमूर्तये नम इदं श्रीदक्षिणामूर्तये}

\fourlineindentedshloka
{देहं प्राणमपीन्द्रियाण्यपि चलां बुद्धिं च शून्यं विदुः}
{स्त्रीबालान्धजडोपमास्त्वहमिति भ्रान्ता भृशं वादिनः}
{मायाशक्तिविलासकल्पितमहाव्यामोहसंहारिणे}
{तस्मै श्रीगुरुमूर्तये नम इदं श्रीदक्षिणामूर्तये}

\fourlineindentedshloka
{राहुग्रस्तदिवाकरेन्दुसदृशो मायासमाच्छादनात्}
{सन्मात्रः करणोपसंहरणतो योऽभूत्सुषुप्तः पुमान्}
{प्रागस्वाप्समिति प्रबोधसमये यः प्रत्यभिज्ञायते}
{तस्मै श्रीगुरुमूर्तये नम इदं श्रीदक्षिणामूर्तये}

\fourlineindentedshloka
{बाल्यादिष्वपि जाग्रदादिषु तथा सर्वास्ववस्थास्वपि}
{व्यावृत्तास्वनुवर्तमानमहमित्यन्तः स्फुरन्तं सदा}
{स्वात्मानं प्रकटीकरोति भजतां यो मुद्रया भद्रया}
{तस्मै श्रीगुरुमूर्तये नम इदं श्रीदक्षिणामूर्तये}

\fourlineindentedshloka
{विश्वं पश्यति कार्यकारणतया स्वस्वामिसम्बन्धतः}
{शिष्याचार्यतया तथैव पितृपुत्राद्यात्मना भेदतः}
{स्वप्ने जाग्रति वा य एष पुरुषो मायापरिभ्रामितः}
{तस्मै श्रीगुरुमूर्तये नम इदं श्रीदक्षिणामूर्तये}

\fourlineindentedshloka*
{भूरम्भांस्यनलोऽनिलोऽम्बरमहर्नाथो हिमांशुः पुमान्}
{इत्याभाति चराचरात्मकमिदं यस्यैव मूर्त्यष्टकम्}
{नान्यत् किञ्चन विद्यते विमृशतां यस्मात्परस्माद्विभोः}
{तस्मै श्रीगुरुमूर्तये नम इदं श्रीदक्षिणामूर्तये}

\fourlineindentedshloka*
{सर्वात्मत्वमिति स्फुटीकृतमिदं यस्मादमुष्मिन् स्तवे}
{तेनास्य श्रवणात्तदर्थमननाद्ध्यानाच्च सङ्कीर्तनात्}
{सर्वात्मत्वमहाविभूतिसहितं स्यादीश्वरत्वं स्वतः}
{सिध्येत् तत्पुनरष्टधा परिणतं चैश्वर्यमव्याहतम्}

॥इति श्रीमच्छङ्कराचार्यविरचितं श्री~दक्षिणामूर्त्यष्टकं सम्पूर्णम्॥

\closesection
\fourlineindentedshloka*
{वटविटपिसमीपे भूमिभागे निषण्णम्}
{सकलमुनिजनानां ज्ञानदातारमारात्}
{त्रिभुवनगुरुमीशं दक्षिणामूर्तिदेवम्}
{जननमरणदुःखच्छेददक्षं नमामि}
% !TeX program = XeLaTeX
% !TeX root = ../../shloka.tex

\sect{तोटकाष्टकम्}

\twolineshloka*
{शङ्करं शङ्कराचार्यं केशवं बादरायणम्}%
{सूत्रभाष्यकृतौ वन्दे भगवन्तौ पुनः पुनः}% 

\fourlineindentedshloka
{नारायणं पद्मभुवं वसिष्ठं शक्तिं च तत्पुत्रपराशरं च}%
{व्यासं शुकं गौडपदं महान्तं गोविन्दयोगीन्द्रमथास्य शिष्यम्}% 
{श्री शङ्कराचार्यमथास्य पद्मपादं च हस्तामलकं च शिष्यम्}%
{तं तोटकं वार्तिककारमन्यानस्मद्गुरून् सन्ततमानतोऽस्मि}% 

\twolineshloka
{विदिताखिलशास्त्रसुधाजलधे महितोपनिषत् कथितार्थनिधे}%
{हृदये कलये विमलं चरणं भव शङ्कर देशिक मे शरणम्}% १}% 


\twolineshloka
{करुणावरुणालय पालय मां भवसागरदुःखविदूनहृदम्}%
{रचयाखिलदर्शनतत्त्वविदं भव शङ्कर देशिक मे शरणम्}% २}% 


\twolineshloka
{भवता जनता सुहिता भविता निजबोधविचारण चारुमते}%
{कलयेश्वरजीवविवेकविदं भव शङ्कर देशिक मे शरणम्}% ३}% 


\twolineshloka
{भव एव भवानिति मे नितरां समजायत चेतसि कौतुकिता}%
{मम वारय मोहमहाजलधिं भव शङ्कर देशिक मे शरणं }% ४}% 


\twolineshloka
{सुकृतेऽधिकृते बहुधा भवतो भविता समदर्शनलालसता}%
{अतिदीनमिमं परिपालय मां भव शङ्कर देशिक मे शरणम्}% ५}% 


\twolineshloka
{जगतीमवितुं कलिताकृतयो विचरन्ति महामहसश्छलतः}%
{अहिमांशुरिवात्र विभासि गुरो भव शङ्कर देशिक मे शरणम्}% ६}% 


\twolineshloka
{गुरुपुङ्गव पुङ्गवकेतन ते समतामयतां न हि कोऽपि सुधीः}%
{शरणागतवत्सल तत्त्वनिधे भव शङ्कर देशिक मे शरणम्}% ७}% 


\twolineshloka
{विदिता न मया विशदैककला न च किञ्चन काञ्चनमस्ति गुरो}%
{द्रुतमेव विधेहि कृपां सहजां भव शङ्कर देशिक मे शरणम्}% ८}% 

॥इति श्री तोटकाचार्यविरचितं श्री तोटकाष्टकं सम्पूर्णम्॥\clearpage
% !TeX program = XeLaTeX
% !TeX root = ../../shloka.tex

\sect{शङ्कराचार्याष्टोत्तरशतनामस्तोत्रम्}

\dnsub{ध्यानम्}
\twolineshloka*
{कैलासाचल-मध्यस्थं कामिताभीष्टदायकम्}
{ब्रह्मादि-प्रार्थना-प्राप्त-दिव्यमानुष-विग्रहम्}

\twolineshloka*
{भक्तानुग्रहणैकान्त-शान्त-स्वान्त-समुज्ज्वलम्}
{संयज्ञं संयमीन्द्राणां सार्वभौमं जगद्गुरुम्}

\twolineshloka*
{किङ्करीभूतभक्तैनः पङ्कजातविशोषणम्}
{ध्यायामि शङ्कराचार्यं सर्वलोकैकशङ्करम्	}

\begin{minipage}{\linewidth}
\centering
\dnsub{स्तोत्रम्}
\twolineshloka
{श्रीशङ्कराचार्यवर्यो ब्रह्मानन्दप्रदायकः}
{अज्ञानतिमिरादित्यः सुज्ञानाम्बुधिचन्द्रमा}
\end{minipage}

\twolineshloka
{वर्णाश्रमप्रतिष्ठाता श्रीमान् मुक्तिप्रदायकः}
{शिष्योपदेशनिरतो भक्ताभीष्टप्रदायकः}

\twolineshloka
{सूक्ष्मतत्त्वरहस्यज्ञः कार्याकार्यप्रबोधकः}
{ज्ञानमुद्राङ्कितकरः शिष्य-हृत्ताप-हारकः}

\twolineshloka
{परिव्राजाश्रमोद्धर्त्ता सर्वतन्त्रस्वतन्त्रधीः}
{अद्वैतस्थापनाचार्यः साक्षाच्छङ्कररूपधृक्}

\twolineshloka
{षण्मतस्थापनाचार्यस्त्रयीमार्गप्रकाशकः}
{वेदवेदान्ततत्त्वज्ञो दुर्वादिमतखण्डनः}

\twolineshloka
{वैराग्यनिरतः शान्तः संसारार्णवतारकः}
{प्रसन्नवदनाम्भोजः परमार्थप्रकाशकः}

\twolineshloka
{पुराणस्मृतिसारज्ञो नित्यतृप्तो महच्छुचिः}
{नित्यानन्दो निरातङ्को निःसङ्गो निर्मलात्मकः}

\twolineshloka
{निर्ममो निरहङ्कारो विश्ववन्द्यपदाम्बुजः}
{सत्त्वप्रदश्च सद्भावः सङ्ख्यातीतगुणोज्ज्वलः}

\twolineshloka
{अनघः सारहृदयः सुधीः सारस्वतप्रदः}
{सत्यात्मा पुण्यशीलश्च साङ्ख्ययोगविचक्षणः}

\twolineshloka
{तपोराशिर्महातेजा गुणत्रयविभागवित्}
{कलिघ्नः कालकर्मज्ञस्तमोगुणनिवारकः}

\twolineshloka
{भगवान् भारतीजेता शारदाह्वानपण्डितः}
{धर्माधर्मविभागज्ञो लक्ष्यभेदप्रदर्शकः}

\twolineshloka
{नादबिन्दुकलाभिज्ञो योगिहृत्पद्मभास्करः}
{अतीन्द्रिय-ज्ञाननिधिर्नित्यानित्यविवेकवान्}

\twolineshloka
{चिदानन्दश्चिन्मयात्मा परकाय-प्रवेशकृत्}
{अमानुष-चरित्राढ्यः क्षेमदायी क्षमाकरः}

\twolineshloka
{भव्यो भद्रप्रदो भूरिमहिमा विश्वरञ्जकः}
{स्वप्रकाशः सदाधारो विश्वबन्धुः शुभोदयः}

\twolineshloka
{विशालकीर्तिर्वागीशः सर्वलोकहितोत्सुकः}
{कैलासयात्रा-सम्प्राप्तश्चन्द्रमौलि-प्रपूजकः}

\twolineshloka
{काञ्च्यां श्रीचक्र-राजाख्य-यन्त्रस्थापन-दीक्षितः}
{श्रीचक्रात्मक-ताटङ्क-तोषिताम्बा-मनोरथः}

\twolineshloka
{श्रीब्रह्मसूत्रोपनिषद्भाष्यादिग्रन्थकल्पकः}
{चतुर्दिक्चतुराम्नायप्रतिष्ठाता महामतिः}

\twolineshloka
{द्विसप्तति-मतोच्छेत्ता सर्वदिग्विजयप्रभुः}
{काषायवसनोपेतो भस्मोद्धूलितविग्रहः}

\twolineshloka
{ज्ञानात्मकैकदण्डाढ्यः कमण्डलुलसत्करः}
{गुरुभूमण्डलाचार्यो भगवत्पादसंज्ञकः}

\twolineshloka
{व्याससन्दर्शनप्रीत ऋश्यशृङ्गपुरेश्वरः}
{सौन्दर्यलहरीमुख्यबहुस्तोत्रविधायकः}

\twolineshloka
{चतुःषष्टिकलाभिज्ञो ब्रह्मराक्षस-मोक्षदः}
{श्रीमन्मण्डनमिश्राख्यस्वयम्भूजयसन्नुतः}

\twolineshloka
{तोटकाचार्यसम्पूज्यः पद्मपादार्चिताङ्घ्रिकः}
{हस्तामलकयोगीन्द्रब्रह्मज्ञानप्रदायकः}

\threelineshloka
{सुरेश्वराख्य-सच्छिष्य-सन्न्यासाश्रम-दायकः}
{नृसिंहभक्तः सद्रत्नगर्भहेरम्बपूजकः}
{व्याख्यासिंहासनाधीशो जगत्पूज्यो जगद्गुरुः}

॥इति श्री शङ्कराचार्याष्टोत्तरशतनामस्तोत्रं सम्पूर्णम्॥

\closesection
\clearpage
\phantomsection\addcontentsline{toc}{chapter}{हनुमत्-स्तोत्राणि}
% !TeX program = XeLaTeX
% !TeX root = ../../shloka.tex

\setlength{\columnsep}{10pt}\newpage
\sect{हनुमान् चालीसा}
\begin{large}
\begin{multicols}{2}
\fourlineindentedshloka*
{श्रीगुरु चरन सरोज रज}
{निज मनु मुकुर सुधार}
{बरनऊँ रघुवर विमल यश}
{जो दायकु फल चार}

\fourlineindentedshloka*
{बुद्धिहीन तनु जानिके}
{सुमिरौं पवनकुमार}
{बल बुद्धि विद्या देहु मोहिं}
{हरहु कलेस विकार}

\dnsub{चौपाई}\resetShloka
\twolineshloka
{जय हनुमान ज्ञान गुण सागर}
{जय कपीश तिहुँ लोक उजागर}

\twolineshloka
{राम दूत अतुलित बल धामा}
{अञ्जनिपुत्र पवनसुत नामा}

\twolineshloka
{महावीर विक्रम बजरङ्गी}
{कुमति निवार सुमति के सङ्गी}

\twolineshloka
{कञ्चन बरन विराज सुवेसा}
{कानन कुण्डल कुञ्चित केशा}

\twolineshloka
{हाथ वज्र औ ध्वजा विराजै}
{काँधे मूँज जनेऊ साजै}

\twolineshloka
{सङ्कर सुवन केसरीनन्दन}
{तेज प्रताप महा जग वन्दन}

\twolineshloka
{विद्यावान गुणी अति चातुर}
{राम काज करिबे को आतुर}

\twolineshloka
{प्रभु चरित्र सुनिबे को रसिया}
{राम लखन सीता मन बसिया}

\twolineshloka*
{राम लक्ष्मण जानकी}
{जय बोलो हनुमान् की}

\twolineshloka
{सूक्ष्म रूप धरि सियहिं दिखावा}
{विकट रूप धरि लङ्क जरावा}

\twolineshloka
{भीम रूप धरि असुर सँहारे}
{रामचन्द्र के काज सँवारे}

\twolineshloka
{लाय सजीवन लखन जियाये}
{श्रीरघुवीर हरषि उर लाये}

\twolineshloka
{रघुपति कीन्ही बहुत बडाई}
{तुम मम प्रिय भरत सम भाई}

\twolineshloka
{सहस वदन तुम्हरो यश गावैं}
{अस कहि श्रीपति कण्ठ लगावैं}

\twolineshloka
{सनकादिक ब्रह्मादि मुनीशा}
{नारद शारद सहित अहीशा}

\twolineshloka
{यम कुबेर दिक्पाल जहाँ ते}
{कवि कोविद कहि सके कहाँ ते}

\twolineshloka
{तुम उपकार सुग्रीवहिं कीन्हा}
{राम मिलाय राज पद दीन्हा}

\twolineshloka*
{राम लक्ष्मण जानकी}
{जय बोलो हनुमान् की}

\twolineshloka
{तुम्हरो मन्त्र विभीषण माना}
{लङ्केश्वर भये सब जग जाना}

\twolineshloka
{युग सहस्र योजन पर भानू}
{लील्यो ताहि मधुर फल जानू}

\twolineshloka
{प्रभु मुद्रिका मेलि मुख माहीं}
{जलधि लाँघि गये अचरज नाहीं}

\twolineshloka
{दुर्गम काज जगत के जेते}
{सुगम अनुग्रह तुम्हरे तेते}

\twolineshloka
{राम दुआरे तुम रखवारे}
{होत न आज्ञा बिन पैसारे}

\twolineshloka
{सब सुख लहै तुम्हारी सरना}
{तुम रक्षक काहू को डर ना}

\twolineshloka
{आपन तेज सम्हारो आपै}
{तीनों लोक हाँक तें काँपै}

\twolineshloka
{भूत पिशाच निकट नहिं आवै}
{महावीर जब नाम सुनावै}

\twolineshloka*
{राम लक्ष्मण जानकी}
{जय बोलो हनुमान् की}

\twolineshloka
{नाशै रोग हरै सब पीरा}
{जपत निरन्तर हनुमत वीरा}

\twolineshloka
{सङ्कट से हनुमान छुडावै}
{मन क्रम वचन ध्यान जो लावै}

\twolineshloka
{सब पर राम तपस्वी राजा}
{तिन के काज सकल तुम साजा}

\twolineshloka
{और मनोरथ जो कोई लावै}
{दासु अमित जीवन फल पावै}

\twolineshloka
{चारों युग प्रताप तुम्हारा}
{है प्रसिद्ध जगत उजियारा}

\twolineshloka
{साधु सन्त के तुम रखवारे}
{असुर निकन्दन राम दुलारे}

\twolineshloka
{अष्ट सिद्धि नव निधि के दाता}
{अस बर दीन जानकी माता}

\twolineshloka
{राम रसायन तुम्हरे पासा}
{सदा रहो रघुपति के दासा}

\twolineshloka*
{राम लक्ष्मण जानकी}
{जय बोलो हनुमान् की}

\twolineshloka
{तुम्हरे भजन राम को पावै}
{जन्म जन्म के दुख बिसरावै}

\twolineshloka
{अन्त काल रघुपति पुर जाई}
{जहाँ जन्मि हरिभक्त कहाई}

\twolineshloka
{और देवता चित्त न धरई}
{हनुमत सेई सर्व सुख करई}

\twolineshloka
{सङ्कट कटै मिटै सब पीरा}
{जो सुमिरै हनुमत बलवीरा}

\twolineshloka
{जै जै जै हनुमान गोसाईं}
{कृपा करहु गुरु देव की नाईं}

\twolineshloka
{यह शत पार पाठ कर कोई}
{छूटहि बंदि महा सुख होई}

\twolineshloka
{यो यह पढ़ै हनुमान् चलीसा}
{होय सिद्धि साखी गौरीसा}

\twolineshloka
{तुलसीदास सदा हरि चेरा}
{कीजै नाथ हृदय मँह डेरा}

\twolineshloka*
{राम लक्ष्मण जानकी}
{जय बोलो हनुमान् की}
\end{multicols}
\end{large}
% !TeX program = XeLaTeX
% !TeX root = ../../shloka.tex

\sect{हनुमत् पञ्चरत्नम्}
\twolineshloka
{वीताखिल-विषयेच्छं जातानन्दाश्रु-पुलकमत्यच्छम्}
{सीतापति-दूताद्यं वातात्मजमद्य भावये हृद्यम्}

\twolineshloka
{तरुणारुण-मुख-कमलं करुणा-रसपूर-पूरितापाङ्गम्}
{सञ्जीवनमाशासे मञ्जुल-महिमानमञ्जना-भाग्यम्}

\twolineshloka
{शम्बरवैरि-शरातिगमम्बुजदल-विपुल-लोचनोदारम्}
{कम्बुगलमनिलदिष्टं बिम्ब-ज्वलितोष्ठमेकमवलम्बे}

\twolineshloka
{दूरीकृत-सीतार्तिः प्रकटीकृत-रामवैभव-स्फूर्तिः}
{दारित-दशमुख-कीर्तिः पुरतो मम भातु हनुमतो मूर्तिः}

\twolineshloka
{वानर-निकराध्यक्षं दानव-कुल-कुमुद-रविकर-सदृशम्}
{दीन-जनावन-दीक्षं पवनतपः पाकपुञ्जमद्राक्षम्}

\twolineshloka
{एतत् पवनसुतस्य स्तोत्रं यः पठति पञ्चरत्नाख्यम्}
{चिरमिह निखिलान् भोगान् भुक्त्वा श्रीराम-भक्तिभाग् भवति}
॥इति श्रीमच्छङ्कराचार्यविरचितं श्री~हनुमत्-पञ्चरत्नं सम्पूर्णम्॥

\twolineshloka*
{यत्र यत्र रघुनाथकीर्तनं तत्र तत्र कृत-मस्तकाञ्जलिम्}
{बाष्पवारिपरिपूर्ण-लोचनं मारुतिं नमत राक्षसान्तकम्‌}

\fourlineindentedshloka*
{उल्लङ्घ्य सिन्धोः सलिलं सलीलम्}
{यः शोकवह्निं जनकात्मजायाः}
{आदाय तेनैव ददाह लङ्काम्}
{नमामि तं प्राञ्जलिराञ्जनेयम्}

\twolineshloka*
{बुद्धिर्बलं यशो धैर्यं निर्भयत्वम् अरोगता}
{अजाड्यं वाक्पटुत्वं च हनुमत्स्मरणाद्भवेत्}

\twolineshloka*
{असाध्यसाधक स्वामिन् असाध्यं तव किं वद}
{रामदूतकृपसिन्धो मत्कार्यं साधय प्रभो}


% !TeX program = XeLaTeX
% !TeX root = ../../shloka.tex

\sect{आञ्जनेयाष्टोत्तरशतनामस्तोत्रम्}

\twolineshloka
{आञ्जनेयो महावीरो हनुमान् मारुतात्मजः}
{तत्त्वज्ञानप्रदः सीतादेवीमुद्राप्रदायकः}

\twolineshloka
{अशोकवनिकाच्छेत्ता सर्वमायाविभञ्जनः}
{सर्वबन्धविमोक्ता च रक्षोविध्वंसकारकः}

\twolineshloka
{परविद्यापरीहर्ता परशौर्यविनाशकः}
{परमन्त्रनिराकर्ता परयन्त्रप्रभेदकः}

\twolineshloka
{सर्वग्रहविनाशी च भीमसेनसहायकृत्}
{सर्वदुःखहरः सर्वलोकचारी मनोजवः}

\twolineshloka
{पारिजातद्रुमूलस्थः सर्वमन्त्रस्वरूपवान्}
{सर्वतन्त्रस्वरूपी च सर्वयन्त्रात्मिकस्तथा}

\twolineshloka
{कपीश्वरो महाकायः सर्वरोगहरः प्रभुः}
{बलसिद्धिकरः सर्वविद्यासम्पत्प्रदायकः}

\twolineshloka
{कपिसेनानायकश्च भविष्यच्चतुराननः}
{कुमारब्रह्मचारी च रत्नकुण्डलदीप्तिमान्}

\twolineshloka
{चञ्चलद्वालसन्नद्धो लम्बमानशिखोज्ज्वलः}
{गन्धर्वविद्यातत्त्वज्ञो महाबलपराक्रमः}

\twolineshloka
{कारागृहविमोक्ता च शृङ्खलाबन्धमोचकः}
{सागरोत्तारकः प्राज्ञो रामदूतः प्रतापवान्}

\twolineshloka
{वानरः केसरीसूनुः सीताशोकनिवारणः}
{अञ्जनागर्भसम्भूतो बालार्कसदृशाननः}

\twolineshloka
{विभीषणप्रियकरो दशग्रीवकुलान्तकः}
{लक्ष्मणप्राणदाता च वज्रकायो महाद्युतिः}

\twolineshloka
{चिरञ्जीवी रामभक्तो दैत्यकार्यविघातकः}
{अक्षहन्ता काञ्चनाभः पञ्चवक्त्रो महातपाः}

\twolineshloka
{लङ्किणीभञ्जनः श्रीमान् सिंहिकाप्राणभञ्जनः}
{गन्धमादनशैलस्थो लङ्कापुरविदाहकः}

\twolineshloka
{सुग्रीवसचिवो धीरः शूरो दैत्यकुलान्तकः}
{सुरार्चितो महातेजो रामचूडामणिप्रदः}

\twolineshloka
{कामरूपी पिङ्गलाक्षो वर्धिमैनाकपूजितः}
{कबलीकृतमार्ताण्डमण्डलो विजितेन्द्रियः}

\twolineshloka
{रामसुग्रीवसन्धाता महिरावणमर्दनः}
{स्फटिकाभो वागधीशो नवव्याकृतिपण्डितः}

\twolineshloka
{चतुर्बाहुर्दीनबन्धुर्महात्मा भक्तवत्सलः}
{सञ्जीवननगाहर्ता शुचिर्वाग्मी धृतव्रतः}

\twolineshloka
{कालनेमिप्रमथनो हरिमर्कटमर्कटः}
{दान्तः शान्तः प्रसन्नात्मा शतकण्ठमदापहः}

\twolineshloka
{योगी रामकथालोलः सीतान्वेषणपण्डितः}
{वज्रदंष्ट्रो वज्रनखो रुद्रवीर्यसमुद्भवः}

\twolineshloka
{इन्द्रजित्प्रहितामोघब्रह्मास्त्रविनिवारकः}
{पार्थध्वजाग्रसंवासी शरपञ्जरहेलकः}

\twolineshloka
{दशबाहुर्लोकपूज्यो जाम्बवत्प्रीतिवर्धनः}
{सीतासमेतश्रीरामपादसेवाधुरन्धरः}

{॥इति श्री आञ्जनेयाष्टोत्तरशतनामस्तोत्रं सम्पूर्णम्॥}


\closesection
\clearpage

\phantomsection\addcontentsline{toc}{chapter}{शास्तास्तोत्राणि}
% !TeX program = XeLaTeX
% !TeX root = ../../shloka.tex

\sect{हरिहरात्मजाष्टकम्}
\fourlineindentedshloka
{हरिवरासनं विश्वमोहनम्}
{हरिदधीश्वरम् आराध्यपादुकम्}
{अरिविमर्दनं नित्यनर्तनम्}
{हरिहरात्मजं देवमाश्रये}

\fourlineindentedshloka
{चरणकीर्तनं भक्तमानसम्}
{भरणलोलुपं नर्तनालसम्}
{अरुणभासुरं भूतनायकम्}
{हरिहरात्मजं देवमाश्रये}

\fourlineindentedshloka
{प्रणयसत्यकं प्राणनायकम्}
{प्रणतकल्पकं सुप्रभञ्चितम्}
{प्रणवमन्दिरं कीर्तनप्रियम्}
{हरिहरात्मजं देवमाश्रये}

\fourlineindentedshloka
{तुरगवाहनं सुन्दराननम्}
{वरगदायुधं वेदवर्णितम्}
{गुरुकृपाकरं कीर्तनप्रियम्}
{हरिहरात्मजं देवमाश्रये}

\fourlineindentedshloka
{त्रिभुवनार्चितं देवतात्मकम्}
{त्रिनयनप्रभुं दिव्यदेशिकम्}
{त्रिदशपूजितं चिन्तितप्रदम्}
{हरिहरात्मजं देवमाश्रये}

\fourlineindentedshloka
{भवभयापहं भावुकावकम्}
{भुवनमोहनं भूतिभूषणम्}
{धवलवाहनं दिव्यवारणम्}
{हरिहरात्मजं देवमाश्रये}

\fourlineindentedshloka
{कलमृदुस्मितं सुन्दराननम्}
{कलभकोमलं गात्रमोहनम्}
{कलभकेसरीं वाजिवाहनम्}
{हरिहरात्मजं देवमाश्रये}

\fourlineindentedshloka
{श्रितजनप्रियं चिन्तितप्रदम्}
{श्रुतिविभूषणं साधुजीवनम्}
{श्रुतिमनोहरं गीतलालसम्}
{हरिहरात्मजं देवमाश्रये}
॥इति श्री हरिहरात्मजाष्टकं सम्पूर्णम्॥
\closesection
\clearpage
\phantomsection\addcontentsline{toc}{chapter}{वेङ्कटेशस्तोत्राणि}
% !TeX program = XeLaTeX
% !TeX root = ../../shloka.tex

\sect{वेङ्कटेश सुप्रभातम्}
\twolineshloka
{कौसल्या सुप्रजा राम पूर्वा सन्ध्या प्रवर्तते}
{उत्तिष्ठ नरशार्दूल कर्तव्यं दैवमाह्निकम्}

\twolineshloka
{उत्तिष्ठोत्तिष्ठ गोविन्द उत्तिष्ठ गरुडध्वज}
{उत्तिष्ठ कमलाकान्त त्रैलोक्यं मङ्गलं कुरु}

\fourlineindentedshloka
{मातः समस्तजगतां मधुकैटभारेः}
{वक्षोविहारिणि मनोहरदिव्यमूर्ते}
{श्रीस्वामिनि श्रितजनप्रियदानशीले}
{श्रीवेङ्कटेशदयिते तव सुप्रभातम्}

\fourlineindentedshloka
{तव सुप्रभातमरविन्दलोचने}
{भवतु प्रसन्नमुखचन्द्रमण्डले}
{विधिशङ्करेन्द्रवनिताभिरर्चिते}
{वृषशैलनाथदयिते दयानिधे}

\fourlineindentedshloka
{अत्र्यादिसप्तऋषयः समुपास्य सन्ध्याम्}
{आकाशसिन्धुकमलानि मनोहराणि}
{आदाय पादयुगमर्चयितुं प्रपन्नाः}
{शेषाद्रिशेखरविभो तव सुप्रभातम्}

\fourlineindentedshloka
{पञ्चाननाब्जभवषण्मुखवासवाद्याः}
{त्रैविक्रमादिचरितं विबुधाः स्तुवन्ति}
{भाषापतिः पठति वासरशुद्धिमारात्}
{शेषाद्रिशेखरविभो तव सुप्रभातम्}

\fourlineindentedshloka
{ईषत्प्रफुल्ल-सरसीरुह-नारिकेल-}
{पूगद्रुमादि-सुमनोहरपालिकानाम्}
{आवाति मन्दमनिलः सह दिव्यगन्धैः}
{शेषाद्रिशेखरविभो तव सुप्रभातम्}

\fourlineindentedshloka
{उन्मील्य नेत्रयुगमुत्तमपञ्जरस्थाः}
{पात्रावशिष्टकदलीफलपायसानि}
{भुक्त्वा सलीलमथ केलिशुकाः पठन्ति}
{शेषाद्रिशेखरविभो तव सुप्रभातम्}

\fourlineindentedshloka
{तन्त्रीप्रकर्षमधुरस्वनया विपञ्च्या}
{गायत्यनन्तचरितं तव नारदोऽपि}
{भाषासमग्रमसकृत्करचाररम्यम्}
{शेषाद्रिशेखरविभो तव सुप्रभातम्}

\fourlineindentedshloka
{भृङ्गावली च मकरन्दरसानुविद्ध-}
{झङ्कारगीत निनदैः सह सेवनाय}
{निर्यात्युपान्तसरसीकमलोदरेभ्यः}
{शेषाद्रिशेखरविभो तव सुप्रभातम्}

\fourlineindentedshloka
{योषागणेन वरदध्निविमथ्यमाने}
{घोषालयेषु दधिमन्थनतीव्रघोषाः}
{रोषात्कलिं विदधते ककुभश्च कुम्भाः}
{शेषाद्रिशेखरविभो तव सुप्रभातम्}

\fourlineindentedshloka
{पद्मेशमित्रशतपत्रगतालिवर्गाः}
{हर्तुं श्रियं कुवलयस्य निजाङ्गलक्ष्म्या}
{भेरीनिनादमिव बिभ्रति तीव्रनादम्}
{शेषाद्रिशेखरविभो तव सुप्रभातम्}

\fourlineindentedshloka
{श्रीमन्नभीष्टवरदाखिललोकबन्धो}
{श्रीश्रीनिवास जगदेकदयैकसिन्धो}
{श्रीदेवतागृहभुजान्तरदिव्यमूर्ते}
{श्रीवेङ्कटाचलपते तव सुप्रभातम्}

\fourlineindentedshloka
{श्रीस्वामिपुष्करिणिकाऽऽप्लवनिर्मलाङ्गाः}
{श्रेयोऽर्थिनो हरविरिञ्चसनन्दनाद्याः}
{द्वारे वसन्ति वरवेत्रहतोत्तमाङ्गाः}
{श्रीवेङ्कटाचलपते तव सुप्रभातम्}

\fourlineindentedshloka
{श्रीशेषशैल-गरुडाचल-वेङ्कटाद्रि-}
{नारायणाद्रि-वृषभाद्रि-वृषाद्रि-मुख्याम्}
{आख्यां त्वदीय वसतेरनिशं वदन्ति}
{श्रीवेङ्कटाचलपते तव सुप्रभातम्}

\fourlineindentedshloka
{सेवापराः शिव-सुरेश-कृशानु-धर्म-}
{रक्षोऽम्बुनाथ-पवमान-धनाधिनाथाः}
{बद्धाञ्जलि-प्रविलसन्निजशीर्ष-देशाः}
{श्रीवेङ्कटाचलपते तव सुप्रभातम्}

\fourlineindentedshloka
{धाटीषु ते विहगराज-मृगाधिराज-}
{नागाधिराज-गजराज-हयाधिराजाः}
{ स्वस्वाधिकार-महिमाऽधिकमर्थयन्ते}
{श्रीवेङ्कटाचलपते तव सुप्रभातम्}

\fourlineindentedshloka
{सूर्येन्दु-भौम-बुध-वाक्पति-काव्य-सौरि-}
{स्वर्भानु-केतु-दिविषत्परिषत्प्रधानाः}
{त्वद्दास-दास-चरमावधि-दासदासाः}
{श्रीवेङ्कटाचलपते तव सुप्रभातम्}

\fourlineindentedshloka
{त्वत् पादधूलिभरितस्फुरितोत्तमाङ्गाः}
{स्वर्गापवर्गनिरपेक्ष-निजान्तरङ्गाः}
{कल्पागमाऽऽकलनयाऽऽकुलतां लभन्ते}
{श्रीवेङ्कटाचलपते तव सुप्रभातम्}

\fourlineindentedshloka
{त्वद्गोपुराग्रशिखराणि निरीक्षमाणाः}
{स्वर्गापवर्गपदवीं परमां श्रयन्तः}
{मर्त्या मनुष्यभुवने मतिमाश्रयन्ते}
{श्रीवेङ्कटाचलपते तव सुप्रभातम्}

\fourlineindentedshloka
{श्रीभूमिनायक दयादिगुणामृताब्धे}
{देवाधिदेव जगदेकशरण्यमूर्ते}
{श्रीमन्ननन्त गरुडादिभिरर्चिताङ्घ्रे}
{श्रीवेङ्कटाचलपते तव सुप्रभातम्}

\fourlineindentedshloka
{श्रीपद्मनाभ पुरुषोत्तम वासुदेव}
{वैकुण्ठ माधव जनार्दन चक्रपाणे}
{श्रीवत्सचिह्न शरणागत-पारिजात}
{श्रीवेङ्कटाचलपते तव सुप्रभातम्}

\fourlineindentedshloka
{कन्दर्पदर्पहरसुन्दरदिव्यमूर्ते}
{कान्ताकुचाम्बुरुह-कुङ्मल-लोलदृष्टे}
{कल्याणनिर्मलगुणाकर दिव्यकीर्ते}
{श्रीवेङ्कटाचलपते तव सुप्रभातम्}

\fourlineindentedshloka
{मीनाकृते कमठ कोल नृसिंह वर्णिन्}
{स्वामिन् परश्वथ तपोधन रामचन्द्र}
{शेषांशराम यदुनन्दन कल्किरूप}
{श्रीवेङ्कटाचलपते तव सुप्रभातम्}

\fourlineindentedshloka
{एला-लवङ्ग-घनसार-सुगन्धि-तीर्थम्}
{दिव्यं वियत्सरिति हेमघटेषु पूर्णम्}
{धृत्वाऽद्य वैदिकशिखामणयः प्रहृष्टाः}
{तिष्ठन्ति वेङ्कटपते तव सुप्रभातम्}

\fourlineindentedshloka
{भास्वानुदेति विकचानि सरोरुहाणि}
{सम्पूरयन्ति निनदैः ककुभो विहङ्गाः}
{श्रीवैष्णवाः सततमर्थित-मङ्गलास्ते}
{धामाऽऽश्रयन्ति तव वेङ्कट सुप्रभातम्}

\fourlineindentedshloka
{ब्रह्मादयः सुरवराः समहर्षयस्ते}
{सन्तः सनन्दन मुखास्तव योगिवर्याः}
{धामान्तिके तव हि मङ्गलवस्तुहस्ताः}
{श्रीवेङ्कटाचलपते तव सुप्रभातम्}

\fourlineindentedshloka
{लक्ष्मीनिवास निरवद्यगुणैकसिन्धो}
{संसार-सागर-समुत्तरणैकसेतो}
{वेदान्तवेद्य निजवैभव भक्तभोग्य}
{श्रीवेङ्कटाचलपते तव सुप्रभातम्}

\fourlineindentedshloka
{इत्थं वृषाचलपतेरिह सुप्रभातम्}
{ये मानवाः प्रतिदिनं पठितुं प्रवृत्ताः}
{तेषां प्रभातसमये स्मृतिरङ्गभाजाम्}
{प्रज्ञां परार्थसुलभां परमां प्रसूते}
॥इति श्री वेङ्कटेश सुप्रभातम् सम्पूर्णम्॥
% !TeX program = XeLaTeX
% !TeX root = ../../shloka.tex

\sect{वेङ्कटेश स्तोत्रम्}
\twolineshloka
{कमलाकुच-चूचुक-कुङ्कुमतो नियतारुणितातुल-नीलतनो}
{कमलायतलोचन लोकपते विजयी भव वेङ्कटशैलपते}

\twolineshloka
{सचतुर्मुख-षण्मुख-पञ्चमुख-प्रमुखाखिलदैवतमौलिमणे}
{शरणागतवत्सल सारनिधे परिपालय मां वृषशैलपते}

\twolineshloka
{अतिवेलतया तव दुर्विषहैरनुवेलकृतैरपराधशतैः}
{भरितं त्वरितं वृषशैलपते परया कृपया परिपाहि हरे}

\twolineshloka
{अधिवेङ्कटशैलमुदारमते जनताभिमताधिकदानरतात्}
{परदेवतया गदितान्निगमैः कमलादयितान्न परं कलये}

\twolineshloka
{कलवेणुरवावशगोपवधू शतकोटिवृतात्स्मरकोटिसमात्}
{प्रतिवल्लविकाभिमतात्सुखदात् वसुदेवसुतान्न परं कलये}

\twolineshloka
{अभिरामगुणाकर दाशरथे जगदेकधनुर्धर धीरमते}
{रघुनायक राम रमेश विभो वरदो भव देव दयाजलधे}

\twolineshloka
{अवनीतनया-कमनीयकरं रजनीकरचारुमुखाम्बुरुहम्}
{रजनीचरराजतमोमिहिरं महनीयमहं रघुराम मये}

\twolineshloka
{सुमुखं सुहृदं सुलभं सुखदं स्वनुजं च सुखायममोघशरम्}
{अपहाय रघूद्वहमन्यमहं न कथञ्चन कञ्चन जातु भजे}

\fourlineindentedshloka
{विना वेङ्कटेशं न नाथो न नाथः}
{सदा वेङ्कटेशं स्मरामि स्मरामि}
{हरे वेङ्कटेश प्रसीद प्रसीद}
{प्रियं वेङ्कटेश प्रयच्छ प्रयच्छ}% (एवं त्रिः)

\fourlineindentedshloka
{अहं दूरतस्ते पदाम्भोजयुग्म}
{प्रणामेच्छयाऽऽगत्य सेवां करोमि}
{सकृत्सेवया नित्यसेवाफलं त्वम्}
{प्रयच्छ प्रयच्छ प्रभो वेङ्कटेश}

\twolineshloka
{अज्ञानिना मया दोषानशेषान् विहितान् हरे}
{क्षमस्व त्वं क्षमस्व त्वं शेषशैल-शिखामणे}
॥इति श्री~वेङ्कटेश स्तोत्रं सम्पूर्णम्॥

% !TeX program = XeLaTeX
% !TeX root = ../../shloka.tex

\sect{वेङ्कटेश प्रपत्तिः}
\fourlineindentedshloka
{ईशानां जगतोऽस्य वेङ्कटपतेर्विष्णोः परां प्रेयसीम्}
{तद्वक्षःस्थल-नित्य-वासरसिकां तत्क्षान्ति-संवर्धिनीम्}
{पद्मालङ्कृतपाणिपल्लवयुगां पद्मासनस्थां श्रियम्}
{वात्सल्यादिगुणोज्ज्वलां भगवतीं वन्दे जगन्मातरम्}

\fourlineindentedshloka
{श्रीमन् कृपाजलनिधे कृतसर्वलोक}
{सर्वज्ञ शक्त नतवत्सल सर्वशेषिन्}
{स्वामिन् सुशील सुलभाश्रितपारिजात}
{श्रीवेङ्कटेश चरणौ शरणं प्रपद्ये}

\fourlineindentedshloka
{आनूपुरार्पितसुजातसुगन्धिपुष्प-}
{सौरभ्यसौरभकरौ समसन्निवेशौ}
{सौम्यौ सदाऽनुभवनेऽपि नवानुभाव्यौ}
{श्रीवेङ्कटेश चरणौ शरणं प्रपद्ये}

\fourlineindentedshloka
{सद्योविकासिसमुदित्वरसान्द्रराग-}
{सौरभ्यनिर्भरसरोरुहसाम्यवार्ताम्}
{सम्यक्षु साहसपदेषु विलेखयन्तौ}
{श्रीवेङ्कटेश चरणौ शरणं प्रपद्ये}

\fourlineindentedshloka
{रेखामयध्वजसुधाकलशातपत्र-}
{वज्राङ्कुशाम्बुरुहकल्पकशङ्खचक्रैः}
{भव्यैरलङ्कृततलौ परतत्त्वचिह्नैः}
{श्रीवेङ्कटेश चरणौ शरणं प्रपद्ये}

\fourlineindentedshloka
{ताम्रोदरद्युतिपराजितपद्मरागौ}
{बाह्यैर्महोभिरभिभूतमहेन्द्रनीलौ}
{उद्यन्नखांशुभिरुदस्तशशाङ्कभासौ}
{श्रीवेङ्कटेश चरणौ शरणं प्रपद्ये}

\fourlineindentedshloka
{सप्रेमभीति कमलाकरपल्लवाभ्याम्}
{संवाहनेऽपि सपदि क्लममादधानौ}
{कान्ताववाङ्मन-सगोचर-सौकुमार्यौ}
{श्रीवेङ्कटेश चरणौ शरणं प्रपद्ये}

\fourlineindentedshloka
{लक्ष्मीमहीतदनुरूपनिजानुभाव-}
{नीलादिदिव्यमहिषीकरपल्लवानाम्}
{आरुण्यसङ्क्रमणतः किल सान्द्ररागौ}
{श्रीवेङ्कटेश चरणौ शरणं प्रपद्ये}

\fourlineindentedshloka
{नित्यानमद्विधिशिवादिकिरीटकोटि-}
{प्रत्युप्त-दीप्त-नवरत्न-महःप्ररोहैः}
{नीराजनाविधिमुदारमुपाददानौ}
{श्रीवेङ्कटेश चरणौ शरणं प्रपद्ये}

\fourlineindentedshloka
{विष्णोः पदे परम इत्युतिदप्रशंसौ}
{यौ मध्व उत्स इति भोग्यतयाऽप्युपात्तौ}
{भूयस्तथेति तव पाणितलप्रदिष्टौ}
{श्रीवेङ्कटेश चरणौ शरणं प्रपद्ये}

\fourlineindentedshloka
{पार्थाय तत्सदृश-सारथिना त्वयैव}
{यौ दर्शितौ स्वचरणौ शरणं व्रजेति}
{भूयोऽपि मह्यमिह तौ करदर्शितौ ते}
{श्रीवेङ्कटेश चरणौ शरणं प्रपद्ये}

\fourlineindentedshloka
{मन्मूर्ध्नि कालियफणे विकटाटवीषु}
{श्रीवेङ्कटाद्रिशिखरे शिरसि श्रुतीनाम्}
{चित्तेऽप्यनन्यमनसां सममाहितौ ते}
{श्रीवेङ्कटेश चरणौ शरणं प्रपद्ये}

\fourlineindentedshloka
{अम्लानहृष्यदवनीतलकीर्णपुष्पौ}
{श्रीवेङ्कटाद्रि-शिखराभरणायमानौ}
{आनन्दिताखिल-मनो-नयनौ तवैतौ}
{श्रीवेङ्कटेश चरणौ शरणं प्रपद्ये}

\fourlineindentedshloka
{प्रायः प्रपन्न-जनता-प्रथमावगाह्यौ}
{मातुः स्तनाविव शिशोरमृतायमानौ}
{प्राप्तौ परस्परतुलामतुलान्तरौ ते}
{श्रीवेङ्कटेश चरणौ शरणं प्रपद्ये}

\fourlineindentedshloka
{सत्त्वोत्तरैः सतत-सेव्यपदाम्बुजेन}
{संसार-तारक-दयार्द्र-दृगञ्चलेन}
{सौम्यौ पयन्तृमुनिना मम दर्शितौ ते}
{श्रीवेङ्कटेश चरणौ शरणं प्रपद्ये}

\fourlineindentedshloka
{श्रीश श्रिया घटिकया त्वदुपायभावे}
{प्राप्ये त्वयि स्वयमुपेयतया स्फुरन्त्या}
{नित्याश्रिताय निरवद्यगुणाय तुभ्यम्}
{स्यां किङ्करो वृषगिरीश न जातु मह्यम्}
॥इति श्रीवेङ्कटेश प्रपत्तिः सम्पूर्णः॥
% !TeX program = XeLaTeX
% !TeX root = ../../shloka.tex

\sect{वेङ्कटेश मङ्गलाशासनम्}
\twolineshloka
{श्रियः कान्ताय कल्याणनिधये निधयेऽर्थिनाम्}
{श्रीवेङ्कटनिवासाय श्रीनिवासाय मङ्गलम्}

\twolineshloka
{लक्ष्मी-सविभ्रमालोक-सुभ्रू-विभ्रमचक्षुषे}
{चक्षुषे सर्वलोकानां वेङ्कटेशाय मङ्गलम्}

\twolineshloka
{श्रीवेङ्कटाद्रि-शृङ्गाग्र-मङ्गलाभरणाङ्घ्रये}
{मङ्गलानां निवासाय श्रीनिवासाय मङ्गलम्}

\twolineshloka
{सर्वावयवसौन्दर्य-सम्पदा सर्वचेतसाम्}
{सदा सम्मोहनायास्तु वेङ्कटेशाय मङ्गलम्}

\twolineshloka
{नित्याय निरवद्याय सत्यानन्दचिदात्मने}
{सर्वान्तरात्मने श्रीमद्-वेङ्कटेशाय मङ्गलम्}

\twolineshloka
{स्वतस्सर्वविदे सर्वशक्तये सर्वशेषिणे}
{सुलभाय सुशीलाय वेङ्कटेशाय मङ्गलम्}

\twolineshloka
{परस्मै ब्रह्मणे पूर्णकामाय परमात्मने}
{प्रयुञ्जे परतत्त्वाय वेङ्कटेशाय मङ्गलम्}

\twolineshloka
{आकालतत्त्वमश्रान्तमात्मनामनुपश्यताम्}
{अतृप्त्यमृतरूपाय वेङ्कटेशाय मङ्गलम्}

\twolineshloka
{प्रायः स्वचरणौ पुंसां शरण्यत्वेन पाणिना}
{कृपयाऽऽदिशते श्रीमद्-वेङ्कटेशाय मङ्गलम्}

\twolineshloka
{दयामृत-तरङ्गिण्यास्तरङ्गैरिव शीतलैः}
{अपाङ्गैः सिञ्चते विश्वं वेङ्कटेशाय मङ्गलम्}

\twolineshloka
{स्रग्भूषाम्बरहेतीनां सुषमावहमूर्तये}
{सर्वार्तिशमनायास्तु वेङ्कटेशाय मङ्गलम्}

\twolineshloka
{श्रीवैकुण्ठविरक्ताय स्वामिपुष्करिणीतटे}
{रमया रममाणाय वेङ्कटेशाय मङ्गलम्}

\twolineshloka
{श्रीमत् सुन्दरजामातृमुनिमानसवासिने}
{सर्वलोकनिवासाय श्रीनिवासाय मङ्गलम्}

\twolineshloka
{मङ्गलाशासनपरैर्मदाचार्य-पुरोगमैः}
{सर्वैश्च पूर्वैराचार्यैः सत्कृतायास्तु मङ्गलम्}

॥इति श्री~वेङ्कटेश मङ्गलाशासनं सम्पूर्णम्॥
% !TeX program = XeLaTeX
% !TeX root = ../../shloka.tex
\sect{वेङ्कटेश करावलम्बस्तोत्रम्}
\fourlineindentedshloka
{श्रीशेषशैल सुनिकेतन दिव्यमूर्ते}
{नारायणाच्युत हरे नलिनायताक्ष}
{लीलाकटाक्ष-परिरक्षित-सर्वलोक}
{श्रीवेङ्कटेश मम देहि करावलम्बम्}

\fourlineindentedshloka
{ब्रह्मादिवन्दितपदाम्बुज शङ्खपाणे}
{श्रीमत्सुदर्शन-सुशोभित-दिव्यहस्त}
{कारुण्यसागर शरण्य सुपुण्यमूर्ते}
{श्रीवेङ्कटेश मम देहि करावलम्बम्}

\fourlineindentedshloka
{वेदान्त-वेद्य भवसागर-कर्णधार}
{श्रीपद्मनाभ कमलार्चितपादपद्म}
{लोकैक-पावन परात्पर पापहारिन्}
{श्रीवेङ्कटेश मम देहि करावलम्बम्}

\fourlineindentedshloka
{लक्ष्मीपते निगमलक्ष्य निजस्वरूप}
{कामादिदोष-परिहारक बोधदायिन्}
{दैत्यादिमर्दन जनार्दन वासुदेव}
{श्रीवेङ्कटेश मम देहि करावलम्बम्}

\fourlineindentedshloka
{तापत्रयं हर विभो रभसा मुरारे}
{संरक्ष मां करुणया सरसीरुहाक्ष}
{मच्छिष्य इत्यनुदिनं परिरक्ष विष्णो}
{श्रीवेङ्कटेश मम देहि करावलम्बम्}

\fourlineindentedshloka
{श्री जातरूपनवरत्न-लसत्किरीट}
{कस्तूरिकातिलकशोभिललाटदेश}
{राकेन्दुबिम्ब-वदनाम्बुज वारिजाक्ष}
{श्रीवेङ्कटेश मम देहि करावलम्बम्}

\fourlineindentedshloka
{वन्दारुलोक-वरदान-वचोविलास}
{रत्नाढ्यहार-परिशोभित-कम्बुकण्ठ}
{केयूररत्न-सुविभासि-दिगन्तराल}
{श्रीवेङ्कटेश मम देहि करावलम्बम्}

\fourlineindentedshloka
{दिव्याङ्गदाञ्चित-भुजद्वय मङ्गलात्मन्}
{केयूरभूषण-सुशोभित-दीर्घबाहो}
{नागेन्द्र-कङ्कण-करद्वय कामदायिन्}
{श्रीवेङ्कटेश मम देहि करावलम्बम्}

\fourlineindentedshloka
{स्वामिन् जगद्धरणवारिधिमध्यमग्नम्}
{मामुद्धराद्य कृपया करुणापयोधे}
{लक्ष्मीं च देहि मम धर्म-समृद्धिहेतुम्}
{श्रीवेङ्कटेश मम देहि करावलम्बम्}

\fourlineindentedshloka
{दिव्याङ्गरागपरिचर्चित-कोमलाङ्ग}
{पीताम्बरावृततनो तरुणार्क-दीप्ते}
{सत्काञ्चनाभ-परिधान-सुपट्टबन्ध}
{श्रीवेङ्कटेश मम देहि करावलम्बम्}

\fourlineindentedshloka
{रत्नाढ्यदाम-सुनिबद्ध-कटि-प्रदेश}
{माणिक्यदर्पण-सुसन्निभ-जानुदेश}
{जङ्घाद्वयेन परिमोहित सर्वलोक}
{श्रीवेङ्कटेश मम देहि करावलम्बम्}

\fourlineindentedshloka
{लोकैकपावन-सरित्परिशोभिताङ्घ्रे}
{त्वत्पाददर्शन दिने च ममाघमीश}
{हार्दं तमश्च सकलं लयमाप भूमन्}
{श्रीवेङ्कटेश मम देहि करावलम्बम्}

\fourlineindentedshloka
{कामादि-वैरि-निवहोऽच्युत मे प्रयातः}
{दारिद्र्यमप्यपगतं सकलं दयालो}
{दीनं च मां समवलोक्य दयार्द्र-दृष्ट्या}
{श्रीवेङ्कटेश मम देहि करावलम्बम्}

\fourlineindentedshloka
{श्रीवेङ्कटेश-पदपङ्कज-षट्पदेन}
{श्रीमन्नृसिंहयतिना रचितं जगत्याम्}
{ये तत्पठन्ति मनुजाः पुरुषोत्तमस्य}
{ते प्राप्नुवन्ति परमां पदवीं मुरारेः}
॥इति~श्री~शृङ्गेरि-जगद्गुरुणा~श्री~नृसिंहभारती-स्वामिना रचितं श्री~वेङ्कटेश~करावलम्बस्तोत्रं सम्पूर्णम्॥
% !TeX program = XeLaTeX
% !TeX root = ../../shloka.tex

\sect{श्रीनिवास गद्यम्}
\begin{flushleft}
श्रीमदखिल-महीमण्डल-मण्डन-धरणिधर-मण्डलाखण्डलस्य'
निखिल-सुरासुर-वन्दित-वराहक्षेत्र-विभूषणस्य' शेषाचल-\\गरुडाचल-वृषभाचल-नारायणाचलाञ्जनाचलादि शिखरिमालाकुलस्य' नादमुख-बोधनिधि-वीधिगुण-साभरण-सत्त्वनिधि-तत्त्वनिधि-भक्तिगुणपूर्ण-श्रीशैलपूर्ण-गुणवशंवद-परमपुरुष-कृपापूर-विभ्रमदतुङ्गशृङ्ग-गलद्गगनगङ्गासमालिङ्गितस्य' सीमातिग गुण रामानुजमुनि नामाङ्कित बहु भूमाश्रय सुरधामालय
वनरामायत वनसीमापरिवृत विशङ्कटतट निरन्तर विजृम्भित भक्तिरस 
निर्झरानन्तार्याहार्य प्रस्रवणधारापूर विभ्रमद-सलिल\-भरभरित महातटाक मण्डितस्य' कलिकर्दम मलमर्दन कलितोद्यम विलसद्यम
नियमादिम मुनिगणनिषेव्यमाण प्रत्यक्षीभवन्निजसलिल मज्जन
नमज्जन निखिलपापनाशन पापनाशन तीर्थाध्यासितस्य'
मुरारिसेवक जरादिपीडित निरार्तिजीवन निराश भूसुर वरातिसुन्दर सुराङ्गनारति कराङ्गसौष्ठव कुमारताकृति कुमारतारक समापनोदय तनूनपातक महापदामय विहापनोदित सकलभुवन विदित कुमारधाराभिधान-तीर्थाधिष्ठितस्य' धरणितल गत सकल हतकलिल शुभसलिल गतबहुळ विविधमल हति चतुर रुचिरतर विलोकनमात्र विदळित विविधमहापातक स्वामिपुष्करिणी समेतस्य' बहुसङ्कट
नरकावट पतदुत्कट कलिकङ्कट कलुषोद्भट जनपातक विनिपातक
रुचिनाटक करहाटक कलशाहृत कमलारत शुभमज्जन जल सज्जन भरित निजदुरित हतिनिरत जनसतत निरर्गळपेपीयमान सलिल
सम्भृत विशङ्कट कटाहतीर्थ विभूषितस्य' एवमादिम भूरिमञ्जिम
सर्वपातक गर्वहातक सिन्धुडम्बर हारिशम्बर विविधविपुल पुण्यतीर्थनिवहनिवासस्य' श्रीमतो वेङ्कटाचलस्य शिखरशेखर-%
महाकल्पशाखी' खर्वीभवदति गर्वीकृत गुरुमेर्वीशगिरि मुखोर्वीधर कुलदर्वीकर दयितोर्वीधर शिखरोर्वी' सतत सदूर्वीकृति चरणघन गर्वचर्वण निपुण तनुकिरणमसृणित गिरिशिखरशेखरतरुनिकर
तिमिरः' वाणीपतिशर्वाणी दयितेन्द्राणीश्वर मुख नाणीयोरसवेणी निभशुभवाणी नुतमहिमाणी' यस्तर कोणी भवदखिलभुवनभवनोदरः' वैमानिकगुरु भूमाधिक गुण रामानुज कृतधामाकर करधामारि दरललामाच्छकनक दामायित निजरामालय' नवकिसलयमय तोरणमालायित वनमालाधरः' कालाम्बुद मालानिभ नीलालक
जालावृत बालाब्ज सलीलामल फालाङ्गसमूलामृत धाराद्वयावधीरण' धीरललिततर विशदतर घन घनसारमयोर्ध्वपुण्ड्ररेखाद्वयरुचिरः' सुविकस्वर दळभास्वर कमलोदर गतमेदुर नवकेसर ततिभासुर परिपिञ्जर कनकाम्बर कलितादर ललितोदर तदालम्ब जम्भरिपु मणिस्तम्भ गम्भीरिमदम्भस्तम्भ समुज्जृम्भमान पीवरोरुयुगळ
तदालम्ब पृथुल कदळी मुकुल मदहरणजङ्घाल जङ्घायुगळः'
नव्यदळ भव्यगल पीतमल शोणिमल सन्मृदुल सत्किसलयाश्रुजल-%
कारि बल शोणतल पदकमल निजाश्रय बलबन्दीकृत शरदिन्दुमण्डली विभ्रमदादभ्र शुभ्र पुनर्भवाधिष्ठिताङ्गुळीगाढ निपीडित पद्मापनः' जानुतलावधि लम्बि विडम्बित वारण शुण्डादण्ड विजृम्भित नीलमणिमय कल्पकशाखा विभ्रमदायि मृणाळलतायत समुज्ज्वलतर कनकवलय वेल्लितैकतर बाहुदण्डयुगळः' युगपदुदित कोटि खरकर हिमकर मण्डल जाज्वल्यमान सुदर्शन पाञ्चजन्य समुत्तुङ्गित शृङ्गापर बाहु युगळः' अभिनवशाण समुत्तेजित महामहा नीलखण्ड मतखण्डन निपुण नवीन परितप्त कार्तस्वर कवचित महनीय पृथुल सालग्राम परम्परा गुम्भित नाभिमण्डल पर्यन्त लम्बमान प्रालम्बदीप्ति समालम्बित विशाल वक्षःस्थलः' गङ्गाझर तुङ्गाकृति भङ्गावळि भङ्गावह सौधावळि बाधावह धारानिभ हारावळि दूराहत गेहान्तर मोहावह महिम मसृणित महातिमिरः' पिङ्गाकृति भृङ्गारु निभाङ्गार दळाङ्गामल नीष्कासित दुष्कार्यघ निष्कावळि दीपप्रभ नीपच्छवि तापप्रद कनकमालिका पिशङ्गित सर्वाङ्गः' नवदळित दळवलित मृदुललित कमलतति मदविहति चतुरतर पृथुलतर सरसतर कनकसरमय रुचिकण्ठिका कमनीयकण्ठः' वाताशनाधिपति शयन कमन परिचरण रतिसमेताखिल फणधरतति मतिकरकनकमय नागाभरण परिवीताखिलाङ्गावगमित शयन भूताहिराज जातातिशयः' रविकोटी परिपाटी धरकोटी रपताटी कितवाटी रसधाटी धर मणिगणकिरण विसरण सततविधुत तिमिरमोह गर्भगेहः' अपरिमित विविधभुवन भरिताखण्ड ब्रह्माण्डमण्डल पिचण्डिलः' आर्यधुर्यानन्तार्य पवित्र खनित्रपात पात्रीकृत निजचुबुक गतव्रणकिण विभूषणवहनसूचित श्रितजनवत्सलतातिशयः' मड्डुडिण्डिम ढमरु जर्झर काहळी पटहावळी मृदुमर्द्दलाशि मृदङ्ग दुन्दुभि ढक्किकामुक हृद्य वाद्यक मधुरमङ्गळ नादमेदुर विसृमर सरस गानरस रुचिर सन्तत सन्तन्यमान नित्योत्सव पक्षोत्सव मासोत्सव संवत्सरोत्सवादि विविधोत्सव कृतानन्दः' श्रीमदानन्दनिलय विमानवासः' सतत पद्मालया पदपद्मरेणु सञ्चितवक्षःस्थल पटवासः' श्रीश्रीनिवासः' सुप्रसन्नो विजयताम्॥१॥

नाटारभि भूपाळ बिलहरि मायामाळव गौळा असावेरी' सावेरी
शुद्धसावेरी देवगान्धारी' धन्यासी बेगड हिन्दुस्थानी कापी तोडी नाटकुरञ्जी' श्रीराग सहन अठाण सारङ्गी दर्बारु पन्तुवराळी वराळी' कल्याणी पूर्वीकल्याणी यमुनाकल्याणी हुसेनी जञ्झोटी कौमारी'
कन्नड खरहरप्रिया कलहंस नादनामक्रिया मुखारी' तोडी पुन्नागवराळी
काम्भोजी भैरवी' यदुकुलकाम्भोजी आनन्दभैरवी शङ्कराभरण मोहन
रेगुप्ती सौराष्ट्री' नीलाम्बरी गुणक्रिया मेघगर्जनी' हंसध्वनि शोकवराळी मध्यमावती जेञ्जुरुटी सुरटी' द्विजावन्ती मलयाम्बरी कापि
परशुधनासरी देशिकतोडी' आहिरी वसन्तगौळी सन्तु केदारगौळा कनकाङ्गी रत्नाङ्गी गानमूर्ति' वनस्पति वाचस्पति दानवती मानरूपी सेनापति' हनुमत्तोडी धेनुका नाटकप्रिया कोकिलप्रिया रूपवती गायकप्रिया' वकुळाभरण चक्रवाक सूर्यकान्त हाटकाम्बरी
झङ्कारध्वनि' नटभैरवी गीर्वाणी हरिकाम्भोजी धीरशङ्कराभरण नागानन्दिनी यागप्रिया' विसृमर सरस गानरसेत्यादि सन्तत
सन्तन्यमान नित्योत्सव पक्षोत्सव मासोत्सव संवत्सरोत्सवादि
विविधोत्सव कृतानन्दः' श्रीमदानन्दनिलयवासः' सतत पद्मालया पदपद्मरेणु सञ्चितवक्षःस्थल पटवासः' श्रीश्रीनिवासः' सुप्रसन्नो विजयताम्॥२॥

श्री अलर्मेल्मङ्गासमेत श्रीश्रीनिवास स्वामी' सुप्रीतः सुप्रसन्नो वरदो
भूत्वा' पनस पाटली पालाश बिल्व पुन्नाग चूत कदळी चन्दन
चम्पक मञ्जुळ मन्दार हिन्तुळादि तिलक मातुलुङ्ग नारिकेळ
क्रौञ्चाशोक माधूकामलक हिन्दुक नागकेतक पूर्णकुन्द पूर्ण गन्ध रस
कन्द वन वञ्जुळ खर्जूर साल कोविदार हिन्ताल पनस विकट
वैकसवरुण तरुधमरण विचुळङ्काश्वत्थ यक्ष वसुध वर्माध मन्त्रिणी'
तिन्त्रिणी बोध न्यग्रोध घटपटल जम्बूमतल्ली वसति वासती जीवनी
पोषणी प्रमुख निखिल सन्दोह तमाल माला महित विराजमान चषक
मयूर हंस भारद्वाज कोकिल चक्रवाक कपोत गरुड नारायण नानाविध पक्षिजाति समूह ब्रह्म-क्षत्रिय-वैश्य-शूद्र-नानाजात्युद्भव देवता
निर्माण' माणिक्य-वज्र-वैडूर्य-गोमेधिक-पुष्यराग-पद्मरागेन्द्र प्रवाळमौक्तिक-स्फटिक-हेम-रत्नखचित धगद्धजायमान रथगज
तुरग पदाति सेवा समूह' भेरी-मद्दळ-मुरवक-झल्लरी-शङ्ख-काहळ नृत्यगीत-ताळवाद्य-कुम्भवाद्य-पञ्चमुखवाद्य अहमीमार्गन्नटीवाद्य किटिकुन्तलवाद्य सुरटीचौण्डोवाद्य तिमिलकविताळवाद्य
तक्कराग्रवाद्य घण्टाताडन ब्रह्मताळ समताळ कोट्टरीताळ ढक्करीताळ ऎक्काळ' धारावाद्य पटह कांस्यवाद्य भरतनाट्यालङ्कार किन्नर किम्पुरुष
रुद्रवीणा मुखवीणा वायुवीणा' तुम्बुरुवीणा गान्धर्ववीणा नारदवीणा' स्वरमण्डल रावणहस्तवीणास्तक्रियालङ्क्रियालङ्कृतानेक-\\विधवाद्य वापीकूपतटाकादि गङ्गा यमुना रेवा वरुणा शोणनदी शोभनदी'
सुवर्णमुखी वेगवती वेत्रवती क्षीरनदी बाहुनदी गरुडनदी कावेरी ताम्रपर्णी प्रमुखा महापुण्यनद्यः' सजलतीर्थैः सहोभयकूलङ्गत सदाप्रवाह ऋग्यजुःसामाथर्वण वेदशास्त्रेतिहासपुराण-सकलविद्याघोष भानुकोटिप्रकाश चन्द्रकोटिसमान नित्यकल्याण परम्परोत्तरोत्तराभिवृद्धिर्भूयादिति' भवन्तो महान्तोऽनुगृह्णन्तु। 
ब्रह्मण्यो राजा धार्मिकोऽस्तु। देशोऽयं निरुपद्रवोऽस्तु।
सर्वे साधुजनाः सुखिनो विलसन्तु। समस्तसन्मङ्गळानि सन्तु। उत्तरोत्तराभिवृद्धिरस्तु। सकलकल्याणसमृद्धिरस्तु॥३॥
\centerline{॥हरिः ॐ॥}
\centerline{॥इति श्री श्रीशैलरङ्गाचार्यविरचितं श्री~श्रीनिवासगद्यं सम्पूर्णम्॥}

\closesection
\end{flushleft}
\closesection
\clearpage
\phantomsection\addcontentsline{toc}{chapter}{नवग्रहस्तोत्राणि}
% !TeX program = XeLaTeX
% !TeX root = ../../shloka.tex

\sect{नवग्रहस्तोत्रम्}

\twolineshloka
{जपाकुसुमसङ्काशं काश्यपेयं महद्युतिम्}
{तमोऽरिं सर्वपापघ्नं प्रणतोऽस्मि दिवाकरम्}

\twolineshloka
{दधिशङ्खतुषाराभं क्षीरोदार्णवसम्भवम्}
{नमामि शशिनं सोमं शम्भोर्मुकुटभूषणम्}

\twolineshloka
{धरणीगर्भसम्भूतं विद्युत्कान्तिसमप्रभम्}
{कुमारं शक्तिहस्तं च मङ्गलं प्रणमाम्यहम्}

\twolineshloka
{प्रियङ्गुकलिकाश्यामं रूपेणाप्रतिमं बुधम्}
{सौम्यं सौम्यगुणोपेतं तं बुधं प्रणमाम्यहम्}

\twolineshloka
{देवानां च ऋषीणां च गुरुं काञ्चनसन्निभम्}
{बुद्धिभूतं त्रिलोकेशं तं नमामि बृहस्पतिम्}

\twolineshloka
{हिमकुन्दमृणालाभं दैत्यानां परमं गुरुम्}
{सर्वशास्त्रप्रवक्तारं भार्गवं प्रणमाम्यहम्}

\twolineshloka
{नीलाञ्जनसमाभासं रविपुत्रं यमाग्रजम्}
{छायामार्तण्डसम्भूतं तं नमामि शनैश्चरम्}

\twolineshloka
{अर्धकायं महावीर्यं चन्द्रादित्यविमर्दनम्}
{सिंहिकागर्भसम्भूतं तं राहुं प्रणमाम्यहम्}

\twolineshloka
{पलाशपुष्पसङ्काशं तारकाग्रहमस्तकम्}
{रौद्रं रौद्रात्मकं घोरं तं केतुं प्रणमाम्यहम्}

\twolineshloka
{इति व्यासमुखोद्गीतं यः पठेत् सुसमाहितः}
{दिवा वा यदि वा रात्रौ विघ्नशान्तिर्भविष्यति}

\twolineshloka
{नरनारीनृपाणां च भवेद्दुःस्वप्ननाशनम्}
{ऐश्वर्यमतुलं तेषामारोग्यं पुष्टिवर्धनम्}

\twolineshloka
{ग्रहनक्षत्रजाः पीडास्तस्कराग्निसमुद्भवाः}
{ताः सर्वाः प्रशमं यान्ति व्यासो ब्रूते न संशयः}

॥इति श्रीव्यासविरचितं नवग्रहस्तोत्रं सम्पूर्णम्॥

% !TeX program = XeLaTeX
% !TeX root = ../../shloka.tex

\sect{नवग्रहपीडाहरस्तोत्रम्}

\twolineshloka
{ग्रहाणामादिरादित्यो लोकरक्षणकारकः}%
{विषमस्थानसम्भूतां पीडां हरतु मे रविः}%। १॥

\twolineshloka
{रोहिणीशः सुधामूर्तिः सुधागात्रः सुधाशनः}%
{विषमस्थानसम्भूतां पीडां हरतु मे विधुः}%। २॥

\twolineshloka
{भूमिपुत्रो महातेजा जगतां भयकृत् सदा}%
{वृष्टिकृद्वृष्टिहर्ता च पीडां हरतु मे कुजः}%। ३॥

\twolineshloka
{उत्पातरूपो जगतां चन्द्रपुत्रो महाद्युतिः}%
{सूर्यप्रियकरो विद्वान् पीडां हरतु मे बुधः}%। ४॥

\twolineshloka
{देवमन्त्री विशालाक्षः सदा लोकहिते रतः}%
{अनेकशिष्यसम्पूर्णः पीडां हरतु मे गुरुः}%। ५॥

\twolineshloka
{दैत्यमन्त्री गुरुस्तेषां प्राणदश्च महामतिः}%
{प्रभुस्ताराग्रहाणां च पीडां हरतु मे भृगुः}%। ६॥

\twolineshloka
{सूर्यपुत्रो दीर्घदेहो विशालाक्षः शिवप्रियः}%
{मन्दचारः प्रसन्नात्मा पीडां हरतु मे शनिः}%। ७॥

\twolineshloka
{महाशिरा महावक्त्रो दीर्घदंष्ट्रो महाबलः}%
{अतनुश्चोर्ध्वकेशश्च पीडां हरतु मे शिखी}%। ८॥

\twolineshloka
{अनेकरूपवर्णैश्च शतशोऽथ सहस्रशः}%
{उत्पातरूपो जगतां पीडां हरतु मे तमः}%। ९॥
॥इति ब्रह्माण्डपुराणोक्तं नवग्रहपीडाहरस्तोत्रं सम्पूर्णम्॥

\closesection

\fourlineindentedshloka*
{आरोग्यं प्रददातु नो दिनकरश्चन्द्रो यशो निर्मलम्}
{भूतिं भूमिसुतः सुधांशुतनयः प्रज्ञां गुरुर्गौरवम्‌}
{काव्यः कोमलवाग्विलासमतुलं मन्दो मुदं सर्वदा}
{राहुर्बाहुबलं विरोधशमनं केतुः कुलस्योन्नतिम्}

\closesection
\clearpage
\renewcommand{\chaptermark}[1]{%
\markboth{\large #1}{}}
\renewcommand{\sect}[1]{\chapt{#1}}
% !TeX program = XeLaTeX
% !TeX root = ../../shloka.tex

\sect{सङ्क्षेपरामायणम्}

\twolineshloka*
{शुक्लाम्बरधरं विष्णुं शशिवर्णं चतुर्भुजम्}
{प्रसन्नवदनं ध्यायेत् सर्वविघ्नोपशान्तये}

\twolineshloka*
{वागीशाद्याः सुमनसः सर्वार्थानामुपक्रमे}
{यं नत्वा कृतकृत्याः स्युस्तं नमामि गजाननम्}

\mbox{}\\
\dnsub{श्री~सरस्वती प्रार्थना}
\fourlineindentedshloka*
{दोर्भिर्युक्ता चतुर्भिं स्फटिकमणिनिभैरक्षमालां दधाना}
{हस्तेनैकेन पद्मं सितमपि च शुकं पुस्तकं चापरेण}
{भासा कुन्देन्दुशङ्खस्फटिकमणिनिभा भासमानाऽसमाना}
{सा मे वाग्देवतेयं निवसतु वदने सर्वदा सुप्रसन्ना}

\mbox{}\\
\dnsub{श्री~वाल्मीकि नमस्क्रिया}

\twolineshloka
{कूजन्तं राम रामेति मधुरं मधुराक्षरम्}
{आरुह्य कविताशाखां वन्दे वाल्मीकिकोकिलम्}

\twolineshloka
{वाल्मीकेर्मुनिसिंहस्य कवितावनचारिणः}
{शृण्वन् रामकथानादं को न याति परां गतिम्}

\twolineshloka
{यः पिबन् सततं रामचरितामृतसागरम्}
{अतृप्तस्तं मुनिं वन्दे प्राचेतसमकल्मषम्}

\begin{minipage}{\linewidth}
\centering
\resetShloka
\dnsub{श्री~हनुमन्नमस्क्रिया}

\twolineshloka
{गोष्पदीकृत-वाराशिं मशकीकृत-राक्षसम्}
{रामायण-महामाला-रत्नं वन्देऽनिलात्मजम्}
\end{minipage}

\twolineshloka
{अञ्जनानन्दनं वीरं जानकीशोकनाशनम्}
{कपीशमक्षहन्तारं वन्दे लङ्काभयङ्करम्}

\twolineshloka
{उल्लङ्घ्य सिन्धोः सलिलं सलीलं यः शोकवह्निं जनकात्मजायाः}
{आदाय तेनैव ददाह लङ्कां नमामि तं प्राञ्जलिराञ्जनेयम्}

\twolineshloka
{आञ्जनेयमतिपाटलाननं काञ्चनाद्रि-कमनीय-विग्रहम्}
{पारिजात-तरुमूल-वासिनं भावयामि पवमान-नन्दनम्}

\twolineshloka
{यत्र यत्र रघुनाथकीर्तनं तत्र तत्र कृतमस्तकाञ्जलिम्}
{बाष्पवारिपरिपूर्णलोचनं मारुतिं नमत राक्षसान्तकम्}

\twolineshloka
{मनोजवं मारुततुल्यवेगं जितेन्द्रियं बुद्धिमतां वरिष्ठम्}
{वातात्मजं वानरयूथमुख्यं श्रीरामदूतं शिरसा नमामि}

\mbox{}\\
\resetShloka
\dnsub{श्री~रामायणप्रार्थना}

\fourlineindentedshloka
{यः कर्णाञ्जलिसम्पुटैरहरहः सम्यक् पिबत्यादरात्}
{वाल्मीकेर्वदनारविन्दगलितं रामायणाख्यं मधु}
{जन्म-व्याधि-जरा-विपत्ति-मरणैरत्यन्त-सोपद्रवम्}
{संसारं स विहाय गच्छति पुमान् विष्णोः पदं शाश्वतम्}

\twolineshloka
{तदुपगत-समास-सन्धियोगं सममधुरोपनतार्थ-वाक्यबद्धम्}
{रघुवरचरितं मुनिप्रणीतं दशशिरसश्च वधं निशामयध्वम्}

\twolineshloka
{वाल्मीकि-गिरिसम्भूता रामसागरगामिनी}
{पुनातु भुवनं पुण्या रामायणमहानदी}

\twolineshloka
{श्लोकसारजलाकीर्णं सर्गकल्लोलसङ्कुलम्}
{काण्डग्राहमहामीनं वन्दे रामायणार्णवम्}

\twolineshloka
{वेदवेद्ये परे पुंसि जाते दशरथात्मजे}
{वेदः प्राचेतसादासीत् साक्षाद्रामायणात्मना}

\mbox{}\\
\resetShloka
\dnsub{श्री~रामध्यानम्}

\fourlineindentedshloka
{वैदेहीसहितं सुरद्रुमतले हैमे महामण्डपे}
{मध्ये पुष्पकमासने मणिमये वीरासने सुस्थितम्}
{अग्रे वाचयति प्रभञ्जनसुते तत्त्वं मुनिभ्यः परम्}
{व्याख्यान्तं भरतादिभिः परिवृतं रामं भजे श्यामलम्}

\fourlineindentedshloka
{वामे भूमिसुता पुरश्च हनुमान् पश्चात् सुमित्रासुतः}
{शत्रुघ्नो भरतश्च पार्श्वदलयोर्वाय्वादिकोणेषु च}
{सुग्रीवश्च विभीषणश्च युवराट् तारासुतो जाम्बवान्}
{मध्ये नीलसरोजकोमलरुचिं रामं भजे श्यामलम्}

\twolineshloka
{नमोऽस्तु रामाय सलक्ष्मणाय देव्यै च तस्यै जनकात्मजायै}
{नमोऽस्तु रुद्रेन्द्रयमानिलेभ्यो नमोऽस्तु चन्द्रार्कमरुद्गणेभ्यः}

\clearpage
\resetShloka
\dnsub{श्रीमद्रामायणम्}
\dnsub{बालकाण्डः}
\dnsub{अथ प्रथमोऽध्यायः}

\twolineshloka
{तपः स्वाध्यायनिरतं तपस्वी वाग्विदां वरम्}
{नारदं परिपप्रच्छ वाल्मीकिर्मुनिपुङ्गवम्}%1

\twolineshloka
{को न्वस्मिन् साम्प्रतं लोके गुणवान् कश्च वीर्यवान्}
{धर्मज्ञश्च कृतज्ञश्च सत्यवाक्यो दृढव्रतः}%2

\twolineshloka
{चारित्रेण च को युक्तः सर्वभूतेषु को हितः}
{विद्वान् कः कः समर्थश्च कश्चैकप्रियदर्शनः}%3

\twolineshloka
{आत्मवान् को जितक्रोधो मतिमान् कोऽनसूयकः}
{कस्य बिभ्यति देवाश्च जातरोषस्य संयुगे}%4

\twolineshloka
{एतदिच्छाम्यहं श्रोतुं परं कौतूहलं हि मे}
{महर्षे त्वं समर्थोऽसि ज्ञातुमेवंविधं नरम्}%5

\twolineshloka
{श्रुत्वा चैतत्त्रिलोकज्ञो वाल्मीकेर्नारदो वचः}
{श्रूयतामिति चऽऽमन्त्र्य प्रहृष्टो वाक्यमब्रवीत्}%6

\twolineshloka
{बहवो दुर्लभाश्चैव ये त्वया कीर्तिता गुणाः}
{मुने वक्ष्याम्यहं बुद्‌ध्वा तैर्युक्तः श्रूयतां नरः}%7

\twolineshloka
{इक्ष्वाकुवंशप्रभवो रामो नाम जनैः श्रुतः}
{नियतात्मा महावीर्यो द्युतिमान् धृतिमान् वशी}%8

\twolineshloka
{बुद्धिमान् नीतिमान् वाग्मी श्रीमान् शत्रुनिबर्हणः}
{विपुलांसो महाबाहुः कम्बुग्रीवो महाहनुः}%9

\twolineshloka
{महोरस्को महेष्वासो गूढजत्रुररिन्दमः}
{आजानुबाहुः सुशिराः सुललाटः सुविक्रमः}%10

\twolineshloka
{समः समविभक्ताङ्गः स्निग्धवर्णः प्रतापवान्}
{पीनवक्षा विशालाक्षो लक्ष्मीवान् शुभलक्षणः}%11

\twolineshloka
{धर्मज्ञः सत्यसन्धश्च प्रजानां च हिते रतः}
{यशस्वी ज्ञानसम्पन्नः शुचिर्वश्यः समाधिमान्}%12

\twolineshloka
{प्रजापतिसमः श्रीमान् धाता रिपुनिषूदनः}
{रक्षिता जीवलोकस्य धर्मस्य परिरक्षिता}%13

\twolineshloka
{रक्षिता स्वस्य धर्मस्य स्वजनस्य च रक्षिता}
{वेदवेदाङ्गतत्त्वज्ञो धनुर्वेदे च निष्ठितः}%14

\twolineshloka
{सर्वशास्त्रार्थतत्त्वज्ञो स्मृतिमान् प्रतिभानवान्}
{सर्वलोकप्रियः साधुरदीनात्मा विचक्षणः}%15

\twolineshloka
{सर्वदाऽभिगतः सद्भिः समुद्र इव सिन्धुभिः}
{आर्यः सर्वसमश्चैव सदैकप्रियदर्शनः}%16

\twolineshloka
{स च सर्वगुणोपेतः कौसल्यानन्दवर्धनः}
{समुद्र इव गाम्भीर्ये धैर्येण हिमवानिव}%17

\twolineshloka
{विष्णुना सदृशो वीर्ये सोमवत् प्रियदर्शनः}
{कालाग्निसदृशः क्रोधे क्षमया पृथिवीसमः}%18

\twolineshloka
{धनदेन समस्त्यागे सत्ये धर्म इवापरः}
{तमेवं गुणसम्पन्नं रामं सत्यपराक्रमम्}%19

\twolineshloka
{ज्येष्ठं श्रेष्ठगुणैर्युक्तं प्रियं दशरथः सुतम्}
{प्रकृतीनां हितैर्युक्तं प्रकृतिप्रियकाम्यया}%20

\twolineshloka
{यौवराज्येन संयोक्तुम् ऐच्छत् प्रीत्या महीपतिः}
{तस्याभिषेकसम्भारान् दृष्ट्वा भार्याऽथ कैकेयी}%21

\twolineshloka
{पूर्वं दत्तवरा देवी वरमेनमयाचत}
{विवासनं च रामस्य भरतस्याभिषेचनम्}%22

\twolineshloka
{स सत्यवचनाद्राजा धर्मपाशेन संयतः}
{विवासयामास सुतं रामं दशरथः प्रियम्}%23

\twolineshloka
{स जगाम वनं वीरः प्रतिज्ञामनुपालयन्}
{पितुर्वचननिर्देशात् कैकेय्याः प्रियकारणात्}%24

\twolineshloka
{तं व्रजन्तं प्रियो भ्राता लक्ष्मणोऽनुजगाम ह}
{स्नेहाद्विनयसम्पन्नः सुमित्रानन्दवर्धनः}%25

\twolineshloka
{भ्रातरं दयितो भ्रातुः सौभ्रात्रमनुदर्शयन्}
{रामस्य दयिता भार्या नित्यं प्राणसमाहिता}%26

\twolineshloka
{जनकस्य कुले जाता देवमायेव निर्मिता}
{सर्वलक्षणसम्पन्ना नारीणामुत्तमा वधूः}%27

\twolineshloka
{सीताऽप्यनुगता रामं शशिनं रोहिणी यथा}
{पौरैरनुगतो दूरं पित्रा दशरथेन च}%28

\twolineshloka
{शृङ्गवेरपुरे सूतं गङ्गाकूले व्यसर्जयत्}
{गुहमासाद्य धर्मात्मा निषादाधिपतिं प्रियम्}%29

\twolineshloka
{गुहेन सहितो रामो लक्ष्मणेन च सीतया}
{ते वनेन वनं गत्वा नदीस्तीर्त्वा बहूदकाः}%30

\twolineshloka
{चित्रकूटमनुप्राप्य भरद्वाजस्य शासनात्}
{रम्यमावसथं कृत्वा रममाणा वने त्रयः}%31

\twolineshloka
{देवगन्धर्वसङ्काशास्तत्र ते न्यवसन् सुखम्}
{चित्रकूटं गते रामे पुत्रशोकातुरस्तथा}%32

\twolineshloka
{राजा दशरथः स्वर्गं जगाम विलपन् सुतम्}
{मृते तु तस्मिन् भरतो वसिष्ठप्रमुखैर्द्विजैः}%33

\twolineshloka
{नियुज्यमानो राज्याय नैच्छद्राज्यं महाबलः}
{स जगाम वनं वीरो रामपादप्रसादकः}%34

\twolineshloka
{गत्वा तु स महात्मानं रामं सत्यपराक्रमम्}
{अयाचत् भ्रातरं रामम् आर्यभावपुरस्कृतः}%35

\twolineshloka
{त्वमेव राजा धर्मज्ञ इति रामं वचोऽब्रवीत्}
{रामोऽपि परमोदारः सुमुखः सुमहायशाः}%36

\twolineshloka
{न चेच्छत् पितुरादेशात् राज्यं रामो महाबलः}
{पादुके चास्य राज्याय न्यासं दत्वा पुनः पुनः}%37

\twolineshloka
{निवर्तयामास ततो भरतं भरताग्रजः}
{स काममनवाप्यैव रामपादावुपस्पृशन्}%38

\twolineshloka
{नन्दिग्रामेऽकरोद्राज्यं रामागमनकाङ्क्षया}
{गते तु भरते श्रीमान् सत्यसन्धो जितेन्द्रियः}%39

\twolineshloka
{रामस्तु पुनरालक्ष्य नागरस्य जनस्य च}
{तत्रऽऽगमनमेकाग्रे दण्डकान् प्रविवेश ह}%40

\twolineshloka
{प्रविश्य तु महारण्यं रामो राजीवलोचनः}
{विराधं राक्षसं हत्वा शरभङ्गं ददर्श ह}%41

\twolineshloka
{सुतीक्ष्णं चाप्यगस्त्यं च अगस्त्यभ्रातरं तथा}
{अगस्त्यवचनाच्चैव जग्राहैन्द्रं शरासनम्}%42

\twolineshloka
{खड्गं च परमप्रीतस्तूणी चाक्षयसायकौ}
{वसतस्तस्य रामस्य वने वनचरैः सह}%43

\twolineshloka
{ऋषयोऽभ्यागमन् सर्वे वधायासुररक्षसाम्}
{स तेषां प्रति शुश्राव राक्षसानां तथा वने}%44

\twolineshloka
{प्रतिज्ञातश्च रामेण वधः संयति रक्षसाम्}
{ऋषीणामग्निकल्पानां दण्डकारण्यवासिनाम्}%45

\twolineshloka
{तेन तत्रैव वसता जनस्थाननिवासिनी}
{विरूपिता शूर्पणखा राक्षसी कामरूपिणी}%46

\twolineshloka
{ततः शूर्पणखावाक्यादुद्युक्तान् सर्वराक्षसान्}
{खरं त्रिशिरसं चैव दूषणं चैव राक्षसम्}%47

\twolineshloka
{निजघान रणे रामस्तेषां चैव पदानुगान्}
{वने तस्मिन् निवसता जनस्थाननिवासिनाम्}%48

\twolineshloka
{रक्षसां निहतान्यासन् सहस्राणि चतुर्दश}
{ततो ज्ञातिवधं श्रुत्वा रावणः क्रोधमूर्छितः}%49

\twolineshloka
{सहायं वरयामास मारीचं नाम राक्षसम्}
{वार्यमाणः सुबहुशो मारीचेन स रावणः}%50

\twolineshloka
{न विरोधो बलवता क्षमो रावण तेन ते}
{अनादृत्य तु तद्वाक्यं रावणः कालचोदितः}%51

\twolineshloka
{जगाम सहमारीचस्तस्यऽऽश्रमपदं तदा}
{तेन मायाविना दूरमपवाह्य नृपात्मजौ}%52

\twolineshloka
{जहार भार्यां रामस्य गृध्रं हत्वा जटायुषम्}
{गृध्रं च निहतं दृष्ट्वा हृतां श्रुत्वा च मैथिलीम्}%53

\twolineshloka
{राघवः शोकसन्तप्तो विललापऽऽकुलेन्द्रियः}
{ततस्तेनैव शोकेन गृध्रं दग्ध्वा जटायुषम्}%54

\twolineshloka
{मार्गमाणो वने सीतां राक्षसं सन्ददर्श ह}
{कबन्धं नाम रूपेण विकृतं घोरदर्शनम्}%55

\twolineshloka
{तं निहत्य महाबाहुर्ददाह स्वर्गतश्च सः}
{स चास्य कथयामास शबरीं धर्मचारिणीम्}%56

\twolineshloka
{श्रमणीं धर्मनिपुणाम् अभिगच्छेति राघव}
{सोऽभ्यगच्छन् महातेजाः शबरीं शत्रुसूदनः}%57

\twolineshloka
{शबर्या पूजितः सम्यग्रामो दशरथात्मजः}
{पम्पातीरे हनुमता सङ्गतो वानरेण ह}%58

\twolineshloka
{हनुमद्वचनाच्चैव सुग्रीवेण समागतः}
{सुग्रीवाय च तत्सर्वं शंसद्रामो महाबलः}%59

\twolineshloka
{आदितस्तत् यथा वृत्तं सीतायाश्च विशेषतः}
{सुग्रीवश्चापि तत्सर्वं श्रुत्वा रामस्य वानरः}%60

\twolineshloka
{चकार सख्यं रामेण प्रीतश्चैवाग्निसाक्षिकम्}
{ततो वानरराजेन वैरानुकथनं प्रति}%61

\twolineshloka
{रामायऽऽवेदितं सर्वं प्रणयाद्दुःखितेन च}
{प्रतिज्ञातं च रामेण तदा वालिवधं प्रति}%62

\twolineshloka
{वालिनश्च बलं तत्र कथयामास वानरः}
{सुग्रीवः शङ्कितश्चासीन्नित्यं वीर्येण राघवे}%63

\twolineshloka
{राघवः प्रत्ययार्थं तु दुन्दुभेः कायमुत्तमम्}
{दर्शयामास सुग्रीवः महापर्वतसन्निभम्}%64

\twolineshloka
{उत्स्मयित्वा महाबाहुः प्रेक्ष्य चास्ति महाबलः}
{पादाङ्गुष्ठेन चिक्षेप सम्पूर्णं दशयोजनम्}%65

\twolineshloka
{बिभेद च पुनः सालान् सप्तैकेन महेषुणा}
{गिरिं रसातलं चैव जनयन् प्रत्ययं तदा}%66

\twolineshloka
{ततः प्रीतमनास्तेन विश्वस्तः स महाकपिः}
{किष्किन्धां रामसहितो जगाम च गुहां तदा}%67

\twolineshloka
{ततोऽगर्जद्धरिवरः सुग्रीवो हेमपिङ्गलः}
{तेन नादेन महता निर्जगाम हरीश्वरः}%68

\twolineshloka
{अनुमान्य तदा तारां सुग्रीवेण समागतः}
{निजघान च तत्रैनं शरेणैकेन राघवः}%69

\twolineshloka
{ततः सुग्रीववचनाद्धत्वा वालिनमाहवे}
{सुग्रीवमेव तद्राज्ये राघवः प्रत्यपादयत्}%70

\twolineshloka
{स च सर्वान् समानीय वानरान् वानरर्षभः}
{दिशः प्रस्थापयामास दिदृक्षुर्जनकात्मजाम्}%71

\twolineshloka
{ततो गृध्रस्य वचनात्सम्पातेर्हनुमान् बली}
{शतयोजनविस्तीर्णं पुप्लुवे लवणार्णवम्}%72

\twolineshloka
{तत्र लङ्कां समासाद्य पुरीं रावणपालिताम्}
{ददर्श सीतां ध्यायन्तीमशोकवनिकां गताम्}%73

\twolineshloka
{निवेदयित्वाऽभिज्ञानं प्रवृत्तिं च निवेद्य च}
{समाश्वास्य च वैदेहीं मर्दयामास तोरणम्}%74

\twolineshloka
{पञ्च सेनाग्रगान् हत्वा सप्त मन्त्रिसुतानपि}
{शूरमक्षं च निष्पिष्य ग्रहणं समुपागमत्}%75

\twolineshloka
{अस्त्रेणोन्मुहमात्मानं ज्ञात्वा पैतामहाद्वरात्}
{मर्षयन् राक्षसान् वीरो यन्त्रिणस्तान् यदृच्छया}%76

\twolineshloka
{ततो दग्ध्वा पुरीं लङ्काम् ऋते सीतां च मैथिलीम्}
{रामाय प्रियमाख्यातुं पुनरायान् महाकपिः}%77

\twolineshloka
{सोऽभिगम्य महात्मानं कृत्वा रामं प्रदक्षिणम्}
{न्यवेदयदमेयात्मा दृष्टा सीतेति तत्त्वतः}%78

\twolineshloka
{ततः सुग्रीवसहितो गत्वा तीरं महोदधेः}
{समुद्रं क्षोभयामास शरैरादित्यसन्निभैः}%79

\twolineshloka
{दर्शयामास चऽऽत्मानं समुद्रः सरितां पतिः}
{समुद्रवचनाच्चैव नलं सेतुमकारयत्}%80

\twolineshloka
{तेन गत्वा पुरीं लङ्कां हत्वा रावणमाहवे}
{रामः सीतामनुप्राप्य परां व्रीडामुपागमत्}%81

\twolineshloka
{तामुवाच ततो रामः परुषं जनसंसदि}
{अमृष्यमाणा सा सीता विवेश ज्वलनं सती}%82

\twolineshloka
{ततोऽग्निवचनात् सीतां ज्ञात्वा विगतकल्मषाम्}
{कर्मणा तेन महता त्रैलोक्यं सचराचरम्}%83

\twolineshloka
{सदेवर्षिगणं तुष्टं राघवस्य महात्मनः}
{बभौ रामः सम्प्रहृष्टः पूजितः सर्वदेवतैः}%84

\twolineshloka
{अभ्यषिच्य च लङ्कायां राक्षसेन्द्रं विभीषणम्}
{कृतकृत्यस्तदा रामो विज्वरः प्रमुमोद ह}%85

\twolineshloka
{देवताभ्यो वरान् प्राप्य समुत्थाप्य च वानरान्}
{अयोध्यां प्रस्थितो रामः पुष्पकेण सुहृद्-वृतः}%86

\twolineshloka
{भरद्वाजाश्रमं गत्वा रामः सत्यपराक्रमः}
{भरतस्यान्तिकं रामो हनूमन्तं व्यसर्जयत्}%87

\twolineshloka
{पुनराख्यायिकां जल्पन् सुग्रीवसहितस्तदा}
{पुष्पकं तत् समारुह्य नन्दिग्रामं ययौ तदा}%88

\twolineshloka
{नन्दिग्रामे जटां हित्वा भ्रातृभिः सहितोऽनघः}
{रामः सीतामनुप्राप्य राज्यं पुनरवाप्तवान्}%89

\twolineshloka
{प्रहृष्टमुदितो लोकस्तुष्टः पुष्टः सुधार्मिकः}
{निरामयो ह्यरोगश्च दुर्भिक्षभयवर्जितः}%90

\twolineshloka
{न पुत्रमरणं केचिद्-द्रक्ष्यन्ति पुरुषाः क्वचित्}
{नार्यश्चाविधवा नित्यं भविष्यन्ति पतिव्रताः}%91

\twolineshloka
{न चाग्निजं भयं किञ्चित् नाप्सु मज्जन्ति जन्तवः}
{न वातजं भयं किञ्चित् नापि ज्वरकृतं तथा}%92

\twolineshloka
{न चापि क्षुद्भयं तत्र न तस्करभयं तथा}
{नगराणि च राष्ट्राणि धनधान्ययुतानि च}%93

\twolineshloka
{नित्यं प्रमुदिताः सर्वे यथा कृतयुगे तथा}
{अश्वमेधशतैरिष्ट्वा तथा बहुसुवर्णकैः}%94

\twolineshloka
{गवां कोट्ययुतं दत्वा विद्वद्‌भ्यो विधिपूर्वकम्}
{असङ्ख्येयं धनं दत्वा ब्राह्मणेभ्यो महायशाः}%95

\twolineshloka
{राजवंशान् शतगुणान् स्थापयिष्यति राघवः}
{चातुर्वर्ण्यं च लोकेऽस्मिन् स्वे स्वे धर्मे नियोक्ष्यति}%96

\twolineshloka
{दशवर्षसहस्राणि दशवर्षशतानि च}
{रामो राज्यमुपासित्वा ब्रह्मलोकं गमिष्यति}%97

\twolineshloka
{इदं पवित्रं पापघ्नं पुण्यं वेदैश्च सम्मितम्}
{यः पठेद्रामचरितं सर्वपापैः प्रमुच्यते}%98

\twolineshloka
{एतदाख्यानमायुष्यं पठन् रामायणं नरः}
{सपुत्रपौत्रः सगणः प्रेत्य स्वर्गे महीयते}%99

\fourlineindentedshloka
{पठन् द्विजो वागृषभत्वमीयात्}
{स्यात् क्षत्रियो भूमिपतित्वमीयात्}
{वणिग्जनः पण्यफलत्वमीयात्}
{जनश्च शूद्रोऽपि महत्त्वमीयात्}%100
{॥इति श्रीमद्वाल्मीकिरामायणे आदिकाव्ये बालकाण्डे प्रथमः सर्गः॥}

\mbox{}\\
\resetShloka
\dnsub{मङ्गलश्लोकाः}
\fourlineindentedshloka
{स्वस्ति प्रजाभ्यः परिपालयन्ताम्}
{न्यायेन मार्गेण महीं महीशाः}
{गोब्राह्मणेभ्यः शुभमस्तु नित्यम्}
{लोकाः समस्ताः सुखिनो भवन्तु}

\twolineshloka
{काले वर्षतु पर्जन्यः पृथिवी सस्यशालिनी}
{देशोऽयं क्षोभरहितो ब्राह्मणाः सन्तु निर्भयाः}

\twolineshloka
{अपुत्राः पुत्रिणः सन्तु पुत्रिणः सन्तु पौत्रिणः}
{अधनाः सधनाः सन्तु जीवन्तु शरदां शतम्}

\twolineshloka
{चरितं रघुनाथस्य शतकोटि-प्रविस्तरम्}
{एकैकमक्षरं पुंसां महापातकनाशनम्}

\twolineshloka
{शृण्वन् रामायणं भक्त्या यः पादं पदमेव वा}
{स याति ब्रह्मणः स्थानं ब्रह्मणा पूज्यते सदा}

\twolineshloka
{रामाय रामभद्राय रामचन्द्राय वेधसे}
{रघुनाथाय नाथाय सीतायाः पतये नमः}

\twolineshloka
{यन्मङ्गलं सहस्राक्षे सर्वदेवनमस्कृते}
{वृत्रनाशे समभवत् तत्ते भवतु मङ्गलम्}

\twolineshloka
{यन्मङ्गलं सुपर्णस्य विनताऽकल्पयत् पुरा}
{अमृतं प्रार्थयानस्य तत्ते भवतु मङ्गलम्}

\twolineshloka
{अमृतोत्पादने दैत्यान् घ्नतो वज्रधरस्य यत्}
{अदितिर्मञ्गलं प्रादात् तत्ते भवतु मङ्गलम्}

\twolineshloka
{त्रीन् विक्रमान् प्रक्रमतो विष्णोरमिततेजसः}
{यदासीन्मङ्गलं राम तत्ते भवतु मङ्गलम्}

\twolineshloka
{ऋषयः सागरा द्वीपा वेदा लोका दिशश्च ते}
{मङ्गलानि महाबाहो दिशन्तु तव सर्वदा}

\twolineshloka
{मङ्गलं कोसलेन्द्राय महनीयगुणाब्धये}
{चक्रवर्तितनूजाय सार्वभौमाय मङ्गलम्}

\twolineshloka*
{कायेन वाचा मनसेन्द्रियैर्वा बुद्‌ध्याऽऽत्मना वा प्रकृतेः स्वभावात्}
{करोमि यद्यत् सकलं परस्मै नारायणायेति समर्पयामि}



\closesection
\end{center}
