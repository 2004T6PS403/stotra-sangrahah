% !TeX program = XeLaTeX
% !TeX root = adhyatma-ramayana-stotras.tex
\setcounter{page}{0}
\pagenumbering{arabic}
\sectionmark{\mbox{}}
\renewcommand{\chaptermark}[1]{%
\markboth{\large #1}{}}

\begin{center}
\chapt{बालकाण्डः}
% !TeX program = XeLaTeX
% !TeX root = ../../shloka.tex

\sect{राम हृदयम्}

\fourlineindentedshloka
{यः पृथिवीभरवारणाय दिविजैः सम्प्रार्थितश्चिन्मयः}
{सञ्जातः पृथिवीतले रविकुले मायामनुष्योऽव्ययः}
{निश्चक्रं हतराक्षसः पुनरगाद् ब्रह्मत्वमाद्यं स्थिराम्}
{कीर्तिं पापहरां विधाय जगतां तं जानकीशं भजे} %1-1

\fourlineindentedshloka
{विश्वोद्भवस्थितिलयादिषु हेतुमेकम्}
{मायाश्रयं विगतमायमचिन्त्यमूर्तिम्}
{आनन्दसान्द्रममलं निजबोधरूपम्}
{सीतापतिं विदिततत्त्वमहं नमामि} %1-2

\fourlineindentedshloka
{पठन्ति ये नित्यमनन्यचेतसः}
{शृण्वन्ति चाध्यात्मिकसंज्ञितं शुभम्}
{रामायणं सर्वपुराणसम्मतम्}
{निर्धूतपापा हरिमेव यान्ति ते} %1-3

\fourlineindentedshloka
{अध्यात्मरामायणमेव नित्यम्}
{पठेद्यदीच्छेद्भवबन्धमुक्तिम्}
{गवां सहस्रायुतकोटिदानात्}
{फलं लभेद्यः शृणुयात्स नित्यम्} %1-4

\twolineshloka
{पुरारिगिरिसम्भूता श्रीरामार्णवसङ्गता}
{अध्यात्मरामगङ्गेयं पुनाति भुवनत्रयम्} %1-5

\fourlineindentedshloka
{कैलासाग्रे कदाचिद्रविशतविमले मन्दिरे रत्नपीठे}
{संविष्टं ध्याननिष्ठं त्रिनयनमभयं सेवितं सिद्धसन्घैः}
{देवी वामाङ्कसंस्था गिरिवरतनया पार्वती भक्तिनम्रा}
{प्राहेदं देवमीशं सकलमलहरं वाक्यमानन्दकन्दम्} %1-6

\uvacha{पार्वत्युवाच}

\fourlineindentedshloka
{नमोऽस्तु ते देव जगन्निवास}
{सर्वात्मदृक् त्वं परमेश्वरोऽसि}
{पृच्छामि तत्त्वं पुरुषोत्तमस्य}
{सनातनं त्वं च सनातनोऽसि} %1-7

\fourlineindentedshloka
{गोप्यं यदत्यन्तमनन्यवाच्यम्}
{वदन्ति भक्तेषु महानुभावाः}
{तदप्यहोऽहं तव देव भक्ता}
{प्रियोऽसि मे त्वं वद यत्तु पृष्टम्} %1-8

\fourlineindentedshloka
{ज्ञानं सविज्ञानमथानुभक्तिवैराग्ययुक्तम्}
{च मितं विभास्वत्}
{जानाम्यहं योषिदपि त्वदुक्तम्}
{यथा तथा ब्रूहि तरन्ति येन} %1-9

\fourlineindentedshloka
{पृच्छामि चान्यच्च परं रहस्यम्}
{तदेव चाग्रे वद वारिजाक्ष}
{श्रीरामचन्द्रेऽखिललोकसारे}
{भक्तिर्दृढा नौर्भवति प्रसिद्धा} %1-10

\fourlineindentedshloka
{भक्तिः प्रसिद्धा भवमोक्षणाय}
{नान्यत्ततः साधनमस्ति किञ्चित्}
{तथाऽपि हृत्संशयबन्धनं मे}
{विभेत्तुमर्हस्यमलोक्तिभिस्त्वम्} %1-11

\fourlineindentedshloka
{वदन्ति रामं परमेकमाद्यम्}
{निरस्तमायागुणसम्प्रवाहम्}
{भजन्ति चाहर्निशमप्रमत्ताः}
{परं पदं यान्ति तथैव सिद्धाः} %1-12

\fourlineindentedshloka
{वदन्ति केचित्परमोऽपि रामः}
{स्वाविद्यया संवृतमात्मसंज्ञम्}
{जानाति नात्मानमतः परेण}
{सम्बोधितो वेद परात्मतत्त्वम्} %1-13

\fourlineindentedshloka
{यदि स्म जानाति कुतो विलापः}
{सीताकृतेऽनेन कृतः परेण}
{जानाति नैवं यदि केन सेव्यः}
{समो हि सर्वैरपि जीवजातैः} %1-14

\twolineshloka
{अत्रोत्तरं किं विदितं भवद्भिः}
{तद्ब्रूत मे संशयभेदि वाक्यम्} %1-15

\uvacha{श्रीमहादेव उवाच}

\fourlineindentedshloka
{धन्यासि भक्तासि परात्मनस्त्वम्}
{यज्ज्ञातुमिच्छा तव रामतत्त्वम्}
{पुरा न केनाप्यभिचोदितोऽहम्}
{वक्तुं रहस्यं परमं निगूढम्} %1-16

\fourlineindentedshloka
{त्वयाऽद्य भक्त्या परिनोदितोऽहम्}
{वक्ष्ये नमस्कृत्य रघूत्तमं ते}
{रामः परात्मा प्रकृतेरनादि-}
{रानन्द एकः पुरुषोत्तमो हि} %1-17

\fourlineindentedshloka
{स्वमायया कृत्स्नमिदं हि सृष्ट्वा}
{नभोवदन्तर्बहिरास्थितो यः}
{सर्वान्तरस्थोऽपि निगूढ आत्मा}
{स्वमायया सृष्टमिदं विचष्टे} %1-18

\fourlineindentedshloka
{जगन्ति नित्यं परितो भ्रमन्ति}
{यत्सन्निधौ चुम्बकलोहवद्धि}
{एतन्न जानन्ति विमूढचित्ताः}
{स्वाविद्यया संवृतमानसा ये} %1-19

\fourlineindentedshloka
{स्वाज्ञानमप्यात्मनि शुद्धबुद्धे}
{स्वारोपयन्तीह निरस्तमाये}
{संसारमेवानुसरन्ति ते वै}
{पुत्रादिसक्ताः पुरुकर्मयुक्ताः} %1-20

\fourlineindentedshloka
{यथाऽप्रकाशो न तु विद्यते रवौ}
{ज्योतिःस्वभावे परमेश्वरे तथा}
{विशुद्धविज्ञानघने रघूत्तमेऽविद्या}
{कथं स्यात्परतः परात्मनि} %1-21

\fourlineindentedshloka
{यथा हि चाक्ष्णा भ्रमता गृहादिकम्}
{विनष्टदृष्टेर्भ्रमतीव दृश्यते}
{तथैव देहेन्द्रियकर्तुरात्मनः}
{कृते परेऽध्यस्य जनो विमुह्यति} %1-22

\fourlineindentedshloka
{नाहो न रात्रिः सवितुर्यथा भवेत्}
{प्रकाशरूपाव्यभिचारतः क्वचित्}
{ज्ञानं तथाऽज्ञानमिदं द्वयं हरौ}
{रामे कथं स्थास्यति शुद्धचिद्घने} %1-23

\fourlineindentedshloka
{तस्मात्परानन्दमये रघूत्तमे}
{विज्ञानरूपे हि न विद्यते तमः}
{अज्ञानसाक्षिण्यरविन्दलोचने}
{मायाश्रयत्वान्न हि मोहकारणम्} %1-24

\twolineshloka
{अत्र ते कथयिष्यामि रहस्यमपि दुर्लभम्}
{सीताराममरुत्सूनुसंवादं मोक्षसाधनम्} %1-25

\twolineshloka
{पुरा रामायणे रामे रावणं देवकण्टकम्}
{हत्वा रणे रणश्लाघी सपुत्रबलवाहनम्} %1-26

\twolineshloka
{सीतया सह सुग्रीवलक्ष्मणाभ्यां समन्वितः}
{अयोध्यामगमद्रामो हनूमत्प्रमुखैर्वृतः} %1-27

\twolineshloka
{अभिषिक्तः परिवृतो वसिष्ठाद्यैर्महात्मभिः}
{सिंहासने समासीनः कोटिसूर्यसमप्रभः} %1-28

\twolineshloka
{दृष्ट्वा तदा हनूमन्तं प्राञ्जलिं पुरतः स्थितम्}
{कृतकार्यं निराकाङ्क्षं ज्ञानापेक्षं महामतिम्} %1-29

\twolineshloka
{रामः सीतामुवाचेदं ब्रूहि तत्त्वं हनूमते}
{निष्कल्मषोऽयं ज्ञानस्य पात्रं नो नित्यभक्तिमान्} %1-30

\twolineshloka
{तथेति जानकी प्राह तत्त्वं रामस्य निश्चितम्}
{हनूमते प्रपन्नाय सीता लोकविमोहिनी} %1-31

\uvacha{सीतोवाच}

\twolineshloka
{रामं विद्धि परं ब्रह्म सच्चिदानन्दमद्वयम्}
{सर्वोपाधिविनिर्मुक्तं सत्तामात्रमगोचरम्} %1-32

\twolineshloka
{आनन्दं निर्मलं शान्तं निर्विकारं निरञ्जनम्}
{सर्वव्यापिनमात्मानं स्वप्रकाशमकल्मषम्} %1-33

\twolineshloka
{मां विद्धि मूलप्रकृतिं सर्गस्थित्यन्तकारिणीम्}
{तस्य सन्निधिमात्रेण सृजामीदमतन्द्रिता} %1-34

\twolineshloka
{तत्सान्निध्यान्मया सृष्टं तस्मिन्नारोप्यतेऽबुधैः}
{अयोध्यानगरे जन्म रघुवंशेऽतिनिर्मले} %1-35

\twolineshloka
{विश्वामित्रसहायत्वं मखसंरक्षणं ततः}
{अहल्याशापशमनं चापभङ्गो महेशितुः} %1-36

\twolineshloka
{मत्पाणिग्रहणं पश्चाद्भार्गवस्य मदक्षयः}
{अयोध्यानगरे वासो मया द्वादशवार्षिकः} %1-37

\twolineshloka
{दण्डकारण्यगमनं विराधवध एव च}
{मायामारीचमरणं मायासीताहृतिस्तथा} %1-38

\twolineshloka
{जटायुषो मोक्षलाभः कबन्धस्य तथैव च}
{शबर्याः पूजनं पश्चात्सुग्रीवेण समागमः} %1-39

\twolineshloka
{वालिनश्च वधः पश्चात्सीतान्वेषणमेव च}
{सेतुबन्धश्च जलधौ लङ्कायाश्च निरोधनम्} %1-40

\twolineshloka
{रावणस्य वधो युद्धे सपुत्रस्य दुरात्मनः}
{विभीषणे राज्यदानं पुष्पकेण मया सह} %1-41

\threelineshloka
{अयोध्यागमनं पश्चाद्राज्ये रामाभिषेचनम्}
{एवमादीनि कर्माणि मयैवाचरितान्यपि}
{आरोपयन्ति रामेऽस्मिन्निर्विकारेऽखिलात्मनि} %1-42

\fourlineindentedshloka
{रामो न गच्छति न तिष्ठति नानुशोचत्याकाङ्क्षते}
{त्यजति नो न करोति किञ्चित्}
{आनन्दमूर्तिरचलः परिणामहीनो}
{मायागुणाननुगतो हि तथा विभाति} %1-43

\twolineshloka
{ततो रामः स्वयं प्राह हनूमन्तमुपस्थितम्}
{शृणु तत्त्वं प्रवक्ष्यामि ह्यात्मानात्मपरात्मनाम्} %1-44

\threelineshloka
{आकाशस्य यथा भेदस्त्रिविधो दृश्यते महान्}
{जलाशये महाकाशस्तदवच्छिन्न एव हि}
{प्रतिबिम्बाख्यमपरं दृश्यते त्रिविधं नभः} %1-45

\twolineshloka
{बुद्ध्यवच्छिन्नचैतन्यमेकं पूर्णमथापरम्}
{आभासस्त्वपरं बिम्बभूतमेवं त्रिधा चितिः} %1-46

\twolineshloka
{साभासबुद्धेः कर्तृत्वमविच्छिन्नेऽविकारिणि}
{साक्षिण्यारोप्यते भ्रान्त्या जीवत्वं च तथा बुधैः} %1-47

\twolineshloka
{आभासस्तु मृषा बुद्धिरविद्याकार्यमुच्यते}
{अविच्छिन्नं तु तद्ब्रह्म विच्छेदस्तु विकल्पतः} %1-48

\twolineshloka
{अविच्छिन्नस्य पूर्णेन एकत्वं प्रतिपाद्यते}
{तत्त्वमस्यादिवाक्यैश्च साभासस्याहमस्तथा} %1-49

\twolineshloka
{ऐक्यज्ञानं यदोत्पन्नं महावाक्येन चात्मनोः}
{तदाऽविद्या स्वकार्यैश्च नश्यत्येव न संशयः} %1-50

\threelineshloka
{एतद्विज्ञाय मद्भक्तो मद्भावायोपपद्यते}
{मद्भक्तिविमुखानां हि शास्त्रगर्तेषु मुह्यताम्}
{न ज्ञानं न च मोक्षः स्यात्तेषां जन्मशतैरपि} %1-51

\fourlineindentedshloka
{इदं रहस्यं हृदयं ममात्मनो}
{मयैव साक्षात्कथितं तवानघ}
{मद्भक्तिहीनाय शठाय न त्वया}
{दातव्यमैन्द्रादपि राज्यतोऽधिकम्} %1-52

\uvacha{श्रीमहादेव उवाच}

\twolineshloka
{एतत्तेऽभिहितं देवि श्रीरामहृदयं मया}
{अतिगुह्यतमं हृद्यं पवित्रं पापशोधनम्} %1-53

\twolineshloka
{साक्षाद्रामेण कथितं सर्ववेदान्तसङ्ग्रहम्}
{यः पठेत्सततं भक्त्या स मुक्तो नात्र संशयः} %1-54

\twolineshloka
{ब्रह्महत्यादि पापानि बहुजन्मार्जितान्यपि}
{नश्यन्त्येव न सन्देहो रामस्य वचनं यथा} %1-55

\fourlineindentedshloka
{योऽतिभ्रष्टोऽतिपापी परधनपरदारेषु नित्योद्यतो वा}
{स्तेयी ब्रह्मघ्नमातापितृवधनिरतो योगिवृन्दापकारी}
{यः सम्पूज्याभिरामं पठति च हृदयं रामचन्द्रस्य भक्त्या}
{योगीन्द्रैरप्यलभ्यं पदमिह लभते सर्वदेवैः स पूज्यम्} %1-56

{॥इति श्रीमदध्यात्मरामायणे उमामहेश्वरसंवादे बालकाण्डे
श्रीरामहृदयं नाम प्रथमः सर्गः॥}

% !TeX program = XeLaTeX
% !TeX root = ../../shloka.tex

\sect{कौसल्याकृत-रामस्तोत्रम्}

\uvacha{कौसल्योवाच}
\addtocounter{shlokacount}{19}
\twolineshloka
{देवदेव नमस्तेऽस्तु शङ्खचक्रगदाधर}
{परमात्माऽच्युतोऽनन्तः पूर्णस्त्वं पुरुषोत्तमः} %3-20

\twolineshloka
{वदन्त्यगोचरं वाचां बुद्ध्यादीनामतीन्द्रियम्}
{त्वां वेदवादिनः सत्तामात्रं ज्ञानैकविग्रहम्} %3-21

\twolineshloka
{त्वमेव मायया विश्वं सृजस्यवसि हंसि च}
{सत्त्वादिगुणसंयुक्तस्तुर्य एवामलः सदा} %3-22

\twolineshloka
{करोषीव न कर्ता त्वं गच्छसीव न गच्छसि}
{शृणोषि न शृणोषीव पश्यसीव न पश्यसि} %3-23

\twolineshloka
{अप्राणो ह्यमनाः शुद्ध इत्यादि श्रुतिरब्रवीत्}
{समः सर्वेषु भूतेषु तिष्ठन्नपि न लक्ष्यसे} %3-24

\twolineshloka
{अज्ञानध्वान्तचित्तानां व्यक्त एव सुमेधसाम्}
{जठरे तव दृश्यन्ते ब्रह्माण्डाः परमाणवः} %3-25

\twolineshloka
{त्वं ममोदरसम्भूत इति लोकान् विडम्बसे}
{भक्तेषु पारवश्यं ते दृष्टं मेऽद्य रघूत्तम} %3-26

\twolineshloka
{संसारसागरे मग्ना पतिपुत्रधनादिषु}
{भ्रमामि मायया तेऽद्य पादमूलमुपागता} %3-27

\twolineshloka
{देव त्वद्रूपमेतन्मे सदा तिष्ठतु मानसे}
{आवृणोतु न मां माया तव विश्वविमोहिनी} %3-28

\threelineshloka
{उपसंहर विश्वात्मन्नदो रूपमलौकिकम्}
{दर्शयस्व महानन्दबालभावं सुकोमलम्}
{ललितालिङ्गनालापैस्तरिष्याम्युत्कटं तमः} %3-29

\uvacha{श्रीभगवानुवाच}

\onelineshloka
{यद्यदिष्टं तवास्त्यम्ब तत्तद्भवतु नान्यथा} %3-30

\twolineshloka
{अहं तु ब्रह्मणा पूर्वं भूमेर्भारापनुत्तये}
{प्रार्थितो रावणं हन्तुं मानुषत्वमुपागतः} %3-31

\twolineshloka
{त्वया दशरथेनाहं तपसाराधितः पुरा}
{मत्पुत्रत्वाभिकाङ्क्षिण्या तथा कृतमनिन्दिते} %3-32

\twolineshloka
{रूपमेतत्त्वया दृष्टं प्राक्तनं तपसः फलम्}
{मद्दर्शनं विमोक्षाय कल्पते ह्यन्यदुर्लभम्} %3-33

\twolineshloka
{संवादमावयोर्यस्तु पठेद्वा शृणुयादपि}
{स याति मम सारूप्यं मरणे मत्स्मृतिं लभेत्} %3-34

{॥इति श्रीमदध्यात्मरामायणे उमामहेश्वरसंवादे बालकाण्डे
तृतीये सर्गे श्री कौसल्याविरचितं श्री~रामस्तोत्रं सम्पूर्णम्॥}

% !TeX program = XeLaTeX
% !TeX root = ../../shloka.tex
\sect{अहल्याकृत-रामस्तोत्रम्}
\uvacha{अहल्योवाच}

\fourlineindentedshloka
{अहो कृतार्थाऽस्मि जगन्निवास ते}
{पादाब्जसंलग्नरजः कणादहम्}
{स्पृशामि यत्पद्मजशङ्करादिभिः}
{विमृग्यते रन्धितमानसैः सदा}% १

\fourlineindentedshloka
{अहो विचित्रं तव राम चेष्टितम्}
{मनुष्यभावेन विमोहितं जगत्}
{चलस्यजस्रं चरणादिवर्जितः}
{सम्पूर्ण आनन्दमयोऽतिमायिकः}% २

\fourlineindentedshloka
{यत्पादपङ्कजपरागपवित्रगात्रा}
{भागीरथी भवविरिञ्चिमुखान्  पुनाति}
{साक्षात्स एव मम दृग्विषयो यदाऽऽस्ते}
{किं वर्ण्यते मम पुराकृतभागधेयम्}% ३

\fourlineindentedshloka
{मर्त्यावतारे मनुजाकृतिं हरिम्}
{रामाभिधेयं रमणीयदेहिनम्}
{धनुर्धरं पद्मविशाललोचनम्}
{भजामि नित्यं न परान्  भजिष्ये}% ४

\fourlineindentedshloka
{यत्पादपङ्कजरजः श्रुतिभिर्विमृग्यम्}
{यन्नाभिपङ्कजभवः कमलासनश्च}
{यन्नामसाररसिको भगवान्पुरारिः}
{तं  रामचन्द्रमनिशं हृदि भावयामि}% ५

\fourlineindentedshloka
{यस्यावतारचरितानि विरिञ्चिलोके}
{गायन्ति नारदमुखा भवपद्मजाद्याः}
{आनन्दजाश्रुपरिषिक्तकुचाग्रसीमा}
{वागीश्वरी च तमहं शरणं प्रपद्ये}% ६

\fourlineindentedshloka
{सोऽयं परात्मा पुरुषः पुराणः}
{एकः स्वयं ज्योतिरनन्त आद्यः}
{मायातनुं लोकविमोहनीयाम्}
{धत्ते परानुग्रह एष रामः}% ७

\fourlineindentedshloka
{अयं हि विश्वोद्भवसंयमानाम्}
{एकः  स्वमायागुणबिम्बितो यः}
{विरिञ्चिविष्ण्वीश्वरनामभेदान्}
{धत्ते स्वतन्त्रः परिपूर्ण आत्मा}% ८

\fourlineindentedshloka
{नमोऽस्तु ते राम तवाङ्घ्रिपङ्कजम्}
{श्रिया धृतं वक्षसि लालितं प्रियात्}
{आक्रान्तमेकेन जगत्त्रयं पुरा}
{ध्येयं मुनीन्द्रैरभिमानवर्जितैः}% ९

\twolineshloka
{जगतामादिभूतस्त्वं जगत्त्वं जगदाश्रयः}
{सर्वभूतेष्वसंयुक्त एको भाति भवान् परः}% १०

\twolineshloka
{ओङ्कारवाच्यस्त्वं राम वाचामविषयः पुमान्}
{वाच्यवाचकभेदेन भवानेव जगन्मयः}% ११

\twolineshloka
{कार्यकारणकर्तृत्वफलसाधनभेदतः}
{एको विभासि राम त्वं मायया बहुरूपया}% १२

\twolineshloka
{त्वन्मायामोहितधियस्त्वां न जानन्ति तत्त्वतः}
{मानुषं त्वाऽभिमन्यन्ते मायिनं परमेश्वरम्}% १३

\twolineshloka
{आकाशवत्त्वं सर्वत्र बहिरन्तर्गतोऽमलः}
{असङ्गो ह्यचलो नित्यः शुद्धो बुद्धः सदव्ययः}% १४

\twolineshloka
{योषिन्मूढाऽहमज्ञा ते तत्त्वं जाने कथं विभो}
{तस्मात्ते शतशो राम नमस्कुर्यामनन्यधीः}% १५

\twolineshloka
{देव मे यत्रकुत्रापि स्थिताया अपि सर्वदा}
{त्वत्पादकमले सक्ता भक्तिरेव  सदाऽस्तु मे}% १६

\twolineshloka
{नमस्ते पुरुषाध्यक्ष नमस्ते भक्तवत्सल}
{नमस्तेऽस्तु हृषीकेश नारायण नमोऽस्तु ते}% १७

\fourlineindentedshloka
{भवभयहरमेकं भानुकोटिप्रकाशम्}
{करधृतशरचापं कालमेघावभासम्}
{कनकरुचिरवस्त्रं रत्नवत्कुण्डलाढ्यम्}
{कमलविशदनेत्रं सानुजं राममीडे}% १८

\twolineshloka
{स्तुत्वैवं पुरुषं साक्षाद्राघवं पुरतः स्थितम्}
{परिक्रम्य प्रणम्याऽऽशु सानुज्ञाता ययौ पतिम्}% १९

\twolineshloka
{अहल्यया कृतं स्तोत्रं यः पठेद्भक्तिसंयुतः}
{स मुच्यतेऽखिलैः पापैः परं ब्रह्माधिगच्छति}% २०

\twolineshloka
{पुत्राद्यर्थे पठेद्भक्त्या रामं हृदि निधाय च}
{संवत्सरेण लभते वन्ध्या अपि सुपुत्रकम्}% २१

\onelineshloka
{सर्वान् कामानवाप्नोति रामचन्द्रप्रसादतः}% २२

\fourlineindentedshloka
{ब्रह्मघ्नो गुरुतल्पगोऽपि पुरुषः स्तेयी सुरापोऽपि वा}
{मातृभ्रातृविहिंसकोऽपि सततं भोगैकबद्धातुरः}
{नित्यं स्तोत्रमिदं जपन्  रघुपतिं भक्त्या हृदिस्थं स्मरन्}
{ध्यायन् मुक्तिमुपैति किं पुनरसौ स्वाचारयुक्तो नरः}% २३

॥इति श्रीमदध्यात्मरामायणे श्री अहल्याविरचितं श्री~रामचन्द्रस्तोत्रं सम्पूर्णम्॥

% !TeX program = XeLaTeX
% !TeX root = ../../shloka.tex

\sect{परशुरामकृत-रामस्तोत्रम्}

\uvacha{परशुराम उवाच}
\addtocounter{shlokacount}{28}
\twolineshloka
{स एव विष्णुस्त्वं राम जातोऽसि ब्रह्मणार्थितः}
{मयि स्थितं तु त्वत्तेजस्त्वयैव पुनराहृतम्} %7-29

\twolineshloka
{अद्य मे सफलं जन्म प्रतीतोऽसि मम प्रभो}
{ब्रह्मादिभिरलभ्यस्त्वं प्रकृतेः पारगो मतः} %7-30

\twolineshloka
{त्वयि जन्मादिषड्भावा न सन्त्यज्ञानसम्भवाः}
{निर्विकारोऽसि पूर्णस्त्वं गमनादिविवर्जितः} %7-31

\twolineshloka
{यथा जले फेनजालं धूमो वह्नौ तथा त्वयि}
{त्वदाधारा त्वद्विषया माया कार्यं सृजत्यहो} %7-32

\twolineshloka
{यावन्मायावृता लोकास्तावत्त्वां न विजानते}
{अविचारितसिद्धैषाऽविद्या विद्याविरोधिनी} %7-33

\twolineshloka
{अविद्याकृतदेहादिसङ्घाते प्रतिबिम्बिता}
{चिच्छक्तिर्जीवलोकेऽस्मिन् जीव इत्यभिधीयते} %7-34

\twolineshloka
{यावद्देहमनःप्राणबुद्ध्यादिष्वभिमानवान्}
{तावत्कर्तृत्वभोक्तृत्वसुखदुःखादिभाग्भवेत्} %7-35

\twolineshloka
{आत्मनःसंसृतिर्नास्ति बुद्धेर्ज्ञानं न जात्विति}
{अविवेकाद्द्वयं युङ्क्त्वा संसारीति प्रवर्तते} %7-36

\twolineshloka
{जडस्य चित्समायोगाच्चित्त्वं भूयाच्चितेस्तथा}
{जडसङ्गाज्जडत्वं हि जलाग्न्योर्मेलनं यथा} %7-37

\twolineshloka
{यावत्त्वत्पादभक्तानां सङ्गसौख्यं न विन्दति}
{तावत्संसारदुःखौघान्न निवर्तेन्नरः सदा} %7-38

\twolineshloka
{तत्सङ्गलब्धया भक्त्या यदा त्वां समुपासते}
{तदा माया शनैर्याति तानवं प्रतिपद्यते} %7-39

\twolineshloka
{ततस्त्वज्ज्ञानसम्पन्नः सद्गुरुस्तेन लभ्यते}
{वाक्यज्ञानं गुरोर्लब्ध्वा त्वत्प्रसादाद्विमुच्यते} %7-40

\twolineshloka
{तस्मात्त्वद्भक्तिहीनानां कल्पकोटिशतैरपि}
{न मुक्तिशङ्का विज्ञानशङ्का नैव सुखं तथा} %7-41

\twolineshloka
{अतस्त्वत्पादयुगले भक्तिर्मे जन्मजन्मनि}
{स्यात्त्वद्भक्तिमतां सङ्गोऽविद्या याभ्यां विनश्यति} %7-42

\twolineshloka
{लोके त्वद्भक्तिनिरतास्त्वद्धर्मामृतवर्षिणः}
{पुनन्ति लोकमखिलं किं पुनः स्वकुलोद्भवान्} %7-43

\twolineshloka
{नमोऽस्तु जगतां नाथ नमस्ते भक्तिभावन}
{नमः कारुणिकानन्त रामचन्द्र नमोऽस्तु ते} %7-44

\twolineshloka
{देव यद्यत्कृतं पुण्यं मया लोकजिगीषया}
{तत्सर्वं तव बाणाय भूयाद्राम नमोऽस्तु ते} %7-45

\twolineshloka
{ततः प्रसन्नो भगवान् श्रीरामः करुणामयः}
{प्रसन्नोऽस्मि तव ब्रह्मन् यत्ते मनसि वर्तते} %7-46

\twolineshloka
{दास्ये तदखिलं कामं मा कुरुष्वात्र संशयम्}
{ततः प्रीतेन मनसा भार्गवो राममब्रवीत्} %7-47

\twolineshloka
{यदि मेऽनुग्रहो राम तवास्ति मधुसूदन}
{त्वद्भक्तसङ्गस्त्वत्पादे दृढा भक्तिः सदास्तु मे} %7-48

\twolineshloka
{स्तोत्रमेतत्पठेद्यस्तु भक्तिहीनोऽपि सर्वदा}
{त्वद्भक्तिस्तस्य विज्ञानं भूयादन्ते स्मृतिस्तव} %7-49

\twolineshloka
{तथेति राघवेणोक्तः परिक्रम्य प्रणम्य तम्}
{पूजितस्तदनुज्ञातो महेन्द्राचलमन्वगात्} %7-50

{॥इति श्रीमदध्यात्मरामायणे उमामहेश्वरसंवादे बालकाण्डे
सप्तमे  सर्गे श्री परशुरामकृतं श्री~रामस्तोत्रं सम्पूर्णम्॥}


\closesection

\clearpage
\chapt{अयोध्याकाण्डः}
% !TeX program = XeLaTeX
% !TeX root = ../../shloka.tex

\sect{नारद-राम-संवादः}

\uvacha{श्री महादेव उवाच}

\twolineshloka
{एकदा सुखमासीनं रामं स्वान्तःपुराजिरे}
{सर्वाभरणसम्पन्नं रत्नसिंहासने स्थितम्} %1-1

\twolineshloka
{नीलोत्पलदलश्यामं कौस्तुभामुक्तकन्धरम्}
{सीतया रत्नदण्डेन चामरेणाथ वीजितम्} %1-2

\twolineshloka
{विनोदयन्तं ताम्बूलचर्वणादिभिरादरात्}
{नारदोऽवतरद्द्रष्टुमम्बराद्यत्र राघवः} %1-3

\twolineshloka
{शुद्धस्फटिकसङ्काशः शरच्चन्द्र इवामलः}
{अतर्कितमुपायातो नारदो दिव्यदर्शनः} %1-4

\twolineshloka
{तं दृष्ट्वा सहसोत्थाय रामः प्रीत्या कृताञ्जलिः}
{ननाम शिरसा भूमौ सीतया सह भक्तिमान्} %1-5

\threelineshloka
{उवाच नारदं रामः प्रीत्या परमया युतः}
{संसारिणां मुनिश्रेष्ठ दुर्लभं तव दर्शनम्}
{अस्माकं विषयासक्तचेतसां नितरां मुने} %1-6

\twolineshloka
{अवाप्तं मे पूर्वजन्मकृतपुण्यमहोदयैः}
{संसारिणाऽपि हि मुने लभ्यते सत्समागमः} %1-7

\twolineshloka
{अतस्त्वद्दर्शनादेव कृतार्थोऽस्मि मुनीश्वर}
{किं कार्यं ते मया कार्यं ब्रूहि तत्करवाणि भोः} %1-8

\twolineshloka
{अथ तं नारदोऽप्याह राघवं भक्तवत्सलम्}
{किं मोहयसि मां राम वाक्यैर्लोकानुसारिभिः} %1-9

\twolineshloka
{संसार्यहमिति प्रोक्तं सत्यमेतत्त्वया विभो}
{जगतामादिभूता या सा माया गृहिणी तव} %1-10

\twolineshloka
{त्वत्सन्निकर्षाज्जायन्ते तस्यां ब्रह्मादयः प्रजाः}
{त्वदाश्रया सदा भाति माया या त्रिगुणात्मिका} %1-11

\twolineshloka
{सूतेऽजस्रं शुक्लकृष्णलोहिताः सर्वदा प्रजाः}
{लोकत्रयमहागेहे गृहस्थस्त्वमुदाहृतः} %1-12

\twolineshloka
{त्वं विष्णुर्जानकी लक्ष्मीः शिवस्त्वं जानकी शिवा}
{ब्रह्मा त्वं जानकी वाणी सूर्यस्त्वं जानकी प्रभा} %1-13

\twolineshloka
{भवान् शशाङ्कः सीता तु रोहिणी शुभलक्षणा}
{शक्रस्त्वमेव पौलोमी सीता स्वाहानलो भवान्} %1-14

\twolineshloka
{यमस्त्वं कालरूपश्च सीता संयमिनी प्रभो}
{निरृतिस्त्वं जगन्नाथ तामसी जानकी शुभा} %1-15

\twolineshloka
{राम त्वमेव वरुणो भार्गवी जानकी शुभा}
{वायुस्त्वं राम सीता तु सदागतिरितीरिता} %1-16

\twolineshloka
{कुबेरस्त्वं राम सीता सर्वसम्पत्प्रकीर्तिता}
{रुद्राणी जानकी प्रोक्ता रुद्रस्त्वं लोकनाशकृत्} %1-17

\twolineshloka
{लोके स्त्रीवाचकं यावत्तत्सर्वं जानकी शुभा}
{पुन्नामवाचकं यावत्तत्सर्वं त्वं हि राघव} %1-18

\onelineshloka
{तस्माल्लोकत्रये देव युवाभ्यां नास्ति किञ्चन} %1-19

\twolineshloka
{त्वदाभासोदिताज्ञानमव्याकृतमितीर्यते}
{तस्मान्महान्स्ततः सूत्रं लिङ्गं सर्वात्मकं ततः} %1-20

\twolineshloka
{अहङ्कारश्च बुद्धिश्च पञ्चप्राणेन्द्रियाणि च}
{लिङ्गमित्युच्यते प्राज्ञैर्जन्ममृत्युसुखादिमत्} %1-21

\twolineshloka
{स एव जीवसंज्ञश्च लोके भाति जगन्मयः}
{अवाच्यानाद्यविद्यैव कारणोपाधिरुच्यते} %1-22

\twolineshloka
{स्थूलं सूक्ष्मं कारणाख्यमुपाधित्रितयं चितेः}
{एतैर्विशिष्टो जीवः स्याद्वियुक्तः परमेश्वरः} %1-23

\twolineshloka
{जाग्रत्स्वप्नसुषुप्त्याख्या संसृतिर्या प्रवर्तते}
{तस्या विलक्षणः साक्षी चिन्मात्रस्त्वं रघूत्तम} %1-24

\twolineshloka
{त्वत्त एव जगज्जातं त्वयि सर्वं प्रतिष्ठितम्}
{त्वय्येव लीयते कृत्स्नं तस्मात्त्वं सर्वकारणम्} %1-25

\twolineshloka
{रज्जावहिमिवात्मानं जीवं ज्ञात्वा भयं भवेत्}
{परात्माहमिति ज्ञात्वा भयदुःखैर्विमुच्यते} %1-26

\twolineshloka
{चिन्मात्रज्योतिषा सर्वाः सर्वदेहेषु बुद्धयः}
{त्वया यस्मात्प्रकाश्यन्ते सर्वस्यात्मा ततो भवान्} %1-27

\onelineshloka
{अज्ञानान्न्यस्यते सर्वं त्वयि रज्जौ भुजङ्गवत्} %1-28

\twolineshloka
{त्वत्पादभक्तियुक्तानां विज्ञानं भवति क्रमात्}
{तस्मात्त्वद्भक्तियुक्ता ये मुक्तिभाजस्त एव हि} %1-29

\twolineshloka
{अहं त्वद्भक्तभक्तानां तद्भक्तानां च किङ्करः}
{अतो मामनुगृह्णीष्व मोहयस्व न मां प्रभो} %1-30

\twolineshloka
{त्वन्नाभिकमलोत्पन्नो ब्रह्मा मे जनकः प्रभो}
{अतस्तवाहं पौत्रोऽस्मि भक्तं मां पाहि राघव} %1-31

\twolineshloka
{इत्युक्त्वा बहुशो नत्वा स्वानन्दाश्रुपरिप्लुतः}
{उवाच वचनं राम ब्रह्मणा नोदितोऽस्म्यहम्} %1-32

\twolineshloka
{रावणस्य वधार्थाय जातोऽसि रघुसत्तम}
{इदानीं राज्यरक्षार्थं पिता त्वामभिषेक्ष्यति} %1-33

\twolineshloka
{यदि राज्याभिसंसक्तो रावणं न हनिष्यसि}
{प्रतिज्ञा ते कृता राम भूभारहरणाय वै} %1-34

\twolineshloka
{तत्सत्यं कुरु राजेन्द्र सत्यसन्धस्त्वमेव हि}
{श्रुत्वैतद्गदितं रामो नारदं प्राह सस्मितम्} %1-35

\twolineshloka
{शृणु नारद मे किञ्चिद्विद्यतेऽविदितं क्वचित्}
{प्रतिज्ञातं च यत्पूर्वं करिष्ये तन्न संशयः} %1-36

\twolineshloka
{किन्तु कालानुरोधेन तत्तत्प्रारब्धसङ्क्षयात्}
{हरिष्ये सर्वभूभारं क्रमेणासुरमण्डलम्} %1-37

\twolineshloka
{रावणस्य विनाशार्थं श्वो गन्ता दण्डकाननम्}
{चतुर्दश समास्तत्र ह्युषित्वा मुनिवेषधृक्} %1-38

\twolineshloka
{सीतामिषेण तं दुष्टं सकुलं नाशयाम्यहम्}
{एवं रामे प्रतिज्ञाते नारदः प्रमुमोद ह} %1-39

\twolineshloka
{प्रदक्षिणत्रयं कृत्वा दण्डवत्प्रणिपत्य तम्}
{अनुज्ञातश्च रामेण ययौ देवगतिं मुनिः} %1-40

\fourlineindentedshloka
{संवादं पठति शृणोति संस्मरेद्वा}
{यो नित्यं मुनिवररामयोः सभक्त्या}
{सम्प्राप्नोत्यमरसुदुर्लभं विमोक्षम्}
{कैवल्यं विरतिपुरःसरं क्रमेण} %1-41

{॥इति श्रीमदध्यात्मरामायणे उमामहेश्वरसंवादे
अयोध्याकाण्डे प्रथमः सर्गे  नारद-राम-संवादः  सम्पूर्णः॥}

% !TeX program = XeLaTeX
% !TeX root = ../../shloka.tex

\sect{राम-सम्भाषणम्}
\addtocounter{shlokacount}{18}

\twolineshloka
{यदिदं दृश्यते सर्वं राज्यं देहादिकं च यत्}
{यदि सत्यं भवेत्तत्र आयासः सफलश्च ते} %4-19

\twolineshloka
{भोगा मेघवितानस्थविद्युल्लेखेव चञ्चलाः}
{आयुरप्यग्निसन्तप्तलोहस्थजलबिन्दुवत्} %4-20

\twolineshloka
{यथा व्यालगलस्थोऽपि भेको दंशानपेक्षते}
{तथा कालाहिना ग्रस्तो लोको भोगानशाश्वतान्} %4-21

\fourlineindentedshloka
{करोति दुःखेन हि कर्मतन्त्रम्}
{शरीरभोगार्थमहर्निशं नरः}
{देहस्तु भिन्नः पुरुषात्समीक्ष्यते}
{को वात्र भोगः पुरुषेण भुज्यते} %4-22

\twolineshloka
{पितृमातृसुतभ्रातृदारबन्ध्वादिसङ्गमः}
{प्रपायामिव जन्तूनां नद्यां काष्ठौघवच्चलः} %4-23

\fourlineindentedshloka
{छायेव लक्ष्मीश्चपला प्रतीता}
{तारुण्यमम्बूर्मिवदध्रुवं च}
{स्वप्नोपमं स्त्रीसुखमायुरल्पम्}
{तथाऽपि जन्तोरभिमान एषः} %4-24

\twolineshloka
{संसृतिः स्वप्नसदृशी सदा रोगादिसङ्कुला}
{गन्धर्वनगरप्रख्या मूढस्तामनुवर्तते} %4-25

\twolineshloka
{आयुष्यं क्षीयते यस्मादादित्यस्य गतागतैः}
{दृष्ट्वाऽन्येषां जरामृत्यू कथञ्चिन्नैव बुध्यते} %4-26

\twolineshloka
{स एव दिवसः सैव रात्रिरित्येव मूढधीः}
{भोगाननुपतत्येव कालवेगं न पश्यति} %4-27

\twolineshloka
{प्रतिक्षणं क्षरत्येतदायुरामघटाम्बुवत्}
{सपत्ना इव रोगौघाः शरीरं प्रहरन्त्यहो} %4-28

\twolineshloka
{जरा व्याघ्रीव पुरतस्तर्जयन्त्यवतिष्ठते}
{मृत्युः सहैव यात्येष समयं सम्प्रतीक्षते} %4-29

\twolineshloka
{देहेऽहम्भावमापन्नो राजाहं लोकविश्रुतः}
{इत्यस्मिन्मनुते जन्तुः कृमिविड्भस्मसंज्ञिते} %4-30

\twolineshloka
{त्वगस्थिमान्सविण्मूत्ररेतोरक्तादिसंयुतः}
{विकारी परिणामी च देह आत्मा कथं वद} %4-31

\twolineshloka
{यमास्थाय भवान्ल्लोकं दग्धुमिच्छति लक्ष्मण}
{देहाभिमानिनः सर्वे दोषाः प्रादुर्भवन्ति हि} %4-32

\twolineshloka
{देहोऽहमिति या बुद्धिरविद्या सा प्रकीर्तिता}
{नाहं देहश्चिदात्मेति बुद्धिर्विद्येति भण्यते} %4-33

\threelineshloka
{अविद्या संसृतेर्हेतुर्विद्या तस्या निवर्तिका}
{तस्माद्यत्नः सदा कार्यो विद्याभ्यासे मुमुक्षुभिः}
{कामक्रोधादयस्तत्र शत्रवः शत्रुसूदन} %4-34

\twolineshloka
{तत्रापि क्रोध एवालं मोक्षविघ्नाय सर्वदा}
{येनाविष्टः पुमान् हन्ति पितृभ्रातृसुहृत्सखीन्} %4-35

\twolineshloka
{क्रोधमूलो मनस्तापः क्रोधः संसारबन्धनम्}
{धर्मक्षयकरः क्रोधस्तस्मात्क्रोधं परित्यज} %4-36

\twolineshloka
{क्रोध एष महान् शत्रुस्तृष्णा वैतरणी नदी}
{सन्तोषो नन्दनवनं शान्तिरेव हि कामधुक्} %4-37

\twolineshloka
{तस्माच्छान्तिं भजस्वाद्य शत्रुरेवं भवेन्न ते}
{देहेन्द्रियमनःप्राणबुद्ध्यादिभ्यो विलक्षणः} %4-38

\twolineshloka
{आत्मा शुद्धः स्वयञ्ज्योतिरविकारी निराकृतिः}
{यावद्देहेन्द्रियप्राणैर्भिन्नत्वं नात्मनो विदुः} %4-39

\twolineshloka
{तावत्संसारदुःखौघैः पीड्यन्ते मृत्युसंयुताः}
{तस्मात्त्वं सर्वदा भिन्नमात्मानं हृदि भावय} %4-40

\twolineshloka
{बुद्ध्यादिभ्यो बहिः सर्वमनुवर्तस्व मा खिदः}
{भुञ्जन् प्रारब्धमखिलं सुखं वा दुःखमेव वा} %4-41

\twolineshloka
{प्रवाहपतितं कार्यं कुर्वन्नपि न लिप्यसे}
{बाह्ये सर्वत्र कर्तृत्वमावहन्नपि राघव} %4-42

\twolineshloka
{अन्तःशुद्धस्वभावस्त्वं लिप्यसे न च कर्मभिः}
{एतन्मयोदितं कृत्स्नं हृदि भावय सर्वदा} %4-43

\twolineshloka
{संसारदुःखैरखिलैर्बाध्यसे न कदाचन}
{त्वमप्यम्ब ममाऽऽदिष्टं हृदि भावय नित्यदा} %4-44

\twolineshloka
{समागमं प्रतीक्षस्व न दुःखैः पीड्यसे चिरम्}
{न सदैकत्र संवासः कर्ममार्गानुवर्तिनाम्} %4-45

\twolineshloka
{यथा प्रवाहपतितप्लवानां सरितां तथा}
{चतुर्दशसमा सङ्ख्या क्षणार्द्धमिव जायते} %4-46

\twolineshloka
{अनुमन्यस्व मामम्ब दुःखं सन्त्यज्य दूरतः}
{एवं चेत्सुखसंवासो भविष्यति वने मम} %4-47

\twolineshloka
{इत्युक्त्वा दण्डवन्मातुः पादयोरपतच्चिरम्}
{उत्थाप्याङ्के समावेश्य आशीर्भिरभ्यनन्दयत्} %4-48

{॥इति श्रीमदध्यात्मरामायणे उमामहेश्वरसंवादे
अयोध्याकाण्डे चतुर्थे सर्गे  राम-सम्भाषणं  सम्पूर्णम्॥}

% !TeX program = XeLaTeX
% !TeX root = ../../shloka.tex

\sect{लक्ष्मण-सम्भाषणम्}
% \addtocounter{shlokacount}{18}
\uvacha{श्री  महादेव उवाच}

\twolineshloka
{सुप्तं रामं समालोक्य गुहः सोऽश्रुपरिप्लुतः}
{लक्ष्मणं प्राह विनयाद् भ्रातः पश्यसि राघवम्} %6-1

\twolineshloka
{शयानं कुशपत्रौघसंस्तरे सीतया सह}
{यः शेते स्वर्णपर्यङ्के स्वास्तीर्णे भवनोत्तमे} %6-2

\twolineshloka
{कैकेयी रामदुःखस्य कारणं विधिना कृता}
{मन्थराबुद्धिमास्थाय कैकेयी पापमाचरत्} %6-3

\twolineshloka
{तच्छ्रुत्वा लक्ष्मणः प्राह सखे शृणु वचो मम}
{कः कस्य हेतुर्दुःखस्य कश्च हेतुः सुखस्य च} %6-4

\onelineshloka
{स्वपूर्वार्जितकर्मैव कारणं सुखदुःखयोः} %6-5

\fourlineindentedshloka
{सुखस्य दुःखस्य न कोऽपि दाता}
{परो ददातीति कुबुद्धिरेषा}
{अहं करोमीति वृथाभिमानः}
{स्वकर्मसूत्रग्रथितो हि लोकः} %6-6

\twolineshloka
{सुहृन्मित्रार्युदासीनद्वेष्यमध्यस्थबान्धवाः}
{स्वयमेवाचरन् कर्म तथा तत्र विभाव्यते} %6-7

\twolineshloka
{सुखं वा यदि वा दुःखं स्वकर्मवशगो नरः}
{यद्यद्यथागतं तत्तद् भुक्त्वा स्वस्थमना भवेत्} %6-8

\twolineshloka
{न मे भोगागमे वाञ्छा न मे भोगविवर्जने}
{आगच्छत्वथ मागच्छत्वभोगवशगो भवेत्} %6-9

\twolineshloka
{स्वस्मिन् देशे च काले च यस्माद्वा येन केन वा}
{कृतं शुभाशुभं कर्म भोज्यं तत्तत्र नान्यथा} %6-10

\twolineshloka
{अलं हर्षविषादाभ्यां शुभाशुभफलोदये}
{विधात्रा विहितं यद्यत्तदलङ्घ्यं सुरासुरैः} %6-11

\twolineshloka
{सर्वदा सुखदुःखाभ्यां नरः प्रत्यवरुध्यते}
{शरीरं पुण्यपापाभ्यामुत्पन्नं सुखदुःखवत्} %6-12

\twolineshloka
{सुखस्यानन्तरं दुःखं दुःखस्यानन्तरं सुखम्}
{द्वयमेतद्धि जन्तूनामलङ्घ्यं दिनरात्रिवत्} %6-13

\twolineshloka
{सुखमध्ये स्थितं दुःखं दुःखमध्ये स्थितं सुखम्}
{द्वयमन्योन्यसंयुक्तं प्रोच्यते जलपङ्कवत्} %6-14

\twolineshloka
{तस्माद्धैर्येण विद्वांस इष्टानिष्टोपपत्तिषु}
{न हृष्यन्ति न मुह्यन्ति समं मायेति भावनात्} %6-15

{॥इति श्रीमदध्यात्मरामायणे उमामहेश्वरसंवादे
अयोध्याकाण्डे षष्ठे सर्गे  लक्ष्मण-सम्भाषणं  सम्पूर्णम्॥}

% !TeX program = XeLaTeX
% !TeX root = ../../shloka.tex

\sect{वाल्मीकीरित-रामस्थानवर्णनम्}
\addtocounter{shlokacount}{51}
\uvacha{श्री वाल्मीकिरुवाच}


\twolineshloka
{त्वमेव सर्वलोकानां निवासस्थानमुत्तमम्}
{तवापि सर्वभूतानि निवाससदनानि हि} %6-52

\threelineshloka
{एवं साधारणं स्थानमुक्तं ते रघुनन्दन}
{सीतया सहितस्येति विशेषं पृच्छतस्तव}
{तद्वक्ष्यामि रघुश्रेष्ठ यत्ते नियतमन्दिरम्} %6-53

\twolineshloka
{शान्तानां समदृष्टीनामद्वेष्टॄणां च जन्तुषु}
{त्वामेव भजतां नित्यं हृदयं तेऽधिमन्दिरम्} %6-54

\twolineshloka
{धर्माधर्मान् परित्यज्य त्वामेव भजतोऽनिशम्}
{सीतया सह ते राम तस्य हृत्सुखमन्दिरम्} %6-55

\twolineshloka
{त्वन्मन्त्रजापको यस्तु त्वामेव शरणं गतः}
{निर्द्वन्द्वो निःस्पृहस्तस्य हृदयं ते सुमन्दिरम्} %6-56

\twolineshloka
{निरहङ्कारिणः शान्ता ये रागद्वेषवर्जिताः}
{समलोष्टाश्मकनकास्तेषां ते हृदयं गृहम्} %6-57

\twolineshloka
{त्वयि दत्तमनोबुद्धिर्यः सन्तुष्टः सदा भवेत्}
{त्वयि सन्त्यक्तकर्मा यस्तन्मनस्ते शुभं गृहम्} %6-58

\twolineshloka
{यो न द्वेष्ट्यप्रियं प्राप्य प्रियं प्राप्य न हृष्यति}
{सर्वं मायेति निश्चित्य त्वां भजेत्तन्मनो गृहम्} %6-59

\twolineshloka
{षड्भावादिविकारान् यो देहे पश्यति नात्मनि}
{क्षुत्तृट् सुखं भयं दुःखं प्राणबुद्ध्योर्निरीक्षते} %6-60

\onelineshloka
{संसारधर्मैर्निर्मुक्तस्तस्य ते मानसं गृहम्} %6-61

\fourlineindentedshloka
{पश्यन्ति ये सर्वगुहाशयस्थम्}
{त्वां चिद्घनं सत्यमनन्तमेकम्}
{अलेपकं सर्वगतं वरेण्यम्}
{तेषां हृदब्जे सह सीतया वस} %6-62

\fourlineindentedshloka
{निरन्तराभ्यासदृढीकृतात्मनाम्}
{त्वत्पादसेवापरिनिष्ठितानाम्}
{त्वन्नामकीर्त्या हतकल्मषाणाम्}
{सीतासमेतस्य गृहं हृदब्जे} %6-63

\twolineshloka
{राम त्वन्नाममहिमा वर्ण्यते केन वा कथम्}
{यत्प्रभावादहं राम ब्रह्मर्षित्वमवाप्तवान्} %6-64

{॥इति श्रीमदध्यात्मरामायणे उमामहेश्वरसंवादे
अयोध्याकाण्डे षष्ठे सर्गे  वाल्मीकीरित-रामस्थानवर्णनं  सम्पूर्णम्॥}

% !TeX program = XeLaTeX
% !TeX root = ../../shloka.tex

\sect{वसिष्ठोपदेशः}

\uvacha{वसिष्ठ  उवाच}

{आत्मा नित्योऽव्ययः शुद्धो जन्मनाशादिवर्जितः॥९५॥} %7-95
\addtocounter{shlokacount}{95}

\twolineshloka
{शरीरं जडमत्यर्थमपवित्रं विनश्वरम्}
{विचार्यमाणे शोकस्य नावकाशः कथञ्चन} %7-96

\twolineshloka
{पिता वा तनयो वाऽपि यदि मृत्युवशं गतः}
{मूढास्तमनुशोचन्ति स्वात्मताडनपूर्वकम्} %7-97

\twolineshloka
{निःसारे खलु संसारे वियोगो ज्ञानिनां यदा}
{भवेद्वैराग्यहेतुः स शान्तिसौख्यं तनोति च} %7-98

\twolineshloka
{जन्मवान् यदि लोकेऽस्मिन्स्तर्हि तं मृत्युरन्वगात्}
{तस्मादपरिहार्योऽयं मृत्युर्जन्मवतां सदा} %7-99

\twolineshloka
{स्वकर्मवशतः सर्वजन्तूनां प्रभवाप्ययौ}
{विजानन्नप्यविद्वान् यः कथं शोचति बान्धवान्} %7-100

\twolineshloka
{ब्रह्माण्डकोटयो नष्टाः सृष्टयो बहुशो गताः}
{शुष्यन्ति सागराः सर्वे कैवास्था क्षणजीविते} %7-101

\twolineshloka
{चलपत्रान्तलग्नाम्बुबिन्दुवत्क्षणभङ्गुरम्}
{आयुस्त्यजत्यवेलायां कस्तत्र प्रत्ययस्तव} %7-102

\twolineshloka
{देही प्राक्तनदेहोत्थकर्मणा देहवान् पुनः}
{तद्देहोत्थेन च पुनरेवं देहः सदात्मनः} %7-103

\twolineshloka
{यथा त्यजति वै जीर्णं वासो गृह्णाति नूतनम्}
{तथा जीर्णं परित्यज्य देही देहं पुनर्नवम्} %7-104

\twolineshloka
{भजत्येव सदा तत्र शोकस्यावसरः कुतः}
{आत्मा न म्रियते जातु जायते न च वर्धते} %7-105

\twolineshloka
{षड्भावरहितोऽनन्तः सत्यप्रज्ञानविग्रहः}
{आनन्दरूपो बुद्ध्यादिसाक्षी लयविवर्जितः} %7-106

\twolineshloka
{एक एव परो ह्यात्मा ह्यद्वितीयः समः स्थितः}
{इत्यात्मानं दृढं ज्ञात्वा त्यक्त्वा शोकं कुरु क्रियाम्} %7-107

\twolineshloka
{तैलद्रोण्याः पितुर्देहमुद्धृत्य सचिवैः सह}
{कृत्यं कुरु यथान्यायमस्माभिः कुलनन्दन} %7-108

{॥इति श्रीमदध्यात्मरामायणे उमामहेश्वरसंवादे
अयोध्याकाण्डे सप्तमे सर्गे  वसिष्ठोपदेशः  सम्पूर्णः॥}


\closesection

\clearpage
\chapt{अरण्यकाण्डः}

\closesection

\clearpage
\chapt{किष्किन्धाकाण्डः}

\closesection

\clearpage
\chapt{सुन्दरकाण्डः}

\closesection

\clearpage
\chapt{युद्धकाण्डः}

\closesection

\clearpage
\chapt{उत्तरकाण्डः}

\closesection

\end{center}
%\clearemptydoublepage
