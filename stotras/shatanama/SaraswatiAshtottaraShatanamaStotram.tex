% !TeX program = XeLaTeX
% !TeX root = ../../shloka.tex

\sect{सरस्वत्यष्टोत्तरशतनामस्तोत्रम्}

\dnsub{ध्यानम्}
\fourlineindentedshloka*
{या कुन्देन्दुतुषारहारधवला या शुभ्रवस्त्रावृता}
{या वीणावरदण्डमण्डितकरा या श्वेतपद्मासना}
{या ब्रह्माच्युतशङ्करप्रभृतिभिर्देवैः सदा पूजिता}
{सा मां पातु सरस्वती भगवती निःशेषजाड्यापहा}

\dnsub{स्तोत्रम्}
\twolineshloka
{सरस्वती महाभद्रा महामाया वरप्रदा}
{श्रीप्रदा पद्मनिलया पद्माक्षी पद्मवक्त्रका}

\twolineshloka
{शिवानुजा पुस्तकभृत् ज्ञानमुद्रा रमा परा}
{कामरूपा महाविद्या महापातकनाशिनी}

\twolineshloka
{महाश्रया मालिनी च महाभोगा महाभुजा}
{महाभागा महोत्साहा दिव्याङ्गा सुरवन्दिता}

\twolineshloka
{महाकाली महापाशा महाकारा महाङ्कुशा}
{पीता च विमला विश्वा विद्युन्माला च वैष्णवी}

\twolineshloka
{चन्द्रिका चन्द्रवदना चन्द्रलेखविभूषिता}
{सावित्री सुरसा देवी दिव्यालङ्कारभूषिता}

\twolineshloka
{वाग्देवी वसुदा तीव्रा महाभद्रा महाबला}
{भोगदा भारती भामा गोविन्दा गोमती शिवा}

\twolineshloka
{जटिला विन्ध्यवासा च विन्ध्याचलविराजिता}
{चण्डिका वैष्णवी ब्राह्मी ब्रह्मज्ञानैकसाधना}

\twolineshloka
{सौदामिनी सुधामूर्तिः सुभद्रा सुरपूजिता}
{सुवासिनी सुनासा च विनिद्रा पद्मलोचना}

\twolineshloka
{विद्यारूपा विशालाक्षी ब्रह्मजाया महाफला}
{त्रयीमूर्ती त्रिकालज्ञा त्रिगुणा शास्त्ररूपिणी}

\twolineshloka
{शुम्भासुरप्रमथिनी शुभदा च स्वरात्मिका}
{रक्तबीजनिहन्त्री च चामुण्डा चाम्बिका तथा}

\twolineshloka
{मुण्डकायप्रहरणा धूम्रलोचनमर्दना}
{सर्वदेवस्तुता सौम्या सुरासुरनमस्कृता}

\twolineshloka
{कालरात्रिः कलाधारा रूपसौभाग्यदायिनी}
{वाग्देवी च वरारोहा वाराही वारिजासना}

\twolineshloka
{चित्राम्बरा चित्रगन्धा चित्रमाल्यविभूषिता}
{कान्ता कामप्रदा वन्द्या विद्याधरसुपूजिता}

\twolineshloka
{श्वेतानना नीलभुजा चतुर्वर्गफलप्रदा}
{चतुराननसाम्राज्या रक्तमध्या निरञ्जना}

\twolineshloka
{हंसासना नीलजङ्घा ब्रह्मविष्णुशिवात्मिका}
{एवं सरस्वतीदेव्या नाम्नामष्टोत्तरं शतम्}

॥इति श्री सरस्वत्यष्टोत्तरशतनामस्तोत्रं सम्पूर्णम्॥