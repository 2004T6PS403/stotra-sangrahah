%!TeX program = Xelatex
%!TeX root = ../../shloka.tex

\sect{नृसिंहाष्टोत्तरशतनामस्तोत्रम्}

\twolineshloka
{नारसिंहो महासिंहो दिव्यसिंहो महाबलः}
{उग्रसिंहो महादेवः स्तम्भजश्चोग्रलोचनः}

\twolineshloka
{रौद्रः सर्वाद्भुतः श्रीमान् योगानन्दस्त्रिविक्रमः}
{हरिः कोलाहलश्चक्री विजयो जयवर्धनः}

\twolineshloka
{पञ्चाननः परब्रह्म अघोरो घोरविक्रमः}
{ज्वालामुखो ज्वालमाली महाज्वालो महाप्रभुः}

\twolineshloka
{निटिलाक्षः सहस्राक्षो दुर्निरीक्ष्यः प्रतापनः}
{महादंष्ट्रायुधः  प्राज्ञश्चण्डकोपी सदाशिवः}

\twolineshloka
{हिरण्यकशिपुध्वंसी दैत्यदानवभञ्जनः}
{गुणभद्रो महाभद्रो बलभद्रो सुभद्रकः}

\twolineshloka
{करालो विकरालश्च विकर्ता सर्वकर्तृकः}
{शिंशुमारस्त्रिलोकात्मा ईशः सर्वेश्वरो विभुः}

\twolineshloka
{भैरवाडम्बरो दिव्यश्चाच्युतः कविमाधवः}
{अधोक्षजोऽक्षरः शर्वो वनमाली वरप्रदः}

\twolineshloka
{विश्वम्भरोऽद्भुतो भव्यो विष्णुश्च पुरुषोत्तमः}
{अमोघास्त्रो नखास्त्रश्च सूर्यज्योतिः सुरेश्वरः}

\twolineshloka
{सहस्रबाहुः सर्वज्ञः सर्वसिद्धिप्रदायकः}
{वज्रदंष्ट्रो वज्रनखो महानादः परन्तपः}

\twolineshloka
{सर्वमन्त्रैकरूपश्च सर्वयन्त्रविदारणः}
{सर्वतन्त्रात्मकोऽव्यक्तः सुव्यक्तो भक्तवत्सलः}

\twolineshloka
{वैशाखशुक्लसम्भूतः शरणागतवत्सलः}
{उदारकीर्तिः पुण्यात्मा महात्मा चण्डविक्रमः}

\twolineshloka
{वेदत्रयप्रपूज्यश्च भगवान् परमेश्वरः}
{श्रीवत्साङ्कः श्रीनिवासो जगद्व्यापी  जगन्मयः}

\twolineshloka
{जगत्पालो जगन्नाथो महाकायो द्विरूपभृत्}
{परमात्मा परञ्ज्योतिर्निर्गुणश्च नृकेसरी}

\twolineshloka
{परतत्त्वं परन्धाम सच्चिदानन्दविग्रहः}
{लक्ष्मीनृसिंहः सर्वात्मा धीरः प्रह्लादपालकः}

॥इति श्री नृसिंहाष्टोत्तरशतनामस्तोत्रं सम्पूर्णम्॥
