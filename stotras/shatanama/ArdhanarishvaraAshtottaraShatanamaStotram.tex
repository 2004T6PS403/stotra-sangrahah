% !TeX program = XeLaTeX
% !TeX root = ../../shloka.tex

\sect{श्री अर्धनारीश्वराष्टोत्तरशतनामस्तोत्रम्}


\twolineshloka
{चामुण्डिकाम्बा श्रीकण्ठः पार्वती परमेश्वरः}
{महाराज्ञी महादेवः सदाराध्या सदाशिवः}% ॥ १ ॥

\twolineshloka
{शिवार्धाङ्गी शिवार्धाङ्गो भैरवी कालभैरवः}
{शक्तित्रितयरूपाढ्या मूर्तित्रितयरूपवान्}% ॥ २ ॥

\twolineshloka
{कामकोटि-सुपीठस्था काशीक्षेत्रसमाश्रयः}
{दाक्षायणी दक्षवैरिः शूलिनी शूलधारकः}% ॥ ३ ॥

\twolineshloka
{ह्रीङ्कारपञ्जरशुकी हरिशङ्कररूपवान्}
{श्रीमद्गणेशजननी षडानन-सुजन्मभूः}% ॥ ४ ॥

\twolineshloka
{पञ्चप्रेतासनारूढा पञ्चब्रह्मस्वरूपभृत्}
{चण्डमुण्डशिरश्छेत्री जलन्धरशिरोहरः}% ॥ ५ ॥

\twolineshloka
{सिंहवाहा वृषारूढः  श्यामाभा स्फटिकप्रभः}
{महिषासुरसंहर्त्री गजासुरविमर्दनः}% ॥ ६ ॥

\twolineshloka
{महाबलाचलावासा महाकैलासवासभूः}
{भद्रकाली वीरभद्रो मीनाक्षी सुन्दरेश्वरः}% ॥ ७ ॥

\twolineshloka
{भण्डासुरादिसंहर्त्री दुष्टान्धकविमर्दनः}
{मधुकैटभसंहर्त्री मधुरापुरनायकः}% ॥ ८ ॥

\twolineshloka
{कालत्रयस्वरूपाढ्या कार्यत्रयविधायकः}
{गिरिजाता गिरीशश्च वैष्णवी विष्णुवल्लभः}% ॥ ९ ॥

\twolineshloka
{विशालाक्षी विश्वनाथः पुष्पास्त्रा विष्णुमार्गणः}
{कौसुम्भवसनोपेता व्याघ्रचर्माम्बरावृतः}% ॥ १० ॥

\twolineshloka
{मूलप्रकृतिरूपाढ्या परब्रह्मस्वरूपवान्}
{रुण्डमालाविभूषाढ्या लसद्रुद्राक्षमालिकः}% ॥ ११ ॥

\twolineshloka
{मनोरूपेक्षुकोदण्डा महामेरुधनुर्धरः}
{चन्द्रचूडा चन्द्रमौलिर्महामाया महेश्वरः}% ॥ १२ ॥

\twolineshloka
{महाकाली महाकालो दिव्यरूपा दिगम्बरः}
{बिन्दुपीठसुखासीना श्रीमदोङ्कारपीठगः}% ॥ १३ ॥

\twolineshloka
{हरिद्राकुङ्कुमालिप्ता भस्मोद्धूलितविग्रहः}
{महापद्माटवीलोला महाबिल्वाटवीप्रियः}% ॥ १४ ॥

\twolineshloka
{सुधामयी विषधरो मातङ्गी मकुटेश्वरः}
{वेदवेद्या वेदवाजी चक्रेशी विष्णुचक्रदः}% ॥ १५ ॥

\twolineshloka
{जगन्मयी जगद्रूपो मृडानी मृत्युनाशनः}
{रामार्चितपदाम्भोजा कृष्णपुत्रवरप्रदः}% ॥ १६ ॥

\twolineshloka
{रमावाणीसुसंसेव्या विष्णुब्रह्मसुसेवितः}
{सूर्यचन्द्राग्निनयना तेजस्त्रयविलोचनः}% ॥ १७ ॥

\twolineshloka
{चिदग्निकुण्डसम्भूता महालिङ्गसमुद्भवः}
{कम्बुकण्ठी कालकण्ठो वज्रेशी वज्रिपूजितः}% ॥ १८ ॥

\twolineshloka
{त्रिकण्टकी त्रिभङ्गीशः भस्मरक्षा स्मरान्तकः}
{हयग्रीववरोद्धात्री मार्कण्डेयवरप्रदः}% ॥ १९ ॥

\twolineshloka
{चिन्तामणिगृहावासा मन्दराचलमन्दिरः}
{विन्ध्याचलकृतावासा विन्ध्यशैलार्यपूजितः}% ॥ २० ॥

\twolineshloka
{मनोन्मनी लिङ्गरूपो जगदम्बा जगत्पिता}
{योगनिद्रा योगगम्यो भवानी भवमूर्तिमान्}% ॥ २१ ॥

\twolineshloka
{श्रीचक्रात्मरथारूढा धरणीधरसंस्थितः}
{श्रीविद्यावेद्यमहिमा निगमागमसंश्रयः}% ॥ २२ ॥

\twolineshloka
{दशशीर्षसमायुक्ता पञ्चविंशतिशीर्षवान्}
{अष्टादशभुजायुक्ता पञ्चाशत्करमण्डितः}% ॥ २३ ॥

\twolineshloka
{ब्राह्म्यादिमातृकारूपा शताष्टेकादशात्मवान्}
{स्थिरा स्थाणुस्तथा बाला सद्योजात उमा मृडः}% ॥ २४ ॥

\twolineshloka
{शिवा शिवश्च रुद्राणी रुद्रश्चैवेश्वरीश्वरः}
{कदम्बकाननावासा दारुकारण्यलोलुपः}% ॥ २५ ॥

\twolineshloka
{नवाक्षरीमनुस्तुत्या पञ्चाक्षरमनुप्रियः}
{नवावरणसम्पूज्या पञ्चायतनपूजितः}% ॥ २६ ॥

\twolineshloka
{देवस्थषट्चक्रदेवी दहराकाशमध्यगः}
{योगिनीगणसंसेव्या भृग्वादिप्रमथावृतः}% ॥ २७ ॥

\twolineshloka
{उग्रताराघोररूपः शर्वाणी शर्वमूर्तिमान्}
{नागवेणी नागभूषो मन्त्रिणी मन्त्रदैवतः}% ॥ २८ ॥

\twolineshloka
{ज्वलज्जिह्वा ज्वलन्नेत्रो दण्डनाथा दृगायुधः}
{पार्थाञ्जनास्त्रसन्दात्री पार्थपाशुपतास्त्रदः}% ॥ २९ ॥

\twolineshloka
{पुष्पवच्चक्रताटङ्का फणिराजसुकुण्डलः}
{बाणपुत्रीवरोद्धात्री बाणासुरवरप्रदः}% ॥ ३० ॥

\twolineshloka
{व्यालकञ्चुकसंवीता व्यालयज्ञोपवीतवान्}
{नवलावण्यरूपाढ्या नवयौवनविग्रहः}% ॥ ३१ ॥

\twolineshloka
{नाट्यप्रिया नाट्यमूर्तिस्त्रिसन्ध्या त्रिपुरान्तकः}
{तन्त्रोपचारसुप्रीता तन्त्रादिमविधायकः}% ॥ ३२ ॥

\twolineshloka
{नववल्लीष्टवरदा नववीरसुजन्मभूः}
{भ्रमरज्या वासुकिज्यो भेरुण्डा भीमपूजितः}% ॥ ३३ ॥

\twolineshloka
{निशुम्भशुम्भदमनी नीचापस्मारमर्दनः}
{सहस्राराम्बुजारूढा सहस्रकमलार्चितः}% ॥ ३४ ॥

\twolineshloka
{गङ्गासहोदरी गङ्गाधरो गौरी त्रियम्बकः}
{श्रीशैलभ्रमराम्बाख्या मल्लिकार्जुनपूजितः}% ॥ ३५ ॥

\twolineshloka
{भवतापप्रशमनी भवरोगनिवारकः}
{चन्द्रमण्डलमध्यस्था मुनिमानसहंसकः}% ॥ ३६ ॥

\twolineshloka
{प्रत्यङ्गिरा प्रसन्नात्मा कामेशी कामरूपवान्}
{स्वयम्प्रभा स्वप्रकाशः कालरात्री कृतान्तहृत्}% ॥ ३७ ॥

\twolineshloka
{सदान्नपूर्णा भिक्षाटो वनदुर्गा वसुप्रदः}
{सर्वचैतन्यरूपाढ्या सच्चिदानन्दविग्रहः}% ॥ ३८ ॥

\twolineshloka
{सर्वमङ्गलरूपाढ्या सर्वकल्याणदायकः}
{राजराजेश्वरी श्रीमद्राजराजप्रियङ्करः}% ॥ ३९ ॥

\twolineshloka
{अर्धनारीश्वरस्येदं नाम्नामष्टोत्तरं शतम्}
{पठन्नर्चन् सदा भक्त्या सर्वसाम्राज्यमाप्नुयात्}% ॥ ४० ॥

॥इति श्रीस्कन्दपुराणे श्री अर्धनारीश्वराष्टोत्तरशतनामस्तोत्रं सम्पूर्णम्॥