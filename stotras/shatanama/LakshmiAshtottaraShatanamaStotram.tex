% !TeX program = XeLaTeX
% !TeX root = ../../shloka.tex

\sect{लक्ष्म्यष्टोत्तरशतनामस्तोत्रम्}
%
%देव्युवाच
%\twolineshloka
%{देवदेव महादेव त्रिकालज्ञ महेश्वर}
%{करुणाकर देवेश भक्तानुग्रहकारक}
%{अष्टोत्तरशतं लक्ष्म्याः श्रोतुमिच्छामि तत्त्वतः॥}
%
%ईश्वर उवाच
%\twolineshloka
%{देवि साधु महाभागे महाभाग्यप्रदायकम्}
%{सर्वैश्वर्यकरं पुण्यं सर्वपापप्रणाशनम्}
%
%\twolineshloka
%{सर्वदारिद्र्यशमनं श्रवणाद्भुक्तिमुक्तिदम्}
%{राजवश्यकरं दिव्यं गुह्याद्गुह्यतमं परम्}
%
%\twolineshloka
%{दुर्लभं सर्वदेवानां चतुःषष्टिकलास्पदम्}
%{पद्मादीनां वरान्तानां विधीनां नित्यदायकम्}
%\twolineshloka
%{समस्तदेवसंसेव्यमणिमाद्यष्टसिद्धिदम्}
%{किमत्र बहुनोक्तेन देवी प्रत्यक्षदायकम्}
%\twolineshloka
%{तव प्रीत्याऽद्य वक्ष्यामि समाहितमनाः शृणुम्}
%{अष्टोत्तरशतस्यास्य महालक्ष्मीस्तु देवता}
%\twolineshloka
%{क्लीम्बीजपदमित्युक्तं शक्तिस्तु भुवनेश्वरी}
%{अङ्गन्यासः करन्यासः स इत्यादिः प्रकीर्तितः}

\dnsub{ध्यानम्}\nopagebreak[4]
\fourlineindentedshloka*
{वन्दे पद्मकरां प्रसन्नवदनां सौभाग्यदां भाग्यदाम्}
{हस्ताभ्यामभयप्रदां मणिगणैर्नानाविधैर्भूषिताम्}
{भक्ताभीष्टफलप्रदां हरिहरब्रह्मादिभिः सेविताम्}
{पार्श्वे पङ्कजशङ्खपद्मनिधिभिर्युक्तां सदा शक्तिभिः}

\twolineshloka*
{सरसिजनिलये सरोजहस्ते धवलतमांशुकगन्धमाल्यशोभे}
{भगवति हरिवल्लभे मनोज्ञे त्रिभुवनभूतिकरि प्रसीद मह्यम्}

\dnsub{स्तोत्रम्}
\resetShloka
\twolineshloka
{प्रकृतिं विकृतिं विद्यां सर्वभूतहितप्रदाम्}
{श्रद्धां विभूतिं सुरभिं नमामि परमात्मिकाम्}

\twolineshloka
{वाचं पद्मालयां पद्मां शुचिं स्वाहां स्वधां सुधाम्}
{धन्यां हिरण्मयीं लक्ष्मीं नित्यपुष्टां विभावरीम्}

\twolineshloka
{अदितिं च दितिं दीप्तां वसुधां वसुधारिणीम्}
{नमामि कमलां कान्तां कामाक्षीं क्रोधसम्भवाम्}

\twolineshloka
{अनुग्रहपदां बुद्धिमनघां हरिवल्लभाम्}
{अशोकाममृतां दीप्तां लोकशोकविनाशिनीम्}

\twolineshloka
{नमामि धर्मनिलयां करुणां लोकमातरम्}
{पद्मप्रियां पद्महस्तां पद्माक्षीं पद्मसुन्दरीम्}

\twolineshloka
{पद्मोद्भवां पद्ममुखीं पद्मनाभप्रियां रमाम्}
{पद्ममालाधरां देवीं पद्मिनीं पद्मगन्धिनीम्}

\twolineshloka
{पुण्यगन्धां सुप्रसन्नां प्रसादाभिमुखीं प्रभाम्}
{नमामि चन्द्रवदनां चन्द्रां चन्द्रसहोदरीम्}

\twolineshloka
{चतुर्भुजां चन्द्ररूपामिन्दिरामिन्दुशीतलाम्}
{आह्लादजननीं पुष्टिं शिवां शिवकरीं सतीम्}

\twolineshloka
{विमलां विश्वजननीं तुष्टिं दारिद्र्यनाशिनीम्}
{प्रीतिपुष्करिणीं शान्तां शुक्लमाल्याम्बरां श्रियम्}

\twolineshloka
{भास्करीं बिल्वनिलयां वरारोहां यशस्विनीम्}
{वसुन्धरामुदाराङ्गां हरिणीं हेममालिनीम्}

\twolineshloka
{धनधान्यकरीं सिद्धिं स्त्रैणसौम्यां शुभप्रदाम्}
{नृपवेश्मगतानन्दां वरलक्ष्मीं वसुप्रदाम्}

\twolineshloka
{शुभां हिरण्यप्राकारां समुद्रतनयां जयाम्}
{नमामि मङ्गलां देवीं विष्णुवक्षःस्थलस्थिताम्}

\twolineshloka
{विष्णुपत्नीं प्रसन्नाक्षीं नारायणसमाश्रिताम्}
{दारिद्र्यध्वंसिनीं देवीं सर्वोपद्रवहारिणीम्}

\twolineshloka
{नवदुर्गां महाकालीं ब्रह्मविष्णुशिवात्मिकाम्}
{त्रिकालज्ञानसम्पन्नां नमामि भुवनेश्वरीम्}

\fourlineindentedshloka
{लक्ष्मीं क्षीरसमुद्रराजतनयां श्रीरङ्गधामेश्वरीम्}
{दासीभूतसमस्तदेववनितां लोकैकदीपाङ्कुराम्}
{श्रीमन्मन्दकटाक्षलब्धविभवब्रह्मेन्द्रगङ्गाधराम्}
{त्वां त्रैलोक्यकुटुम्बिनीं सरसिजां वन्दे मुकुन्दप्रियाम्}

\fourlineindentedshloka
{मातर्नमामि कमले कमलायताक्षि}
{श्रीविष्णुहृत्कमलवासिनि विश्वमातः}
{क्षीरोदजे कमलकोमलगर्भगौरि}
{लक्ष्मि प्रसीद सततं नमतां शरण्ये}

\dnsub{फलश्रुतिः}\nopagebreak[4]
\twolineshloka
{त्रिकालं यो जपेद्विद्वान् षण्मासं विजितेन्द्रियः}
{दारिद्र्यध्वंसनं कृत्वा सर्वमाप्नोत्ययत्नतः}

\twolineshloka
{देवीनामसहस्रेषु पुण्यमष्टोत्तरं शतम्}
{येन श्रियमवाप्नोति कोटिजन्मदरिद्रितः}

\twolineshloka
{भृगुवारे शतं धीमान् पठेद्वत्सरमात्रकम्}
{अष्टैश्वर्यमवाप्नोति कुबेर इव भूतले}

\twolineshloka
{दारिद्र्यमोचनं नाम स्तोत्रमम्बापरं शतम्}
{येन श्रियमवाप्नोति कोटिजन्मदरिद्रितः}

\threelineshloka
{भुक्त्वा तु विपुलान् भोगानस्याः सायुज्यमाप्नुयात्}
{प्रातःकाले पठेन्नित्यं सर्वदुःखोपशान्तये}
{पठंस्तु चिन्तयेद्देवीं सर्वाभरणभूषिताम्}
॥इति श्री लक्ष्म्यष्टोत्तरशतनामस्तोत्रं सम्पूर्णम्॥