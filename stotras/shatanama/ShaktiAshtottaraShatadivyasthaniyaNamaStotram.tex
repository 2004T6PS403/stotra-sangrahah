% !TeX program = XeLaTeX
% !TeX root = ../../shloka.tex
\sect{शक्त्यष्टोत्तरशतदिव्यस्थानीयनामस्तोत्रम्}

\uvacha{दक्ष उवाच}
\twolineshloka*
{एवमुक्तोऽब्रवीद्दक्षः केषु केषु मयाऽनघे}
{तीर्थेषु च त्वं द्रष्टव्या स्तोतव्या कैश्च नामभिः}

\uvacha{देव्युवाच}
\twolineshloka*
{देवीः सर्वदा सर्वभूतेषु द्रष्टव्या सर्वतो भुवि}
{सप्तलोकेषु यत्किञ्चिद्रहितं न मया हि तत्}

\twolineshloka*
{तथापि येषु स्थानेषु द्रष्टव्या सिद्धि मीप्सुभिः}
{स्मर्तव्या भूतिकामैर्वा तानि वक्ष्यामि तत्त्वतः}

\dnsub{स्तोत्रम्}
\twolineshloka
{वाराणस्यां विशालाक्षी नैमिषे लिङ्गधारिणी}
{प्रयागे ललिता देवी कामाक्षी गन्धमादने}

\twolineshloka
{मानसे कुमुदा नाम विश्वकाया तथाम्बरे}
{गोमन्ते गोमती नाम मन्दरे कामचारिणी}

\twolineshloka
{मदोत्कटा चैत्ररथे जयन्ती हस्तिनापुरे}
{कान्यकुब्जे तथा गौरी रम्भा मलयपर्वते}

\twolineshloka
{एकाम्रके कीर्तिमती विश्वे विश्वेश्वरीं विदुः}
{पुष्करे पुरुहूतेति केदारे मार्गदायी}

\twolineshloka
{नन्दा हिमवतःपृष्ठे गोकर्णे भद्रकर्णिका}
{स्थानेश्वरे भवानी तु बिल्वके बिल्वपत्रिका}

\twolineshloka
{श्रीशैले माधवी नाम भद्रा भद्रेश्वरे तथा}
{जया वराहशैले तु कमला कमलालये}

\twolineshloka
{रुद्रकोट्यां च रुद्राणी काली कालञ्जरे गिरौ}
{महालिङ्गे तु कपिला मर्कोटे मुकुटेश्वरी}

\twolineshloka
{शालग्रामे महादेवी शिवलिङ्गे जलप्रिया}
{मायापुर्यां कुमारी तु सन्ताने ललिता तथा}

\twolineshloka
{उत्पलाक्षी सहस्राक्षे कमलाक्षे महोत्पला}
{गङ्गायां मङ्गला नाम विमला पुरुषोत्तमे}

\twolineshloka
{विपाशायाममोघाक्षी पाटला पुण्ड्रवर्धने}
{नारायणी सुपार्श्वे तु विकूटे भद्रसुन्दरी}

\twolineshloka
{विपुले विपुला नाम कल्याणी मलयाचले}
{कोटवी कोटितीर्थे तु सुगन्धा माधवे वने}

\twolineshloka
{कुब्जाम्रके त्रिसन्ध्या तु गङ्गाद्वारे रतिप्रिया}
{शिवकुण्डे सुनन्दा तु नन्दिनी देविकातटे}

\twolineshloka
{रुक्मिणी द्वारवत्यां तु राधा वृन्दावने वने}
{देविका मथूरायां तु पाताले परमेश्वरी}

\twolineshloka
{चित्रकूटे तथा सीता विन्ध्ये विन्ध्याधिवासिनी}
{सह्याद्रावेकवीरा तु हरिश्चन्द्रे तु चन्द्रिका}

\twolineshloka
{रमणा रामतीर्थे तु यमुनायां मृगावती}
{करवीरे महालक्ष्मीरुमादेवी विनायके}

\twolineshloka
{अरोगा वैद्यनाथे तु महाकाले महेश्वरी}
{अभयेत्युष्णतीर्थेषु चामृता विन्ध्यकन्दरे}

\twolineshloka
{माण्डव्ये माण्डवी नाम स्वाहा महेश्वरे पुरे}
{छागलाण्डे प्रचण्डा तु चण्डिका मकरन्दके}

\twolineshloka
{सोमेश्वरे वरारोहा प्रभासे पुष्करावती}
{देवमाता सरस्वत्यां पारावारतटे मता}

\twolineshloka
{महालये महाभागा पयोष्ण्यां पिङ्गलेश्वरी}
{सिंहिका कृतशौचे तु कार्तिकेये यशस्करी}

\twolineshloka
{उत्पलावर्तके लोला सुभद्रा शोणसङ्गमे}
{माता सिद्धपुरे लक्ष्मीरङ्गना भरताश्रमे}

\twolineshloka
{जालन्धरे विश्वमुखी तारा किष्किन्धपर्वते}
{देवदारुवने पुष्टिर्मेधा काश्मीर मण्डले}

\twolineshloka
{भीमादेवी हिमाद्रौ तु पुष्टिर्विश्वेश्वरे तथा}
{कपालमोचने शुद्धिर्माता कायावरोहणे}

\twolineshloka
{शङ्खोद्धारे ध्वनिर्नाम धृतिः पिण्डारके तथा}
{काला तु चद्रभागायामच्चोदे शिवकारिणी}

\twolineshloka
{वेणायाममृता नाम बदर्यां उर्वशी तथा}
{औषधी चोत्तरकुरौ कृशद्वीपे कुशोदका}

\twolineshloka
{मन्मथा हेमकूटे तु मुकुटे सत्यवादिनी}
{अश्वत्थे वन्दनीया तु निधिर्वैश्रवणालये}

\twolineshloka
{गायत्री वेदवदने पार्वती शिवसन्निधौ}
{देवलोके तथेन्द्राणी ब्रह्मास्येषु सरस्वती}

\twolineshloka
{सूर्यबिम्बे प्रभा नाम मातॄणां वैष्णवी तथा}
{अरुन्धती सतीनां तु रामासु च तिलोत्तमा}

\twolineshloka
{चित्ते ब्रह्मकलानामशक्तिः सर्वशरीरिणाम्}
{एतदुद्देशतः प्रोक्तं नामाष्टशतमुत्तमम्}

\twolineshloka
{अष्टोत्तरं च तीर्थानां शतमेतदुदाहृतम्}
{यः पठेच्छृणुयाद्वाऽपि सर्वपापैः प्रमुच्यते}

\twolineshloka
{एषु तीर्थेषु यः कृत्वा स्नानं पश्यन्ति मां नरः}
{सर्वपापविनिर्मुक्तः कल्पं शिवपुरे वसेत्}

॥इति श्रीमत्स्यमहापुराणे श्री शक्त्यष्टोत्तरशतदिव्यस्थानीयनामस्तोत्रं सम्पूर्णम्॥