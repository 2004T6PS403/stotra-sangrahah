% !TeX program = XeLaTeX
% !TeX root = ../../shloka.tex
\sect{अन्नपूर्णाष्टोत्तरशतनामस्तोत्रम्}

\dnsub{ध्यानम्}
\fourlineindentedshloka*
{सिन्दूराभां त्रिनेत्राममृतशशिकलां खेचरीं रत्नवस्त्राम्}
{पीनोत्तुङ्गस्तनाढ्यामभिनवविलसद्यौवनारम्भरम्याम्}
{नानालङ्कारयुक्तां सरसिजनयनामिन्दुसङ्क्रान्तमूर्तिम्}
{देवीं पाशाङ्कुशाढ्यामभयवरकरामन्नपूर्णां नमामि}
\dnsub{स्तोत्रम्}


\twolineshloka
{अन्नपूर्णा शिवा देवी भीमा पुष्टिः सरस्वती}
{सर्वज्ञा पार्वती दुर्गा शर्वाणी शिववल्लभा}

\twolineshloka
{वेदविद्या महाविद्या विद्यादात्री विशारदा}
{कुमारी त्रिपुरा बाला लक्ष्मीः श्रीर्भयहारिणी}

\twolineshloka
{भवानी विष्णुजननी ब्रह्मादिजननी तथा}
{गणेशजननी शक्तिः कुमारजननी शुभा}

\twolineshloka
{भोगप्रदा भगवती भक्ताभीष्टप्रदायिनी}
{भवरोगहरा भव्या शुभ्रा परममङ्गला}

\twolineshloka
{भवानी चञ्चला गौरी चारुचन्द्रकलाधरा}
{विशालाक्षी विश्वमाता विश्ववन्द्या विलासिनी}

\twolineshloka
{आर्या कल्याणनिलाया रुद्राणी कमलासना}
{शुभप्रदा शुभावर्ता वृत्तपीनपयोधरा}

\twolineshloka
{अम्बा संहारमथनी मृडानी सर्वमङ्गला}
{विष्णुसंसेविता सिद्धा ब्रह्माणी सुरसेविता}

\twolineshloka
{परमानन्ददा शान्तिः परमानन्दरूपिणी}
{परमानन्दजननी परानन्दप्रदायिनी}

\twolineshloka
{परोपकारनिरता परमा भक्तवत्सला}
{पूर्णचन्द्राभवदना पूर्णचन्द्रनिभांशुका}

\twolineshloka
{शुभलक्षणसम्पन्ना शुभानन्दगुणार्णवा}
{शुभसौभाग्यनिलया शुभदा च रतिप्रिया}

\twolineshloka
{चण्डिका चण्डमथनी चण्डदर्पनिवारिणी}
{मार्तण्डनयना साध्वी चन्द्राग्निनयना सती}

\twolineshloka
{पुण्डरीकहरा पूर्णा पुण्यदा पुण्यरूपिणी}
{मायातीता श्रेष्ठमाया श्रेष्ठधर्मात्मवन्दिता}

\twolineshloka
{असृष्टिः सङ्गरहिता सृष्टिहेतुः कपर्दिनी}
{वृषारूढा शूलहस्ता स्थितिसंहारकारिणी}

\twolineshloka
{मन्दस्मिता स्कन्दमाता शुद्धचित्ता मुनिस्तुता}
{महाभगवती दक्षा दक्षाध्वरविनाशिनी}

\twolineshloka
{सर्वार्थदात्री सावित्री सदाशिवकुटुम्बिनी}
{नित्यसुन्दरसर्वाङ्गी सच्चिदानन्दलक्षणा}

\twolineshloka
{नाम्नामष्टोत्तरशतमम्बायाः पुण्यकारणम्}
{सर्वसौभाग्यसिद्ध्यर्थं जपनीयं प्रयत्नतः}

\twolineshloka
{एतानि दिव्यनामानि श्रुत्वा ध्यात्वा निरन्तरम्}
{स्तुत्वा देवीं च सततं सर्वान् कामानवाप्नुयात्}

{॥इति श्रीशिवरहस्ये श्री अन्नपूर्णाष्टोत्तरशतनामस्तोत्रं सम्पूर्णम्॥}
