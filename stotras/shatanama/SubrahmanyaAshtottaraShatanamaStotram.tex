% !TeX program = XeLaTeX
% !TeX root = ../../shloka.tex

\sect{सुब्रह्मण्याष्टोत्तरशतनामस्तोत्रम्}
\dnsub{ध्यानम्}
\twolineshloka*
{शक्तिहस्तं विरूपाक्षं शिखिवाहं षडाननम्}
{दारुणं रिपुरोगघ्नं भावये कुक्कुटध्वजम्}

\dnsub{स्तोत्रम्}
\twolineshloka
{स्कन्दो गुहः षण्मुखश्च फालनेत्रसुतः प्रभुः}
{पिङ्गलः कृत्तिकासूनुः शिखिवाहो द्विषड्भुजः}%१॥

\twolineshloka
{द्विषण्णेत्रः शक्तिधरः पिशिताशप्रभञ्जनः}
{तारकासुरसंहारी रक्षोबलविमर्दनः}%२॥

\twolineshloka
{मत्तः प्रमत्तोन्मत्तश्च सुरसैन्यसुरक्षकः}
{देवसेनापतिः प्राज्ञः कृपालुर्भक्तवत्सलः}%३॥

\twolineshloka
{उमासुतः शक्तिधरः कुमारः क्रौञ्चदारणः}
{सेनानीरग्निजन्मा च विशाखः शङ्करात्मजः}%४॥

\twolineshloka
{शिवस्वामी गणस्वामी सर्वस्वामी सनातनः}
{अनन्तमूर्तिरक्षोभ्यः पार्वतीप्रियनन्दनः}%५॥

\twolineshloka
{गङ्गासुतः शरोद्भूत आहूतः पावकात्मजः}
{जृम्भः प्रजृम्भ उज्जृम्भः कमलासनसंस्तुतः}%६॥

\twolineshloka
{एकवर्णो द्विवर्णश्च त्रिवर्णः सुमनोहरः}
{चतुर्वर्णः पञ्चवर्णः प्रजापतिरहःपतिः}%७॥

\twolineshloka
{अग्निगर्भः शमीगर्भो विश्वरेता सुरारिहा}
{हरिद्वर्णः शुभकरो वटुश्च पटुवेषभृत्}%८॥

\twolineshloka
{पूषा गभस्तिर्गहनश्चन्द्रवर्णः कलाधरः}
{मायाधरो महामायी कैवल्यः शङ्करात्मजः}%९॥

\twolineshloka
{विश्वयोनिरमेयात्मा तेजोयोनिरनामयः}
{परमेष्ठी परब्रह्म वेदगर्भो विराट्सुतः}%१०॥

\twolineshloka
{पुलिन्दकन्याभर्ता च महासारस्वतावृतः}
{आश्रिताखिलदाता च चोरघ्नो रोगनाशनः}%११॥

\twolineshloka
{अनन्तमूर्तिरानन्दः शिखण्डी-कृतकेतनः}
{डम्भः परमडम्भश्च महाडम्भो वृषाकपिः}%१२॥

\twolineshloka
{कारणोत्पत्ति-देहश्च कारणातीत-विग्रहः}
{अनीश्वरोऽमृतः प्राणः प्राणायामपरायणः}%१३॥

\threelineshloka
{विरुद्धहन्तो वीरघ्नो रक्तश्यामगलोऽपि च}
{सुब्रह्मण्यो गुहः प्रीतो ब्रह्मण्यो ब्राह्मणप्रियः}
{वंशवृद्धिकरो वेदवेद्योऽक्षयफलप्रदः}%१४॥

॥इति श्री सुब्रह्मण्याष्टोत्तरशतनामस्तोत्रं सम्पूर्णम्॥