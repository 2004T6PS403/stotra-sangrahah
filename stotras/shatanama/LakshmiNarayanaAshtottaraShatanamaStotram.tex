%!TeX program = Xelatex
%!TeX root = ../../shloka.tex
\sect{लक्ष्मीनारायणाष्टोत्तरशतनामस्तोत्रम्}

\twolineshloka
{श्रीर्विष्णुः कमला शार्ङ्गी लक्ष्मीर्वैकुण्ठनायकः}
{पद्मालया चतुर्बाहुः क्षीराब्धितनयाऽच्युतः}% १}%}

\twolineshloka
{इन्दिरा पुण्डरीकाक्षो रमा गरुडवाहनः}
{भार्गवी शेषपर्यङ्को विशालाक्षी जनार्दनः}% २}%}

\twolineshloka
{स्वर्णाङ्गी वरदो देवी हरिरिन्दुमुखी प्रभुः}
{सुन्दरी नरकध्वंसी लोकमाता मुरान्तकः}% ३}%}

\twolineshloka
{भक्तप्रिया दानवारिरम्बिका मधुसूदनः}
{वैष्णवी देवकीपुत्रो रुक्मिणी केशिमर्दनः}% ४}%}

\twolineshloka
{वरलक्ष्मी जगन्नाथः कीरवाणी हलायुधः}
{नित्या सत्यव्रतो गौरी शौरिः कान्ता सुरेश्वरः}

\twolineshloka
{नारायणी हृषीकेशः पद्महस्ता त्रिविक्रमः}
{माधवी पद्मनाभश्च स्वर्णवर्णा निरीश्वरः}

\twolineshloka
{सती पीताम्बरः शान्ता वनमाली क्षमाऽनघः}
{जयप्रदा बलिध्वंसी वसुधा पुरुषोत्तमः}

\twolineshloka
{राज्यप्रदाऽखिलाधारो माया कंसविदारणः}
{महेश्वरी महादेवो परमा पुण्यविग्रहः}


\twolineshloka
{रमा मुकुन्दः सुमुखी मुचुकुन्दवरप्रदः}
{वेदवेद्याऽब्धि-जामाता सुरूपाऽर्केन्दुलोचनः }

\twolineshloka
{पुण्याङ्गना पुण्यपादो पावनी पुण्यकीर्तनः}
{विश्वप्रिया विश्वनाथो वाग्रूपी वासवानुजः}

\twolineshloka
{सरस्वती स्वर्णगर्भो गायत्री गोपिकाप्रियः}
{यज्ञरूपा यज्ञभोक्ता भक्ताभीष्टप्रदा गुरुः}

\twolineshloka
{स्तोत्रक्रिया स्तोत्रकारः सुकुमारी सवर्णकः}
{मानिनी मन्दरधरो सावित्री जन्मवर्जितः}

\twolineshloka
{मन्त्रगोप्त्री महेष्वासो योगिनी योगवल्लभः}
{जयप्रदा जयकरो रक्षित्री सर्वरक्षकः}

\twolineshloka
{अष्टोत्तरशतं नाम्नां लक्ष्म्या नारायणस्य च}
{यः पठेत् प्रातरुत्थाय सर्वदा विजयी भवेत्}

॥इति श्री लक्ष्मीनारायणाष्टोत्तरशतनामस्तोत्रं सम्पूर्णम्॥