% !TeX document-id = {04540bd7-0671-43b8-9d03-b7b92320c9d3}
% !TeX program = XeLaTeX
% !TeX root = ../../shloka.tex

\sect{आदित्याष्टोत्तरशतनामस्तोत्रम्}

\twolineshloka
{नवग्रहाणां सर्वेषां सूर्यादीनां पृथक् पृथक्}
{पीडा च दुःसहा राजन् जायते सततं नृणाम्}

\twolineshloka
{पीडानाशाय राजेन्द्र नामानि शृणु भास्वतः}
{सूर्यादीनां च सर्वेषां पीडा नश्यति शृण्वतः}

\twolineshloka
{आदित्यः सविता सूर्यः पूषाऽर्कः शीघ्रगो रविः}
{भगस्त्वष्टाऽर्यमा हंसो हेलिस्तेजोनिधिर्हरिः}

\twolineshloka
{दिननाथो दिनकरः सप्तसप्तिः प्रभाकरः}
{विभावसुर्वेदकर्ता वेदाङ्गो वेदवाहनः}

\twolineshloka
{हरिदश्वः कालवक्त्रः कर्मसाक्षी जगत्पतिः}
{पद्मिनीबोधको भानुर्भास्करः करुणाकरः}

\twolineshloka
{द्वादशात्मा विश्वकर्मा लोहिताङ्गस्तमोनुदः}
{जगन्नाथोऽरविन्दाक्षः कालात्मा कश्यपात्मजः}

\twolineshloka
{भूताश्रयो ग्रहपतिः सर्वलोकनमस्कृतः}
{जपाकुसुमसङ्काशो भास्वानदितिनन्दनः}

\twolineshloka
{ध्वान्तेभसिंहः सर्वात्मा लोकनेत्रो विकर्तनः}
{मार्तण्डो मिहिरः सूरस्तपनो लोकतापनः}

\twolineshloka
{जगत्कर्ता जगत्साक्षी शनैश्चरपिता जयः}
{सहस्ररश्मिस्तरणिर्भगवान् भक्तवत्सलः}

\twolineshloka
{विवस्वानादिदेवश्च देवदेवो दिवाकरः}
{धन्वन्तरिर्व्याधिहर्ता दद्रुकुष्ठविनाशनः}

\twolineshloka
{चराचरात्मा मैत्रेयोऽमितो विष्णुर्विकर्तनः}
{लोकशोकापहर्ता च कमलाकर आत्मभूः}

\twolineshloka
{नारायणो महादेवो रुद्रः पुरुष ईश्वरः}
{जीवात्मा परमात्मा च सूक्ष्मात्मा सर्वतोमुखः}

\twolineshloka
{इन्द्रोऽनलो यमश्चैव नैरृतो वरुणोऽनिलः}
{श्रीद ईशान इन्दुश्च भौमः सौम्यो गुरुः कविः}

\twolineshloka
{सौरिर्विधुन्तुदः केतुः कालः कालात्मको विभुः}
{सर्वदेवमयो देवः कृष्णः कामप्रदायकः}

\twolineshloka
{य एतैर्नामभिर्मर्त्यो भक्त्या स्तौति दिवाकरम्}
{सर्वपापविनिर्मुक्तः सर्वरोगविवर्जितः}

\twolineshloka
{पुत्रवान् धनवाञ्छ्रीमान् जायते स न संशयः}
{रविवारे पठेद्यस्तु नामान्येतानि भास्वतः}

\twolineshloka
{पीडाशान्तिर्भवेत्तस्य ग्रहाणां च विशेषतः}
{सद्यः सुखमवाप्नोति चाऽऽयुर्दीर्घं च नीरुजम्}

॥इति श्री भविष्यपुराणे आदित्य अष्टोत्तरशतनामस्तोत्रं सम्पूर्णम्॥