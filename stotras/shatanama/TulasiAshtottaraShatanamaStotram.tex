% !TeX program = XeLaTeX
% !TeX root = ../../shloka.tex
\sect{तुलस्यष्टोत्तरशतनामस्तोत्रम्}

% \dnsub{ध्यानम्}
% \fourlineindentedshloka*
% {}
% {}
% {}
% {}


\dnsub{स्तोत्रम्}
{तुलसी पावनी पूज्या वृन्दावननिवासिनी}
{ज्ञानदात्री ज्ञानमयी निर्मला सर्वपूजिता}% ॥१॥

\twolineshloka
{सती पतिव्रता वृन्दा क्षीराब्धिमथनोद्भवा}
{कृष्णवर्णा रोगहन्त्री त्रिवर्णा सर्वकामदा}% ॥२॥

\twolineshloka
{लक्ष्मीसखी नित्यशुद्धा सुदती भूमिपावनी}
{हरिद्रान्नैकनिरता हरिपादकृतालया}% ॥३॥

\twolineshloka
{पवित्ररूपिणी धन्या सुगन्धिन्यमृतोद्भवा}
{सुरूपाऽऽरोग्यदा तुष्टा शक्तित्रितयरूपिणी}% ॥४॥

\twolineshloka
{देवी देवर्षिसंस्तुत्या कान्ता विष्णुमनःप्रिया}
{भूतवेतालभीतिघ्नी महापातकनाशिनी}% ॥५॥

\twolineshloka
{मनोरथप्रदा मेधा कान्तिर्विजयदायिनी}
{शङ्खचक्रगदापद्मधारिणी कामरूपिणी}% ॥६॥

\twolineshloka
{अपवर्गप्रदा श्यामा कृशमध्या सुकेशिनी}
{वैकुण्ठवासिनी नन्दा बिम्बोष्ठी कोकिलस्वरा}% ॥७॥

\twolineshloka
{कपिला निम्नगाजन्मभूमिरायुष्यदायिनी}
{वनरूपा दुःखनाशिन्यविकारा चतुर्भुजा}% ॥८॥

\twolineshloka
{गरुत्मद्वाहना शान्ता दान्ता विघ्ननिवारिणी}
{श्रीविष्णुमूलिका पुष्टिस्त्रिवर्गफलदायिनी}% ॥९॥

\twolineshloka
{महाशक्तिर्महामाया लक्ष्मीवाणीसुपूजिता}
{सुमङ्गल्यर्चनप्रीता सौमङ्गल्यविवर्धिनी}% ॥१०॥

\twolineshloka
{चातुर्मास्योत्सवाराध्या विष्णुसान्निध्यदायिनी}
{उत्थानद्वादशीपूज्या सर्वदेवप्रपूजिता}% ॥११॥

\twolineshloka
{गोपीरतिप्रदा नित्या निर्गुणा पार्वतीप्रिया}
{अपमृत्युहरा राधाप्रिया मृगविलोचना}% ॥१२॥

\twolineshloka
{अम्लाना हंसगमना कमलासनवन्दिता}
{भूलोकवासिनी शुद्धा रामकृष्णादिपूजिता}% ॥१३॥

\twolineshloka
{सीतापूज्या राममनःप्रिया नन्दनसंस्थिता}
{सर्वतीर्थमयी मुक्ता लोकसृष्टिविधायिनी}% ॥१४॥

\twolineshloka
{प्रातर्दृश्या ग्लानिहन्त्री वैष्णवी सर्वसिद्धिदा}
{नारायणी सन्ततिदा मूलमृद्धारिपावनी}% ॥१५॥

\twolineshloka
{अशोकवनिकासंस्था सीताध्याता निराश्रया}
{गोमतीसरयूतीररोपिता कुटिलालका}% ॥१६॥

\twolineshloka
{अपात्रभक्ष्यपापघ्नी दानतोयविशुद्धि}
{श्रुतिधारणसुप्रीता शुभा सर्वेष्टदायिनी}% ॥१७॥

\threelineshloka
{नाम्नां शतं साष्टकं तत्तुलस्याः सर्वमङ्गलम्}
{सौमङ्गल्यप्रदं प्रातः पठेद्भक्त्या सुभाग्यदम्} ।
{लक्ष्मीपतिप्रसादेन सर्वविद्याप्रदं नृणाम्}% ॥ १८॥

{॥इति श्री तुलस्यष्टोत्तरशतनामस्तोत्रं सम्पूर्णम्॥}