% !TeX program = XeLaTeX
% !TeX root = ../../shloka.tex

\sect{शङ्कराचार्याष्टोत्तरशतनामस्तोत्रम्}

\dnsub{ध्यानम्}
\twolineshloka*
{कैलासाचल-मध्यस्थं कामिताभीष्टदायकम्}
{ब्रह्मादि-प्रार्थना-प्राप्त-दिव्यमानुष-विग्रहम्}

\twolineshloka*
{भक्तानुग्रहणैकान्त-शान्त-स्वान्त-समुज्ज्वलम्}
{संयज्ञं संयमीन्द्राणां सार्वभौमं जगद्गुरुम्}

\twolineshloka*
{किङ्करीभूतभक्तैनः पङ्कजातविशोषणम्}
{ध्यायामि शङ्कराचार्यं सर्वलोकैकशङ्करम् }

\dnsub{स्तोत्रम्}
\twolineshloka
{श्रीशङ्कराचार्यवर्यो ब्रह्मानन्दप्रदायकः}
{अज्ञानतिमिरादित्यः सुज्ञानाम्बुधिचन्द्रमा}

\twolineshloka
{वर्णाश्रमप्रतिष्ठाता श्रीमान् मुक्तिप्रदायकः}
{शिष्योपदेशनिरतो भक्ताभीष्टप्रदायकः}

\twolineshloka
{सूक्ष्मतत्त्वरहस्यज्ञः कार्याकार्यप्रबोधकः}
{ज्ञानमुद्राञ्चितकरः शिष्य-हृत्ताप-हारकः}

\twolineshloka
{परिव्राज्याश्रमोद्धर्ता सर्वतन्त्रस्वतन्त्रधीः}
{अद्वैतस्थापनाचार्यः साक्षाच्छङ्कररूपभृत्}

\twolineshloka
{षण्मतस्थापनाचार्यस्त्रयीमार्गप्रकाशकः}
{वेदवेदान्ततत्त्वज्ञो दुर्वादिमतखण्डनः}

\twolineshloka
{वैराग्यनिरतः शान्तः संसारार्णवतारकः}
{प्रसन्नवदनाम्भोजः परमार्थप्रकाशकः}

\twolineshloka
{पुराणस्मृतिसारज्ञो नित्यतृप्तो महच्छुचिः}
{नित्यानन्दो निरातङ्को निःसङ्गो निर्मलात्मकः}

\twolineshloka
{निर्ममो निरहङ्कारो विश्ववन्द्यपदाम्बुजः}
{सत्त्वप्रधानः सद्भावः सङ्ख्यातीतगुणोज्ज्वलः}

\twolineshloka
{अनघः सारहृदयः सुधीः सारस्वतप्रदः}
{सत्यात्मा पुण्यशीलश्च साङ्ख्ययोगविचक्षणः}

\twolineshloka
{तपोराशिर्महातेजा गुणत्रयविभागवित्}
{कलिघ्नः कालकर्मज्ञस्तमोगुणनिवारकः}

\twolineshloka
{भगवान् भारतीजेता शारदाह्वानपण्डितः}
{धर्माधर्मविभागज्ञो लक्ष्यभेदप्रदर्शकः}

\twolineshloka
{नादबिन्दुकलाभिज्ञो योगिहृत्पद्मभास्करः}
{अतीन्द्रिय-ज्ञाननिधिर्नित्यानित्यविवेकवान्}

\twolineshloka
{चिदानन्दश्चिन्मयात्मा परकाय-प्रवेशकृत्}
{अमानुष-चरित्राढ्यः क्षेमदायी क्षमाकरः}

\twolineshloka
{भव्यो भद्रप्रदो भूरिमहिमा विश्वरञ्जकः}
{स्वप्रकाशः सदाधारो विश्वबन्धुः शुभोदयः}

\twolineshloka
{विशालकीर्तिर्वागीशः सर्वलोकहितोत्सुकः}
{कैलासयात्रा-सम्प्राप्त-चन्द्रमौलि-प्रपूजकः}

\twolineshloka
{काञ्च्यां श्रीचक्र-राजाख्य-यन्त्रस्थापन-दीक्षितः}
{श्रीचक्रात्मक-ताटङ्क-पोषिताम्बा-मनोरथः}

\twolineshloka
{श्रीब्रह्मसूत्रोपनिषद्भाष्यादिग्रन्थकल्पकः}
{चतुर्दिक्चतुराम्नायप्रतिष्ठाता महामतिः}

\twolineshloka
{द्विसप्तति-मतोच्छेत्ता सर्वदिग्विजयप्रभुः}
{काषायवसनोपेतो भस्मोद्धूलितविग्रहः}

\twolineshloka
{ज्ञानात्मकैकदण्डाढ्यः कमण्डलुलसत्करः}
{व्याससन्दर्शनप्रीतो भगवत्पादसंज्ञकः}

\twolineshloka
{चतुःषष्टिकलाभिज्ञो ब्रह्मराक्षस-मोक्षदः}
{सौन्दर्यलहरीमुख्यबहुस्तोत्रविधायकः}

\twolineshloka
{श्रीमन्मण्डनमिश्राख्यस्वयम्भूजयसन्नुतः}
{तोटकाचार्यसम्पूज्यः पद्मपादार्चिताङ्घ्रिकः}

\twolineshloka
{हस्तामलकयोगीन्द्रब्रह्मज्ञानप्रदायकः}
{सुरेश्वरादि-षट्-शिष्य-सन्न्यासाश्रम-दायकः}

\threelineshloka
{निर्व्याजकरुणामूर्तिर्जगत्पूज्यो जगद्गुरुः}
{भेरीपटहवाद्यादिराजलक्षणलक्षितः}
{सकृत्स्मरणसन्तुष्टः सर्वज्ञो ज्ञानदायकः}

॥इति श्री शङ्कराचार्याष्टोत्तरशतनामस्तोत्रं सम्पूर्णम्॥
