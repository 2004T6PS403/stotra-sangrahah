% !TeX program = XeLaTeX
% !TeX root = ../../shloka.tex
\sect{विट्ठलाष्टोत्तरशतनामस्तोत्रम्}

\twolineshloka
{श्रुत्वा नामसहस्रं श्रीविट्ठलस्य गुणान्वितम्}
{पार्वती परिपप्रच्छ शङ्करं लोकशङ्करम्}

{श्री पार्वत्युवाच}
\twolineshloka
{देवदेव महादेव देवानुग्रहविग्रह}
{अष्टोत्तरशतं नाम्नां तस्य देवस्य मे वद}

\threelineshloka
{जनः कलिमलादिग्धोऽलसः कामाविलो जडः}
{पठाद्यस्य सकृद्वापि सौशील्यादिरतो भवेत्}
{तदहं श्रोतुमिच्छामि तवाज्ञानं न विद्यते}

{सूत उवाच}
\twolineshloka
{इत्युक्तो भार्यया शम्भुः सम्स्मरन् पुरुषोत्तमम्}
{कटिस्थितकरद्वन्द्वं जगाद नगपुत्रिकाम्}

{श्री शङ्कर उवाच}
\twolineshloka
{देवि लोकोपकाराय कृतः प्रश्नस्त्वयानघे}
{हिताय सर्वजन्तूनां नाम्नामष्टोत्तरं शतम्}

\threelineshloka
{श्रीविट्ठलस्य देवस्य वाग्मिसिद्धिप्रदं नृणाम्}
{अष्टोत्तरशतस्याहमृषिः प्रोक्तो मनीषिभिः}
{छन्दोऽनुष्टुब्देवता श्रीविट्ठलः परिकीर्तितः}

\dnsub{ध्यानम्}
\fourlineindentedshloka
{ध्यायेच्छ्रीविट्ठलाख्यं समपदकमलं पद्मपत्रायताक्षम्}
{गम्भीरस्निग्धहासं कटिनिहितकरं नीलमेघावभासम्}
{विद्युद्वासो वसानं मणिमयमुकुटं कुण्डलोद्भासिगण्डम्}
{मायूरस्रग्विभूषाभयवरसहितं कौस्तुभोद्भासिताङ्गम्}

\dnsub{स्तोत्रम्}
\twolineshloka
{{ॐ क्लीं} विट्ठलः पुण्डरीकाक्षः पुण्डरीकनिभेक्षणः}
{पुण्डरीकाश्रमपदः पुण्डरीकजलाप्लुतः}

\twolineshloka
{पुण्डरीकक्षेत्रवासः पुण्डरीकवरप्रदः}
{शारदाधिष्ठितद्वारः शारदेन्दुनिभाननः}

\twolineshloka
{नारदाधिष्ठितद्वारो नारदेशप्रपूजितः}
{भुवनाधीश्वरीद्वारो भुवनाधीश्वरीश्वरः}

\twolineshloka
{दुर्गाश्रितोत्तरद्वारो दुर्गमागमसंवृतः}
{क्षुल्लपेशीपिनद्धोरुर्गोपेष्ट्याश्लिष्टजानुकः}

\twolineshloka
{कटिस्थितकरद्वन्द्वो वरदाभयमुद्रितः}
{त्रेतातोरणपालस्थत्रिविक्रम इतीरितः}

\twolineshloka
{तित ऊक्षेत्रपोश्वत्थकोटीश्वरवरप्रदः}
{स एव करवीरस्थो नारीनारायणीति च}

\twolineshloka
{नीरासङ्गमसंस्थश्च सैकतः प्रतिमार्चितः}
{वेणुनादेन देवानां मनःश्रवणमङ्गलः}

{॥इति श्रीपद्मपुराणे पञ्चोनषष्टिसाहस्र्यां संहितायां हररहस्ये उमामहेश्वरसंवादे श्रीविट्ठलाष्टोत्तरशतनामस्तोत्रं सम्पूर्णम्॥}