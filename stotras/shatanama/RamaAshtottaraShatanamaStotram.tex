% !TeX program = XeLaTeX
% !TeX root = ../../shloka.tex

\sect{रामाष्टोत्तरशतनामस्तोत्रम्}

\dnsub{ध्यानम्}
\twolineshloka*
{श्रीराघवं दशरथात्मजमप्रमेयं सीतापतिं रघुकुलान्वयरत्नदीपम्}
{आजानुबाहुमरविन्ददलायताक्षं रामं निशाचरविनाशकरं नमामि}

\dnsub{स्तोत्रम्}\nopagebreak[4]
\twolineshloka
{श्रीरामो रामभद्रश्च रामचन्द्रश्च शाश्वतः}
{राजीवलोचनः श्रीमान् राजेन्द्रो रघुपुङ्गवः}

\twolineshloka
{जानकीवल्लभो जैत्रो जितामित्रो जनार्दनः}
{विश्वामित्रप्रियो दान्तः शरणत्राणतत्परः}

\twolineshloka
{वालिप्रमथनो वाग्मी सत्यवाक् सत्यविक्रमः}
{सत्यव्रतो व्रतधरः सदा हनुमदाश्रितः}

\twolineshloka
{कौसलेयः खरध्वंसी विराधवधपण्डितः}
{विभीषणपरित्राता हरकोदण्डखण्डनः}

\twolineshloka
{सप्ततालप्रभेत्ता च दशग्रीवशिरोहरः}
{जामदग्न्यमहादर्पदलनस्ताटकान्तकः}

\twolineshloka
{वेदान्तसारो वेदात्मा भवरोगस्य भेषजम्}
{दूषणत्रिशिरोहन्ता त्रिमूर्तिस्त्रिगुणात्मकः}

\twolineshloka
{त्रिविक्रमस्त्रिलोकात्मा पुण्यचारित्रकीर्तनः}
{त्रिलोकरक्षको धन्वी दण्डकारण्यकर्तनः}

\twolineshloka
{अहल्याशापशमनः पितृभक्तो वरप्रदः}
{जितेन्द्रियो जितक्रोधो जितामित्रो जगद्गुरुः}

\twolineshloka
{ऋक्षवानरसङ्घाती चित्रकूटसमाश्रयः}
{जयन्तत्राणवरदः सुमित्रापुत्रसेवितः}

\twolineshloka
{सर्वदेवादिदेवश्च मृतवानरजीवनः}
{मायामारीचहन्ता च महादेवो महाभुजः}

\twolineshloka
{सर्वदेवस्तुतः सौम्यो ब्रह्मण्यो मुनिसंस्तुतः}
{महायोगो महोदारः सुग्रीवेप्सितराज्यदः}

\twolineshloka
{सर्वपुण्याधिकफलः स्मृतसर्वाघनाशनः}
{अनादिरादिपुरुषो महापूरुष एव च}

\twolineshloka
{पुण्योदयो दयासारः पुराणपुरुषोत्तमः}
{स्मितवक्त्रो मितभाषी पूर्वभाषी च राघवः}

\twolineshloka
{अनन्तगुणगम्भीरो धीरोदात्तगुणोत्तमः}
{मायामानुषचारित्रो महादेवादिपूजितः}

\twolineshloka
{सेतुकृज्जितवारीशः सर्वतीर्थमयो हरिः}
{श्यामाङ्गः सुन्दरः शूरः पीतवासा धनुर्धरः}

\twolineshloka
{सर्वयज्ञाधिपो यज्वा जरामरणवर्जितः}
{शिवलिङ्गप्रतिष्ठाता सर्वापगुणवर्जितः}

\threelineshloka
{परमात्मा परं ब्रह्म सच्चिदानन्दविग्रहः}
{परञ्ज्योतिः परन्धाम पराकाशः परात्परः}
{परेशः पारगः पारः सर्वदेवात्मकः परः}

{॥इति~श्रीपद्मपुराणे~उत्तरखण्डे\\ श्रीरामाष्टोत्तरशतनामस्तोत्रं सम्पूर्णम्॥}
