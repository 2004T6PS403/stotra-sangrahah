% !TeX program = XeLaTeX
% !TeX root = ../../shloka.tex

\sect{गणेशाष्टोत्तरशतनामस्तोत्रम्}
\twolineshloka
{विनायको विघ्नराजो गौरीपुत्रो गणेश्वरः}
{स्कन्दाग्रजोऽव्ययो पूतो दक्षोऽध्यक्षो द्विजप्रियः}

\twolineshloka
{अग्निगर्भच्छिदिन्द्रश्रीप्रदो वाणीबलप्रदः}
{सर्वसिद्धिप्रदः शर्वतनयः शर्वरीप्रियः}

\twolineshloka
{सर्वात्मकः सृष्टिकर्ता देवोऽनेकार्चितः शिवः}
{शुद्धो बुद्धिप्रियः शान्तो ब्रह्मचारी गजाननः}

\twolineshloka
{द्वैमात्रेयो मुनिस्तुत्यो भक्तविघ्नविनाशनः}
{एकदन्तश्चतुर्बाहुश्चतुरः शक्तिसंयुतः}

\twolineshloka
{लम्बोदरः शूर्पकर्णो हरिर्ब्रह्मविदुत्तमः}
{कालो ग्रहपतिः कामी सोमसूर्याग्निलोचनः}

\twolineshloka
{पाशाङ्कुशधरश्चण्डो गुणातीतो निरञ्जनः}
{अकल्मषः स्वयंसिद्धः सिद्धार्चितपदाम्बुजः}

\twolineshloka
{बीजपूरफलासक्तो वरदः शाश्वतः कृतिः}
{विद्वत्प्रियो वीतभयो गदी चक्रीक्षुचापधृत्}

\twolineshloka
{श्रीदोऽजोत्पलकरः श्रीपतिः स्तुतिहर्षितः}
{कुलाद्रिभेत्ता जटिलः कलिकल्मषनाशनः}

\twolineshloka
{चन्द्रचूडामणिः कान्तः पापहारी समाहितः}
{आश्रितः श्रीकरः सौम्यो भक्तवाञ्छितदायकः}

\twolineshloka
{शान्तः कैवल्यसुखदः सच्चिदानन्दविग्रहः}
{ज्ञानी दयायुतो दान्तो ब्रह्म द्वेषविवर्जितः}

\twolineshloka
{प्रमत्तदैत्यभयदः श्रीकण्ठो विबुधेश्वरः}
{रमार्चितो विधिर्नागराजयज्ञोपवीतवान्}

\twolineshloka
{स्थूलकण्ठ: स्वयङ्कर्ता सामघोषप्रियो परः}
{स्थूलतुण्डोऽग्रणीर्धीरो वागीशः सिद्धिदायकः}

\twolineshloka
{दूर्वाबिल्वप्रियोऽव्यक्तमूर्तिरद्भुतमूर्तिमान्}
{शैलेन्द्रतनुजोत्सङ्गखेलनोत्सुकमानसः}

\threelineshloka
{स्वलावण्यसुधासारजितमन्मथविग्रहः}
{समस्तजगदाधारो मायी मूषिकवाहनः}
{हृष्टस्तुष्टः प्रसन्नात्मा सर्वसिद्धिप्रदायकः}

{॥इति श्री गणेशाष्टोत्तरशतनामस्तोत्रं सम्पूर्णम्॥}
