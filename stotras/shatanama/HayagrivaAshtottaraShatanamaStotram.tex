% !TeX program = XeLaTeX
% !TeX root = ../../shloka.tex

\sect{हयग्रीवाष्टोत्तरशतनामस्तोत्रम्}
\twolineshloka
{हयग्रीवो महाविष्णुः केशवो मधुसूदनः}
{गोविन्दः पुण्डरीकाक्षो विष्णुर्विश्वम्भरो हरिः}
\twolineshloka
{आदित्यः सर्ववागीशः सर्वाधारः सनातनः}
{निराधारो निराकारो निरीशो निरुपद्रवः}
\twolineshloka
{निरञ्जनो  निष्कलङ्को नित्यतृप्तो निरामयः}
{चिदानन्दमयः साक्षी शरण्यः सर्वदायकः}
\twolineshloka
{श्रीमान् लोकत्रयाधीशः शिवः सारस्वतप्रदः}
{वेदोद्धर्त्ता वेदनिधिर्वेदवेद्यः प्रभूतनः}
\twolineshloka
{पूर्णः पूरयिता पुण्य़ः पुण्यकीर्तिः परात्परः}
{परमात्मा परञ्ज्योतिः परेशः पारगः परः}
\twolineshloka
{र्सर्ववेदात्मको विद्वान् वेदवेदाङ्गपारगः}
{सकलोपनिषद्वेद्यो निष्कलः सर्वशास्त्रकृत्}
\twolineshloka
{अक्षमालाज्ञानमुद्रायुक्तहस्तो वरप्रदः}
{पुराणपुरुषः श्रेष्ठः शरण्यः परमेश्वरः }
\twolineshloka
{शान्तो दान्तो जितक्रोधो जितामित्रो जगन्मयः}
{जगन्मृत्युहरो जीवो जयदो जाड्यनाशनः}
\twolineshloka
{जनप्रियो जनस्तुत्यो जापकप्रियकृत्प्रभुः}
{विमलो विश्वरूपश्च विश्वगोप्ता विधिस्तुतः}
\twolineshloka
{विधीन्द्रशिवसंस्तुत्यः शान्तिदः क्षान्तिपारगः}
{श्रेयप्रदः श्रुतिमयः श्रेयसां पतिरीश्वरः}
\twolineshloka
{अच्युतोऽनन्तरूपश्च प्राणदः पृथिवीपतिः}
{अव्यक्तो व्यक्तरूपश्च सर्वसाक्षी तमोहरः}
\twolineshloka
{अज्ञाननाशको ज्ञानी पूर्णचन्द्रसमप्रभः}
{ज्ञानदो वाक्पतिर्योगी योगीशः सर्वकामदः}
\twolineshloka
{महायोगी महामौनी मौनीशः श्रेयसां पतिः}
{हंसः परमहंसश्च विश्वगोप्ता विराट् स्वराट्}
\twolineshloka
{शुद्धस्फटिकसङ्काशो जटामण्डलसंयुतः}
{आदिमध्यान्तरहितः सर्ववागीश्वरेश्वरः}

॥इति श्री हयग्रीवाष्टोत्तरशतनामस्तोत्रं सम्पूर्णम्॥
