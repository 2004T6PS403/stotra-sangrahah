% !TeX program = XeLaTeX
% !TeX root = ../../shloka.tex
\sect{गङ्गाष्टोत्तरशतनामस्तोत्रम्}

\dnsub{ध्यानम्}
\fourlineindentedshloka*
{सितमकरनिषण्णां शुभ्रवर्णां त्रिनेत्राम्}
{करधृतकलशोद्यत्सोत्पलामत्यभीष्टाम्}
{विधिहरिहररूपां सेन्दुकोटीरचूडाम्}
{कलितसितदुकूलां जाह्नवीं तां नमामि}

\dnsub{स्तोत्रम्}

\uvacha{श्री नारद उवाच}
\twolineshloka
{गङ्गा नाम परं पुण्यं कथितं परमेश्वर}
{नामानि कति शस्तानि गङ्गायाः प्रणिशंस मे}

\uvacha{श्री महादेव उवाच}
\twolineshloka
{नाम्नां सहस्रमध्ये तु नामाष्टशतमुत्तमम्}
{जाह्नव्या मुनिशार्दूल तानि मे शृणु तत्त्वतः}

\twolineshloka
{गङ्गा त्रिपथगा देवी शम्भुमौलिविहारिणी}
{जाह्नवी पापहन्त्री च महापातकनाशिनी}

\twolineshloka
{पतितोद्धारिणी स्रोतस्वती परमवेगिनी}
{विष्णुपादाब्जसम्भूता विष्णुदेहकृतालया}

\twolineshloka
{स्वर्गाब्धिनिलया साध्वी स्वर्णदी सुरनिम्नगा}
{मन्दाकिनी महावेगा स्वर्णशृङ्गप्रभेदिनी}

\twolineshloka
{देवपूज्यतमा दिव्या दिव्यस्थान निवासिनी}
{सुचारुनीररुचिरा महापर्वतभेदिनी}

\twolineshloka
{भागीरथी भगवती महामोक्षप्रदायिनी}
{सिन्धुसङ्गगता शुद्धा रसातलनिवासिनी}

\twolineshloka
{महाभोगा भोगवती सुभगानन्ददायिनी}
{महापापहरा पुण्या परमाह्लाददायिनी}

\twolineshloka
{पार्वती शिवपत्नी च शिवशीर्षगतालया}
{शम्भोर्जटामध्यगता निर्मला निर्मलानना}

\twolineshloka
{महाकलुषहन्त्री च जह्नुपुत्री जगत्प्रिया}
{त्रैलोक्यपावनी पूर्णा पूर्णब्रह्मस्वरूपिणी}

\twolineshloka
{जगत्पूज्यतमा चारुरूपिणी जगदम्बिका}
{लोकानुग्रहकर्त्री च सर्वलोकदयापरा}

\twolineshloka
{याम्यभीतिहरा तारा पारा संसारतारिणी}
{ब्रह्माण्डभेदिनी ब्रह्मकमण्डलुकृतालया}

\twolineshloka
{सौभाग्यदायिनी पुंसां निर्वाणपददायिनी}
{अचिन्त्यचरिता चारुरुचिरातिमनोहरा}

\twolineshloka
{मर्त्यस्था मृत्युभयहा स्वर्गमोक्षप्रदायिनी}
{पापापहारिणी दूरचारिणी वीचिधारिणी}

\twolineshloka
{कारुण्यपूर्णा करुणामयी दुरितनाशिनी}
{गिरिराजसुता गौरीभगिनी गिरिशप्रिया}

\twolineshloka
{मेनकागर्भसम्भूता मैनाकभगिनीप्रिया}
{आद्या त्रिलोकजननी त्रैलोक्यपरिपालिनी}

\twolineshloka
{तीर्थश्रेष्ठतमा श्रेष्ठा सर्वतीर्थमयी शुभा}
{चतुर्वेदमयी सर्वा पितृसन्तृप्तिदायिनी}

\twolineshloka
{शिवदा शिवसायुज्यदायिनी शिववल्लभा}
{तेजस्विनी त्रिनयना त्रिलोचनमनोरमा}

\twolineshloka
{सप्तधारा शतमुखी सगरान्वयतारिणी}
{मुनिसेव्या मुनिसुता जह्नुजानुप्रभेदिनी}

\twolineshloka
{मकरस्था सर्वगता सर्वाशुभनिवारिणी}
{सुदृश्या चाक्षुषीतृप्तिदायिनी मकरालया}

\twolineshloka
{सदानन्दमयी नित्यानन्ददा नगपूजिता}
{सर्वदेवाधिदेवैश्च परिपूज्यपदाम्बुजा}

\twolineshloka
{एतानि मुनिशार्दूल नामानि कथितानि ते}
{शस्तानि जाह्नवीदेव्याः सर्वपापहराणि च}

\twolineshloka
{य इदं पठते भक्त्या प्रातरुत्थाय नारद}
{गङ्गायाः परमं पुण्यं नामाष्टशतमेव हि}

\twolineshloka
{तस्य पापानि नश्यन्ति ब्रह्महत्यादिकान्यपि}
{आरोग्यमतुलं सौख्यं लभते नात्र संशयः}

\twolineshloka
{यत्र कुत्रापि संस्नायात्पठेत्स्तोत्रमनुत्तमम्}
{तत्रैव गङ्गास्नानस्य फलं प्राप्नोति निश्चितम्}

\twolineshloka
{प्रत्यहं प्रपठेदेतद् गङ्गानामशताष्टकम्}
{सोऽन्ते गङ्गामनुप्राप्य प्रयाति परमं पदम्}

\twolineshloka
{गङ्गायां स्नानसमये यः पठेद्भक्तिसंयुतः}
{सोऽश्वमेधसहस्राणां फलमाप्नोति मानवः}

\twolineshloka
{गवामयुतदानस्य यत्फलं समुदीरितम्}
{तत्फलं समवाप्नोति पञ्चम्यां प्रपठन्नरः}

\twolineshloka
{कार्त्तिक्यां पौर्णमास्यां तु स्नात्वा सगरसङ्गमे}
{यः पठेत्स महेशत्वं याति सत्यं न संशयः}

{॥इति~श्रीमद्भागवते~महापुराणे श्रीगङ्गाष्टोत्तरशतनामस्तोत्रं सम्पूर्णम्॥}
\closesection