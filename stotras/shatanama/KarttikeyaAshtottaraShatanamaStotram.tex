% !TeX program = XeLaTeX
% !TeX root = ../../shloka.tex
\sect{कार्त्तिकेयाष्टोत्तरशतनामस्तोत्रम्}

\dnsub{ध्यानम्}
\fourlineindentedshloka*
{सिन्दूरारुणकान्तिमिन्दुवदनं केयूरहारादिभिः}
{दिव्यैराभरणैर्विभूषिततनुं स्वर्गस्य सौख्यप्रदम्}
{अम्भोजाभयशक्तिकुक्कुटधरं रत्नाङ्गरागांशुकम्}
{सुब्रह्मण्यमुपास्महे प्रणमतां भीतिप्रणाशोद्यतम्}

\dnsub{स्तोत्रम्}
\twolineshloka
{विश्वामित्रस्तु भगवान् कुमारं शरणं गतः}
{स्तवं दिव्यं सम्प्रचक्रे महासेनस्य चापि सः}

\twolineshloka
{अष्टोत्तरशतनाम्नां शृणु त्वं तानि फाल्गुन}
{जपेन येषां पापानि यान्ति ज्ञानमवाप्नुयात्}

\twolineshloka
{त्वं ब्रह्मवादी त्वं ब्रह्मा ब्रह्मब्राह्मणवत्सलः}
{ब्रह्मण्यो ब्रह्मदेवश्च ब्रह्मदो ब्रह्मसङ्ग्रहः}

\twolineshloka
{त्वं परं परमं तेजो मङ्गलानां च मङ्गलम्}
{अप्रमेयगुणश्चैव मन्त्राणां मन्त्रगो भवान्}

\twolineshloka
{त्वं सावित्रीमयो देवः सर्वत्रैवापराजितः}
{मन्त्रः सर्वात्मको देवः षडक्षरवतां वरः}

\twolineshloka
{गवां पुत्रः सुरारिघ्नः सम्भवो भवभावनः}
{पिनाकी शत्रुहा चैव कूटः स्कन्दः सुराग्रणीः}

\twolineshloka
{द्वादशो भूर्भुवो भावी भुवःपुत्रो नमस्कृतः}
{नागराजः सुधर्मात्मा नाकपृष्ठः सनातनः}

\twolineshloka
{हेमगर्भो महागर्भो जयश्च विजयेश्वरः}
{त्वं कर्ता त्वं विधाता च नित्योऽनित्योऽरिमर्दनः}

\twolineshloka
{महासेनो महातेजा वीरसेनश्चमूपतिः}
{सुरसेनः सुराध्यक्षो भीमसेनो निरामयः}

\twolineshloka
{शौरिर्यदुर्महातेजा वीर्यवान् सत्यविक्रमः}
{तेजोगर्भोऽसुररिपुः सुरमूर्तिः सुरोर्जितः}

\twolineshloka
{कृतज्ञो वरदः सत्यः शरण्यः साधुवत्सलः}
{सुव्रतः सूर्यसङ्काशो वह्निगर्भो रणोत्सुकः}

\twolineshloka
{पिप्पली शीघ्रगो रौद्रिर्गाङ्गेयो रिपुदारणः}
{कार्त्तिकेयः प्रभुः क्षान्तो नीलदंष्ट्रो महामनाः}

\twolineshloka
{निग्रहो निग्रहाणां च नेता त्वं दैत्यसूदनः}
{प्रग्रहः परमानन्दः क्रोधघ्नस्तारकोऽच्छिदः}

\twolineshloka
{कुक्कुटी बहुलो वादी कामदो भूरिवर्धनः}
{अमोघोऽमृतदो ह्यग्निः शत्रुघ्नः सर्वबोधनः}

\twolineshloka
{अनघो ह्यमरः श्रीमानुन्नतो ह्यग्निसम्भवः}
{पिशाचराजः सूर्याभः शिवात्मा त्वं सनातनः}

\twolineshloka
{एवं स सर्वभूतानां संस्तुतः परमेश्वरः}
{नाम्नामष्टशतेनायं विश्वामित्रमहर्षिणा}

\twolineshloka
{प्रसन्नमूर्तिराहेदं मुनीन्द्र व्रियतामिति}
{मम त्वया द्विजश्रेष्ठ स्तुतिरेषा विनिर्मिता}

\twolineshloka
{भविष्यति मनोभीष्टप्राप्तये प्राणिनां भुवि}
{विवर्धते कुले लक्ष्मीस्तस्य यः प्रपठेदिमम्}

\twolineshloka
{न राक्षसाः पिशाचा वा न भूतानि न चाऽऽपदः}
{विघ्नकारीणि तद्गेहे यत्रैवं संस्तुवन्ति माम्}

\twolineshloka
{दुःस्वप्नं न च पश्येत्स बद्धो मुच्येत बन्धनात्}
{स्तवस्यास्य प्रभावेण दिव्यभावः पुमान्भवेत्}

{॥इति श्रीस्कन्दमहापुराणे माहेश्वरखण्डान्तर्गते कुमारिकाखण्डे श्रीकार्त्तिकेयाष्टोत्तरशतनामस्तोत्रं सम्पूर्णम्॥}