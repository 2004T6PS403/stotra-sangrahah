% !TeX program = XeLaTeX
% !TeX root = ../../shloka.tex
\sect{विष्णोरष्टोत्तरशतदिव्यस्थानीयनामस्तोत्रम्}

\twolineshloka
{अष्टोत्तरशतस्थानेष्वाविर्भूतं जगत्पतिम्}
{नमामि जगतामीशं नारायणमनन्यधीः}

\twolineshloka
{श्रीवैकुण्ठे वासुदेवमामोदे कर्षणाह्वयम्}
{प्रद्युम्नं च प्रमोदाख्ये सम्मोदे चानिरुद्धकम्}

\twolineshloka
{सत्यलोके तथा विष्णुं पद्माक्षं सूर्यमण्डले}
{क्षीराब्धौ शेषशयनं श्वेतद्वीपेतु तारकम्}

\twolineshloka
{नारायणं बदर्याख्ये नैमिषे हरिमव्ययम्}
{शालग्रामं हरिक्षेत्रे अयोध्यायां रघूत्तमम्}

\twolineshloka
{मथुरायां बालकृष्णं मायायां मधुसूदनम्}
{काश्यां तु भोगशयनमवन्त्यामवनीपतिम्}

\twolineshloka
{द्वारवत्यां यादवेन्द्रं व्रजे गोपीजनप्रियम्}
{वृन्दावने नन्दसूनुं गोविन्दं कालियह्रदे}

\twolineshloka
{गोवर्धने गोपवेषं भवघ्नं भक्तवत्सलम्}
{गोमन्तपर्वते शौरिं हरिद्वारे जगत्पतिम्}

\twolineshloka
{प्रयागे माधवं चैव गयायां तु गदाधरम्}
{गङ्गासागरगे विष्णुं चित्रकूटे तु राघवम्}

\twolineshloka
{नन्दिग्रामे राक्षसघ्नं प्रभासे विश्वरूपिणम्}
{श्रीकूर्मे कूर्ममचलं नीलाद्रौ पुरुषोत्तमम्}

\twolineshloka
{सिंहाचले महासिंहं गदिनं तुलसीवने}
{घृतशैले पापहरं श्वेताद्रौ सिंहरूपिणम्}

\twolineshloka
{योगानन्दं धर्मपुर्यां काकुले त्वान्ध्रनायकम्}
{अहोबिले गारुडाद्रौ हिरण्यासुरमर्दनम्}

\twolineshloka
{विट्ठलं पाण्डुरङ्गे तु वेङ्कटाद्रौ रमासखम्}
{नारायणं यादवाद्रौ नृसिंहं घटिकाचले}

\twolineshloka
{वरदं वारणगिरौ काञ्च्यां कमललोचनम्}
{यथोक्तकारिणं चैव परमेशपुराश्रयम्}

\twolineshloka
{पाण्डवानां तथा दूतं त्रिविक्रममथोन्नतम्}
{कामासिक्यां नृसिंहं च तथाष्टभुजसज्ञकम्}

\twolineshloka
{मेघाकारं शुभाकारं शेषाकारं तु शोभनम्}
{अन्तरा शितिकण्ठस्य कामकोट्यां शुभप्रदम्}

\twolineshloka
{कालमेघं खगारूढं कोटिसूर्यसमप्रभम्}
{दिव्यं दीपप्रकाशं च देवानामधिपं मुने}

\twolineshloka
{प्रवालवर्णं दीपाभं काञ्च्यामष्टादशस्थितम्}
{श्रीगृध्रसरसस्तीरे भान्तं विजयराघवम्}

\twolineshloka
{वीक्षारण्ये महापुण्ये शयानं वीरराघवम्}
{तोताद्रौ तुङ्गशयनं गजार्तिघ्नं गजस्थले}

\twolineshloka
{महाबलं बलिपुरे भक्तिसारे जगत्पतिम्}
{महावराहं श्रीमुष्णे महीन्द्रे पद्मलोचनम्}

\twolineshloka
{श्रीरङ्गे तु जगन्नाथं श्रीधामे जानकीप्रियम्}
{सारक्षेत्रे सारनाथं खण्डने हरचापहम्}

\twolineshloka
{श्रीनिवासस्थले पूर्णं सुवर्णं स्वर्णमन्दिरे}
{व्याघ्रपुर्यां महाविष्णुं भक्तिस्थाने तु भक्तिदम्}

\twolineshloka
{श्वेतह्रदे शान्तमूर्तिमग्निपुर्यां सुरप्रियम्}
{भर्गाख्यं भार्गवस्थाने वैकुण्ठाख्ये तु माधवम्}

\twolineshloka
{पुरुषोत्तमे भक्तसखं चक्रतीर्थे सुदर्शनम्}
{कुम्भकोणे चक्रपाणिं भूतस्थाने तु शार्ङ्गिणम्}

\twolineshloka
{कपिस्थले गजार्तिघ्नं गोविन्दं चित्रकूटके}
{अनुत्तमं चोत्तमायां श्वेताद्रौ पद्मलोचनम्}

\twolineshloka
{पार्थस्थले परब्रह्म कृष्णाकोट्यां मधुद्विषम्}
{नन्दपुर्यां महानन्दं वृद्धपुर्यां वृषाश्रयम्}

\twolineshloka
{असङ्गं सङ्गमग्रामे शरण्ये शरणं महत्}
{दक्षिणद्वारकायां तु गोपालं जगतां पतिम्}

\twolineshloka
{सिंहक्षेत्रे महासिंहं मल्लारिं मणिमण्डपे}
{निबिडे निबिडाकारं धानुष्के जगदीश्वरम्}

\twolineshloka
{मौहूरे कालमेघं तु मधुरायां तु सुन्दरम्}
{वृषभाद्रौ महापुण्ये परमस्वामिसज्ञकम्}

\twolineshloka
{श्रीमद्वरगुणे नाथं कुरुकायां रमासखम्}
{गोष्ठीपुरे गोष्ठपतिं शयानं दर्भसंस्तरे}

\twolineshloka
{धन्विमङ्गलके शौरिं बलाढ्यं भ्रमरस्थले}
{कुरङ्गे तु तथा पूर्णं कृष्णामेकं वटस्थले}

\twolineshloka
{अच्युतं क्षुद्रनद्यां तु पद्मनाभमनन्तके}
{एतानि विष्णोः स्थानानि पूजितानि महात्मभिः}

\twolineshloka
{अधिष्ठितानि देवेश तत्रासीनं च माधवम्}
{यः स्मरेत्सततं भक्त्या चेतसानन्यगामिना}

\twolineshloka
{स विधूयातिसंसारबन्धं याति हरेः पदम्}
{अष्टोत्तरशतं विष्णोः स्थानानि पठता स्वयम्}

\twolineshloka
{अधीताः सकला वेदाः कृताश्च विविधा मखाः}
{सम्पादिता तथा मुक्तिः परमानन्ददायिनी}

\threelineshloka
{अवगाढानि तीर्थानि ज्ञातः स भगवान् हरिः}
{आद्यमेतत्स्वयं व्यक्तं विमानं रङ्गसज्ञकम्}
{श्रीमुष्णं वेङ्कटाद्रिं च शालग्रामं च नैमिषम्}

\twolineshloka
{तोताद्रिं पुष्करं चैव नरनारायणाश्रमम्}
{अष्टौ मे मूर्तयः सन्ति स्वयं व्यक्ता महीतले}

{॥ इति श्रीविष्णोरष्टोत्तरशतदिव्यस्थानीयनामस्तोत्रं सम्पूर्णम् ॥}