% !TeX program = XeLaTeX
% !TeX root = ../../shloka.tex

\sect{दुर्गाष्टोत्तरशतनामस्तोत्रम्}
\dnsub{न्यासः}
अस्य श्रीदुर्गाष्टोत्तरशतनामास्तोत्रमालामन्त्रस्य\\
महाविष्णुमहेश्वराः ऋषयः। अनुष्टुप् छन्दः।\\
श्रीदुर्गापरमेश्वरी देवता।\\
ह्रां बीजम्। ह्रीं शक्तिः। ह्रूं कीलकम्।\\
सर्वाभीष्टसिद्ध्यर्थे जपहोमार्चने विनियोगः।\\

\dnsub{स्तोत्रम्}
\twolineshloka
{सत्या साध्या भवप्रीता भवानी भवमोचनी}
{आर्या दुर्गा जया चऽऽद्या त्रिनेत्रा शूलधारिणी}

\twolineshloka
{पिनाकधारिणी चित्रा चण्डघण्टा महातपाः}
{मनो बुद्धिरहङ्कारा चिद्रूपा च चिदाकृतिः}

\twolineshloka
{अनन्ता भाविनी भव्या ह्यभव्या च सदागतिः}
{शाम्भवी देवमाता च चिन्ता रत्नप्रिया तथा}

\twolineshloka
{सर्वविद्या दक्षकन्या दक्षयज्ञविनाशिनी}
{अपर्णाऽनेकवर्णा च पाटला पाटलावती}

\twolineshloka
{पट्टाम्बरपरीधाना कलमञ्जीररञ्जिनी}
{ईशानी च महाराज्ञी ह्यप्रमेयपराक्रमा}

\twolineshloka
{रुद्राणी क्रूररूपा च सुन्दरी सुरसुन्दरी}
{वनदुर्गा च मातङ्गी मतङ्गमुनिकन्यका}

\twolineshloka
{ब्राह्मी माहेश्वरी चैन्द्री कौमारी वैष्णवी तथा}
{चामुण्डा चैव वाराही लक्ष्मीश्च पुरुषाकृतिः}

\twolineshloka
{विमला ज्ञानरूपा च क्रिया नित्या च बुद्धिदा}
{बहुला बहुलप्रेमा महिषासुरमर्दिनी}

\twolineshloka
{मधुकैटभहन्त्री च चण्डमुण्डविनाशिनी}
{सर्वशास्त्रमयी चैव सर्वदानवघातिनी}

\twolineshloka
{अनेकशस्त्रहस्ता च सर्वशस्त्रास्त्रधारिणी}
{भद्रकाली सदाकन्या कैशोरी युवतिर्यतिः}

\twolineshloka
{प्रौढाऽप्रौढा वृद्धमाता घोररूपा महोदरी}
{बलप्रदा घोररूपा महोत्साहा महाबला}

\twolineshloka
{अग्निज्वाला रौद्रमुखी कालरात्री तपस्विनी}
{नारायणी महादेवी विष्णुमाया शिवात्मिका}

\twolineshloka
{शिवदूती कराली च ह्यनन्ता परमेश्वरी}
{कात्यायनी महाविद्या महामेधास्वरूपिणी}

\twolineshloka
{गौरी सरस्वती चैव सावित्री ब्रह्मवादिनी}
{सर्वतत्त्वैकनिलया वेदमन्त्रस्वरूपिणी}

\dnsub{फलश्रुतिः}
\twolineshloka
{इदं स्तोत्रं महादेव्या नाम्नाम् अष्टोत्तरं शतम्}
{यः पठेत् प्रयतो नित्यं भक्तिभावेन चेतसा}

\twolineshloka
{शत्रुभ्यो न भयं तस्य तस्य शत्रुक्षयं भवेत्}
{सर्वदुःखदरिद्राच्च सुसुखं मुच्यते ध्रुवम्}

\twolineshloka
{विद्यार्थी लभते विद्यां धनार्थी लभते धनम्}
{कन्यार्थी लभते कन्यां कन्या च लभते वरम्}

\twolineshloka
{ऋणी ऋणाद्विमुच्येत ह्यपुत्रो लभते सुतम्}
{रोगाद्विमुच्यते रोगी सुखमत्यन्तमश्नुते}

\twolineshloka
{भूमिलाभो भवेत् तस्य सर्वत्र विजयी भवेत्}
{सर्वान् कामानवाप्नोति महादेवीप्रसादतः}

\twolineshloka
{कुङ्कुमैर्बिल्वपत्रैश्च सुगन्धै रक्तपुष्पकैः}
{रक्तपत्रैर्विशेषेण पूजयन् भद्रमश्नुते}
॥इति श्री~दुर्गाष्टोत्तरशतनामस्तोत्रं सम्पूर्णम्॥
