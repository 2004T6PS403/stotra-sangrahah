% !TeX program = XeLaTeX
% !TeX root = ../../shloka.tex

\sect{रामस्तवराजस्तोत्रम्}
अस्य श्रीरामचन्द्रस्तवराजस्तोत्रमन्त्रस्य सनत्कुमारऋषिः। \\
श्रीरामो देवता। अनुष्टुप् छन्दः। सीता बीजम्। हनुमान् शक्तिः। 
श्रीरामप्रीत्यर्थे जपे विनियोगः॥

सूत उवाच\nopagebreak[4]

\twolineshloka
{सर्वशास्त्रार्थतत्त्वज्ञं व्यासं सत्यवतीसुतम्}
{धर्मपुत्रः प्रहृष्टात्मा प्रत्युवाच मुनीश्वरम्}% ॥ १॥

युधिष्ठिर उवाच\nopagebreak[4]

\twolineshloka
{भगवन् योगिनां श्रेष्ठ सर्वशास्त्रविशारद}
{किं तत्त्वं किं परं जाप्यं किं ध्यानं मुक्तिसाधनम्}% ॥ २॥

{श्रोतुमिच्छामि तत्सर्वं ब्रूहि मे मुनिसत्तम।}\\\nopagebreak[4]
{वेदव्यास उवाच}\\\nopagebreak[4]
{धर्मराज महाभाग शृणु वक्ष्यामि तत्त्वतः॥३॥}% 

\refstepcounter{shlokacount}
\twolineshloka
{यत्परं यद्गुणातीतं यज्ज्योतिरमलं शिवम्}
{तदेव परमं तत्त्वं कैवल्यपदकारणम्}% ॥ ४॥

\twolineshloka
{श्रीरामेति परं जाप्यं तारकं ब्रह्मसंज्ञकम्}
{ब्रह्महत्यादिपापघ्नमिति वेदविदो विदुः}% ॥ ५॥

\twolineshloka
{श्रीराम रामेति जना ये जपन्ति च सर्वदा}
{तेषां भुक्तिश्च मुक्तिश्च भविष्यति न संशयः}% ॥ ६॥

\twolineshloka
{स्तवराजं पुरा प्रोक्तं नारदेन च धीमता}
{तत्सर्वं सम्प्रवक्ष्यामि हरिध्यानपुरःसरम्}% ॥ ७॥

\twolineshloka
{तापत्रयाग्निशमनं सर्वाघौघनिकृन्तनम्}
{दारिद्र्यदुःखशमनं सर्वसम्पत्करं शिवम्}% ॥ ८॥

\twolineshloka
{विज्ञानफलदं दिव्यं मोक्षैकफलसाधनम्}
{नमस्कृत्य प्रवक्ष्यामि रामं कृष्णं जगन्मयम्}% ॥ ९॥

\twolineshloka
{अयोध्यानगरे रम्ये रत्नमण्डपमध्यगे}
{स्मरेत्कल्पतरोर्मूले रत्नसिंहासनं शुभम्}% ॥ १०॥

\twolineshloka
{तन्मध्येऽष्टदलं पद्मं नानारत्नैश्च वेष्टितम्}
{स्मरेन्मध्ये दाशरथिं सहस्रादित्यतेजसम्}% ॥ ११॥

\twolineshloka
{पितुरङ्कगतं राममिन्द्रनीलमणिप्रभम्}
{कोमलाङ्गं विशालाक्षं विद्युद्वर्णाम्बरावृतम्}% ॥ १२॥

\twolineshloka
{भानुकोटिप्रतीकाशकिरीटेन विराजितम्}
{रत्नग्रैवेयकेयूररत्नकुण्डलमण्डितम्}% ॥ १३॥

\twolineshloka
{रत्नकङ्कणमञ्जीरकटिसूत्रैरलङ्कृतम्}
{श्रीवत्सकौस्तुभोरस्कं मुक्ताहारोपशोभितम्}% ॥ १४॥

\twolineshloka
{दिव्यरत्नसमायुक्तमुद्रिकाभिरलङ्कृतम्}
{राघवं द्विभुजं बालं राममीषत्स्मिताननम्}% ॥ १५॥

\twolineshloka
{तुलसीकुन्दमन्दारपुष्पमाल्यैरलङ्कृतम्}
{कर्पूरागुरुकस्तूरीदिव्यगन्धानुलेपनम्}% ॥ १६॥

\twolineshloka
{योगशास्त्रेष्वभिरतं योगेशं योगदायकम्}
{सदा भरतसौमित्रशत्रुघ्नैरुपशोभितम्}% ॥ १७॥

\twolineshloka
{विद्याधरसुराधीशसिद्धगन्धर्वकिन्नरैः}
{योगीन्द्रैर्नारदाद्यैश्च स्तूयमानमहर्निशम्}% ॥ १८॥

\twolineshloka
{विश्वामित्रवसिष्ठादिमुनिभिः परिसेवितम्}
{सनकादिमुनिश्रेष्ठैर्योगिवृन्दैश्च सेवितम्}% ॥ १९॥

\twolineshloka
{रामं रघुवरं वीरं धनुर्वेदविशारदम्}
{मङ्गलायतनं देवं रामं राजीवलोचनम्}% ॥ २०॥

\twolineshloka
{सर्वशास्त्रार्थतत्त्वज्ञमानन्दकरसुन्दरम्}
{कौसल्यानन्दनं रामं धनुर्बाणधरं हरिम्}% ॥ २१॥

\twolineshloka
{एवं सञ्चिन्तयन् विष्णुं यज्ज्योतिरमलं विभुम्}
{प्रहृष्टमानसो भूत्वा मुनिवर्यः स नारदः}% ॥ २२॥

\twolineshloka
{सर्वलोकहितार्थाय तुष्टाव रघुनन्दनम्}
{कृताञ्जलिपुटो भूत्वा चिन्तयन्नद्भुतं हरिम्}% ॥ २३॥

\twolineshloka
{यदेकं यत्परं नित्यं यदनन्तं चिदात्मकम्}
{यदेकं व्यापकं लोके तद्रूपं चिन्तयाम्यहम्}% ॥ २४॥

\fourlineindentedshloka
{विज्ञानहेतुं विमलायताक्षम्}{प्रज्ञानरूपं स्वसुखैकहेतुम्}
{श्रीरामचन्द्रं हरिमादिदेवम्}{परात्परं राममहं भजामि}% ॥ २५॥

\fourlineindentedshloka
{कविं पुराणं पुरुषं पुरस्तात्}{सनातनं योगिनमीशितारम्}
{अणोरणीयांसमनन्तवीर्यम्}{प्राणेश्वरं राममसौ ददर्श}% ॥ २६॥

नारद उवाच\nopagebreak[4]
\twolineshloka
{नारायणं जगन्नाथमभिरामं जगत्पतिम्}
{कविं पुराणं वागीशं रामं दशरथात्मजम्}% ॥ २७॥

\twolineshloka
{राजराजं रघुवरं कौसल्यानन्दवर्धनम्}
{भर्गं वरेण्यं विश्वेशं रघुनाथं जगद्गुरुम्}% ॥ २८॥

\twolineshloka
{सत्यं सत्यप्रियं श्रेष्ठं जानकीवल्लभं विभुम्}
{सौमित्रिपूर्वजं शान्तं कामदं कमलेक्षणम्}% ॥ २९॥

\twolineshloka
{आदित्यं रविमीशानं घृणिं सूर्यमनामयम्}
{आनन्दरूपिणं सौम्यं राघवं करुणामयम्}% ॥ ३०॥

\twolineshloka
{जामदग्न्यं तपोमूर्तिं रामं परशुधारिणम्}
{वाक्पतिं वरदं वाच्यं श्रीपतिं पक्षिवाहनम्}% ॥ ३१॥

\twolineshloka
{श्रीशार्ङ्गधारिणं रामं चिन्मयानन्दविग्रहम्}
{हलधृग्विष्णुमीशानं बलरामं कृपानिधिम्}% ॥ ३२॥

\twolineshloka
{श्रीवल्लभं कृपानाथं जगन्मोहनमच्युतम्}
{मत्स्यकूर्मवराहादिरूपधारिणमव्ययम्}% ॥ ३३॥

\twolineshloka
{वासुदेवं जगद्योनिमनादिनिधनं हरिम्}
{गोविन्दं गोपतिं विष्णुं गोपीजनमनोहरम्}% ॥ ३४॥

\twolineshloka
{गोगोपालपरीवारं गोपकन्यासमावृतम्}
{विद्युत्पुञ्जप्रतीकाशं रामं कृष्णं जगन्मयम्}% ॥ ३५॥

\twolineshloka
{गोगोपिकासमाकीर्णं वेणुवादनतत्परम्}
{कामरूपं कलावन्तं कामिनीकामदं विभुम्}% ॥ ३६॥

\twolineshloka
{मन्मथं मथुरानाथं माधवं मकरध्वजम्}
{श्रीधरं श्रीकरं श्रीशं श्रीनिवासं परात्परम्}% ॥ ३७॥

\twolineshloka
{भूतेशं भूपतिं भद्रं विभूतिं भूमिभूषणम्}
{सर्वदुःखहरं वीरं दुष्टदानववैरिणम्}% ॥ ३८॥

\twolineshloka
{श्रीनृसिंहं महाबाहुं महान्तं दीप्ततेजसम्}
{चिदानन्दमयं नित्यं प्रणवं ज्योतिरूपिणम्}% ॥ ३९॥

\twolineshloka
{आदित्यमण्डलगतं निश्चितार्थस्वरूपिणम्}
{भक्तिप्रियं पद्मनेत्रं भक्तानामीप्सितप्रदम्}% ॥ ४०॥

\twolineshloka
{कौसल्येयं कलामूर्तिं काकुत्स्थं कमलाप्रियम्}
{सिंहासने समासीनं नित्यव्रतमकल्मषम्}% ॥ ४१॥

\twolineshloka
{विश्वामित्रप्रियं दान्तं स्वदारनियतव्रतम्}
{यज्ञेशं यज्ञपुरुषं यज्ञपालनतत्परम्}% ॥ ४२॥

\twolineshloka
{सत्यसन्धं जितक्रोधं शरणागतवत्सलम्}
{सर्वक्लेशापहरणं विभीषणवरप्रदम्}% ॥ ४३॥

\twolineshloka
{दशग्रीवहरं रौद्रं केशवं केशिमर्दनम्}
{वालिप्रमथनं वीरं सुग्रीवेप्सितराज्यदम्}% ॥ ४४॥

\twolineshloka
{नरवानरदेवैश्च सेवितं हनुमत्प्रियम्}
{शुद्धं सूक्ष्मं परं शान्तं तारकं ब्रह्मरूपिणम्}% ॥ ४५॥

\twolineshloka
{सर्वभूतात्मभूतस्थं सर्वाधारं सनातनम्}
{सर्वकारणकर्तारं निदानं प्रकृतेः परम्}% ॥ ४६॥

\twolineshloka
{निरामयं निराभासं निरवध्यं निरञ्जनम्}
{नित्यानन्दं निराकारमद्वैतं तमसः परम्}% ॥ ४७॥

\twolineshloka
{परात्परतरं तत्त्वं सत्यानन्दं चिदात्मकम्}
{मनसा शिरसा नित्यं प्रणमामि रघूत्तमम्}% ॥ ४८॥

\twolineshloka
{सूर्यमण्डलमध्यस्थं रामं सीतासमन्वितम्}
{नमामि पुण्डरीकाक्षममेयं गुरुतत्परम्}% ॥ ४९॥

\twolineshloka
{नमोऽस्तु वासुदेवाय ज्योतिषां पतये नमः}
{नमोऽस्तु रामदेवाय जगदानन्दरूपिणे}% ॥ ५०॥

\twolineshloka
{नमो वेदान्तनिष्ठाय योगिने ब्रह्मवादिने}
{मायामयनिरासाय प्रपन्नजनसेविने}% ॥ ५१॥ 

\twolineshloka
{वन्दामहे महेशानचण्डकोदण्डखण्डनम्}
{जानकीहृदयानन्दवर्धनं रघुनन्दनम्}% ॥ ५२॥

\fourlineindentedshloka
{उत्फुल्लामलकोमलोत्पलदलश्यामाय रामाय ते} 
{कामाय प्रमदामनोहरगुणग्रामाय रामात्मने}
{योगारूढमुनीन्द्रमानससरोहंसाय संसारवि-}
{ध्वंसाय स्फुरदोजसे रघुकुलोत्तंसाय पुंसे नमः}% ॥ ५३॥

\fourlineindentedshloka
{भवोद्भवं वेदविदां वरिष्ठम्}
{आदित्यचन्द्रानलसुप्रभावम्}
{सर्वात्मकं सर्वगतस्वरूपम्}
{नमामि रामं तमसः परस्तात्}% ॥ ५४॥

\fourlineindentedshloka
{निरञ्जनं निष्प्रतिमं निरीहम्} 
{निराश्रयं निष्कलमप्रपञ्चम्}
{नित्यं ध्रुवं निर्विषयस्वरूपम्}
{निरन्तरं राममहं भजामि}% ॥ ५५॥

\fourlineindentedshloka
{भवाब्धिपोतं भरताग्रजं तम्} 
{भक्तिप्रियं भानुकुलप्रदीपम्}
{भूतत्रिनाथं भुवनाधिपं तम्}
{भजामि रामं भवरोगवैद्यम्}% ॥ ५६॥

\fourlineindentedshloka
{सर्वाधिपत्यं समराङ्गधीरम्} 
{सत्यं चिदानन्दमयस्वरूपम्}
{सत्यं शिवं शान्तिमयं शरण्यम्}
{सनातनं राममहं भजामि}% ॥ ५७॥

\fourlineindentedshloka
{कार्यक्रियाकारणमप्रमेयम्} 
{कविं पुराणं कमलायताक्षम्}
{कुमारवेद्यं करुणामयं तम्}
{कल्पद्रुमं राममहं भजामि}% ॥ ५८॥

\fourlineindentedshloka
{त्रैलोक्यनाथं सरसीरुहाक्षम्} 
{दयानिधिं द्वन्द्वविनाशहेतुम्}
{महाबलं वेदनिधिं सुरेशम्}
{सनातनं राममहं भजामि}% ॥ ५९॥

\fourlineindentedshloka
{वेदान्तवेद्यं कविमीशितारम्}
{अनादिमध्यान्तमचिन्त्यमाद्यम्}
{अगोचरं निर्मलमेकरूपम्}
{नमामि रामं तमसः परस्तात्}% ॥ ६०॥

\fourlineindentedshloka
{अशेषवेदात्मकमादिसंज्ञम्} 
{अजं हरिं विष्णुमनन्तमाद्यम्}
{अपारसंवित्सुखमेकरूपम्}
{परात्परं राममहं भजामि}% ॥ ६१॥

\fourlineindentedshloka
{तत्त्वस्वरूपं पुरुषं पुराणम्} 
{स्वतेजसा पूरितविश्वमेकम्}
{राजाधिराजं रविमण्डलस्थम्}
{विश्वेश्वरं राममहं भजामि}% ॥ ६२॥

\fourlineindentedshloka
{लोकाभिरामं रघुवंशनाथम्} 
{हरिं चिदानन्दमयं मुकुन्दम्}
{अशेषविद्याधिपतिं कवीन्द्रम्}
{नमामि रामं तमसः परस्तात्}% ॥ ६३॥

\fourlineindentedshloka
{योगीन्द्रसङ्घैश्च सुसेव्यमानम्} 
{नारायणं निर्मलमादिदेवम्}
{नतोऽस्मि नित्यं जगदेकनाथम्}
{आदित्यवर्णं तमसः परस्तात्}% ॥ ६४॥

\fourlineindentedshloka
{विभूतिदं विश्वसृजं विरामम्} 
{राजेन्द्रमीशं रघुवंशनाथम्}
{अचिन्त्यमव्यक्तमनन्तमूर्तिम्}
{ज्योतिर्मयं राममहं भजामि}% ॥ ६५॥

\fourlineindentedshloka
{अशेषसंसारविहारहीनम्}
{आदित्यगं पूर्णसुखाभिरामम्}
{समस्तसाक्षिं तमसः परस्तात्}
{नारायणं विष्णुमहं भजामि}% ॥ ६६॥

\fourlineindentedshloka
{मुनीन्द्रगुह्यं परिपूर्णकामम्} 
{कलानिधिं कल्मषनाशहेतुम्}
{परात्परं यत्परमं पवित्रम्}
{नमामि रामं महतो महान्तम्}% ॥ ६७॥

\twolineshloka
{ब्रह्मा विष्णुश्च रुद्रश्च देवेन्द्रो देवतास्तथा}
{आदित्यादिग्रहाश्चैव त्वमेव रघुनन्दन}% ॥ ६८॥

\twolineshloka
{तापसा ऋषयः सिद्धाः साध्याश्च मरुतस्तथा}
{विप्रा वेदास्तथा यज्ञाः पुराणं धर्मसंहिताः}% ॥ ६९॥

\twolineshloka
{वर्णाश्रमास्तथा धर्मा वर्णधर्मास्तथैव च}
{यक्षराक्षसगन्धर्वा दिक्पाला दिग्गजादयः}% ॥ ७०॥

\twolineshloka
{सनकादिमुनिश्रेष्ठास्त्वमेव रघुपुङ्गव}
{वसवोऽष्टौ त्रयः काला रुद्रा एकादश स्मृताः}% ॥ ७१॥

\twolineshloka
{तारका दश दिक् चैव त्वमेव रघुनन्दन}
{सप्तद्वीपाः समुद्राश्च नगा नद्यस्तथा द्रुमाः}% ॥ ७२॥

\twolineshloka
{स्थावरा जङ्गमाश्चैव त्वमेव रघुनायक}
{देवतिर्यङ्मनुष्याणां दानवानां तथैव च}% ॥ ७३॥

\twolineshloka
{माता पिता तथा भ्राता त्वमेव रघुवल्लभ}
{सर्वेषां त्वं परं ब्रह्म त्वन्मयं सर्वमेव हि}% ॥ ७४॥

\twolineshloka
{त्वमक्षरं परं ज्योतिस्त्वमेव पुरुषोत्तम}
{त्वमेव तारकं ब्रह्म त्वत्तोऽन्यन्नैव किञ्चन}% ॥ ७५॥

\twolineshloka
{शान्तं सर्वगतं सूक्ष्मं परं ब्रह्म सनातनम्}
{राजीवलोचनं रामं प्रणमामि जगत्पतिम्}% ॥ ७६॥

व्यास उवाच

\twolineshloka
{ततः प्रसन्नः श्रीरामः प्रोवाच मुनिपुङ्गवम्}
{तुष्टोऽस्मि मुनिशार्दूल वृणीष्व वरमुत्तमम्}% ॥ ७७॥

नारद उवाच

\twolineshloka
{यदि तुष्टोऽसि सर्वज्ञ श्रीराम करुणानिधे}
{त्वन्मूर्तिदर्शनेनैव कृतार्थोऽहं च सर्वदा}% ॥ ७८॥

\threelineshloka
{धन्योऽहं कृतकृत्योऽहं पुण्योऽहं पुरुषोत्तम}
{अद्य मे सफलं जन्म जीवितं सफलं च मे}% ॥ ७९॥
{अद्य मे सफलं ज्ञानमद्य मे सफलं तपः}
\twolineshloka
{अद्य मे सफलं कर्म त्वत्पादाम्भोजदर्शनात्}
{अद्य मे सफलं सर्वं त्वन्नामस्मरणं तथा}% ॥ ८०॥

\twolineshloka
{त्वत्पादाम्भोरुहद्वन्द्वसद्भक्तिं देहि राघव}
{ततः परमसम्प्रीतः स रामः प्राह नारदम्}% ॥ ८१॥

श्रीराम उवाच

\twolineshloka
{मुनिवर्य महाभाग मुने त्विष्टं ददामि ते}
{यत्त्वया चेप्सितं सर्वं मनसा तद्भविष्यति}% ॥ ८२॥

नारद उवाच\nopagebreak[4]

\fourlineindentedshloka
{परं न याचे रघुनाथ युष्मत्}
{पादाब्जभक्तिः सततं ममास्तु}
{इदं प्रियं नाथ वरं प्रयाचे}
{पुनः पुनस्त्वामिदमेव याचे}% ॥ ८३॥

व्यास उवाच

\twolineshloka
{इत्येवमीडितो रामः प्रादात् तस्मै वरान्तरम्}
{वीरो रामो महातेजाः सच्चिदानन्दविग्रहः}% ॥ ८४॥

\twolineshloka
{अद्वैतममलं ज्ञानं स्वनामस्मरणं तथा}
{अन्तर्दधौ जगन्नाथः पुरतस्तस्य राघवः}% ॥ ८५॥

\twolineshloka
{इति श्रीरघुनाथस्य स्तवराजमनुत्तमम्}
{सर्वसौभाग्यसम्पत्तिदायकं मुक्तिदं शुभम्}% ॥ ८६॥

\twolineshloka
{कथितं ब्रह्मपुत्रेण वेदानां सारमुत्तमम्}
{गुह्याद्गुह्यतमं दिव्यं तव स्नेहात्प्रकीर्तितम्}% ॥ ८७॥

\twolineshloka
{यः पठेच्छृणुयाद्वाऽपि त्रिसन्ध्यं श्रद्धयान्वितः}
{ब्रह्महत्यादिपापानि तत्समानि बहूनि च}% ॥ ८८॥

\twolineshloka
{स्वर्णस्तेयं सुरापानं गुरुतल्पगतिस्तथा}
{गोवधाद्युपपापानि अनृतात्सम्भवानि च}% ॥ ८९॥

\twolineshloka
{सर्वैः प्रमुच्यते पापैः कल्पायुतशतोद्भवैः}
{मानसं वाचिकं पापं कर्मणा समुपार्जितम्}% ॥ ९०॥

\twolineshloka
{श्रीरामस्मरणेनैव तत्क्षणान्नश्यति ध्रुवम्}
{इदं सत्यमिदं सत्यं सत्यमेतदिहोच्यते}% ॥ ९१॥

\twolineshloka
{रामं सत्यं परं ब्रह्म रामात् किञ्चिन्न विद्यते}
{तस्माद्रामस्वरूपं हि सत्यं सत्यमिदं जगत्}% ॥ ९२॥

\fourlineindentedshloka
{श्रीरामचन्द्र रघुपुङ्गव राजवर्य} 
{राजेन्द्र राम रघुनायक राघवेश}
{राजाधिराज रघुनन्दन रामचन्द्र}
{दासोऽहमद्य भवतः शरणागतोऽस्मि}% ॥ ९३॥

\fourlineindentedshloka
{वैदेहीसहितं सुरद्रुमतले हैमे महामण्डपे}
{मध्ये पुष्पकमासने मणिमये वीरासने सुस्थितम्}
{अग्रे वाचयति प्रभञ्जनसुते तत्त्वं मुनीभ्यः परम्}
{व्याख्यान्तं भरतादिभिः परिवृतं रामं भजे श्यामलम्}% ॥ ९४॥

\fourlineindentedshloka
{रामं रत्नकिरीटकुण्डलयुतं केयूरहारान्वितम्} 
{सीतालङ्कृतवामभागममलं सिंहासनस्थं विभुम्} 
{सुग्रीवादिहरीश्वरैः सुरगणैः संसेव्यमानं सदा}
{विश्वामित्रपराशरादिमुनिभिः संस्तूयमानं प्रभुम्}% ॥ ९५॥

\fourlineindentedshloka
{सकलगुणनिधानं योगिभिः स्तूयमानम्} 
{भुजविजितसमानं राक्षसेन्द्रादिमानम्} 
{महितनृपभयानं सीतया शोभमानम्}
{स्मर हृदयविमानं ब्रह्म रामाभिधानम्}% ॥ ९६॥

\fourlineindentedshloka
{रघुवर तव मूर्तिर्मामके मानसाब्जे} 
{नरकगतिहरं ते नामधेयं मुखे मे}
{अनिशमतुलभक्त्या मस्तकं त्वत्पादाब्जे}
{भवजलनिधिमग्नं रक्ष मामार्तबन्धो}% ॥ ९७॥

\twolineshloka
{रामरत्नमहं वन्दे चित्रकूटपतिं हरिम्}
{कौसल्याभक्तिसम्भूतं जानकीकण्ठभूषणम्}% ॥ ९८॥

॥इति श्रीसनत्कुमारसंहितायां नारदोक्तं श्रीरामस्तवराजस्तोत्रं सम्पूर्णम्॥