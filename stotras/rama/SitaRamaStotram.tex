% !TeX program = XeLaTeX
% !TeX root = ../../shloka.tex

\sect{सीतारामस्तोत्रम्}
\twolineshloka
{अयोध्यापुरनेतारं मिथिलापुरनायिकाम्}
{राघवाणामलङ्कारं वैदेहानामलङ्क्रियाम्}

\twolineshloka
{रघूणां कुलदीपं च निमीनां कुलदीपिकाम्}
{सूर्यवंशसमुद्भूतं सोमवंशसमुद्भवाम्}

\twolineshloka
{पुत्रं दशरथस्याद्यं पुत्रीं जनकभूपतेः}
{वसिष्ठानुमताचारं शतानन्दमतानुगाम्}

\twolineshloka
{कौसल्यागर्भसम्भूतं वेदिगर्भोदितां स्वयम्}
{पुण्डरीकविशालाक्षं स्फुरदिन्दीवरेक्षणाम्}

\twolineshloka
{चन्द्रकान्ताननाम्भोजं चन्द्रबिम्बोपमाननाम्}
{मत्तमातङ्गगमनं मत्तहंसवधूगताम्}

\twolineshloka
{चन्दनार्द्रभुजामध्यं कुङ्कुमार्द्रकुचस्थलीम्}
{चापालङ्कृतहस्ताब्जं पद्मालङ्कृतपाणिकाम्}

\twolineshloka
{शरणागतगोप्तारं प्रणिपातप्रसादिकाम्}
{कालमेघनिभं रामं कार्तस्वरसमप्रभाम्}

\twolineshloka
{दिव्यसिंहासनासीनं दिव्यस्रग्वस्त्रभूषणाम्}
{अनुक्षणं कटाक्षाभ्यां अन्योन्येक्षणकाङ्क्षिणौ}

\twolineshloka
{अन्योन्यसदृशाकारौ त्रैलोक्यगृहदम्पती}
{इमौ युवां प्रणम्याहं भजाम्यद्य कृतार्थताम्}

\twolineshloka
{अनेन स्तौति यत्स्तुत्यं रामं सीतां च भक्तितः}
{तस्य तौ तनुतां पुण्याः सम्पदः सकलार्थदाः}

\twolineshloka
{एवं श्रीरामचन्द्रस्य जानक्याश्च विशेषतः}
{कृतं हनुमता पुण्यं स्तोत्रं सद्यो विमुक्तिदम्}
{यः पठेत् प्रातरुत्थाय सर्वान् कामानवाप्नुयात्॥}

॥इति श्री हनुमत्कृतं श्री~सीतारामस्तोत्रं सम्पूर्णम्॥