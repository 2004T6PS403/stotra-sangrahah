% !TeX program = XeLaTeX
% !TeX root = ../../shloka.tex

\sect{रामरक्षास्तोत्रम्}
अस्य श्रीरामरक्षास्तोत्रमन्त्रस्य।\\
बुधकौशिक ऋषिः। श्रीसीतारामचन्द्रो देवता।\\
अनुष्टुप् छन्दः। सीता शक्तिः। श्रीमद्-हनुमान कीलकम्।\\
श्रीरामचन्द्रप्रीत्यर्थे रामरक्षास्तोत्रजपे विनियोगः॥

\dnsub{ध्यानम्}
\fourlineindentedshloka*
{ध्यायेदाजानुबाहुं धृतशरधनुषं बद्धपद्मासनस्थम्}
{पीतं वासो वसानं नवकमलदलस्पर्धिनेत्रं प्रसन्नम्}
{वामाङ्कारूढ-सीतामुखकमलमिलल्लोचनं नीरदाभम्}
{नानालङ्कारदीप्तं दधतमुरुजटामण्डनं रामचन्द्रम्}

%\dnsub{स्तोत्रम्}
\twolineshloka*
{चरितं रघुनाथस्य शतकोटि-प्रविस्तरम्}
{एकैकमक्षरं पुंसां महापातकनाशनम्}

\fourlineindentedshloka*
{ध्यात्वा नीलोत्पलश्यामं रामं राजीवलोचनम्}
{जानकीलक्ष्मणोपेतं जटामुकुटमण्डितम्}
{सासितूणधनुर्बाणपाणिं नक्तञ्चरान्तकम्}
{स्वलीलया जगत्त्रातुम् आविर्भूतम् अजं विभुम्}

{रामरक्षां पठेत्प्राज्ञः पापघ्नीं सर्वकामदाम्॥}

\dnsub{कवचम्}\resetShloka

\twolineshloka
{शिरो मे राघवः पातु भालं दशरथात्मजः}
{कौसल्येयो दृशौ पातु विश्वामित्रप्रियः श्रुती}

\twolineshloka
{घ्राणं पातु मखत्राता मुखं सौमित्रिवत्सलः}
{जिह्वां विद्यानिधिः पातु कण्ठं भरतवन्दितः}

\twolineshloka
{स्कन्धौ दिव्यायुधः पातु भुजौ भग्नेशकार्मुकः}
{करौ सीतापतिः पातु हृदयं जामदग्न्यजित्}

\twolineshloka
{मध्यं पातु खरध्वंसी नाभिं जाम्बवदाश्रयः}
{गुह्यं जितेन्द्रियः पातु पृष्ठः पातु रघूत्तमः}

\twolineshloka
{वक्षः पातु कबन्धारिः स्तनौ गीर्वाणवन्दितः}
{पार्श्वो कुलपतिः पातु कुक्षिमिक्ष्वाकुनन्दनः}

\twolineshloka
{सुग्रीवेशः कटी पातु सक्थिनी हनुमत्प्रभुः}
{ऊरू रघूत्तमः पातु रक्षःकुलविनाशकृत्}

\twolineshloka
{जानुनी सेतुकृत् पातु जङ्घे दशमुखान्तकः}
{पादौ विभीषणश्रीदः पातु रामोऽखिलं वपुः}

%\dnsub{फलश्रुतिः}
\twolineshloka
{एतां रामबलोपेतां रक्षां यः सुकृती पठेत्}
{स चिरायुः सुखी पुत्री विजयी विनयी भवेत्}

\twolineshloka
{पातालभूतलव्योमचारिणश्छद्मचारिणः}
{न द्रष्टुमपि शक्तास्ते रक्षितं रामनामभिः}

\twolineshloka
{रामेति रामभद्रेति रामचन्द्रेति वा स्मरन्}
{नरो न लिप्यते पापैर्भुक्तिं मुक्तिं च विन्दति}

\twolineshloka
{जगजैत्रैकमन्त्रेण रामनाम्नाऽभिरक्षितम्}
{यः कण्ठे धारयेत्तस्य करस्थाः सर्वसिद्धयः}

\twolineshloka
{वज्रपञ्जरनामेदं यो रामकवचं स्मरेत्}
{अव्याहताज्ञः सर्वत्र लभते जयमङ्गलम्}

\twolineshloka
{आदिष्टवान् यथा स्वप्ने रामरक्षामिमां हरः}
{तथा लिखितवान् प्रातः प्रबुद्धो बुधकौशिकः}

\twolineshloka
{आरामः कल्पवृक्षाणां विरामः सकलापदाम्}
{अभिरामस्त्रिलोकानां रामः श्रीमान् स नः प्रभुः}

\twolineshloka
{तरुणौ रूपसम्पन्नौ सुकुमारौ महाबलौ}
{पुण्डरीकविशालाक्षौ चीरकृष्णाजिनाम्बरौ}

\twolineshloka
{फलमूलाशनौ दान्तौ तापसौ ब्रह्मचारिणौ}
{पुत्रौ दशरथस्यैतौ भ्रातरौ रामलक्ष्मणौ}

\twolineshloka
{शरण्यौ सर्वसत्त्वानां श्रेष्ठौ सर्वधनुष्मताम्}
{रक्षः कुलनिहन्तारौ त्रायेतां नो रघूत्तमौ}

\twolineshloka
{आत्तसज्जधनुषाविषुस्पृशौ अक्षयाशुगनिषङ्गसङ्गिनौ}
{रक्षणाय मम रामलक्ष्मणौ अग्रतः पथि सदैव गच्छताम्}

\twolineshloka
{सन्नद्धः कवची खड्गी चापबाणधरो युवा}
{यच्छन्मनोरथोऽस्माकं रामः पातु सलक्ष्मणः}

\twolineshloka
{रामो दाशरथिः शूरो लक्ष्मणानुचरो बली}
{काकुत्स्थः पुरुषः पूर्णः कौसल्येयो रघूत्तमः}

\twolineshloka
{वेदान्तवेद्यो यज्ञेशः पुराणपुरुषोत्तमः}
{जानकीवल्लभः श्रीमान् अप्रमेयपराक्रमः}

\twolineshloka
{इत्येतानि जपन्नित्यं मद्भक्तः श्रद्धयान्वितः}
{अश्वमेधाधिकं पुण्यं सम्प्राप्नोति न संशयः}
॥इति पद्मपुराणे वेदव्यासकृतौ भगवद्वसिष्ठ-श्रीबुधकौशिकप्रणीतं वज्रपञ्जरं नाम श्री~रामकवचं सम्पूर्णम्॥
\resetShloka
\twolineshloka
{रामं दूर्वादलश्यामं पद्माक्षं पीतवाससम्}
{स्तुवन्ति नामभिर्दिव्यैर्न ते संसारिणो नरः}

\fourlineindentedshloka
{रामं लक्ष्मणपूर्वजं रघुवरं सीतापतिं सुन्दरम्}
{काकुत्स्थं करुणार्णवं गुणनिधिं विप्रप्रियं धार्मिकम्}
{राजेन्द्रं सत्यसन्धं दशरथतनयं श्यामलं शान्तमूर्तिम्}
{वन्दे लोकाभिरामं रघुकुलतिलकं राघवं रावणारिम्}

\twolineshloka
{रामाय रामभद्राय रामचन्द्राय वेधसे}
{रघुनाथाय नाथाय सीतायाः पतये नमः}

\fourlineindentedshloka
{श्रीराम राम रघुनन्दन राम राम}
{श्रीराम राम भरताग्रज राम राम}
{श्रीराम राम रणकर्कश राम राम}
{श्रीराम राम शरणं भव राम राम}

\fourlineindentedshloka
{श्रीरामचन्द्रचरणौ मनसा स्मरामि}
{श्रीरामचन्द्रचरणौ वचसा गृह्णामि}
{श्रीरामचन्द्रचरणौ शिरसा नमामि}
{श्रीरामचन्द्रचरणौ शरणं प्रपद्ये}

\fourlineindentedshloka
{माता रामो मत्पिता रामचन्द्रः}
{स्वामी रामो मत्सखा रामचन्द्रः}
{सर्वस्वं मे रामचन्द्रो दयालुः}
{नान्यं जाने नैव जाने न जाने}

\twolineshloka
{दक्षिणे लक्ष्मणो यस्य वामे तु जनकात्मजा}
{पुरतो मारुतिर्यस्य तं वन्दे रघुनन्दनम्}

\twolineshloka
{लोकाभिरामं रणरङ्गधीरं राजीवनेत्रं रघुवंशनाथम्}
{कारुण्यरूपं करुणाकरं तं श्रीरामचन्द्रं शरणं प्रपद्ये}

\twolineshloka
{मनोजवं मारुततुल्यवेगं जितेन्द्रियं बुद्धिमतां वरिष्ठम्}
{वातात्मजं वानरयूथमुख्यं श्रीरामदूतं शरणं प्रपद्ये}

\twolineshloka
{कूजन्तं राम रामेति मधुरं मधुराक्षरम्}
{आरुह्य कविताशाखां वन्दे वाल्मीकिकोकिलम्}

\twolineshloka
{आपदाम् अपहर्तारं दातारं सर्वसम्पदाम्}
{लोकाभिरामं श्रीरामं भूयो भूयो नमाम्यहम्}

\twolineshloka
{भर्जनं भवबीजानाम् अर्जनं सुखसम्पदाम्}
{तर्जनं यमदूतानां राम रामेति गर्जनम्}

\fourlineindentedshloka
{रामो राजमणिः सदा विजयते रामं रमेशं भजे}
{रामेणाभिहता निशाचरचमू रामाय तस्मै नमः}
{रामान्नास्ति परायणं परतरं रामस्य दासोऽस्म्यहम्}
{रामे चित्तलयः सदा भवतु मे भो राम मामुद्धर}

\twolineshloka
{राम रामेति रामेति रमे रामे मनोरमे}
{सहस्रनाम तत्तुल्यं रामनाम वरानने}
%॥इति श्रीबुधकौशिकविरचितं श्रीरामरक्षास्तोत्रं सम्पूर्णम्॥
॥श्री सीतारामचन्द्रार्पणमस्तु॥

\closesection
\twolineshloka*
{मङ्गलं कोसलेन्द्राय महनीयगुणाब्धये}
{चक्रवर्तितनूजाय सार्वभौमाय मङ्गलम्}