% !TeX program = XeLaTeX
% !TeX root = ../../shloka.tex

\sect{रामाष्टकम्}
\twolineshloka
{भजे विशेषसुन्दरं समस्तपापखण्डनम्}
{स्वभक्तचित्तरञ्जनं सदैव राममद्वयम्}

\twolineshloka
{जटाकलापशोभितं समस्तपापनाशकम्}
{स्वभक्तभीतिभञ्जनं भजे ह राममद्वयम्}

\twolineshloka
{निजस्वरूपबोधकं कृपाकरं भवापहम्}
{समं शिवं निरञ्जनं भजे ह राममद्वयम्}

\twolineshloka
{सहप्रपञ्चकल्पितं ह्यनामरूपवास्तवम्}
{निराकृतिं निरामयं भजे ह राममद्वयम्}

\twolineshloka
{निष्प्रपञ्चनिर्विकल्पनिर्मलं निरामयम्}
{चिदेकरूपसन्ततं भजे ह राममद्वयम्}

\twolineshloka
{भवाब्धिपोतरूपकं ह्यशेषदेहकल्पितम्}
{गुणाकरं कृपाकरं भजे ह राममद्वयम्}

\twolineshloka
{महासुवाक्यबोधकैर्विराजमानवाक्पदैः}
{परं ब्रह्मसद्व्यापकं भजे ह राममद्वयम्}

\twolineshloka
{शिवप्रदं सुखप्रदं भवच्छिदं भ्रमापहम्}
{विराजमानदेशिकं भजे ह राममद्वयम्}

\fourlineindentedshloka
{रामाष्टकं पठति यः सुखदं सुपुण्यम्}
{व्यासेन भाषितमिदं शृणुते मनुष्यः}
{विद्यां श्रियं विपुलसौख्यमनन्तकीर्तिम्}
{सम्प्राप्य देहविलये लभते च मोक्षम्}
  
॥इति श्री व्यासविरचितं श्री रामाष्टकं सम्पूर्णम्॥