% !TeX program = XeLaTeX
% !TeX root = ../../shloka.tex

\setlength{\columnsep}{10pt}\newpage
\sect{हनुमान् चालीसा}
\begin{large}
\begin{multicols}{2}
\fourlineindentedshloka*
{श्रीगुरु चरन सरोज रज}
{निज मनु मुकुर सुधार}
{बरनऊँ रघुवर विमल यश}
{जो दायकु फल चार}

\fourlineindentedshloka*
{बुद्धिहीन तनु जानिके}
{सुमिरौं पवनकुमार}
{बल बुद्धि विद्या देहु मोहिं}
{हरहु कलेस विकार}

\dnsub{चौपाई}\resetShloka
\twolineshloka
{जय हनुमान ज्ञान गुण सागर}
{जय कपीश तिहुँ लोक उजागर}

\twolineshloka
{राम दूत अतुलित बल धामा}
{अञ्जनिपुत्र पवनसुत नामा}

\twolineshloka
{महावीर विक्रम बजरङ्गी}
{कुमति निवार सुमति के सङ्गी}

\twolineshloka
{कञ्चन बरन विराज सुवेसा}
{कानन कुण्डल कुञ्चित केशा}

\twolineshloka
{हाथ वज्र औ ध्वजा विराजै}
{काँधे मूँज जनेऊ साजै}

\twolineshloka
{सङ्कर सुवन केसरीनन्दन}
{तेज प्रताप महा जग वन्दन}

\twolineshloka
{विद्यावान गुणी अति चातुर}
{राम काज करिबे को आतुर}

\twolineshloka
{प्रभु चरित्र सुनिबे को रसिया}
{राम लखन सीता मन बसिया}

\twolineshloka*
{राम लक्ष्मण जानकी}
{जय बोलो हनुमान् की}

\twolineshloka
{सूक्ष्म रूप धरि सियहिं दिखावा}
{विकट रूप धरि लङ्क जरावा}

\twolineshloka
{भीम रूप धरि असुर सँहारे}
{रामचन्द्र के काज सँवारे}

\twolineshloka
{लाय सजीवन लखन जियाये}
{श्रीरघुवीर हरषि उर लाये}

\twolineshloka
{रघुपति कीन्ही बहुत बडाई}
{तुम मम प्रिय भरत सम भाई}

\twolineshloka
{सहस वदन तुम्हरो यश गावैं}
{अस कहि श्रीपति कण्ठ लगावैं}

\twolineshloka
{सनकादिक ब्रह्मादि मुनीशा}
{नारद शारद सहित अहीशा}

\twolineshloka
{यम कुबेर दिक्पाल जहाँ ते}
{कवि कोविद कहि सके कहाँ ते}

\twolineshloka
{तुम उपकार सुग्रीवहिं कीन्हा}
{राम मिलाय राज पद दीन्हा}

\twolineshloka*
{राम लक्ष्मण जानकी}
{जय बोलो हनुमान् की}

\twolineshloka
{तुम्हरो मन्त्र विभीषण माना}
{लङ्केश्वर भये सब जग जाना}

\twolineshloka
{युग सहस्र योजन पर भानू}
{लील्यो ताहि मधुर फल जानू}

\twolineshloka
{प्रभु मुद्रिका मेलि मुख माहीं}
{जलधि लाँघि गये अचरज नाहीं}

\twolineshloka
{दुर्गम काज जगत के जेते}
{सुगम अनुग्रह तुम्हरे तेते}

\twolineshloka
{राम दुआरे तुम रखवारे}
{होत न आज्ञा बिन पैसारे}

\twolineshloka
{सब सुख लहै तुम्हारी सरना}
{तुम रक्षक काहू को डर ना}

\twolineshloka
{आपन तेज सम्हारो आपै}
{तीनों लोक हाँक तें काँपै}

\twolineshloka
{भूत पिशाच निकट नहिं आवै}
{महावीर जब नाम सुनावै}

\twolineshloka*
{राम लक्ष्मण जानकी}
{जय बोलो हनुमान् की}

\twolineshloka
{नाशै रोग हरै सब पीरा}
{जपत निरन्तर हनुमत वीरा}

\twolineshloka
{सङ्कट से हनुमान छुडावै}
{मन क्रम वचन ध्यान जो लावै}

\twolineshloka
{सब पर राम तपस्वी राजा}
{तिन के काज सकल तुम साजा}

\twolineshloka
{और मनोरथ जो कोई लावै}
{दासु अमित जीवन फल पावै}

\twolineshloka
{चारों युग प्रताप तुम्हारा}
{है प्रसिद्ध जगत उजियारा}

\twolineshloka
{साधु सन्त के तुम रखवारे}
{असुर निकन्दन राम दुलारे}

\twolineshloka
{अष्ट सिद्धि नव निधि के दाता}
{अस बर दीन जानकी माता}

\twolineshloka
{राम रसायन तुम्हरे पासा}
{सदा रहो रघुपति के दासा}

\twolineshloka*
{राम लक्ष्मण जानकी}
{जय बोलो हनुमान् की}

\twolineshloka
{तुम्हरे भजन राम को पावै}
{जन्म जन्म के दुख बिसरावै}

\twolineshloka
{अन्त काल रघुपति पुर जाई}
{जहाँ जन्मि हरिभक्त कहाई}

\twolineshloka
{और देवता चित्त न धरई}
{हनुमत सेई सर्व सुख करई}

\twolineshloka
{सङ्कट कटै मिटै सब पीरा}
{जो सुमिरै हनुमत बलवीरा}

\twolineshloka
{जै जै जै हनुमान गोसाईं}
{कृपा करहु गुरु देव की नाईं}

\twolineshloka
{यह शत पार पाठ कर कोई}
{छूटहि बंदि महा सुख होई}

\twolineshloka
{यो यह पढ़ै हनुमान् चलीसा}
{होय सिद्धि साखी गौरीसा}

\twolineshloka
{तुलसीदास सदा हरि चेरा}
{कीजै नाथ हृदय मँह डेरा}

\twolineshloka*
{राम लक्ष्मण जानकी}
{जय बोलो हनुमान् की}
\end{multicols}
\end{large}