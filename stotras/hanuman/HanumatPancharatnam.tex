% !TeX program = XeLaTeX
% !TeX root = ../../shloka.tex

\sect{हनुमत् पञ्चरत्नम्}
\twolineshloka
{वीताखिल-विषयेच्छं जातानन्दाश्रु-पुलकमत्यच्छम्}
{सीतापति-दूताद्यं वातात्मजमद्य भावये हृद्यम्}

\twolineshloka
{तरुणारुण-मुख-कमलं करुणा-रसपूर-पूरितापाङ्गम्}
{सञ्जीवनमाशासे मञ्जुल-महिमानमञ्जना-भाग्यम्}

\twolineshloka
{शम्बरवैरि-शरातिगमम्बुजदल-विपुल-लोचनोदारम्}
{कम्बुगलमनिलदिष्टं बिम्ब-ज्वलितोष्ठमेकमवलम्बे}

\twolineshloka
{दूरीकृत-सीतार्तिः प्रकटीकृत-रामवैभव-स्फूर्तिः}
{दारित-दशमुख-कीर्तिः पुरतो मम भातु हनुमतो मूर्तिः}

\twolineshloka
{वानर-निकराध्यक्षं दानव-कुल-कुमुद-रविकर-सदृशम्}
{दीन-जनावन-दीक्षं पवनतपः पाकपुञ्जमद्राक्षम्}

\twolineshloka
{एतत् पवनसुतस्य स्तोत्रं यः पठति पञ्चरत्नाख्यम्}
{चिरमिह निखिलान् भोगान् भुक्त्वा श्रीराम-भक्तिभाग् भवति}
॥इति श्रीमच्छङ्कराचार्यविरचितं श्री~हनुमत्-पञ्चरत्नं सम्पूर्णम्॥

\twolineshloka*
{यत्र यत्र रघुनाथकीर्तनं तत्र तत्र कृत-मस्तकाञ्जलिम्}
{बाष्पवारिपरिपूर्ण-लोचनं मारुतिं नमत राक्षसान्तकम्‌}

\fourlineindentedshloka*
{उल्लङ्घ्य सिन्धोः सलिलं सलीलम्}
{यः शोकवह्निं जनकात्मजायाः}
{आदाय तेनैव ददाह लङ्काम्}
{नमामि तं प्राञ्जलिराञ्जनेयम्}

\twolineshloka*
{बुद्धिर्बलं यशो धैर्यं निर्भयत्वम् अरोगता}
{अजाड्यं वाक्पटुत्वं च हनुमत्स्मरणाद्भवेत्}

\twolineshloka*
{असाध्यसाधक स्वामिन् असाध्यं तव किं वद}
{रामदूतकृपसिन्धो मत्कार्यं साधय प्रभो}

