% !TeX program = XeLaTeX
% !TeX root = ../../shloka.tex

\sect{सरस्वतीस्तोत्रम् अगस्त्यमुनि प्रोक्तम्}

\fourlineindentedshloka
{या कुन्देन्दुतुषारहारधवला या शुभ्रवस्त्रावृता}
{या वीणावरदण्डमण्डितकरा या श्वेतपद्मासना}
{या ब्रह्माच्युतशङ्करप्रभृतिभिर्देवैः सदा पूजिता }
{सा मां पातु सरस्वती भगवती निःशेषजाड्यापहा}

\fourlineindentedshloka
{दोर्भिर्युक्ता चतुर्भिं स्फटिकमणिनिभैरक्षमालान्दधाना}
{हस्तेनैकेन पद्मं सितमपि च शुकं पुस्तकं चापरेण}
{भासा कुन्देन्दुशङ्खस्फटिकमणिनिभा भासमानाऽसमाना}
{सा मे वाग्देवतेयं निवसतु वदने सर्वदा सुप्रसन्ना}

\fourlineindentedshloka
{सुरासुरसेवितपादपङ्कजा}
{करे विराजत्कमनीयपुस्तका}
{विरिञ्चिपत्नी कमलासनस्थिता}
{सरस्वती नृत्यतु वाचि मे सदा}

\fourlineindentedshloka
{सरस्वती सरसिजकेसरप्रभा}
{तपस्विनी सितकमलासनप्रिया}
{घनस्तनी कमलविलोललोचना}
{मनस्विनी भवतु वरप्रसादिनी}

\twolineshloka
{सरस्वति नमस्तुभ्यं वरदे कामरूपिणि}
{विद्यारम्भं करिष्यामि सिद्धिर्भवतु मे सदा}

\twolineshloka
{सरस्वति नमस्तुभ्यं सर्वदेवि नमो नमः}
{शान्तरूपे शशिधरे सर्वयोगे नमो नमः}

\twolineshloka
{नित्यानन्दे निराधारे निष्कलायै नमो नमः}
{विद्याधरे विशालाक्षि शुद्धज्ञाने नमो नमः}

\twolineshloka
{शुद्धस्फटिकरूपायै सूक्ष्मरूपे नमो नमः}
{शब्दब्रह्मि चतुर्हस्ते सर्वसिद्‌ध्यै नमो नमः}

\twolineshloka
{मुक्तालङ्कृत-सर्वाङ्ग्यै मूलाधारे नमो नमः}
{मूलमन्त्रस्वरूपायै मूलशक्त्यै नमो नमः}

\twolineshloka
{मनो मणिमहायोगे वागीश्वरि नमो नमः}
{वाग्भ्यै वरदहस्तायै वरदायै नमो नमः}

\twolineshloka
{वेदायै वेदरूपायै वेदान्तायै नमो नमः}
{गुणदोषविवर्जिन्यै गुणदीप्त्यै नमो नमः}

\twolineshloka
{सर्वज्ञाने सदानन्दे सर्वरूपे नमो नमः}
{सम्पन्नायै कुमार्यै च सर्वज्ञे ते नमो नमः}

\twolineshloka
{योगानार्य उमादेव्यै योगानन्दे नमो नमः}
{दिव्यज्ञान त्रिनेत्रायै दिव्यमूर्त्यै नमो नमः}

\twolineshloka
{अर्धचन्द्रजटाधारि चन्द्रबिम्बे नमो नमः}
{चन्द्रादित्यजटाधारि चन्द्रबिम्बे नमो नमः}

\twolineshloka
{अणुरूपे महारूपे विश्वरूपे नमो नमः}
{अणिमाद्यष्टसिद्धायै आनन्दायै नमो नमः}

\twolineshloka
{ज्ञान-विज्ञान-रूपायै ज्ञानमूर्ते नमो नमः}
{नानाशास्त्र-स्वरूपायै नानारूपे नमो नमः}

\twolineshloka
{पद्मदा पद्मवंशा च पद्मरूपे नमो नमः}
{परमेष्ठ्यै परामूर्त्यै नमस्ते पापनाशिनि}

\twolineshloka
{महादेव्यै महाकाल्यै महालक्ष्म्यै नमो नमः}
{ब्रह्मविष्णुशिवायै च ब्रह्मनार्यै नमो नमः}

\twolineshloka
{कमलाकरपुष्पा च कामरूपे नमो नमः}
{कपालि कर्मदीप्तायै कर्मदायै नमो नमः}

\twolineshloka
{सायं प्रातः पठेन्नित्यं षण्मासात् सिद्धिरुच्यते}
{चोरव्याघ्रभयं नास्ति पठतां शृण्वतामपि}

\twolineshloka
{इत्थं सरस्वतीस्तोत्रम् अगस्त्यमुनिवाचकम्}
{सर्वसिद्धिकरं नॄणां सर्वपापप्रणाशनम्}

॥इति श्री अगस्त्यमुनि-प्रोक्तं श्री~सरस्वतीस्तोत्रं सम्पूर्णम्॥