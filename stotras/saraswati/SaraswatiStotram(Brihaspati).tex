% !TeX program = XeLaTeX
% !TeX root = ../../shloka.tex

\sect{सरस्वतीस्तोत्रं बृहस्पतिविरचितम्}
%\vspace{-1cm}
बृहस्पतिरुवाच
\twolineshloka
{सरस्वति नमस्यामि चेतनां हृदि संस्थिताम्}
{कण्ठस्थां पद्मयोनिं त्वां ह्रीङ्कारां सुप्रियां सदा}

\twolineshloka
{मतिदां वरदां चैव सर्वकामफलप्रदाम्}
{केशवस्य प्रियां देवीं वीणाहस्तां वरप्रदाम्}

\twolineshloka
{मन्त्रप्रियां सदा हृद्यां कुमतिध्वंसकारिणीम्}
{स्वप्रकाशां निरालम्बामज्ञानतिमिरापहाम्}

\twolineshloka
{मोक्षप्रियां शुभां नित्यां सुभगां शोभनप्रियाम्}
{पद्मोपविष्टां कुण्डलिनीं शुक्लवस्त्रां मनोहराम्}

\twolineshloka
{आदित्यमण्डले लीनां प्रणमामि जनप्रियाम्}
{ज्ञानाकारां जगद्द्वीपां भक्तविघ्नविनाशिनीम्}

\twolineshloka
{इति सत्यं स्तुता देवी वागीशेन महात्मना}
{आत्मानं दर्शयामास शरदिन्दुसमप्रभाम्}

श्रीसरस्वत्युवाच\\
वरं वृणीष्व भद्रं त्वं यत्ते मनसि वर्तते।\\
बृहस्पतिरुवाच\\
प्रसन्ना यदि मे देवि परं ज्ञानं प्रयच्छ मे॥७॥\\
\refstepcounter{shlokacount}
श्रीसरस्वत्युवाच\\
\twolineshloka
{दत्तं ते निर्मलं ज्ञानं कुमतिध्वंसकारकम्}
{स्तोत्रेणानेन मां भक्त्या ये स्तुवन्ति सदा नराः}

\twolineshloka
{लभन्ते परमं ज्ञानं मम तुल्यपराक्रमाः}
{कवित्वं मत्प्रसादेन प्राप्नुवन्ति मनोगतम्}

\twolineshloka
{त्रिसन्ध्यं प्रयतो भूत्वा यस्त्विमं पठते नरः}
{तस्य कण्ठे सदा वासं करिष्यामि न संशयः}
॥इति~श्री~रुद्रयामले श्री~बृहस्पतिविरचितं श्री~सरस्वतीस्तोत्रं सम्पूर्णम्॥