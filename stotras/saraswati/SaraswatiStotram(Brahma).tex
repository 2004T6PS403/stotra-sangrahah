% !TeX program = XeLaTeX
% !TeX root = ../../shloka.tex

\setlength{\shlokaspaceskip}{12pt}
\sect{सिद्धसरस्वतीस्तोत्रं श्रीमद्-ब्रह्मविरचितम्}
\dnsub{न्यासः}
ॐ अस्य श्रीसरस्वतीस्तोत्रमन्त्रस्य।\\
ब्रह्मा ऋषिः। गायत्री छन्दः। श्रीसरस्वती देवता।\\
धर्मार्थकाममोक्षार्थे जपे विनियोगः।\\

\dnsub{स्तोत्रम्}
\fourlineindentedshloka
{आरूढा श्वेतहंसे भ्रमति च गगने दक्षिणे चाक्षसूत्रम्}
{वामे हस्ते च दिव्याम्बरकनकमयं पुस्तकं ज्ञानगम्या}
{सा वीणां वादयन्ती स्वकरकरजपैः शास्त्रविज्ञानशब्दैः}
{क्रीडन्ती दिव्यरूपा करकमलधरा भारती सुप्रसन्ना}

\threelineshloka
{श्वेतपद्मासना देवी श्वेतगन्धानुलेपना}
{अर्चिता मुनिभिः सर्वैरृषिभिः स्तूयते सदा}
{एवं ध्यात्वा सदा देवीं वाञ्छितं लभते नरः}

\fourlineindentedshloka
{शुक्लां ब्रह्मविचारसारपरमामाद्यां जगद्व्यापिनीम्}
{वीणापुस्तकधारिणीमभयदां जाड्यान्धकारापहाम्}
{हस्ते स्फटिकमालिकां विदधतीं पद्मासने संस्थितां}
{वन्दे तां परमेश्वरीं भगवतीं बुद्धिप्रदां शारदाम्}

\fourlineindentedshloka
{या कुन्देन्दुतुषारहारधवला या शुभ्रवस्त्रावृता}
{या वीणावरदण्डमण्डितकरा या श्वेतपद्मासना}
{या ब्रह्माच्युतशङ्करप्रभृतिभिर्देवैः सदा वन्दिता}
{सा मां पातु सरस्वती भगवती निःशेषजाड्यापहा}

\fourlineindentedshloka
{ह्रीं ह्रौं हृद्यैकबीजे शशिरुचिकमले कल्पविस्पष्टशोभे}
{भव्ये भव्यानुकूले कुमतिवनदवे विश्ववन्द्याङ्घ्रिपद्मे}
{पद्मे पद्मोपविष्टे प्रणजनमनोमोदसम्पादयित्रि}
{प्रोत्फुल्लज्ञानकूटे हरिनिजदयिते देवि संहारसारे}

\fourlineindentedshloka
{ऐं ऐं ऐं दृष्टमन्त्रे कमलभवमुखाम्भोजभूतस्वरूपे}
{रूपारूपप्रकाशे सकलगुणमये निर्गुणे निर्विकारे}
{न स्थूले नैव सूक्ष्मेऽप्यविदितविभवे नापि विज्ञानतत्त्वे}
{विश्वे विश्वान्तरात्मे सुरवरनमिते निष्कले नित्यशुद्धे}

\fourlineindentedshloka
{ह्रीं ह्रीं ह्रीं जाप्यतुष्टे हिमरुचिमुकुटे वल्लकीव्यग्रहस्ते}
{मातर्मातर्नमस्ते दह दह जडतां देहि बुद्धिं प्रशस्ताम्}
{विद्ये वेदान्तवेद्ये परिणतपठिते मोक्षदे मुक्तिमार्गे}
{मार्गातीतस्वरूपे भव मम वरदा शारदे शुभ्रहारे}

\fourlineindentedshloka
{धीं धीं धीं धारणाख्ये धृतिमतिनतिभिर्नामभिः कीर्तनीये}
{नित्येऽनित्ये निमित्ते मुनिगणनमिते नूतने वै पुराणे}
{पुण्ये पुण्यप्रवाहे हरिहरनमिते नित्यशुद्धे सुवर्णे}
{मातर्मात्रार्धतत्त्वे मतिमति मतिदे माधवप्रीतिमोदे}

\fourlineindentedshloka
{हूं हूं हूं स्वरूपे दह दह दुरितं पुस्तकव्यग्रहस्ते}
{सन्तुष्टाकारचित्ते स्मितमुखि सुभगे जृम्भिणि स्तम्भविद्ये}
{मोहे मुग्धप्रवाहे कुरु मम विमतिध्वान्तविध्वंसमीडे}
{गीर्गौर्वाग्भारति त्वं कविवररसनासिद्धिदे सिद्धिसाध्ये}

\fourlineindentedshloka
{स्तौमि त्वां त्वां च वन्दे मम खलु रसनां नो कदाचित् त्यजेथा}
{मा मे बुद्धिर्विरुद्धा भवतु न च मनो देवि मे यातु पापम्}
{मा मे दुःखं कदाचित् क्वचिदपि विषयेऽप्यस्तु मे नाकुलत्वम्}
{शास्त्रे वादे कवित्वे प्रसरतु मम धीर्माऽस्तु कुण्ठा कदाऽपि}

\fourlineindentedshloka
{इत्येतैः श्लोकमुख्यैः प्रतिदिनमुषसि स्तौति यो भक्तिनम्रो}
{वाणी वाचस्पतेरप्यविदितविभवो वाक्पटुर्मृष्टकण्ठः}
{स्यादिष्टाद्यर्थलाभैः सुतमिव सततं पातितं सा च देवी}
{सौभाग्यं तस्य लोके प्रभवति कविता विघ्नमस्तं प्रयाति}

\fourlineindentedshloka
{निर्विघ्नं तस्य विद्या प्रभवति सततं चाश्रुतग्रन्थबोधः}
{कीर्तिस्रैलोक्यमध्ये निवसति वदने शारदा तस्य साक्षात्}
{दीर्घायुर्लोकपूज्यः सकलगुणनिधिः सन्ततं राजमान्यो~-}
{वाग्देव्याः सम्प्रसादात् त्रिजगति विजयी जायते सत्सभासु}

\twolineshloka
{ब्रह्मचारी व्रती मौनी त्रयोदश्यां निरामिषः}
{सारस्वतो जनः पाठात् सकृदिष्टार्थलाभवान्}

\twolineshloka
{पक्षद्वये त्रयोदश्याम् एकविंशतिसङ्ख्यया}
{अविच्छिन्नः पठेद्धीमान् ध्यात्वा देवीं सरस्वतीम्}

\twolineshloka
{सर्वपापविनिर्मुक्तः सुभगो लोकविश्रुतः}
{वाञ्छितं फलमाप्नोति लोकेऽस्मिन्नात्र संशयः}

\twolineshloka
{ब्रह्मणेति स्वयं प्रोक्तं सरस्वत्याः स्तवं शुभम्}
{प्रयत्नेन पठेन्नित्यं सोऽमृतत्वाय कल्पते}
॥इति श्रीमद्ब्रह्मणा विरचितं श्री~सिद्धसरस्वतीस्तोत्रं सम्पूर्णम्॥
\setlength{\shlokaspaceskip}{24pt}