% !TeX program = XeLaTeX
% !TeX root = ../../shloka.tex
\sect{नवग्रहमङ्गलाष्टकम्}

\fourlineindentedshloka
{भास्वानर्कसमिच्च रक्तकिरणः सिंहाधिपः काश्यपो -}
{गुर्विन्दोश्च कुजस्य मित्रमरिगः त्रिस्थः शुभः प्राङ्मुखः}
{शत्रुर्भार्गवसौरयोः प्रियः कुजः कालिङ्गदेशाधिपो -}
{मध्ये वर्तुलमण्डले स्थितिमितः कुर्यात्सदा मङ्गलम्}

\fourlineindentedshloka
{चन्द्रः कर्कटकप्रभुः सितरुचिश्चात्रेयगोत्रोद्भवः}
{चाग्नेये चतुरश्रकोऽपरमुखो गौर्यर्चया तर्पितः}
{षट्सप्ताग्निदशाद्यशोभनफलो शत्रुर्बुधार्कप्रियः}
{सौम्यो यामुनदेशपर्णजसमित्कुर्यात्सदा मङ्गलम्}

\fourlineindentedshloka
{भौमो दक्षिणदिक्त्रिकोणनिलयोऽवन्तीपतिः खादिर-}
{प्रीतो वृश्चिकमेषयोरधिपतिर्गुर्वर्कचन्द्रप्रियः}
{ज्ञारिः षट्त्रिशुभप्रदश्च वसुधादाता गुहाधीश्वरो}
{भारद्वाजकुलाधिपोऽरुणरुचिः कुर्यात्सदा मङ्गलम्}

\fourlineindentedshloka
{सौम्यः पीत उदङ्मुखः समिदपामार्गोऽत्रिगोत्रोद्भवो -}
{बाणेशानगतः सुहृद्रविसुतो वैरीकृतानुष्णरुक्}
{कन्यायुग्मपतिर्दशाष्टमचतुःषण्णेत्रगः शोभनो -}
{विष्ण्वाराधनतर्पितो मगधपः कुर्यात्सदा मङ्गलम्}

\fourlineindentedshloka
{जीवश्चोत्तरदिङ्मुखोत्तरककुभ्जातोऽङ्गिरो गोत्रदः}
{पीतोऽश्वत्थसमिच्च सिन्ध्वधिपतिः चापर्क्षमीनाधिपः}।
{सूर्येन्दुक्षितिजप्रियः सितबुधारातिः समो भानुजे -}
{सप्तापत्यतपोऽर्थगः शुभकरः कुर्यात्सदा मङ्गलम्}

\fourlineindentedshloka
{शुक्रो भार्गवगोत्रजः सितरुचिः पूर्वाननः पूर्वदिक्}
{काम्बोजाधिपतिस्तुलावृषभगश्चौदुम्बरैस्तर्पितः}
{सौम्यर्क्योः सुहृदम्बिकास्तुतिवशात् प्रीतोर्कचन्द्राहितो -}
{नारीभोगकरः शुभो भृगुसुतः कुर्यात्सदा मङ्गलम्}

\fourlineindentedshloka
{सौरिः कृष्णरुचिश्च पश्चिममुखः सौराष्ट्रपः काश्यपो -}
{नाथः कुम्भमृगर्क्षयोः प्रियसुहृत् शुक्रज्ञयोर्रुद्रगः}
{षट्त्रिस्थः शुभदो शुभो  धनुगतिश्चापाकृतौ मण्डले}
{सन्तिष्ठन् चिरजीवितादिफलदः कुर्यात्सदा मङ्गलम्}

\fourlineindentedshloka
{राहुर्बर्बरदेशपो  निरृतौ कृष्णाङ्गशूर्पासनो}
{याम्याशाभिमुखश्च चन्द्ररविरुध् पैडीनसिः क्रौर्यवान्}
{षट्त्रिस्थः शुभकृत् करालवदनः प्रीतश्च दूर्वाहुतौ}
{दुर्गापूजनतः प्रसन्नहृदयः कुर्यात्सदा मङ्गलम्}

\fourlineindentedshloka
{केतुर्जैमिनिगोत्रजः कुशसमिद्वायव्यकोणेस्थितः}
{चित्राङ्कध्वजलाञ्छनो हि भगवान् याम्याननः शोभनः}
{सन्तुष्टो गणनाथपूजनवशात् गङ्गादितीर्थप्रदः}
{षट्त्रिस्थः शुभकृच्च चित्रिततनुः कुर्यात्सदा मङ्गलम्}

॥इति श्री नवग्रहमङ्गलाष्टकं सम्पूर्णम्॥