 % !TeX program = XeLaTeX
 % !TeX root = ../../shloka.tex

\sect{सूर्यस्तवराजः}

\uvacha{वसिष्ठ उवाच}

\twolineshloka
{स्तुवंस्तत्र ततः साम्बः कृशो धमनि सन्ततः}
{राजन् नामसहस्रेण सहस्रांशुं दिवाकरम्} %॥१॥

\twolineshloka
{खिद्यमानं तु तं दृष्ट्वा सूर्यः कृष्णात्मजं तदा}
{स्वप्ने तु दर्शनं दत्त्वा पुनर्वचनमब्रवीत्} %॥२॥

\uvacha{श्री सूर्य उवाच}

\twolineshloka
{साम्ब साम्ब महाबाहो शृणु जाम्बवतीसुत}
{अलं नामसहस्रेण पठंस्त्वेवं स्तवं शुभम्} %॥३॥

\twolineshloka
{यानि नामानि गुह्यानि पवित्राणि शुभानि च}
{तानि ते कीर्तयिष्यामि श्रुत्वा तमवधारय} %॥४॥

\twolineshloka
{ॐ विकर्तनो विवस्वांश्च मार्तण्डो भास्करो रविः}
{लोकप्रकाशकः श्रीमाँल्लोकचक्षुर्ग्रहेश्वरः} %॥५॥

\twolineshloka
{लोकसाक्षी त्रिलोकेशः कर्ता हर्ता तमिस्रहा}
{तपनस्तापनश्चैव शुचिः सप्ताश्ववाहनः} %॥६॥

\twolineshloka
{गभस्तिहस्तो ब्रह्मा च सर्वदेवनमस्कृतः}
{एकविंशतिरित्येष स्तव इष्टः सदा मम} %॥७॥

\twolineshloka
{शरीरारोग्यदश्चैव धनवृद्धियशस्करः}
{स्तवराज इति ख्यातस्त्रिषु लोकेषु विश्रुतः} %॥८॥

\twolineshloka
{य एतेन महाबाहो द्वे सन्ध्येऽस्तमनोदये}
{स्तौति मां प्रणतो भूत्वा सर्वपापैः प्रमुच्यते} %॥९॥

\twolineshloka
{कायिकं वाचिकं चापि मानसं यच्च दुष्कृतम्}
{तत् सर्वमेकजाप्येन प्रणश्यति ममाग्रतः} %॥१०॥

\twolineshloka
{एष जप्यश्च होमश्च सन्ध्योपासनमेव च}
{बलिमन्त्रोऽर्घ्यमन्त्रश्च धूपमन्त्रस्तथैव च} %॥११॥

\twolineshloka
{अन्नप्रदाने स्नाने च प्रणिपाते प्रदक्षिणे}
{पूजितोऽयं महामन्त्रः सर्वव्याधिहरः शुभः} %॥१२॥

\twolineshloka
{एवमक्त्वा तु भगवान् भास्करो जगदीश्वरः}
{आमन्त्र्य कृष्णतनयं तत्रैवान्तरधीयत} %॥१३॥

\twolineshloka
{साम्बोऽपि स्तवराजेन स्तुत्वा सप्ताश्ववाहनम्}
{पूतात्मा नीरुजः श्रीमांस्तस्माद्रोगाद्विमुक्तवान्} %॥१४॥


॥इति साम्बपुराणे रोगापनयने श्रीसूर्यवक्त्रविनिर्गतः स्तवराजः सम्पूर्णः॥