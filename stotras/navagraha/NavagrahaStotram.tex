% !TeX program = XeLaTeX
% !TeX root = ../../shloka.tex

\sect{नवग्रहस्तोत्रम्}

\twolineshloka
{जपाकुसुमसङ्काशं काश्यपेयं महद्युतिम्}
{तमोऽरिं सर्वपापघ्नं प्रणतोऽस्मि दिवाकरम्}

\twolineshloka
{दधिशङ्खतुषाराभं क्षीरोदार्णवसम्भवम्}
{नमामि शशिनं सोमं शम्भोर्मुकुटभूषणम्}

\twolineshloka
{धरणीगर्भसम्भूतं विद्युत्कान्तिसमप्रभम्}
{कुमारं शक्तिहस्तं च मङ्गलं प्रणमाम्यहम्}

\twolineshloka
{प्रियङ्गुकलिकाश्यामं रूपेणाप्रतिमं बुधम्}
{सौम्यं सौम्यगुणोपेतं तं बुधं प्रणमाम्यहम्}

\twolineshloka
{देवानां च ऋषीणां च गुरुं काञ्चनसन्निभम्}
{बुद्धिभूतं त्रिलोकेशं तं नमामि बृहस्पतिम्}

\twolineshloka
{हिमकुन्दमृणालाभं दैत्यानां परमं गुरुम्}
{सर्वशास्त्रप्रवक्तारं भार्गवं प्रणमाम्यहम्}

\twolineshloka
{नीलाञ्जनसमाभासं रविपुत्रं यमाग्रजम्}
{छायामार्तण्डसम्भूतं तं नमामि शनैश्चरम्}

\twolineshloka
{अर्धकायं महावीर्यं चन्द्रादित्यविमर्दनम्}
{सिंहिकागर्भसम्भूतं तं राहुं प्रणमाम्यहम्}

\twolineshloka
{पलाशपुष्पसङ्काशं तारकाग्रहमस्तकम्}
{रौद्रं रौद्रात्मकं घोरं तं केतुं प्रणमाम्यहम्}

\twolineshloka
{इति व्यासमुखोद्गीतं यः पठेत् सुसमाहितः}
{दिवा वा यदि वा रात्रौ विघ्नशान्तिर्भविष्यति}

\twolineshloka
{नरनारीनृपाणां च भवेद्दुःस्वप्ननाशनम्}
{ऐश्वर्यमतुलं तेषामारोग्यं पुष्टिवर्धनम्}

\twolineshloka
{ग्रहनक्षत्रजाः पीडास्तस्कराग्निसमुद्भवाः}
{ताः सर्वाः प्रशमं यान्ति व्यासो ब्रूते न संशयः}

॥इति श्रीव्यासविरचितं नवग्रहस्तोत्रं सम्पूर्णम्॥
