% !TeX program = XeLaTeX
% !TeX root = ../../shloka.tex
\sect{गायत्री स्तवनम्}

\twolineshloka
{यन्मण्डलं दीप्तिकरं विशालं रत्नप्रभं तीव्रमनादिरूपम्}
{दारिद्र्यदुःखक्षयकारणं च पुनातु मां तत्सवितुर्वरेण्यम्}

\twolineshloka
{यन्मण्डलं देवगणैः सुपूजितं   विप्रैः स्तुतं मानवमुक्तिकोविदम्}
{तं देवदेवं प्रणमामि सूर्यं   पुनातु मां तत्सवितुर्वरेण्यम्}

\twolineshloka
{यन्मण्डलं ज्ञानघनं त्वगम्यं   त्रैलोक्यपूज्यं त्रिगुणात्मरूपम्}
{समस्त-तेजोमय-दिव्यरूपं   पुनातु मां तत्सवितुर्वरेण्यम्}

\twolineshloka
{यन्मण्डलं गूढमतिप्रबोधं   धर्मस्य वृद्धिं कुरुते जनानाम्}
{यत्सर्वपापक्षयकारणं च   पुनातु मां तत्सवितुर्वरेण्यम्}

\twolineshloka
{यन्मण्डलं व्याधिविनाशदक्षं   यदृग्यजुःसामसु सम्प्रगीतम् }
{प्रकाशितं येन च भूर्भुवः स्वः   पुनातु मां तत्सवितुर्वरेण्यम्}

\twolineshloka
{यन्मण्डलं वेदविदो वदन्ति   गायन्ति यच्चारण-सिद्धसङ्घाः}
{यद्योगिनो योगजुषां च सङ्घाः   पुनातु मां तत्सवितुर्वरेण्यम्}

\twolineshloka
{यन्मण्डलं सर्वजनेषु पूजितं   ज्योतिश्च कुर्यादिह मर्त्यलोके}
{यत्कालकल्पक्षयकारणं च   पुनातु मां तत्सवितुर्वरेण्यम्}

\twolineshloka
{यन्मण्डलं विश्वसृजां प्रसिद्धमुत्पत्ति-रक्षा-प्रलय-प्रगल्भम्}
{यस्मिञ्जगत्संहरतेऽखिलं च   पुनातु मां तत्सवितुर्वरेण्यम् }

\twolineshloka
{यन्मण्डलं सर्वजनस्य विष्णोरात्मा   परं धाम विशुद्धतत्त्वम्}
{सूक्ष्मान्तरैर्योगपथानुगम्यं   पुनातु मां तत्सवितुर्वरेण्यम्}

\twolineshloka
{यन्मण्डलं ब्रह्मविदो वदन्ति   गायन्ति यच्चारण-सिद्धसङ्घाः}
{यन्मण्डलं वेदविदः स्मरन्ति   पुनातु मां तत्सवितुर्वरेण्यम्}

\twolineshloka
{यन्मण्डलं वेदविदोपगीतं   यद्योगिनां योगपथानुगम्यम्}
{तत्सर्ववेदं प्रणमामि सूर्यं   पुनातु मां तत्सवितुर्वरेण्यम्}
