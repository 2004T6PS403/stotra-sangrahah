% !TeX program = XeLaTeX
% !TeX root = ../../shloka.tex

\sect{आदित्यहृदयम्}
\twolineshloka
{ततो युद्धपरिश्रान्तं समरे चिन्तया स्थितम्}
{रावणं चाग्रतो दृष्ट्वा युद्धाय समुपस्थितम्}

\twolineshloka
{दैवतैश्च समागम्य द्रष्टुमभ्यागतो रणम्}
{उपागम्याब्रवीद्रामम् अगस्त्यो भगवान् ऋषिः}

\twolineshloka
{राम राम महाबाहो शृणु गुह्यं सनातनम्}
{येन सर्वानरीन् वत्स समरे विजयिष्यसि}

\twolineshloka
{आदित्यहृदयं पुण्यं सर्वशत्रुविनाशनम्}
{जयावहं जपेन्नित्यम् अक्षय्यं परमं शिवम्}

\twolineshloka
{सर्वमङ्गलमाङ्गल्यं सर्वपापप्रणाशनम्}
{चिन्ताशोकप्रशमनम् आयुर्वर्धनमुत्तमम्}

\twolineshloka
{रश्मिमन्तं समुद्यन्तं देवासुरनमस्कृतम्}
{पूजयस्व विवस्वन्तं भास्करं भुवनेश्वरम्}

\twolineshloka
{सर्वदेवात्मको ह्येष तेजस्वी रश्मिभावनः}
{एष देवासुरगणान् लोकान् पाति गभस्तिभिः}

\twolineshloka
{एष ब्रह्मा च विष्णुश्च शिवः स्कन्दः प्रजापतिः}
{महेन्द्रो धनदः कालो यमः सोमो ह्यपां पतिः}

\twolineshloka
{पितरो वसवः साध्या ह्यश्विनौ मरुतो मनुः}
{वायुर्वह्निः प्रजाप्राण ऋतुकर्ता प्रभाकरः}

\twolineshloka
{आदित्यः सविता सूर्यः खगः पूषा गभस्तिमान्}
{सुवर्णसदृशो भानुर्हिरण्यरेता दिवाकरः}

\twolineshloka
{हरिदश्वः सहस्रार्चिः सप्तसप्तिर्मरीचिमान्}
{तिमिरोन्मथनः शम्भुस्त्वष्टा मार्ताण्ड अंशुमान्}

\twolineshloka
{हिरण्यगर्भः शिशिरस्तपनो भास्करो रविः}
{अग्निगर्भोऽदितेः पुत्रः शङ्खः शिशिरनाशनः}

\twolineshloka
{व्योमनाथस्तमोभेदी ऋग्यजुस्सामपारगः}
{घनवृष्टिरपां मित्रो विन्ध्यवीथीप्लवङ्गमः}

\twolineshloka
{आतपी मण्डली मृत्युः पिङ्गलः सर्वतापनः}
{कविर्विश्वो महातेजा रक्तः सर्वभवोद्भवः}

\twolineshloka
{नक्षत्रग्रहताराणाम् अधिपो विश्वभावनः}
{तेजसामपि तेजस्वी द्वादशात्मन् नमोऽस्तुते}

\twolineshloka
{नमः पूर्वाय गिरये पश्चिमायाद्रये नमः}
{ज्योतिर्गणानां पतये दिनाधिपतये नमः}

\twolineshloka
{जयाय जयभद्राय हर्यश्वाय नमो नमः}
{नमो नमः सहस्रांशो आदित्याय नमो नमः}

\twolineshloka
{नम उग्राय वीराय सारङ्गाय नमो नमः}
{नमः पद्मप्रबोधाय मार्ताण्डाय नमो नमः}

\twolineshloka
{ब्रह्मेशानाच्युतेशाय सूर्यायादित्यवर्चसे}
{भास्वते सर्वभक्षाय रौद्राय वपुषे नमः}

\twolineshloka
{तमोघ्नाय हिमघ्नाय शत्रुघ्नायामितात्मने}
{कृतघ्नघ्नाय देवाय ज्योतिषां पतये नमः}

\twolineshloka
{तप्तचामीकराभाय वह्नये विश्वकर्मणे}
{नमस्तमोऽभिनिघ्नाय रुचये लोकसाक्षिणे}

\twolineshloka
{नाशयत्येष वै भूतं तदेव सृजति प्रभुः}
{पायत्येष तपत्येष वर्षत्येष गभस्तिभिः}

\twolineshloka
{एष सुप्तेषु जागर्ति भूतेषु परिनिष्ठितः}
{एष एवाग्निहोत्रं च फलं चैवाग्निहोत्रिणाम्}

\twolineshloka
{वेदाश्च क्रतवश्चैव क्रतूनां फलमेव च}
{यानि कृत्यानि लोकेषु सर्व एष रविः प्रभुः}

%\dnsub{फलश्रुतिः}
\twolineshloka
{एनमापत्सु कृच्छ्रेषु कान्तारेषु भयेषु च}
{कीर्तयन् पुरुषः कश्चिन्नावसीदति राघव}

\twolineshloka
{पूजयस्वैनमेकाग्रो देवदेवं जगत्पतिम्}
{एतत् त्रिगुणितं जप्त्वा युद्धेषु विजयिष्यसि}

\twolineshloka
{अस्मिन् क्षणे महाबाहो रावणं त्वं वधिष्यसि}
{एवमुक्त्वा तदाऽगस्त्यो जगाम च यथाऽऽगतम्}

\twolineshloka
{एतच्छ्रुत्वा महातेजा नष्टशोकोऽभवत्तदा}
{धारयामास सुप्रीतो राघवः प्रयतात्मवान्}

\twolineshloka
{आदित्यं प्रेक्ष्य जप्त्वा तु परं हर्षमवाप्तवान्}
{त्रिराचम्य शुचिर्भूत्वा धनुरादाय वीर्यवान्}

\twolineshloka
{रावणं प्रेक्ष्य हृष्टात्मा युद्धाय समुपागमत्}
{सर्वयत्नेन महता वधे तस्य धृतोऽभवत्}

\twolineshloka
{अथ रविरवदन्निरीक्ष्य रामं मुदितमनाः परमं प्रहृष्यमाणः}
{निशिचरपतिसङ्क्षयं विदित्वा सुरगणमध्यगतो वचस्त्वरेति}
॥इत्यार्षे श्रीमद्रामायणे वाल्मीकीये आदिकाव्ये युद्धकाण्डे आदित्यहृदयं नाम सप्तोत्तरशततमः सर्गः॥