% !TeX program = XeLaTeX
% !TeX root = ../../shloka.tex
% Corrected using https://youtu.be/32TGOFRIVmE (one shloka missing there has been retained)
\sect{सूर्यमण्डल स्तोत्रम्}
\twolineshloka*
{नमोऽस्तु सूर्याय सहस्ररश्मये सहस्रशाखान्वितसम्भवात्मने}
{सहस्रयोगोद्भवभावभागिने सहस्रसङ्ख्यायुगधारिणे नमः}

\twolineshloka
{यन्मण्डलं दीप्तिकरं विशालं रत्नप्रभं तीव्रमनादिरूपम्}
{दारिद्र्यदुःखक्षयकारणं च पुनातु मां तत्सवितुर्वरेण्यम्}

\twolineshloka
{यन्मण्डलं देवगणैः सुपूजितं   विप्रैः स्तुतं भावनमुक्तिकोविदम्}
{तं देवदेवं प्रणमामि सूर्यं   पुनातु मां तत्सवितुर्वरेण्यम्}

\twolineshloka
{यन्मण्डलं ज्ञानघनं त्वगम्यं   त्रैलोक्यपूज्यं त्रिगुणात्मरूपम्}
{समस्त-तेजोमय-दिव्यरूपं   पुनातु मां तत्सवितुर्वरेण्यम्}

\twolineshloka
{यन्मण्डलं गूढमतिप्रबोधं   धर्मस्य वृद्धिं कुरुते जनानाम्}
{यत्सर्वपापक्षयकारणं च   पुनातु मां तत्सवितुर्वरेण्यम्}

\twolineshloka
{यन्मण्डलं व्याधिविनाशदक्षं   यदृग्यजुःसामसु सम्प्रगीतम्}
{प्रकाशितं येन च भूर्भुवः स्वः   पुनातु मां तत्सवितुर्वरेण्यम्}

\twolineshloka
{यन्मण्डलं वेदविदो वदन्ति   गायन्ति यच्चारण-सिद्धसङ्घाः}
{यद्योगिनो योगजुषां च सङ्घाः   पुनातु मां तत्सवितुर्वरेण्यम्}

\twolineshloka
{यन्मण्डलं सर्वजनैश्च पूजितं   ज्योतिश्च कुर्यादिह मर्त्यलोके}
{यत्कालकालाद्यमनादिरूपं   पुनातु मां तत्सवितुर्वरेण्यम्}

\twolineshloka
{यन्मण्डलं विष्णुचतुर्मुखाख्यं यदक्षरं पापहरं जनानाम्}
{यत्कालकल्पक्षयकारणं च   पुनातु मां तत्सवितुर्वरेण्यम्}

\twolineshloka
{यन्मण्डलं विश्वसृजं प्रसिद्धमुत्पत्ति-रक्षा-प्रलय-प्रगल्भम्}
{यस्मिञ्जगत्संहरतेऽखिलं च   पुनातु मां तत्सवितुर्वरेण्यम्}

\twolineshloka
{यन्मण्डलं सर्वगतस्य विष्णोरात्मा   परं धाम विशुद्धतत्त्वम्}
{सूक्ष्मान्तरैर्योगपथानुगम्यं   पुनातु मां तत्सवितुर्वरेण्यम्}

\twolineshloka
{यन्मण्डलं वेदविदो वदन्ति   गायन्ति यच्चारण-सिद्धसङ्घाः}
{यन्मण्डलं वेदविदः स्मरन्ति   पुनातु मां तत्सवितुर्वरेण्यम्}

\twolineshloka
{यन्मण्डलं वेदविदोपगीतं   यद्योगिनां योगपथानुगम्यम्}
{तत्सर्ववेद्यं  प्रणमामि सूर्यं   पुनातु मां तत्सवितुर्वरेण्यम्}

\twolineshloka
{सूर्यमण्डलसुस्त्रोत्रं यः पठेत् सततं नरः}
{सर्वपापविशुद्धात्मा  सूर्यलोके महीयते}

\closesection

\twolineshloka*
{यो देवः सविताऽस्माकं धियो धर्माधि-गोचरः}
{प्रेरयेत् तस्य यद्भर्गस्तद्वरेण्यमुपास्महे}

॥इति श्री भविष्योत्तरपुराणे श्रीकृष्णार्जुनसंवादे सूर्यमण्डलस्तोत्रं सम्पूर्णम्॥