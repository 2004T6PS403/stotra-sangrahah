% !TeX program = XeLaTeX
% !TeX root = ../../shloka.tex

\sect{शुक्रचतुर्विंशतिनामस्तोत्रम्}

\twolineshloka
{शृण्वन्तु मुनयः सर्वे शुक्रस्तोत्रमिदं शुभम्।}
{रहस्यं सर्वभूतानां शुक्रप्रीतिकरं शुभम्॥ १॥}

\twolineshloka
{येषां सङ्कीर्तनान्नित्यं सर्वान् कामानवाप्नुयात्}
{तानि शुक्रस्य नामानि कथयामि शुभानि च}

\twolineshloka
{शुक्रः शुभग्रहः श्रीमान् वर्षकृद्वर्षविघ्नकृत्}
{तेजोनिधिर्ज्ञानदाता योगी योगविदां वरः}

\twolineshloka
{दैत्यसञ्जीवनो धीरो दैत्यनेतोशना कविः}
{नीतिकर्ता ग्रहाधीशो विश्वात्मा लोकपूजितः}

\twolineshloka
{शुक्लमाल्याम्बरधरः श्रीचन्दनसमप्रभः}
{अक्षमालाधरः काव्यः तपोमूर्तिर्धनप्रदः}

\twolineshloka
{चतुर्विंशतिनामानि अष्टोत्तरशतं यथा}
{देवस्याग्रे विशेषेण पूजां कृत्वा विधानतः}

\twolineshloka
{य इदं पठति स्तोत्रं भार्गवस्य महात्मनः}
{विषमस्थोऽपि भगवान् तुष्टः स्यान्नात्र संशयः}

\fourlineindentedshloka
{स्तोत्रं भृगोरिदमनन्तगुणप्रदं यो}
{भक्त्या पठेच्च मनुजो नियतः शुचिः सन्}
{प्राप्नोति नित्यमतुलां श्रियमीप्सितार्थान्}
{राज्यं समस्तधनधान्ययुतां समृद्धिम्}

॥इति श्रीस्कान्दपुराणे श्री~शुक्रचतुर्विंशतिनामस्तोत्रं सम्पूर्णम्॥
