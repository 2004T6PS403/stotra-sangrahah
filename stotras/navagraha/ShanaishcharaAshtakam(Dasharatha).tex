% !TeX program = XeLaTeX
% !TeX root = ../../shloka.tex

\sect{दशरथकृत शनैश्चराष्टकम्}

अस्य श्रीशनैश्चरस्तोत्रमन्त्रस्य दशरथ ऋषिः। शनैश्चरो देवता। त्रिष्टुप् छन्दः। शनैश्चरप्रीत्यर्थे जपे विनियोगः।

दशरथ उवाच

\twolineshloka
{कोणोन्तको रौद्र यमोऽथ बभ्रुः कृष्णः शनिः पिङ्गलमन्दसौरिः}
{नित्यं स्मृतो यो हरते च पीडां तस्मै नमः श्रीरविनन्दनाय}

\twolineshloka
{सुरासुराः किम्पुरुषोरगेन्द्रा गन्धर्वविद्याधरपन्नगाश्च}
{पीड्यन्ति सर्वे विषमस्थितेन तस्मै नमः श्रीरविनन्दनाय}

\twolineshloka
{नरा नरेन्द्राः पशवो मृगेन्द्रा वन्याश्च ये कीटपतङ्गभृङ्गाः}
{पीड्यन्ति सर्वे विषमस्थितेन तस्मै नमः श्रीरविनन्दनाय}

\twolineshloka
{देशाश्च दुर्गाणि वनानि यत्र सेनानिषेशाः पुरपत्तनानि}
{पीड्यन्ति सर्वे विषमस्थितेन तस्मै नमः श्रीरविनन्दनाय}

\twolineshloka
{तिलैर्यवैर्माषगुडान्नदानैर्लोहेन नीलाम्बरदानतो वा}
{प्रीणाति मन्त्रैर्निजवासरे च तस्मै नमः श्रीरविनन्दनाय}

\twolineshloka
{प्रयागकूले यमुनातटे च सरस्वतीपुण्यजले गुहायाम्}
{यो योगिनां ध्यानगतोऽपि सूक्ष्मस्तस्मै नमः श्रीरविनन्दनाय}
\twolineshloka
{अन्यप्रदेशात्स्वगृहं प्रविष्टस्तदीयवारे स नरः सुखी स्यात्}
{गृहाद्गतो यो न पुनः प्रयाति तस्मै नमः श्रीरविनन्दनाय}

\twolineshloka
{स्रष्टा स्वयम्भूर्भुवनत्रयस्य त्राता हरीशो हरते पिनाकी}
{एकस्त्रिधा ऋग्यजुस्साममूर्तिस्तस्मै नमः श्रीरविनन्दनाय}

\twolineshloka
{शन्यष्टकं यः प्रयतः प्रभाते नित्यं सुपुत्रैः पशुबान्धवैश्च}
{पठेत्तु सौख्यं भुवि भोगयुक्तः प्राप्नोति निर्वाणपदं तदन्ते}

\twolineshloka
{कोणस्थः पिङ्गलो बभ्रुः कृष्णो रौद्रोऽन्तको यमः}
{सौरिः शनैश्चरो मन्दः पिप्पलादेन संस्तुतः}

\twolineshloka
{एतानि दशनामानि प्रातरुत्थाय यः पठेत्}
{शनैश्चरकृता पीडा न कदाचिद्भविष्यति}

॥इति श्री दशरथकृतं श्री शनैश्चराष्टकं सम्पूर्णम्॥

