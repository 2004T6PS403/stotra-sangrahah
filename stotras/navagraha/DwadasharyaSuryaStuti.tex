% !TeX program = XeLaTeX
% !TeX root = ../../shloka.tex

\sect{द्वादशार्यासूर्यस्तुतिः}

\twolineshloka
{उद्यन्नद्य विवस्वान् आरोहन्नुत्तरां दिवं देवः}
{हृद्रोगं मम सूर्यो हरिमाणं चऽऽशु नाशयतु}

\twolineshloka
{निमिषार्धेनैकेन द्वे च शते द्वे सहस्रे द्वे}
{क्रममाण योजनानां नमोऽस्तु ते नळिननाथाय}

\twolineshloka
{कर्म ज्ञान ख दशकं मनश्च जीव इति विश्वसर्गाय}
{द्वादशधा यो विचरति स द्वादशमूर्तिरस्तु मोदाय}

\twolineshloka
{त्वं हि यजू ऋक् सामस्त्वमागमस्त्वं वषट्कारः}
{त्वं विश्वं त्वं हंसस्त्वं भानो परमहंसश्च}

\twolineshloka
{शिवरूपात् ज्ञानमहं त्वत्तो मुक्तिं जनार्दनाकारात्}
{शिखिरूपादैश्वर्यं त्वत्तश्चारोग्यमिच्छामि}

\twolineshloka
{त्वचि दोषा दृशि दोषाः हृदि दोषा येऽखिलेन्द्रियजदोषाः}
{तान् पूषा हतदोषः किञ्चित् रोषाग्निना दहतु}

\twolineshloka
{धर्मार्थकाममोक्ष प्रतिरोधानुग्रतापवेग करान्}
{बन्दीकृतेन्द्रियगणान् गदान् विखण्डयतु चण्डांशुः}

\twolineshloka
{येन विनेदं तिमिरं जगदेत्य ग्रसति चरमचरमखिलम्}
{धृतबोधं तं नळिनीभर्तारं हर्तारमापदामीडे}

\twolineshloka
{यस्य सहस्राभीशोरभीशुलेशो हिमांशुबिम्बगतः}
{भासयति नक्तमखिलं भेदयतु विपद्गणानरुणः}

\twolineshloka
{तिमिरमिव नेत्रतिमिरं पटलमिवाऽशेषरोगपटलं नः}
{काशमिवाधिनिकायं कालपिता रोगयुक्ततां हरतात्}

\twolineshloka
{वाताश्मरीगदार्शस्त्वग्दोषमहोदरप्रमेहांश्च}
{ग्रहणीभगन्दराख्या महतीस्त्वं मे रुजो हंसि}

\twolineshloka
{त्वं माता त्वं शरणं त्वं धाता त्वं धनं त्वमाचार्यः}
{त्वं त्राता त्वं हर्ता विपदाम् अर्क प्रसीद मम भानो}

\twolineshloka*
{इत्यार्याद्वादशकं साम्बस्य पुरो नभःस्थलात्पतितम्}
{पठतां भाग्यसमृद्धिः समस्तरोगक्षयश्च स्यात्}

॥इति श्री द्वादशार्यासूर्यस्तुतिः सम्पूर्णः॥