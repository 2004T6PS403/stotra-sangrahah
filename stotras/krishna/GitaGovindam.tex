% !TeX program = XeLaTeX
% !TeX root = ../../shloka.tex

\bigskip
\sect{गीतगोविन्दम्}
\bigskip
\dnsub{श्री जयदेव ध्यानम्}
\fourlineindentedshloka*
{श्रीगोपालविलासिनी वलयसद्रत्नादिमुग्धाकृति}
{श्रीराधापतिपादपद्मभजनानन्दाब्धिमग्नोऽनिशम्}
{लोके सत्कविराजराज इति यः ख्यातो दयाम्भोनिधिः}
{तं वन्दे जयदेवसद्गुरुमहं पद्मावतीवल्लभम्}

%\clearpage
प्रलयपयोधिजले केशव धृतवानसि वेदम्।\\
विहितवहित्रचरित्रमखेदम्॥\\
केशव धृतमीनशरीर जय जगदीश हरे॥१॥

\smallskip
क्षितिरतिविपुलतरे केशव तव तिष्ठति पृष्ठे।\\
धरणिधरणकिणचक्रगरिष्ठे॥\\
केशव धृतकच्छपरूप जय जगदीश हरे॥२॥

\smallskip
वसति दशनशिखरे केशव धरणी तव लग्ना।\\
शशिनि कलङ्ककलेव निमग्ना॥\\
केशव धृतसूकररूप जय जगदीश हरे॥३॥

\smallskip
तव करकमलवरे केशव नखमद्भुतशृङ्गम्।\\
दलितहिरण्यकशिपुतनुभृङ्गम्॥\\
केशव धृतनरहरिरूप जय जगदीश हरे॥४॥

\smallskip
छलयसि विक्रमणे केशव बलिमद्भुतवामन।\\
पदनखनीरजनितजनपावन॥\\
केशव धृतवामनरूप जय जगदीश हरे॥५॥

\smallskip
क्षत्रियरुधिरमये केशव जगदपगतपापम्।\\
स्नपयसि पयसि शमितभवतापम्॥\\
केशव धृतभृगुपतिरूप जय जगदीश हरे॥६॥

\smallskip
वितरसि दिक्षु रणे केशव दिक्पतिकमनीयम्।\\
दशमुखमौलिबलिं रमणीयम्॥\\
केशव धृतरामशरीर जय जगदीश हरे॥७॥

\smallskip
वहसि वपुषि विशदे केशव वसनं जलदाभम्।\\
हलहतिभीतिमिलितयमुनाभम्॥\\
केशव धृतहलधररूप जय जगदीश हरे॥८॥

\smallskip
निन्दसि यज्ञविधेः केशव अहह श्रुतिजातम्।\\
सदयहृदयदर्शितपशुघातम्॥\\
केशव धृतबुद्धशरीर जय जगदीश हरे॥९॥

\smallskip
म्लेच्छनिवहनिधने केशव कलयसि करवालम्।\\
धूमकेतुमिव किमपि करालम्॥\\
केशव धृतकल्किशरीर जय जगदीश हरे॥१०॥

\smallskip
श्रीजयदेवकवेः केशव इदमुदितमुदारम्।\\
शृणु शुभदं सुखदं भवसारम्॥\\
केशव धृतदशविधरूप जय जगदीश हरे॥

\smallskip
\fourlineindentedshloka*
{वेदानुद्धरते जगन्निवहते भूगोलमुद्बिभ्रते}
{दैत्यं दारयते बलिं छलयते क्षत्रक्षयं कुर्वते}
{पौलस्त्यं जयते हलं कलयते कारुण्यमातन्वते}
{म्लेच्छान्मूर्च्छयते दशाकृतिकृते कृष्णाय तुभ्यं नमः}
॥इति श्री जयदेवविरचितं दशावतार-गीतगोविन्दं सम्पूर्णम्॥
