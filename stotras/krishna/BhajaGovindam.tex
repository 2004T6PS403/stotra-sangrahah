% !TeX program = XeLaTeX
% !TeX root = ../../shloka.tex

\clearpage
\sect{भज गोविन्दम्}
\setlength{\shlokaspaceskip}{6pt}
\setlength{\columnseprule}{0pt}
\setlength{\columnsep}{10pt}
\begin{large}
\begin{multicols}{2}
\begin{flushleft}
\fourlineindentedshloka
{भज गोविन्दं भज गोविन्दम्}
{गोविन्दं भज मूढमते}
{सम्प्राप्ते सन्निहिते काले}
{न हि न हि रक्षति डुकृञ् करणे}

\fourlineindentedshloka
{मूढ जहीहि धनागमतृष्णाम्}
{कुरु सद्बुद्धिं मनसि वितृष्णाम्}
{यल्लभसे निजकर्मोपात्तम्}
{वित्तं तेन विनोदय चित्तम्}

\fourlineindentedshloka
{नारीस्तनभरनाभीदेशं}
{दृष्ट्वा मा गा मोहावेशम्}
{एतन्मांसावसादि विकारम्}
{मनसि विचिन्तय वारं वारम्}

\fourlineindentedshloka
{नलिनीदलगतजलमतितरलम्}
{तद्वज्जीवितमतिशयचपलम्}
{विद्धि व्याध्यभिमानग्रस्तम्}
{लोकं शोकहतं च समस्तम्}

\fourlineindentedshloka
{यावद्वित्तोपार्जन-सक्तः}
{तावन्निज-परिवारो रक्तः}
{पश्चाज्जीवति जर्जरदेहे}
{वार्तां कोऽपि न पृच्छति गेहे}

\fourlineindentedshloka
{यावत् पवनो निवसति देहे}
{तावत् पृच्छति कुशलं गेहे}
{गतवति वायौ देहापाये}
{भार्या बिभ्यति तस्मिन् काये}

\fourlineindentedshloka
{बालस्तावत्क्रीडासक्तः}
{तरुणस्तावत्तरुणीसक्तः}
{वृद्धस्तावच्चिन्तासक्तः}
{परे ब्रह्मणि कोऽपि न सक्तः}

\fourlineindentedshloka
{का ते कान्ता कस्ते पुत्रः}
{संसारोऽयमतीव विचित्रः}
{कस्य त्वं कः कुत आयातः}
{तत्त्वं चिन्तय तदिह भ्रातः}

\fourlineindentedshloka
{सत्सङ्गत्वे निःसङ्गत्वम्}
{निःसङ्गत्वे निर्मोहत्वम्}
{निर्मोहत्वे निश्चलितत्त्वम्}
{निश्चलितत्त्वे जीवन-मुक्तिः}

\fourlineindentedshloka
{वयसि गते कः कामविकारः}
{शुष्के नीरे कः कासारः}
{क्षीणे वित्ते कः परिवारः}
{ज्ञाते तत्त्वे कः संसारः}

\fourlineindentedshloka
{मा कुरु धनजनयौवनगर्वम्}
{हरति निमेषात् कालः सर्वम्}
{मायामयमिदमखिलं हित्वा}
{ब्रह्मपदं त्वं प्रविश विदित्वा}

\fourlineindentedshloka
{दिनयामिन्यौ सायं प्रातः}
{शिशिरवसन्तौ पुनरायातः}
{कालः क्रीडति गच्छत्यायुः}
{तदपि न मुञ्चत्याशावायुः}

\fourlineindentedshloka*
{द्वादशमञ्जरिकाभिरशेषः}
{कथितो वैयाकरणस्यैषः}
{उपदेशो भूद्विद्यानिपुणैः}
{श्रीमच्छङ्करभगवच्छरणैः}

\fourlineindentedshloka
{का ते कान्ता धनगतचिन्ता}
{वातुल किं तव नास्ति नियन्ता}
{त्रिजगति सज्जनसङ्गतिरेका}
{भवति भवार्णवतरणे नौका}

\fourlineindentedshloka
{जटिलो मुण्डी लुञ्छितकेशः}
{काषायाम्बरबहुकृतवेषः}
{पश्यन्नपि चन पश्यति मूढः}
{उदरनिमित्तं बहुकृतवेषः}

\fourlineindentedshloka
{अङ्गं गलितं पलितं मुण्डम्}
{दशनविहीनं जातं तुण्डम्}
{वृद्धो याति गृहीत्वा दण्डम्}
{तदपि न मुञ्चत्याशापिण्डम्}

\fourlineindentedshloka
{अग्रे वह्निः पृष्ठे भानुः}
{रात्रौ चुबुकसमर्पितजानुः}
{करतलभिक्षस्तरुतलवासः}
{तदपि न मुञ्चत्याशापाशः}

\fourlineindentedshloka
{कुरुते गङ्गासागरगमनं}
{व्रतपरिपालनमथवा दानम्}
{ज्ञानविहीनः सर्वमतेन}
{मुक्तिं न भजति जन्मशतेन}

\fourlineindentedshloka
{सुरमन्दिर-तरुमूल-निवासः}
{शय्या भूतलमजिनं वासः}
{सर्व-परिग्रह भोगत्यागः}
{कस्य सुखं न करोति विरागः}

\fourlineindentedshloka
{योगरतो वा भोगरतो वा}
{सङ्गरतो वा सङ्गविहीनः}
{यस्य ब्रह्मणि रमते चित्तं}
{नन्दति नन्दति नन्दत्येव}

\fourlineindentedshloka
{भगवद्गीता किञ्चिदधीता}
{गङ्गाजललव-कणिका पीता}
{सकृदपि येन मुरारि समर्चा}
{क्रियते तस्य यमेन न चर्चा}

\fourlineindentedshloka
{पुनरपि जननं पुनरपि मरणम्}
{पुनरपि जननी-जठरे शयनम्}
{इह संसारे बहुदुस्तारे}
{कृपयाऽपारे पाहि मुरारे}

\fourlineindentedshloka
{रथ्या-चर्पट-विरचित-कन्थः}
{पुण्यापुण्य-विवर्जित-पन्थः}
{योगी योगनियोजित चित्तो}
{रमते बालोन्मत्तवदेव}

\fourlineindentedshloka
{कस्त्वं कोऽहं कुत आयातः}
{का मे जननी को मे तातः}
{इति परिभावय सर्वमसारम्}
{विश्वं त्यक्त्वा स्वप्नविचारम्}

\fourlineindentedshloka
{त्वयि मयि चान्यत्रैको विष्णुः}
{व्यर्थं कुप्यसि मय्यसहिष्णुः}
{सर्वस्मिन्नपि पश्यात्मानं}
{सर्वत्रोत्सृज भेदाज्ञानम्}

\fourlineindentedshloka
{शत्रौ मित्रे पुत्रे बन्धौ}
{मा कुरु यत्नं विग्रहसन्धौ}
{भव समचित्तः सर्वत्र त्वम्}
{वाञ्छस्यचिराद्यदि विष्णुत्वम्}

\fourlineindentedshloka
{कामं क्रोधं लोभं मोहं}
{त्यक्त्वाऽऽत्मानं भावय कोऽहम्}
{आत्मज्ञानविहीना मूढाः}
{ते पच्यन्ते नरकनिगूढाः}

\fourlineindentedshloka
{गेयं गीता नामसहस्रं}
{ध्येयं श्रीपति-रूपमजस्रम्}
{नेयं सज्जन-सङ्गे चित्तं}
{देयं दीनजनाय च वित्तम्}

\fourlineindentedshloka
{सुखतः क्रियते रामाभोगः}
{पश्चाद्धन्त शरीरे रोगः}
{यद्यपि लोके मरणं शरणं}
{तदपि न मुञ्चति पापाचरणम्}

\fourlineindentedshloka
{अर्थमनर्थं भावय नित्यं}
{नास्ति ततः सुखलेशः सत्यम्}
{पुत्रादपि धनभाजां भीतिः}
{सर्वत्रैषा विहिता रीतिः}

\fourlineindentedshloka
{प्राणायामं प्रत्याहारं}
{नित्यानित्य-विवेकविचारम्}
{जाप्यसमेत-समाधिविधानं}
{कुर्ववधानं महदवधानम्}

\fourlineindentedshloka
{गुरुचरणाम्बुज-निर्भर-भक्तः}
{संसारादचिराद्भव मुक्तः}
{सेन्द्रियमानस-नियमादेवं}
{द्रक्ष्यसि निजहृदयस्थं देवम्}

\fourlineindentedshloka
{मूढः कश्चन वैयाकरणो}
{डुकृञ्करणाध्ययन-धुरिणः}
{श्रीमच्छङ्कर-भगवच्छिष्यैः}
{बोधित आसिच्छोधितकरणः}

\fourlineindentedshloka
{भज गोविन्दं भज गोविन्दम्}
{गोविन्दं भज मूढमते}
{नामस्मरणादन्यमुपायं}
{न हि पश्यामो भवतरणे}

\fourlineshloka{}{}{}{}
\fourlineshloka{}{}{}{}
\end{flushleft}
\end{multicols}
॥इति श्रीमच्छङ्कराचार्यविरचितं भज गोविन्दं सम्पूर्णम्॥
\end{large}
\setlength{\shlokaspaceskip}{24pt}
\setlength{\columnseprule}{1pt}
\setlength{\columnsep}{30pt}