% !TeX program = XeLaTeX
% !TeX root = ../../shloka.tex
\sect{भज गोविन्दम्}
\setlength{\shlokaspaceskip}{6pt}
\setlength{\columnseprule}{0pt}
\setlength{\columnsep}{10pt}
\begin{large}
\begin{multicols}{2}
\begin{flushleft}
\fourlineindentedshloka
{भज गोविन्दं भज गोविन्दम्}
{गोविन्दं भज मूढमते}
{सम्प्राप्ते सन्निहिते काले}
{न हि न हि रक्षति डुकृञ् करणे}

\fourlineindentedshloka
{मूढ जहीहि धनागमतृष्णाम्}
{कुरु सद्बुद्धिं मनसि वितृष्णाम्}
{यल्लभसे निजकर्मोपात्तम्}
{वित्तं तेन विनोदय चित्तम्}

\fourlineindentedshloka
{यावद्वित्तोपार्जन-सक्तः}
{तावन्निज-परिवारो रक्तः}
{पश्चाज्जीवति जर्जरदेहे}
{वार्तां कोऽपि न पृच्छति गेहे}

\fourlineindentedshloka
{मा कुरु धनजनयौवनगर्वम्}
{हरति निमेषात् कालः सर्वम्}
{मायामयमिदमखिलं हित्वा}
{ब्रह्मपदं त्वं प्रविश विदित्वा}

\fourlineindentedshloka
{सुरमन्दिर-तरुमूल-निवासः}
{शय्या भूतलमजिनं वासः}
{सर्व-परिग्रह भोगत्यागः}
{कस्य सुखं न करोति विरागः}

\fourlineindentedshloka
{भगवद्गीता किञ्चिदधीता}
{गङ्गाजललव-कणिका पीता}
{सकृदपि येन मुरारि समर्चा}
{क्रियते तस्य यमेन न चर्चा}

\fourlineindentedshloka
{पुनरपि जननं पुनरपि मरणम्}
{पुनरपि जननी-जठरे शयनम्}
{इह संसारे बहुदुस्तारे}
{कृपयाऽपारे पाहि मुरारे}

\fourlineindentedshloka
{गेयं गीता नामसहस्रं}
{ध्येयं श्रीपति-रूपमजस्रम्}
{नेयं सज्जन-सङ्गे चित्तं}
{देयं दीनजनाय च वित्तम्}

\fourlineindentedshloka
{अर्थमनर्थं भावय नित्यं}
{नास्ति ततः सुखलेशः सत्यम्}
{पुत्रादपि धनभाजां भीतिः}
{सर्वत्रैषा विहिता रीतिः}

\fourlineindentedshloka
{गुरुचरणाम्बुज-निर्भर-भक्तः}
{संसारादचिराद्भव मुक्तः}
{सेन्द्रियमानस-नियमादेवं}
{द्रक्ष्यसि निजहृदयस्थं देवम्}

\end{flushleft}
\end{multicols}
॥इति श्रीमच्छङ्कराचार्यविरचितं भज गोविन्दं सम्पूर्णम्॥
\end{large}
\setlength{\shlokaspaceskip}{24pt}
\setlength{\columnseprule}{1pt}
\setlength{\columnsep}{30pt}