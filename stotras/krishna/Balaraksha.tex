% !TeX program = XeLaTeX
% !TeX root = ../../shloka.tex

\sect{बालरक्षा}

\twolineshloka*
{नमोऽस्तु ते व्यास विशालबुद्धे फुल्लारविन्दायतपत्रनेत्र}
{येन त्वया भारततैलपूर्णः प्रज्वालितो ज्ञानमयः प्रदीपः}

\fourlineindentedshloka*
{कस्तूरीतिलकं ललाटपटले वक्षःस्थले कौस्तुभम्}
{नासाग्रे वरमौक्तिकं करतले वेणुः करे कङ्कणम्}
{सर्वाङ्गे हरिचन्दनं सुललितं कण्ठे च मुक्तावली}
{गोपस्त्रीपरिवेष्टितो विजयते गोपालचूडामणिः}

\fourlineindentedshloka*
{अस्ति स्वस्तरुणीकराग्रविगलत् कल्पप्रसूनाप्लुतम्}
{वस्तुप्रस्तुतवेणुनादलहरी निर्वाणनिर्व्याकुलम्}
{स्रस्तस्रस्तनिबद्धनीविविलसत् गोपीसहस्रावृतम्}
{हस्तन्यस्तनतापवर्गमखिलोदारं किशोराकृति}

गोप्य ऊचुः

\fourlineindentedshloka
{अव्यादजोऽङ्घ्रि मणिमांस्तव जान्वथोरू}
{यज्ञोऽच्युतः कटितटं जठरं हयास्यः}
{हृत्केशवस्त्वदुर ईश इनस्तु कण्ठम्}
{विष्णुर्भुजं मुखमुरुक्रम ईश्वरः कम्}

\fourlineindentedshloka
{चक्र्यग्रतः सहगदो हरिरस्तु पश्चात्}
{त्वत्पार्श्वयोर्धनुरसी मधुहाऽजनश्च}
{कोणेषु शङ्ख उरुगाय उपर्युपेन्द्रः}
{तार्क्ष्यः क्षितौ हलधरः पुरुषः समन्तात्}

\twolineshloka
{इन्द्रियाणि हृषीकेशः प्राणान्नारायणोऽवतु}
{श्वेतद्वीपपतिश्चित्तं मनो योगेश्वरोऽवतु}

\twolineshloka
{पृश्निगर्भस्तु ते बुद्धिमात्मानं भगवान्परः}
{क्रीडन्तं पातु गोविन्दः शयानं पातु माधवः}

\twolineshloka
{व्रजन्तमव्याद्वैकुण्ठ आसीनं त्वां श्रियः पतिः}
{भुञ्जानं यज्ञभुक्पातु सर्वग्रहभयङ्करः}

\twolineshloka
{डाकिन्यो यातुधान्यश्च कुष्माण्डा येऽर्भकग्रहाः}
{भूतप्रेतपिशाचाश्च यक्षरक्षोविनायकाः}

\twolineshloka
{कोटरा रेवती ज्येष्ठा पूतना मातृकादयः}
{उन्मादा ये ह्यपस्मारा देहप्राणेन्द्रियद्रुहः}

\twolineshloka
{स्वप्नदृष्टा महोत्पाता वृद्धबालग्रहाश्च ये}
{सर्वे नश्यन्तु ते विष्णोर्नामग्रहणभीरवः}

{॥इति श्रीमद्भागवते महापुराणे पारमहंस्यां संहितायां दशमस्कन्धे षष्ठमेऽध्याये गोपीकृतबालरक्षा सम्पूर्णा॥}