% !TeX program = XeLaTeX
% !TeX root = ../../shloka.tex

\sect{कालनेमिसम्भाषणम्}


\addtocounter{shlokacount}{45}

\uvacha{कालनेमिरुवाच}

\twolineshloka
{सीतां प्रयच्छ रामाय राज्यं देहि विभीषणे}
{वनं याहि महाबाहो रम्यं मुनिगणाश्रयम्} %6-46

\twolineshloka
{स्नात्वा प्रातः शुभजले कृत्वा सन्ध्यादिकाः क्रियाः}
{तत एकान्तमाश्रित्य सुखासनपरिग्रहः} %6-47

\twolineshloka
{विसृज्य सर्वतः सङ्गमितरान् विषयान् बहिः}
{बहिःप्रवृत्ताक्षगणं शनैः प्रत्यक् प्रवाहय} %6-48

\twolineshloka
{प्रकृतेर्भिन्नमात्मानं विचारय सदाऽनघ}
{चराचरं जगत्कृत्स्नं देहबुद्धीन्द्रियादिकम्} %6-49

\twolineshloka
{आब्रह्मस्तम्बपर्यन्तं दृश्यते श्रूयते च यत्}
{सैषा प्रकृतिरित्युक्ता सैव मायेति कीर्तिता} %6-50

\twolineshloka
{सर्गस्थितिविनाशानां जगद्वृक्षस्य कारणम्}
{लोहितश्वेतकृष्णादि प्रजाः सृजति सर्वदा} %6-51

\twolineshloka
{कामक्रोधादिपुत्राद्यान् हिंसातृष्णादिकन्यकाः}
{मोहयन्त्यनिशं देवमात्मानं स्वैर्गुणैर्विभुम्} %6-52

\twolineshloka
{कर्तृत्वभोक्तृत्वमुखान् स्वगुणानात्मनीश्वरे}
{आरोप्य स्ववशं कृत्वा तेन क्रीडति सर्वदा} %6-53

\twolineshloka
{शुद्धोऽप्यात्मा यया युक्तः पश्यतीव सदा बहिः}
{विस्मृत्य च स्वमात्मानं मायागुणविमोहितः} %6-54

\twolineshloka
{यदा सद्गुरुणा युक्तो बोध्यते बोधरूपिणा}
{निवृत्तदृष्टिरात्मानं पश्यत्येव सदा स्फुटम्} %6-55

\twolineshloka
{जीवन्मुक्तः सदा देही मुच्यते प्राकृतैर्गुणैः}
{त्वमप्येवं सदाऽऽत्मानं विचार्य नियतेन्द्रियः} %6-56

\twolineshloka
{प्रकृतेरन्यमात्मानं ज्ञात्वा मुक्तो भविष्यसि}
{ध्यातुं यद्यसमर्थोऽसि सगुणं देवमाश्रय} %6-57

\twolineshloka
{हृत्पद्मकर्णिके स्वर्णपीठे मणिगणान्विते}
{मृदुश्लक्ष्णतरे तत्र जानक्या सह संस्थितम्} %6-58

\twolineshloka
{वीरासनं विशालाक्षं विद्युत्पुञ्जनिभाम्बरम्}
{किरीटहारकेयूरकौस्तुभादिभिरन्वितम्} %6-59

\twolineshloka
{नूपुरैः कटकैर्भान्तं तथैव वनमालया}
{लक्ष्मणेन धनुर्द्वन्द्वकरेण परिसेवितम्} %6-60

\twolineshloka
{एवं ध्यात्वा सदाऽऽत्मानं रामं सर्वहृदि स्थितम्}
{भक्त्या परमया युक्तो मुच्यते नात्र संशयः} %6-61

\threelineshloka
{शृणु वै चरितं तस्य भक्तैर्नित्यमनन्यधीः}
{एवं चेत्कृतपूर्वाणि पापानि च महान्त्यपि}
{क्षणादेव विनश्यन्ति यथाऽग्नेस्तूलराशयः} %6-62

\fourlineindentedshloka
{भजस्व रामं परिपूर्णमेकम्}
{विहाय वैरं निजभक्तियुक्तः}
{हृदा सदा भावितभावरूपम्}
{अनामरूपं पुरुषं पुराणम्} %6-63

{॥इति श्रीमदध्यात्मरामायणे उमामहेश्वरसंवादे युद्धकाण्डे
षष्ठे सर्गे कालनेमिसम्भाषणम् सम्पूर्णम्॥}
