% !TeX program = XeLaTeX
% !TeX root = ../../shloka.tex

\sect{लक्ष्मण-सम्भाषणम्}
% \addtocounter{shlokacount}{18}
\uvacha{श्री  महादेव उवाच}

\twolineshloka
{सुप्तं रामं समालोक्य गुहः सोऽश्रुपरिप्लुतः}
{लक्ष्मणं प्राह विनयाद् भ्रातः पश्यसि राघवम्} %6-1

\twolineshloka
{शयानं कुशपत्रौघसंस्तरे सीतया सह}
{यः शेते स्वर्णपर्यङ्के स्वास्तीर्णे भवनोत्तमे} %6-2

\twolineshloka
{कैकेयी रामदुःखस्य कारणं विधिना कृता}
{मन्थराबुद्धिमास्थाय कैकेयी पापमाचरत्} %6-3

\twolineshloka
{तच्छ्रुत्वा लक्ष्मणः प्राह सखे शृणु वचो मम}
{कः कस्य हेतुर्दुःखस्य कश्च हेतुः सुखस्य च} %6-4

\onelineshloka
{स्वपूर्वार्जितकर्मैव कारणं सुखदुःखयोः} %6-5

\fourlineindentedshloka
{सुखस्य दुःखस्य न कोऽपि दाता}
{परो ददातीति कुबुद्धिरेषा}
{अहं करोमीति वृथाभिमानः}
{स्वकर्मसूत्रग्रथितो हि लोकः} %6-6

\twolineshloka
{सुहृन्मित्रार्युदासीनद्वेष्यमध्यस्थबान्धवाः}
{स्वयमेवाचरन् कर्म तथा तत्र विभाव्यते} %6-7

\twolineshloka
{सुखं वा यदि वा दुःखं स्वकर्मवशगो नरः}
{यद्यद्यथागतं तत्तद् भुक्त्वा स्वस्थमना भवेत्} %6-8

\twolineshloka
{न मे भोगागमे वाञ्छा न मे भोगविवर्जने}
{आगच्छत्वथ मागच्छत्वभोगवशगो भवेत्} %6-9

\twolineshloka
{स्वस्मिन् देशे च काले च यस्माद्वा येन केन वा}
{कृतं शुभाशुभं कर्म भोज्यं तत्तत्र नान्यथा} %6-10

\twolineshloka
{अलं हर्षविषादाभ्यां शुभाशुभफलोदये}
{विधात्रा विहितं यद्यत्तदलङ्घ्यं सुरासुरैः} %6-11

\twolineshloka
{सर्वदा सुखदुःखाभ्यां नरः प्रत्यवरुध्यते}
{शरीरं पुण्यपापाभ्यामुत्पन्नं सुखदुःखवत्} %6-12

\twolineshloka
{सुखस्यानन्तरं दुःखं दुःखस्यानन्तरं सुखम्}
{द्वयमेतद्धि जन्तूनामलङ्घ्यं दिनरात्रिवत्} %6-13

\twolineshloka
{सुखमध्ये स्थितं दुःखं दुःखमध्ये स्थितं सुखम्}
{द्वयमन्योन्यसंयुक्तं प्रोच्यते जलपङ्कवत्} %6-14

\twolineshloka
{तस्माद्धैर्येण विद्वांस इष्टानिष्टोपपत्तिषु}
{न हृष्यन्ति न मुह्यन्ति समं मायेति भावनात्} %6-15

{॥इति श्रीमदध्यात्मरामायणे उमामहेश्वरसंवादे
अयोध्याकाण्डे षष्ठे सर्गे  लक्ष्मण-सम्भाषणं  सम्पूर्णम्॥}
