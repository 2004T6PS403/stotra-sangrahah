% !TeX program = XeLaTeX
% !TeX root = ../../shloka.tex

\sect{कुम्भकर्णसम्भाषणम्}


\addtocounter{shlokacount}{56}

\uvacha{कुम्भकर्ण  उवाच}

\twolineshloka
{पुरा मन्त्रविचारे ते गदितं यन्मया नृप}
{तदद्य त्वामुपगतं फलं पापस्य कर्मणः} %7-57

\twolineshloka
{पूर्वमेव मया प्रोक्तो रामो नारायणः परः}
{सीता च योगमायेति बोधितोऽपि न बुध्यसे} %7-58

\twolineshloka
{एकदाऽहं वने सानौ विशालायां स्थितो निशि}
{दृष्टो मया मुनिः साक्षान्नारदो दिव्यदर्शनः} %7-59

\twolineshloka
{तमब्रवं महाभाग कुतो गन्तासि मे वद}
{इत्युक्तो नारदः प्राह देवानां मन्त्रणे स्थितः} %7-60

\twolineshloka
{तत्रोत्पन्नमुदन्तं ते वक्ष्यामि शृणु तत्त्वतः}
{युवाभ्यां पीडिता देवाः सर्वे विष्णुमुपागताः} %7-61

\twolineshloka
{ऊचुस्ते देवदेवेशं स्तुत्वा भक्त्या समाहिताः}
{जहि रावणमक्षोभ्यं देव त्रैलोक्यकण्टकम्} %7-62

\twolineshloka
{मानुषेण मृतिस्तस्य कल्पिता ब्रह्मणा पुरा}
{अतस्त्वं मानुषो भूत्वा जहि रावणकण्टकम्} %7-63

\twolineshloka
{तथेत्याह महाविष्णुः सत्यसङ्कल्प ईश्वरः}
{जातो रघुकुले देवो राम इत्यभिविश्रुतः} %7-64

\twolineshloka
{स हनिष्यति वः सर्वानित्युक्त्वा प्रययौ मुनिः}
{अतो जानीहि रामं त्वं परं ब्रह्म सनातनम्} %7-65

\twolineshloka
{त्यज वैरं भजस्वाद्य मायामानुषविग्रहम्}
{भजतो भक्तिभावेन प्रसीदति रघूत्तमः} %7-66

\twolineshloka
{भक्तिर्जनित्री ज्ञानस्य भक्तिर्मोक्षप्रदायिनी}
{भक्तिहीनेन यत्किञ्चित्कृतं सर्वमसत्समम्} %7-67

\twolineshloka
{अवताराः सुबहवो विष्णोर्लीलानुकारिणः}
{तेषां सहस्रसदृशो रामो ज्ञानमयः शिवः} %7-68

\twolineshloka
{रामं भजन्ति निपुणा मनसा वचसाऽनिशम्}
{अनायासेन संसारं तीर्त्वा यान्ति हरेः पदम्} %7-69

\fourlineindentedshloka
{ये राममेव सततं भुवि शुद्धसत्त्वा}
{ध्यायन्ति तस्य चरितानि पठन्ति सन्तः}
{मुक्तास्त एव भवभोगमहाहिपाशैः}
{सीतापतेः पदमनन्तसुखं प्रयान्ति} %7-70

{॥इति श्रीमदध्यात्मरामायणे उमामहेश्वरसंवादे युद्धकाण्डे सप्तमे
 सर्गे कुम्भकर्णसम्भाषणम् सम्पूर्णम्॥}
