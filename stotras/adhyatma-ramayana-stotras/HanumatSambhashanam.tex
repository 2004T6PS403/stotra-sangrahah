% !TeX program = XeLaTeX
% !TeX root = ../../shloka.tex

\sect{हनूमत्सम्भाषणम्}


\addtocounter{shlokacount}{14}

\fourlineindentedshloka
{विचार्य लोकस्य विवेकतो गतिम्}
{न राक्षसीं बुद्धिमुपैहि रावण}
{दैवीं गतिं संसृतिमोक्षहैतुकीम्}
{समाश्रयात्यन्तहिताय देहिनः} %4-15

\fourlineindentedshloka
{त्वं ब्रह्मणो ह्युत्तमवंशसम्भवः}
{पौलस्त्यपुत्रोऽसि कुबेरबान्धवः}
{देहात्मबुद्ध्याऽपि च पश्य राक्षसो}
{नास्यात्मबुद्ध्या किमु राक्षसो नहि} %4-16

\fourlineindentedshloka
{शरीरबुद्धीन्द्रियदुःखसन्ततिः}
{न ते न च त्वं तव निर्विकारतः}
{अज्ञानहेतोश्च तथैव सन्ततेः}
{असत्त्वमस्याः स्वपतो हि दृश्यवत्} %4-17

\fourlineindentedshloka
{इदं तु सत्यं तव नास्ति विक्रिया}
{विकारहेतुर्न च तेऽद्वयत्वतः}
{यथा नभः सर्वगतं न लिप्यते}
{तथा भवान् देहगतोऽपि सूक्ष्मकः}

\fourlineindentedshloka
{देहेन्द्रियप्राणशरीरसङ्गतः}
{त्वात्मेति बद्ध्वाखिलबन्धभाग्भवेत्} %4-18
{चिन्मात्रमेवाहमजोऽहमक्षरो}
{ह्यानन्दभावोऽहमिति प्रमुच्यते}

\fourlineindentedshloka
{देहोऽप्यनात्मा पृथिवीविकारजो}
{न प्राण आत्माऽनिल एष एव सः} %4-19
{मनोऽप्यहङ्कारविकार एव नो}
{न चापि बुद्धिः प्रकृतेर्विकारजा}

\fourlineindentedshloka
{आत्मा चिदानन्दमयोऽविकारवान्}
{देहादिसङ्घाद्व्यतिरिक्त ईश्वरः} %4-20
{निरञ्जनो मुक्त उपाधितः सदा}
{ज्ञात्वैवमात्मानमितो विमुच्यते}

\fourlineindentedshloka
{अतोऽहमात्यन्तिकमोक्षसाधनम्}
{वक्ष्ये शृणुष्वावहितो महामते} %4-21
{विष्णोर्हि भक्तिः सुविशोधनं धियः}
{ततो भवेज्ज्ञानमतीव निर्मलम्}

\fourlineindentedshloka
{विशुद्धतत्त्वानुभवो भवेत्ततः}
{सम्यग्विदित्वा परमं पदं व्रजेत्} %4-22
{अतो भजस्वाद्य हरिं रमापतिम्}
{रामं पुराणं प्रकृतेः परं विभुम्}

\fourlineindentedshloka
{विसृज्य मौर्ख्यं हृदि शत्रुभावनाम्}
{भजस्व रामं शरणागतप्रियम्}
{सीतां पुरस्कृत्य सपुत्रबान्धवो}
{रामं नमस्कृत्य विमुच्यसे भयात्} %4-23

\fourlineindentedshloka
{रामं परात्मानमभावयन् जनो}
{भक्त्या हृदिस्थं सुखरूपमद्वयम्}
{कथं परं तीरमवाप्नुयाज्जनो}
{भवाम्बुधेर्दुःखतरङ्गमालिनः} %4-24

\fourlineindentedshloka
{नो चेत्त्वमज्ञानमयेन वह्निना}
{ज्वलन्तमात्मानमरक्षितारिवत्}
{नयस्यधोऽधः स्वकृतैश्च पातकैः}
{विमोक्षशङ्का न च ते भविष्यति} %4-25

{॥इति श्रीमदध्यात्मरामायणे उमामहेश्वरसंवादे सुन्दरकाण्डे
चतुर्थे सर्गे हनूमत्सम्भाषणं सम्पूर्णम्॥}
