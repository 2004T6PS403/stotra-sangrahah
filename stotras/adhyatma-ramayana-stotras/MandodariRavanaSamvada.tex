% !TeX program = XeLaTeX
% !TeX root = ../../shloka.tex

\sect{मन्दोदरी-रावण-संवादः}


\addtocounter{shlokacount}{35}

\uvacha{श्री  महादेव  उवाच}

\threelineshloka
{रावणस्तु ततो भार्यामुवाच परिसान्त्वयन्}
{दैवाधीनमिदं भद्रे जीवता किं न दृश्यते}
{त्यज शोकं विशालाक्षि ज्ञानमालम्ब्य निश्चितम्} %10-36

\twolineshloka
{अज्ञानप्रभवः शोकः शोको ज्ञानविनाशकृत्}
{अज्ञानप्रभवाहन्धीः शरीरादिष्वनात्मसु} %10-37

\twolineshloka
{तन्मूलः पुत्रदारादिसम्बन्धः संसृतिस्ततः}
{हर्षशोकभयक्रोधलोभमोहस्पृहादयः} %10-38

\twolineshloka
{अज्ञानप्रभवा ह्येते जन्ममृत्युजरादयः}
{आत्मा तु केवलं शुद्धो व्यतिरिक्तो ह्यलेपकः} %10-39

\twolineshloka
{आनन्दरूपो ज्ञानात्मा सर्वभावविवर्जितः}
{न संयोगो वियोगो वा विद्यते केनचित्सतः} %10-40

\twolineshloka
{एवं ज्ञात्वा स्वमात्मानं त्यज शोकमनिन्दिते}
{इदानीमेव गच्छामि हत्वा रामं सलक्ष्मणम्} %10-41

\twolineshloka
{आगमिष्यामि नो चेन्मां दारयिष्यति सायकैः}
{श्रीरामो वज्रकल्पैश्च ततो गच्छामि तत्पदम्} %10-42

\twolineshloka
{तदा त्वया मे कर्तव्या क्रिया मच्छासनात्प्रिये}
{सीतां हत्वा मया सार्धं त्वं प्रवेक्ष्यसि पावकम्} %10-43

\twolineshloka
{एवं श्रुत्वा वचस्तस्य रावणस्यातिदुःखिता}
{उवाच नाथ मे वाक्यं शृणु सत्यं तथा कुरु} %10-44

\twolineshloka
{शक्यो न राघवो जेतुं त्वया चान्यैः कदाचन}
{रामो देववरः साक्षात्प्रधानपुरुषेश्वरः} %10-45

\twolineshloka
{मत्स्यो भूत्वा पुरा कल्पे मनुं वैवस्वतं प्रभुः}
{ररक्ष सकलापद्भ्यो राघवो भक्तवत्सलः} %10-46

\twolineshloka
{रामः कूर्मोऽभवत्पूर्वं लक्षयोजनविस्तृतः}
{समुद्रमथने पृष्ठे दधार कनकाचलम्} %10-47

\twolineshloka
{हिरण्याक्षोऽतिदुर्वृत्तो हतोऽनेन महात्मना}
{क्रोडरूपेण वपुषा क्षोणीमुद्धरता क्वचित्} %10-48

\twolineshloka
{त्रिलोककण्टकं दैत्यं हिरण्यकशिपुं पुरा}
{हतवान्नारसिंहेन वपुषा रघुनन्दनः} %10-49

\twolineshloka
{विक्रमैस्त्रिभिरेवासौ बलिं बद्ध्वा जगत्त्रयम्}
{आक्रम्यादात्सुरेन्द्राय भृत्याय रघुसत्तमः} %10-50

\twolineshloka
{राक्षसाः क्षत्रियाकारा जाता भूमेर्भरावहाः}
{तान् हत्वा बहुशो रामो भुवं जित्वा ह्यदान्मुनेः} %10-51

\twolineshloka
{स एव साम्प्रतं जातो रघुवंशे परात्परः}
{भवदर्थे रघुश्रेष्ठो मानुषत्वमुपागतः} %10-52

\twolineshloka
{तस्य भार्या किमर्थं वा हृता सीता वनाद्बलात्}
{मम पुत्रविनाशार्थं स्वस्यापि निधनाय च} %10-53

\twolineshloka
{इतः परं वा वैदेहीं प्रेषयस्व रघूत्तमे}
{विभीषणाय राज्यं तु दत्त्वा गच्छामहे वनम्} %10-54

\twolineshloka
{मन्दोदरीवचः श्रुत्वा रावणो वाक्यमब्रवीत्}
{कथं भद्रे रणे पुत्रान् भ्रातॄन् राक्षसमण्डलम्} %10-55

\twolineshloka
{घातयित्वा राघवेण जीवामि वनगोचरः}
{रामेण सह योत्स्यामि रामबाणैः सुशीघ्रगैः} %10-56

\threelineshloka
{विदार्यमाणो यास्यामि तद्विष्णोः परमं पदम्}
{जानामि राघवं विष्णुं लक्ष्मीं जानामि जानकीम्}
{ज्ञात्वैव जानकी सीता मयाऽऽनीता वनाद्बलात्} %10-57

\twolineshloka
{रामेण निधनं प्राप्य यास्यामीति परं पदम्}
{विमुच्य त्वां तु संसाराद्गमिष्यामि सह प्रिये} %10-58

\twolineshloka
{परानन्दमयी शुद्धा सेव्यते या मुमुक्षुभिः}
{तां गतिं तु गमिष्यामि हतो रामेण संयुगे} %10-59

\onelineshloka
{प्रक्षाल्य कल्मषाणीह मुक्तिं यास्यामि दुर्लभाम्} %10-60

\fourlineindentedshloka
{क्लेशादिपञ्चकतरङ्गयुतं भ्रमाढ्यम्}
{दारात्मजाप्तधनबन्धुझषाभियुक्तम्}
{और्वानलाभनिजरोषमनङ्गजालम्}
{संसारसागरमतीत्य हरिं व्रजामि} %10-61

{॥इति श्रीमदध्यात्मरामायणे उमामहेश्वरसंवादे युद्धकाण्डे दशमे  सर्गे  मन्दोदरी-रावण-संवादः  सम्पूर्णः}
