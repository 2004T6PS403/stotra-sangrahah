% !TeX program = XeLaTeX
% !TeX root = ../../shloka.tex

\sect{अगस्त्यद्वारा  रामस्वरूपकथनम्}

\uvacha{अगस्त्य उवाच}
\addtocounter{shlokacount}{62}

\twolineshloka
{भवान्नारायणः साक्षाज्जगतामादिकृद्विभुः}
{त्वत्स्वरूपमिदं सर्वं जगत्स्थावरजङ्गमम्} %2-63

\twolineshloka
{त्वन्नाभिकमलोत्पन्नो ब्रह्मा लोकपितामहः}
{अग्निस्ते मुखतो जातो वाचा सह रघूत्तम} %2-64

\twolineshloka
{बाहुभ्यां लोकपालौघाश्चक्षुर्भ्यां चन्द्रभास्करौ}
{दिशश्च विदिशश्चैव कर्णाभ्यां ते समुत्थिताः} %2-65

\twolineshloka
{घ्राणात्प्राणः समुत्पन्नश्चाश्विनौ देवसत्तमौ}
{जङ्घाजानूरुजघनाद्भुवर्लोकादयोऽभवन्} %2-66

\twolineshloka
{कुक्षिदेशात्समुत्पन्नाश्चत्वारः सागरा हरे}
{स्तनाभ्यामिन्द्रवरुणौ वालखिल्याश्च रेतसः} %2-67

\twolineshloka
{मेढ्राद्यमो गुदान्मृत्युर्मन्यो रुद्रस्त्रिलोचनः}
{अस्थिभ्यः पर्वता जाताः केशेभ्यो मेघसंहतिः} %2-68

\twolineshloka
{ओषध्यस्तव रोमेभ्यो नखेभ्यश्च खरादयः}
{त्वं विश्वरूपः पुरुषो मायाशक्तिसमन्वितः} %2-69

\twolineshloka
{नानारूप इवाऽऽभासि गुणव्यतिकरे सति}
{त्वामाश्रित्यैव विबुधाः पिबन्त्यमृतमध्वरे} %2-70

\twolineshloka
{त्वया सृष्टमिदं सर्वं विश्वं स्थावरजङ्गमम्}
{त्वामाश्रित्यैव जीवन्ति सर्वे स्थावरजङ्गमाः} %2-71

\twolineshloka
{त्वद्युक्तमखिलं वस्तु व्यवहारेऽपि राघव}
{क्षीरमध्यगतं सर्पिर्यथा व्याप्याखिलं पयः} %2-72

\twolineshloka
{त्वद्भासा भासतेऽर्कादि न त्वं तेनावभाससे}
{सर्वगं नित्यमेकं त्वां ज्ञानचक्षुर्विलोकयेत्} %2-73

\twolineshloka
{नाज्ञानचक्षुस्त्वां पश्येदन्धदृग् भास्करं यथा}
{योगिनस्त्वां विचिन्वन्ति स्वदेहे परमेश्वरम्} %2-74

\twolineshloka
{अतन्निरसनमुखैर्वेदशीर्षैरहर्निशम्}
{त्वत्पादभक्तिलेशेन गृहीता यदि योगिनः} %2-75

\threelineshloka
{विचिन्वन्तो हि पश्यन्ति चिन्मात्रं त्वां न चान्यथा}
{मया प्रलपितं किञ्चित्सर्वज्ञस्य तवाग्रतः}
{क्षन्तुमर्हसि देवेश तवानुग्रहभागहम्} %2-76

\fourlineindentedshloka
{दिग्देशकालपरिहीनमनन्यमेकम्}
{चिन्मात्रमक्षरमजं चलनादिहीनम्}
{सर्वज्ञमीश्वरमनन्तगुणं व्युदस्त-}
{मायं भजे रघुपतिं भजतामभिन्नम्} %2-77


{॥इति श्रीमदध्यात्मरामायणे उमामहेश्वरसंवादे उत्तरकाण्डे द्वितीये  सर्गे 
अगस्त्यद्वारा  रामस्वरूपकथनं सम्पूर्णम्॥}
