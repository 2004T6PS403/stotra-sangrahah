% !TeX program = XeLaTeX
% !TeX root = ../../shloka.tex

\sect{परशुरामकृत-रामस्तोत्रम्}

\uvacha{परशुराम उवाच}
\addtocounter{shlokacount}{28}
\twolineshloka
{स एव विष्णुस्त्वं राम जातोऽसि ब्रह्मणार्थितः}
{मयि स्थितं तु त्वत्तेजस्त्वयैव पुनराहृतम्} %7-29

\twolineshloka
{अद्य मे सफलं जन्म प्रतीतोऽसि मम प्रभो}
{ब्रह्मादिभिरलभ्यस्त्वं प्रकृतेः पारगो मतः} %7-30

\twolineshloka
{त्वयि जन्मादिषड्भावा न सन्त्यज्ञानसम्भवाः}
{निर्विकारोऽसि पूर्णस्त्वं गमनादिविवर्जितः} %7-31

\twolineshloka
{यथा जले फेनजालं धूमो वह्नौ तथा त्वयि}
{त्वदाधारा त्वद्विषया माया कार्यं सृजत्यहो} %7-32

\twolineshloka
{यावन्मायावृता लोकास्तावत्त्वां न विजानते}
{अविचारितसिद्धैषाऽविद्या विद्याविरोधिनी} %7-33

\twolineshloka
{अविद्याकृतदेहादिसङ्घाते प्रतिबिम्बिता}
{चिच्छक्तिर्जीवलोकेऽस्मिन् जीव इत्यभिधीयते} %7-34

\twolineshloka
{यावद्देहमनःप्राणबुद्ध्यादिष्वभिमानवान्}
{तावत्कर्तृत्वभोक्तृत्वसुखदुःखादिभाग्भवेत्} %7-35

\twolineshloka
{आत्मनःसंसृतिर्नास्ति बुद्धेर्ज्ञानं न जात्विति}
{अविवेकाद्द्वयं युङ्क्त्वा संसारीति प्रवर्तते} %7-36

\twolineshloka
{जडस्य चित्समायोगाच्चित्त्वं भूयाच्चितेस्तथा}
{जडसङ्गाज्जडत्वं हि जलाग्न्योर्मेलनं यथा} %7-37

\twolineshloka
{यावत्त्वत्पादभक्तानां सङ्गसौख्यं न विन्दति}
{तावत्संसारदुःखौघान्न निवर्तेन्नरः सदा} %7-38

\twolineshloka
{तत्सङ्गलब्धया भक्त्या यदा त्वां समुपासते}
{तदा माया शनैर्याति तानवं प्रतिपद्यते} %7-39

\twolineshloka
{ततस्त्वज्ज्ञानसम्पन्नः सद्गुरुस्तेन लभ्यते}
{वाक्यज्ञानं गुरोर्लब्ध्वा त्वत्प्रसादाद्विमुच्यते} %7-40

\twolineshloka
{तस्मात्त्वद्भक्तिहीनानां कल्पकोटिशतैरपि}
{न मुक्तिशङ्का विज्ञानशङ्का नैव सुखं तथा} %7-41

\twolineshloka
{अतस्त्वत्पादयुगले भक्तिर्मे जन्मजन्मनि}
{स्यात्त्वद्भक्तिमतां सङ्गोऽविद्या याभ्यां विनश्यति} %7-42

\twolineshloka
{लोके त्वद्भक्तिनिरतास्त्वद्धर्मामृतवर्षिणः}
{पुनन्ति लोकमखिलं किं पुनः स्वकुलोद्भवान्} %7-43

\twolineshloka
{नमोऽस्तु जगतां नाथ नमस्ते भक्तिभावन}
{नमः कारुणिकानन्त रामचन्द्र नमोऽस्तु ते} %7-44

\twolineshloka
{देव यद्यत्कृतं पुण्यं मया लोकजिगीषया}
{तत्सर्वं तव बाणाय भूयाद्राम नमोऽस्तु ते} %7-45

\twolineshloka
{ततः प्रसन्नो भगवान् श्रीरामः करुणामयः}
{प्रसन्नोऽस्मि तव ब्रह्मन् यत्ते मनसि वर्तते} %7-46

\twolineshloka
{दास्ये तदखिलं कामं मा कुरुष्वात्र संशयम्}
{ततः प्रीतेन मनसा भार्गवो राममब्रवीत्} %7-47

\twolineshloka
{यदि मेऽनुग्रहो राम तवास्ति मधुसूदन}
{त्वद्भक्तसङ्गस्त्वत्पादे दृढा भक्तिः सदास्तु मे} %7-48

\twolineshloka
{स्तोत्रमेतत्पठेद्यस्तु भक्तिहीनोऽपि सर्वदा}
{त्वद्भक्तिस्तस्य विज्ञानं भूयादन्ते स्मृतिस्तव} %7-49

\twolineshloka
{तथेति राघवेणोक्तः परिक्रम्य प्रणम्य तम्}
{पूजितस्तदनुज्ञातो महेन्द्राचलमन्वगात्} %7-50

{॥इति श्रीमदध्यात्मरामायणे उमामहेश्वरसंवादे बालकाण्डे
सप्तमे  सर्गे श्री परशुरामकृतं श्री~रामस्तोत्रं सम्पूर्णम्॥}
