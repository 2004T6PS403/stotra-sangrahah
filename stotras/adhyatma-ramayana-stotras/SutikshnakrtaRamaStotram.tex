% !TeX program = XeLaTeX
% !TeX root = ../../shloka.tex

\sect{सुतीक्ष्णकृतं रामस्तोत्रम्}


\addtocounter{shlokacount}{26}
\uvacha{सुतीक्ष्ण उवाच}

\fourlineindentedshloka
{त्वन्मन्त्रजाप्यहमनन्तगुणाप्रमेय}
{सीतापते शिवविरिञ्चिसमाश्रिताङ्घ्रे}
{संसारसिन्धुतरणामलपोतपाद}
{रामाभिराम सततं तव दासदासः} %2-27

\fourlineindentedshloka
{मामद्य सर्वजगतामविगोचरस्त्वम्}
{त्वन्मायया सुतकलत्रगृहान्धकूपे}
{मग्नं निरीक्ष्य मलपुद्गलपिण्डमोह-}
{पाशानुबद्धहृदयं स्वयमागतोऽसि} %2-28

\fourlineindentedshloka
{त्वं सर्वभूतहृदयेषु कृतालयोऽपि}
{त्वन्मन्त्रजाप्यविमुखेषु तनोषि मायाम्}
{त्वन्मन्त्रसाधनपरेष्वपयाति माया}
{सेवानुरूपफलदोऽसि यथा महीपः} %2-29

\fourlineindentedshloka
{विश्वस्य सृष्टिलयसंस्थितिहेतुरेकः}
{त्वं मायया त्रिगुणया विधिरीशविष्णू}
{भासीश मोहितधियां विविधाकृतिस्त्वम्}
{यद्वद्रविः सलिलपात्रगतो ह्यनेकः} %2-30

\fourlineindentedshloka
{प्रत्यक्षतोऽद्य भवतश्चरणारविन्दम्}
{पश्यामि राम तमसः परतः स्थितस्य}
{दृग्रूपतस्त्वमसतामविगोचरोऽपि}
{त्वन्मन्त्रपूतहृदयेषु सदा प्रसन्नः} %2-31

\fourlineindentedshloka
{पश्यामि राम तव रूपमरूपिणोऽपि}
{मायाविडम्बनकृतं सुमनुष्यवेषम्}
{कन्दर्पकोटिसुभगं कमनीयचापबाणम्}
{दयार्द्रहृदयं स्मितचारुवक्त्रम्} %2-32

\fourlineindentedshloka
{सीतासमेतमजिनाम्बरमप्रधृष्यम्}
{सौमित्रिणा नियतसेवितपादपद्मम्}
{नीलोत्पलद्युतिमनन्तगुणं प्रशान्तम्}
{मद्भागधेयमनिशं प्रणमामि रामम्} %2-33

\fourlineindentedshloka
{जानन्तु राम तव रूपमशेषदेश-}
{कालाद्युपाधिरहितं घनचित्प्रकाशम्}
{प्रत्यक्षतोऽद्य मम गोचरमेतदेव}
{रूपं विभातु हृदये न परं विकाङ्क्षे} %2-34

{॥इति श्रीमदध्यात्मरामायणे उमामहेश्वरसंवादे अरण्यकाण्डे
प्रथमे सर्गे सुतीक्ष्णकृतं श्री~रामस्तोत्रं सम्पूर्णम्॥}
