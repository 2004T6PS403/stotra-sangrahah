% !TeX program = XeLaTeX
% !TeX root = ../../shloka.tex

\sect{शिवकृत-रामस्तुतिः}

\uvacha{श्री महादेव उवाच}
\addtocounter{shlokacount}{50}

\fourlineindentedshloka
{नमोऽस्तु रामाय सशक्तिकाय}
{नीलोत्पलश्यामलकोमलाय}
{किरीटहाराङ्गदभूषणाय}
{सिंहासनस्थाय महाप्रभाय} %15-51

\fourlineindentedshloka
{त्वमादिमध्यान्तविहीन एकः}
{सृजस्यवस्यत्सि च लोकजातम्}
{स्वमायया तेन न लिप्यसे त्वम्}
{यत्स्वे सुखेऽजस्ररतोऽनवद्यः} %15-52

\fourlineindentedshloka
{लीलां विधत्से गुणसंवृतस्त्वम्}
{प्रपन्नभक्तानुविधानहेतोः}
{नानावतारैः सुरमानुषाद्यैः}
{प्रतीयसे ज्ञानिभिरेव नित्यम्} %15-53

\fourlineindentedshloka
{स्वांशेन लोकं सकलं विधाय तम्}
{बिभर्षि च त्वं तदधः फणीश्वरः}
{उपर्यधो भान्वनिलोडुपौषधि-}
{प्रवर्षरूपोऽवसि नैकधा जगत्} %15-54

\fourlineindentedshloka
{त्वमिह देहभृतां शिखिरूपः}
{पचसि भुक्तमशेषमजस्रम्}
{पवनपञ्चकरूपसहायो}
{जगदखण्डमनेन बिभर्षि} %15-55

\fourlineindentedshloka
{चन्द्रसूर्यशिखिमध्यगतं यत्}
{तेज ईश चिदशेषतनूनाम्}
{प्राभवत्तनुभृतामिव धैर्यम्}
{शौर्यमायुरखिलं तव सत्त्वम्} %15-56

\fourlineindentedshloka
{त्वं विरिञ्चिशिवविष्णुविभेदात्}
{कालकर्मशशिसूर्यविभागात्}
{वादिनां पृथगिवेश विभासि}
{ब्रह्म निश्चितमनन्यदिहैकम्} %15-57

\fourlineindentedshloka
{मत्स्यादिरूपेण यथा त्वमेकः}
{श्रुतौ पुराणेषु च लोकसिद्धः}
{तथैव सर्वं सदसद्विभाग-}
{स्त्वमेव नान्यद्भवतो विभाति} %15-58

\fourlineindentedshloka
{यद्यत्समुत्पन्नमनन्तसृष्टा-}
{वुत्पत्स्यते यच्च भवच्च यच्च}
{न दृश्यते स्थावरजङ्गमादौ}
{त्वया विनातःपरतः परस्त्वम्} %15-59

\fourlineindentedshloka
{तत्त्वं न जानन्ति परात्मनस्ते}
{जनाः समस्तास्तव माययातः}
{त्वद्भक्तसेवाऽमलमानसानाम्}
{विभाति तत्त्वं परमेकमैशम्} %15-60

\fourlineindentedshloka
{ब्रह्मादयस्ते न विदुः स्वरूपम्}
{चिदात्मतत्त्वं बहिरर्थभावाः}
{ततो बुधस्त्वामिदमेव रूपम्}
{भक्त्या भजन्मुक्तिमुपैत्यदुःखः} %15-61

\fourlineindentedshloka
{अहं भवन्नाम गृणन् कृतार्थो}
{वसामि काश्यामनिशं भवान्या}
{मुमूर्षमाणस्य विमुक्तयेऽहम्}
{दिशामि मन्त्रं तव राम नाम} %15-62

\fourlineindentedshloka
{इमं स्तवं नित्यमनन्यभक्त्या}
{शृण्वन्ति गायन्ति लिखन्ति ये वै}
{ते सर्वसौख्यं परमं च लब्ध्वा}
{भवत्पदं यान्तु भवत्प्रसादात्} %15-63

{॥इति श्रीमदध्यात्मरामायणे उमामहेश्वरसंवादे युद्धकाण्डे पञ्चदशे  सर्गे 
शिवकृत-रामस्तुतिः  सम्पूर्णः॥}
