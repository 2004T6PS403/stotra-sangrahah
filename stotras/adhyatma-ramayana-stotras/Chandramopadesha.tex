% !TeX program = XeLaTeX
% !TeX root = ../../shloka.tex

\sect{चन्द्रमोपदेशः}


\addtocounter{shlokacount}{11}

\twolineshloka
{शृणु वत्स वचो मेऽद्य श्रुत्वा कुरु यथेप्सितम्}
{देहमूलमिदं दुःखं देहः कर्मसमुद्भवः} %8-12

\twolineshloka
{कर्म प्रवर्तते देहेऽहम्बुद्ध्या पुरुषस्य हि}
{अहङ्कारस्त्वनादिः स्यादविद्यासम्भवो जडः} %8-13

\twolineshloka
{चिच्छायया सदा युक्तस्तप्तायःपिण्डवत् सदा}
{तेन देहस्य तादात्म्याद्देहश्चेतनवान् भवेत्} %8-14

\twolineshloka
{देहोऽहमिति बुद्धिः स्यादात्मनोऽहङ्कृतेर्बलात्}
{तन्मूल एष संसारः सुखदुःखादिसाधकः} %8-15

\twolineshloka
{आत्मनो निर्विकारस्य मिथ्या तादात्म्यतः सदा}
{देहोऽहं कर्मकर्ताऽहमिति सङ्कल्प्य सर्वदा} %8-16

\twolineshloka
{जीवः करोति कर्माणि तत्फलैर्बद्ध्यतेऽवशः}
{ऊर्ध्वाधो भ्रमते नित्यं पापपुण्यात्मकः स्वयम्} %8-17

\twolineshloka
{कृतं मयाऽधिकं पुण्यं यज्ञदानादि निश्चितम्}
{स्वर्गं गत्वा सुखं भोक्ष्य इति सङ्कल्पवान् भवेत्} %8-18

\twolineshloka
{तथैवाध्यासतस्तत्र चिरं भुक्त्वा सुखं महत्}
{क्षीणपुण्यः पतत्यर्वागनिच्छन् कर्मचोदितः} %8-19

\twolineshloka
{पतित्वा मण्डले चेन्दोस्ततो नीहारसंयुतः}
{भूमौ पतित्वा व्रीह्यादौ तत्र स्थित्वा चिरं पुनः} %8-20

\twolineshloka
{भूत्वा चतुर्विधं भोज्यं पुरुषैर्भुज्यते ततः}
{रेतो भूत्वा पुनस्तेन ऋतौ स्त्रीयोनिसिञ्चितः} %8-21

\twolineshloka
{योनिरक्तेन संयुक्तं जरायुपरिवेष्टितम्}
{दिनेनैकेन कललं भूत्वा रूढत्वमाप्नुयात्} %8-22

\twolineshloka
{तत्पुनः पञ्चरात्रेण बुद्बुदाकारतामियात्}
{सप्तरात्रेण तदपि मांसपेशित्वमाप्नुयात्} %8-23

\twolineshloka
{पक्षमात्रेण सा पेशी रुधिरेण परिप्लुता}
{तस्या एवाङ्कुरोत्पत्तिः पञ्चविंशतिरात्रिषु} %8-24

\twolineshloka
{ग्रीवा शिरश्च स्कन्धश्च पृष्ठवंशस्तथोदरम्}
{पञ्चधाङ्गानि चैकैकं जायन्ते मासतः क्रमात्} %8-25

\twolineshloka
{पाणिपादौ तथा पार्श्वः कटिर्जानु तथैव च}
{मासद्वयात् प्रजायन्ते क्रमेणैव न चान्यथा} %8-26

\twolineshloka
{त्रिभिर्मासैः प्रजायन्ते अङ्गानां सन्धयः क्रमात्}
{सर्वाङ्गुल्यः प्रजायन्ते क्रमान्मासचतुष्टये} %8-27

\twolineshloka
{नासा कर्णौ च नेत्रे च जायन्ते पञ्चमासतः}
{दन्तपङ्क्तिर्नखा गुह्यं पञ्चमे जायते तथा} %8-28

\twolineshloka
{अर्वाक् षण्मासतश्छिद्रं कर्णयोर्भवति स्फुटम्}
{पायुर्मेढ्रमुपस्थं च नाभिश्चापि भवेन्नृणाम्} %8-29

\twolineshloka
{सप्तमे मासि रोमाणि शिरः केशास्तथैव च}
{विभक्तावयवत्वं च सर्वं सम्पद्यतेऽष्टमे} %8-30

\twolineshloka
{जठरे वर्धते गर्भः स्त्रिया एवं विहङ्गम}
{पञ्चमे मासि चैतन्यं जीवः प्राप्नोति सर्वशः} %8-31

\twolineshloka
{नाभिसूत्राल्परन्ध्रेण मातृभुक्तान्नसारतः}
{वर्धते गर्भतः पिण्डो न म्रियेत स्वकर्मतः} %8-32

\twolineshloka
{स्मृत्वा सर्वाणि जन्मानि पूर्वकर्माणि सर्वशः}
{जठरानलतप्तोऽयमिदं वचनमब्रवीत्} %8-33

\twolineshloka
{नानायोनिसहस्रेषु जायमानोऽनुभूतवान्}
{पुत्रदारादिसम्बन्धं कोटिशः पशुबान्धवान्} %8-34

\twolineshloka
{कुटुम्बभरणासक्त्या न्यायान्यायैर्धनार्जनम्}
{कृतं नाकरवं विष्णुचिन्तां स्वप्नेऽपि दुर्भगः} %8-35

\twolineshloka
{इदानीं तत्फलं भुञ्जे गर्भदुःखं महत्तरम्}
{अशाश्वते शाश्वतवद्देहे तृष्णासमन्वितः} %8-36

\twolineshloka
{अकार्याण्येव कृतवान्न कृतं हितमात्मनः}
{इत्येवं बहुधा दुःखमनुभूय स्वकर्मतः} %8-37

\twolineshloka
{कदा निष्क्रमणं मे स्याद्गर्भान्निरयसन्निभात्}
{इत ऊर्ध्वं नित्यमहं विष्णुमेवानुपूजये} %8-38

\twolineshloka
{इत्यादि चिन्तयन् जीवो योनियन्त्रप्रपीडितः}
{जायमानोऽतिदुःखेन नरकात्पातकी यथा} %8-39

\twolineshloka
{पूतिव्रणान्निपतितः कृमिरेष इवापरः}
{ततो बाल्यादिदुःखानि सर्व एवं विभुञ्जते} %8-40

\twolineshloka
{त्वया चैवानुभूतानि सर्वत्र विदितानि च}
{न वर्णितानि मे गृध्र यौवनादिषु सर्वतः} %8-41

\twolineshloka
{एवं देहोऽहमित्यस्मादभ्यासान्निरयादिकम्}
{गर्भवासादिदुःखानि भवन्त्यभिनिवेशतः} %8-42

\twolineshloka
{तस्माद्देहद्वयादन्यमात्मानं प्रकृतेः परम्}
{ज्ञात्वा देहादिममतां त्यक्त्वाऽऽत्मज्ञानवान् भवेत्} %8-43

\twolineshloka
{जाग्रदादिविनिर्मुक्तं सत्यज्ञानादिलक्षणम्}
{शुद्धं बुद्धं सदा शान्तमात्मानमवधारयेत्} %8-44

\twolineshloka
{चिदात्मनि परिज्ञाते नष्टे मोहेऽज्ञसम्भवे}
{देहः पततु वाऽरब्धकर्मवेगेन तिष्ठतु} %8-45

\twolineshloka
{योगिनो न हि दुःखं वा सुखं वाऽज्ञानसम्भवम्}
{तस्माद्देहेन सहितो यावत्प्रारब्धसङ्क्षयः} %8-46

{तावत्तिष्ठ सुखेन त्वं धृतकञ्चुकसर्पवत्।}

{॥इति श्रीमदध्यात्मरामायणे उमामहेश्वरसंवादे किष्किन्धाकाण्डे
अष्टमे सर्गे चन्द्रमोपदेशः सम्पूर्णः॥}
