% !TeX program = XeLaTeX
% !TeX root = ../../shloka.tex

\sect{रामोपदेशः}

\addtocounter{shlokacount}{52}
\uvacha{श्री महादेव उवाच}

\twolineshloka
{एकान्ते ध्याननिरते एकदा राघवे सति}
{ज्ञात्वा नारायणं साक्षात्कौसल्या प्रियवादिनी} %7-53

\twolineshloka
{भक्त्याऽऽगत्य प्रसन्नं तं प्रणता प्राह हृष्टधीः}
{राम त्वं जगतामादिरादिमध्यान्तवर्जितः} %7-54

\twolineshloka
{परमात्मा परानन्दः पूर्णः पुरुष ईश्वरः}
{जातोऽसि मे गर्भगृहे मम पुण्यातिरेकतः} %7-55

\twolineshloka
{अवसाने ममाप्यद्य समयोऽभूद्रघूत्तम}
{नाद्याप्यबोधजः कृत्स्नो भवबन्धो निवर्तते} %7-56

\twolineshloka
{इदानीमपि मे ज्ञानं भवबन्धनिवर्तकम्}
{यथा सङ्क्षेपतो भूयात्तथा बोधय मां विभो} %7-57

\twolineshloka
{निर्वेदवादिनीमेवं मातरं मातृवत्सलः}
{दयालुः प्राह धर्मात्मा जराजर्जरितां शुभाम्} %7-58

\twolineshloka
{मार्गास्त्रयो मया प्रोक्ताः पुरा मोक्षाप्तिसाधकाः}
{कर्मयोगो ज्ञानयोगो भक्तियोगश्च शाश्वतः} %7-59

\twolineshloka
{भक्तिर्विभिद्यते मातस्त्रिविधा गुणभेदतः}
{स्वभावो यस्य यस्तेन तस्य भक्तिर्विभिद्यते} %7-60

\twolineshloka
{यस्तु हिंसां समुद्दिश्य दम्भं मात्सर्यमेव वा}
{भेददृष्टिश्च संरम्भी भक्तो मे तामसः स्मृतः} %7-61

\twolineshloka
{फलाभिसन्धिर्भोगार्थी धनकामो यशस्तथा}
{अर्चादौ भेदबुद्ध्या मां पूजयेत्स तु राजसः} %7-62

\twolineshloka
{परस्मिन्नर्पितं यस्तु कर्मनिर्हरणाय वा}
{कर्तव्यमिति वा कुर्याद्भेदबुद्ध्या स सात्त्विकः} %7-63

\twolineshloka
{मद्गुणाश्रयणादेव मय्यनन्तगुणालये}
{अविच्छिन्ना मनोवृत्तिर्यथा गङ्गाम्बुनोऽम्बुधौ} %7-64

\twolineshloka
{तदेव भक्तियोगस्य लक्षणं निर्गुणस्य हि}
{अहैतुक्यव्यवहिता या भक्तिर्मयि जायते} %7-65

\twolineshloka
{सा मे सालोक्यसामीप्यसार्ष्टिसायुज्यमेव वा}
{ददात्यपि न गृह्णन्ति भक्ता मत्सेवनं विना} %7-66

\twolineshloka
{स एवात्यन्तिको योगो भक्तिमार्गस्य भामिनि}
{मद्भावं प्राप्नुयात्तेन अतिक्रम्य गुणत्रयम्} %7-67

\twolineshloka
{महता कामहीनेन स्वधर्माचरणेन च}
{कर्मयोगेन शस्तेन वर्जितेन विहिंसनात्} %7-68

\twolineshloka
{मद्दर्शनस्तुतिमहापूजाभिः स्मृतिवन्दनैः}
{भूतेषु मद्भावनया सङ्गेनासत्यवर्जनैः} %7-69

\twolineshloka
{बहुमानेन महतां दुःखिनामनुकम्पया}
{स्वसमानेषु मैत्र्या च यमादीनां निषेवया} %7-70

\twolineshloka
{वेदान्तवाक्यश्रवणान्मम नामानुकीर्तनात्}
{सत्सङ्गेनार्जवेनैव ह्यहमः परिवर्जनात्} %7-71

\twolineshloka
{काङ्क्षया मम धर्मस्य परिशुद्धान्तरो जनः}
{मद्गुणश्रवणादेव याति मामञ्जसा जनः} %7-72

\twolineshloka
{यथा वायुवशाद्गन्धः स्वाश्रयाद्\mbox{}घ्राणमाविशेत्}
{योगाभ्यासरतं चित्तमेवमात्मानमाविशेत्} %7-73

\twolineshloka
{सर्वेषु प्राणिजातेषु ह्यहमात्मा व्यवस्थितः}
{तमज्ञात्वा विमूढात्मा कुरुते केवलं बहिः} %7-74

\twolineshloka
{क्रियोत्पन्नैर्नैकभेदैर्द्रव्यैर्मे नाम्ब तोषणम्}
{भूतावमानिनार्चायामर्चितोऽहं न पूजितः} %7-75

\twolineshloka
{तावन्मामर्चयेद्देवं प्रतिमादौ स्वकर्मभिः}
{यावत्सर्वेषु भूतेषु स्थितं चात्मनि न स्मरेत्} %7-76

\twolineshloka
{यस्तु भेदं प्रकुरुते स्वात्मनश्च परस्य च}
{भिन्नदृष्टेर्भयं मृत्युस्तस्य कुर्यान्न संशयः} %7-77

\twolineshloka
{मामतः सर्वभूतेषु परिच्छिन्नेषु संस्थितम्}
{एकं ज्ञानेन मानेन मैत्र्या चार्चेदभिन्नधीः} %7-78

\twolineshloka
{चेतसैवानिशं सर्वभूतानि प्रणमेत्सुधीः}
{ज्ञात्वा मां चेतनं शुद्धं जीवरूपेण संस्थितम्} %7-79

\twolineshloka
{तस्मात्कदाचिन्नेक्षेत भेदमीश्वरजीवयोः}
{भक्तियोगो ज्ञानयोगो मया मातरुदीरितः} %7-80

\twolineshloka
{आलम्ब्यैकतरं वाऽपि पुरुषः शुभमृच्छति}
{ततो मां भक्तियोगेन मातः सर्वहृदि स्थितम्} %7-81

\twolineshloka
{पुत्ररूपेण वा नित्यं स्मृत्वा शान्तिमवाप्स्यसि}
{श्रुत्वा रामस्य वचनं कौसल्याऽऽनन्दसंयुता} %7-82

\twolineshloka
{रामं सदा हृदि ध्यात्वा छित्त्वा संसारबन्धनम्}
{अतिक्रम्य गतीस्तिस्रोऽप्यवाप परमां गतिम्} %7-83

\fourlineindentedshloka
{कैकेयी चापि योगं रघुपतिगदितं पूर्वमेवाधिगम्य}
{श्रद्धाभक्तिप्रशान्ता हृदि रघुतिलकं भावयन्ती गतासुः}
{गत्वा स्वर्गं स्फुरन्ती दशरथसहिता मोदमानावतस्थे}
{माता श्रीलक्ष्मणस्याप्यतिविमलमतिः प्राप भर्तुः समीपम्} %7-84

{॥इति श्रीमदध्यात्मरामायणे उमामहेश्वरसंवादे उत्तरकाण्डे सप्तमः
सर्गः सम्पूरणः॥}
