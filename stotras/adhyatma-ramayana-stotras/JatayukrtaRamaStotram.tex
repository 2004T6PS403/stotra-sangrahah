% !TeX program = XeLaTeX
% !TeX root = ../../shloka.tex

\sect{जटायुकृत-रामस्तोत्रम्}


\addtocounter{shlokacount}{43}
\uvacha{जटायुरुवाच}

\fourlineindentedshloka
{अगणितगुणमप्रमेयमाद्यम्}
{सकलजगत्स्थितिसंयमादिहेतुम्}
{उपरमपरमं परात्मभूतम्}
{सततमहं प्रणतोऽस्मि रामचन्द्रम्} %8-44

\fourlineindentedshloka
{निरवधिसुखमिन्दिराकटाक्षम्}
{क्षपितसुरेन्द्रचतुर्मुखादिदुःखम्}
{नरवरमनिशं नतोऽस्मि रामम्}
{वरदमहं वरचापबाणहस्तम्} %8-45

\fourlineindentedshloka
{त्रिभुवनकमनीयरूपमीड्यम्}
{रविशतभासुरमीहितप्रदानम्}
{शरणदमनिशं सुरागमूले}
{कृतनिलयं रघुनन्दनं प्रपद्ये} %8-46

\fourlineindentedshloka
{भवविपिनदवाग्निनामधेयम्}
{भवमुखदैवतदैवतं दयालुम्}
{दनुजपतिसहस्रकोटिनाशम्}
{रवितनयासदृशं हरिं प्रपद्ये} %8-47

\fourlineindentedshloka
{अविरतभवभावनातिदूरम्}
{भवविमुखैर्मुनिभिः सदैव दृश्यम्}
{भवजलधिसुतारणाङ्घ्रिपोतम्}
{शरणमहं रघुनन्दनं प्रपद्ये} %8-48

\fourlineindentedshloka
{गिरिशगिरिसुतामनोनिवासम्}
{गिरिवरधारिणमीहिताभिरामम्}
{सुरवरदनुजेन्द्रसेविताङ्घ्रिम्}
{सुरवरदं रघुनायकं प्रपद्ये} %8-49

\fourlineindentedshloka
{परधनपरदारवर्जितानाम्}
{परगुणभूतिषु तुष्टमानसानाम्}
{परहितनिरतात्मनां सुसेव्यम्}
{रघुवरमम्बुजलोचनं प्रपद्ये} %8-50

\fourlineindentedshloka
{स्मितरुचिरविकासिताननाब्ज-}
{मतिसुलभं सुरराजनीलनीलम्}
{सितजलरुहचारुनेत्रशोभम्}
{रघुपतिमीशगुरोर्गुरुं प्रपद्ये} %8-51

\fourlineindentedshloka
{हरिकमलजशम्भुरूपभेदात्}
{त्वमिह विभासि गुणत्रयानुवृत्तः}
{रविरिव जलपूरितोदपात्रे-}
{ष्वमरपतिस्तुतिपात्रमीशमीडे} %8-52

\fourlineindentedshloka
{रतिपतिशतकोटिसुन्दराङ्गम्}
{शतपथगोचरभावनाविदूरम्}
{यतिपतिहृदये सदा विभातम्}
{रघुपतिमार्तिहरं प्रभुं प्रपद्ये} %8-53

\twolineshloka
{इत्येवं स्तुवतस्तस्य प्रसन्नोऽभूद्रघूत्तमः}
{उवाच गच्छ भद्रं ते मम विष्णोः परं पदम्} %8-54

\twolineshloka
{शृणोति य इदं स्तोत्रं लिखेद्वा नियतः पठेत्}
{स याति मम सारूप्यं मरणे मत्स्मृतिं लभेत्} %8-55

\fourlineindentedshloka
{इति राघवभाषितं तदा}
{श्रुतवान् हर्षसमाकुलो द्विजः}
{रघुनन्दनसाम्यमास्थितः}
{प्रययौ ब्रह्मसुपूजितं पदम्} %8-56

{॥इति श्रीमदध्यात्मरामयणे उमामहेश्वरसंवादे
अरण्यकाण्डे अष्टमे  सर्गे  जटायुकृतं  श्री  रामस्तोत्रं  सम्पूर्णम्॥}
