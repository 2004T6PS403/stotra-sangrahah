% !TeX program = XeLaTeX
% !TeX root = ../../shloka.tex

\sect{वाल्मीक्योपदेशः}

\addtocounter{shlokacount}{39}
\uvacha{वाल्मीकिरुवाच}

\twolineshloka
{शृणु वक्ष्यामि ते सर्वं सङ्क्षेपाद्बन्धमोक्षयोः}
{स्वरूपं साधनं चापि मत्तः श्रुत्वा यथोदितम्} %6-40

\twolineshloka
{तथैवाऽऽचर भद्रं ते जीवन्मुक्तो भविष्यसि}
{देह एव महागेहमदेहस्य चिदात्मनः} %6-41

\twolineshloka
{तस्याहङ्कार एवास्मिन्मन्त्री तेनैव कल्पितः}
{देहगेहाभिमानं स्वं समारोप्य चिदात्मनि} %6-42

\twolineshloka
{तेन तादात्म्यमापन्नः स्वचेष्टितमशेषतः}
{विदधाति चिदानन्दे तद्वासितवपुः स्वयम्} %6-43

\twolineshloka
{तेन सङ्कल्पितो देही सङ्कल्पनिगडावृतः}
{पुत्रदारगृहादीनि सङ्कल्पयति चानिशम्} %6-44

\twolineshloka
{सङ्कल्पयन् स्वयं देही परिशोचति सर्वदा}
{त्रयस्तस्याहमो देहा अधमोत्तममध्यमाः} %6-45

\twolineshloka
{तमः सत्त्वरजः संज्ञा जगतः कारणं स्थितेः}
{तमोरूपाद्धि सङ्कल्पान्नित्यं तामसचेष्टया} %6-46

\twolineshloka
{अत्यन्तं तामसो भूत्वा कृमिकीटत्वमाप्नुयात्}
{सत्त्वरूपो हि सङ्कल्पो धर्मज्ञानपरायणः} %6-47

\twolineshloka
{अदूरमोक्षसाम्राज्यः सुखरूपो हि तिष्ठति}
{रजोरूपो हि सङ्कल्पो लोके स व्यवहारवान्} %6-48

\twolineshloka
{परितिष्ठति संसारे पुत्रदारानुरञ्जितः}
{त्रिविधं तु परित्यज्य रूपमेतन्महामते} %6-49

\twolineshloka
{सङ्कल्पं परमाप्नोति पदमात्मपरिक्षये}
{दृष्टीः सर्वाः परित्यज्य नियम्य मनसा मनः} %6-50

\twolineshloka
{सबाह्याभ्यन्तरार्थस्य सङ्कल्पस्य क्षयं कुरु}
{यदि वर्षसहस्राणि तपश्चरसि दारुणम्} %6-51

\twolineshloka
{पातालस्थस्य भूस्थस्य स्वर्गस्थस्यापि तेऽनघ}
{नान्यः कश्चिदुपायोऽस्ति सङ्कल्पोपशमादृते} %6-52

\twolineshloka
{अनाबाधेऽविकारे स्वे सुखे परमपावने}
{सङ्कल्पोपशमे यत्नं पौरुषेण परं कुरु} %6-53

\twolineshloka
{सङ्कल्पतन्तौ निखिला भावाः प्रोताः किलानघ}
{छिन्ने तन्तौ न जानीमः क्व यान्ति विभवाः पराः} %6-54

\twolineshloka
{निःसङ्कल्पो यथाप्राप्तव्यवहारपरो भव}
{क्षये सङ्कल्पजालस्य जीवो ब्रह्मत्वमाप्नुयात्} %6-55

\twolineshloka
{अधिगतपरमार्थतामुपेत्य प्रसभमपास्य विकल्पजालमुच्चैः}
{अधिगमय पदं तदद्वितीयं विततसुखाय सुषुप्तचित्तवृत्तिः} %6-56

{॥इति श्रीमदध्यात्मरामायणे उमामहेश्वरसंवादे उत्तरकाण्डे षष्ठे
सर्गे वाल्मीक्योपदेशः  सम्पूर्णः॥}
