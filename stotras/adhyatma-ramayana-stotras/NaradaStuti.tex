% !TeX program = XeLaTeX
% !TeX root = ../../shloka.tex

\sect{नारदस्तुतिः}


\addtocounter{shlokacount}{33}

\uvacha{नारद उवाच}

\twolineshloka
{देवदेव जगन्नाथ परमात्मन् सनातन}
{नारायणाखिलाधार विश्वसाक्षिन्नमोऽस्तु ते} %8-34

\twolineshloka
{विशुद्धज्ञानरूपोऽपि त्वं लोकानतिवञ्चयन्}
{मायया मनुजाकारः सुखदुःखादिमानिव} %8-35

\twolineshloka
{त्वं मायया गुह्यमानः सर्वेषां हृदि संस्थितः}
{स्वयञ्ज्योतिः स्वभावस्त्वं व्यक्त एवामलात्मनाम्} %8-36

\twolineshloka
{उन्मीलयन् सृजस्येतन्नेत्रे राम जगत्त्रयम्}
{उपसंह्रियते सर्वं त्वया चक्षुर्निमीलनात्} %8-37

\twolineshloka
{यस्मिन् सर्वमिदं भाति यतश्चैतच्चराचरम्}
{यस्मान्न किञ्चिल्लोकेऽस्मिंस्तस्मै ते ब्रह्मणे नमः} %8-38

\twolineshloka
{प्रकृतिं पुरुषं कालं व्यक्ताव्यक्तस्वरूपिणम्}
{यं जानन्ति मुनिश्रेष्ठास्तस्मै रामाय ते नमः} %8-39

\twolineshloka
{विकाररहितं शुद्धं ज्ञानरूपं श्रुतिर्जगौ}
{त्वां सर्वजगदाकारमूर्तिं चाप्याह सा श्रुतिः} %8-40

\twolineshloka
{विरोधो दृश्यते देव वैदिको वेदवादिनाम्}
{निश्चयं नाधिगच्छन्ति त्वत्प्रसादं विना बुधाः} %8-41

\twolineshloka
{मायया क्रीडतो देव न विरोधो मनागपि}
{रश्मिजालं रवेर्यद्वद्दृश्यते जलवद्\mbox{}भ्रमात्} %8-42

\twolineshloka
{भ्रान्तिज्ञानात्तथा राम त्वयि सर्वं प्रकल्प्यते}
{मनसोऽविषयो देव रूपं ते निर्गुणं परम्} %8-43

\twolineshloka
{कथं दृश्यं भवेद्देव दृश्याभावे भजेत्कथम्}
{अतस्तवावतारेषु रूपाणि निपुणा भुवि} %8-44

\twolineshloka
{भजन्ति बुद्धिसम्पन्नास्तरन्त्येव भवार्णवम्}
{कामक्रोधादयस्तत्र बहवः परिपन्थिनः} %8-45

\twolineshloka
{भीषयन्ति सदा चेतो मार्जारा मूषकं यथा}
{त्वन्नाम स्मरतां नित्यं त्वद्रूपमपि मानसे} %8-46

\twolineshloka
{त्वत्पूजानिरतानां ते कथामृतपरात्मनाम्}
{त्वद्भक्तसङ्गिनां राम संसारो गोष्पदायते} %8-47

\twolineshloka
{अतस्ते सगुणं रूपं ध्यात्वाऽहं सर्वदा हृदि}
{मुक्तश्चरामि लोकेषु पूज्योऽहं सर्वदैवतैः} %8-48

\twolineshloka
{राम त्वया महत्कार्यं कृतं देवहितेच्छया}
{कुम्भकर्णवधेनाद्य भूभारोऽयं गतः प्रभो} %8-49

\twolineshloka
{श्वो हनिष्यति सौमित्रिरिन्द्रजेतारमाहवे}
{हनिष्यसेऽथ राम त्वं परश्वो दशकन्धरम्} %8-50

\twolineshloka
{पश्यामि सर्वं देवेश सिद्धैः सह नभोगतः}
{अनुगृह्णीष्व मां देव गमिष्यामि सुरालयम्} %8-51

\twolineshloka
{इत्युक्त्वा राममामन्त्र्य नारदो भगवानृषिः}
{ययौ देवैः पूज्यमानो ब्रह्मलोकमकल्मषम्} %8-52

{॥इति श्रीमदध्यात्मरामायणे उमामहेश्वरसंवादे युद्धकाण्डे
अष्टमे सर्गे नारदस्तुतिः सम्पूर्णः॥}
