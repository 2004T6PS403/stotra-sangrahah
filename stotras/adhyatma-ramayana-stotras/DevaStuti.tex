% !TeX program = XeLaTeX
% !TeX root = ../../shloka.tex

\sect{ब्रह्मेन्द्रदेवतादि  स्तुतिः}

\uvacha{श्री महादेव उवाच}
\addtocounter{shlokacount}{9}

\twolineshloka
{ततः शक्रः सहस्राक्षो यमश्च वरुणस्तथा}
{कुबेरश्च महातेजाः पिनाकी वृषवाहनः} %13-1

\twolineshloka
{ब्रह्मा ब्रह्मविदां श्रेष्ठो मुनिभिः सिद्धचारणैः}
{ऋषयः पितरः साध्या गन्धर्वाप्सरसोरगाः} %13-2

\twolineshloka
{एते चान्ये विमानाग्र्यैराजग्मुर्यत्र राघवः}
{अब्रुवन् परमात्मानं रामं प्राञ्जलयश्च ते} %13-3

\twolineshloka
{कर्ता त्वं सर्वलोकानां साक्षी विज्ञानविग्रहः}
{वसूनामष्टमोऽसि त्वं रुद्राणां शङ्करो भवान्} %13-4

\twolineshloka
{आदिकर्ताऽसि लोकानां ब्रह्मा त्वं चतुराननः}
{अश्विनौ घ्राणभूतौ ते चक्षुषी चन्द्रभास्करौ} %13-5

\twolineshloka
{लोकानामादिरन्तोऽसि नित्य एकः सदोदितः}
{सदा शुद्धः सदा बुद्धः सदा मुक्तोऽगुणोऽद्वयः} %13-6

\twolineshloka
{त्वन्मायासंवृतानां त्वं भासि मानुषविग्रहः}
{त्वन्नाम स्मरतां राम सदा भासि चिदात्मकः} %13-7

\twolineshloka
{रावणेन हृतं स्थानमस्माकं तेजसा सह}
{त्वयाऽद्य निहतो दुष्टः पुनः प्राप्तं पदं स्वकम्} %13-8

\twolineshloka
{एवं स्तुवत्सु देवेषु ब्रह्मा साक्षात्पितामहः}
{अब्रवीत्प्रणतो भूत्वा रामं सत्यपथे स्थितम्} %13-9


\uvacha{ब्रह्मोवाच}

\fourlineindentedshloka
{वन्दे देवं विष्णुमशेषस्थितिहेतुम्}
{त्वामध्यात्मज्ञानिभिरन्तर्हृदि भाव्यम्}
{हेयाहेयद्वन्द्वविहीनं परमेकम्}
{सत्तामात्रं सर्वहृदिस्थं दृशिरूपम्} %13-10

\fourlineindentedshloka
{प्राणापानौ निश्चयबुद्ध्या हृदि रुद्ध्वा}
{छित्वा सर्वं संशयबन्धं विषयौघान्}
{पश्यन्तीशं यं गतमोहा यतयस्तम्}
{वन्दे रामं रत्नकिरीटं रविभासम्} %13-11

\fourlineindentedshloka
{मायातीतं माधवमाद्यं जगदादिम्}
{मानातीतं मोहविनाशं मुनिवन्द्यम्}
{योगिध्येयं योगविधानं परिपूर्णम्}
{वन्दे रामं रञ्जितलोकं रमणीयम्} %13-12

\fourlineindentedshloka
{भावाभावप्रत्ययहीनं भवमुख्यैः}
{योगासक्तैरर्चितपादाम्बुजयुग्मम्}
{नित्यं शुद्धं बुद्धमनन्तं प्रणवाख्यम्}
{वन्दे रामं वीरमशेषासुरदावम्} %13-13

\fourlineindentedshloka
{त्वं मे नाथो नाथितकार्याखिलकारी}
{मानातीतो माधवरूपोऽखिलधारी}
{भक्त्या गम्यो भावितरूपो भवहारी}
{योगाभ्यासैर्भावितचेतःसहचारी} %13-14

\fourlineindentedshloka
{त्वामाद्यन्तं लोकततीनां परमीशम्}
{लोकानां नो लौकिकमानैरधिगम्यम्}
{भक्तिश्रद्धाभावसमेतैर्भजनीयम्}
{वन्दे रामं सुन्दरमिन्दीवरनीलम्} %13-15

\fourlineindentedshloka
{को वा ज्ञातुं त्वामतिमानं गतमानम्}
{मायासक्तो माधव शक्तो मुनिमान्यम्}
{वृन्दारण्ये वन्दितवृन्दारकवृन्दम्}
{वन्दे रामं भवमुखवन्द्यं सुखकन्दम्} %13-16

\fourlineindentedshloka
{नानाशास्त्रैर्वेदकदम्बैः प्रतिपाद्यम्}
{नित्यानन्दं निर्विषयज्ञानमनादिम्}
{मत्सेवार्थं मानुषभावं प्रतिपन्नम्}
{वन्दे रामं मरकतवर्णं मथुरेशम्} %13-17

\fourlineindentedshloka
{श्रद्धायुक्तो यः पठतीमं स्तवमाद्यम्}
{ब्राह्मं ब्रह्मज्ञानविधानं भुवि मर्त्यः}
{रामं श्यामं कामितकामप्रदमीशम्}
{ध्यात्वा ध्याता पातकजालैर्विगतः स्यात्} %13-18

\fourlineindentedshloka
{श्रुत्वा स्तुतिं लोकगुरोर्विभावसुः}
{स्वाङ्के समादाय विदेहपुत्रिकाम्}
{विभ्राजमानां विमलारुणद्युतिम्}
{रक्ताम्बरां दिव्यविभूषणान्विताम्} %13-19

\fourlineindentedshloka
{प्रोवाच साक्षी जगतां रघूत्तमम्}
{प्रपन्नसर्वार्तिहरं हुताशनः}
{गृहाण देवीं रघुनाथ जानकीम्}
{पुरा त्वया मय्यवरोपितां वने} %13-20

\fourlineindentedshloka
{विधाय मायाजनकात्मजां हरे}
{दशाननप्राणविनाशनाय च}
{हतो दशास्यः सह पुत्रबान्धवैः}
{निराकृतोऽनेन भरो भुवः प्रभो} %13-21

\fourlineindentedshloka
{तिरोहिता सा प्रतिबिम्बरूपिणी}
{कृता यदर्थं कृतकृत्यतां गता}
{ततोऽतिहृष्टां परिगृह्य जानकीम्}
{रामः प्रहृष्टः प्रतिपूज्य पावकम्} %13-22

\begin{minipage}{\linewidth}
\centering%major manual alignment to make it look good!
{\hspace{-7ex}स्वाङ्के समावेश्य सदाऽनपायिनीम्}\\
{\hspace{-1ex}श्रियं त्रिलोकीजननीं श्रियः पतिः।}
\fourlineindentedshloka
{दृष्ट्वाऽथ\hspace{1.3ex}रामं\hspace{1.3ex}जनकात्मजायुतम्}
{श्रिया स्फुरन्तं सुरनायको मुदा}
{भक्त्या गिरा गद्गदया समेत्य}
{कृताञ्जलिः \hspace{1.5ex}स्तोतुमथोपचक्रमे} %13-23
\end{minipage}

\begin{minipage}{\linewidth}
\uvacha{इन्द्र उवाच}

\fourlineindentedshloka
{भजेऽहं सदा राममिन्दीवराभम्}
{भवारण्यदावानलाभाभिधानम्}
{भवानीहृदा भावितानन्दरूपम्}
{भवाभावहेतुं भवादिप्रपन्नम्} %13-24
\end{minipage}

\fourlineindentedshloka
{सुरानीकदुःखौघनाशैकहेतुम्}
{नराकारदेहं निराकारमीड्यम्}
{परेशं परानन्दरूपं वरेण्यम्}
{हरिं राममीशं भजे भारनाशम्} %13-25

\fourlineindentedshloka
{प्रपन्नाखिलानन्ददोहं प्रपन्नम्}
{प्रपन्नार्तिनिःशेषनाशाभिधानम्}
{तपोयोगयोगीशभावाभिभाव्यम्}
{कपीशादिमित्रं भजे राममित्रम्} %13-26

\fourlineindentedshloka
{सदा भोगभाजां सुदूरे विभान्तम्}
{सदा योगभाजामदूरे विभान्तम्}
{चिदानन्दकन्दं सदा राघवेशम्}
{विदेहात्मजानन्दरूपं प्रपद्ये} %13-27

\fourlineindentedshloka
{महायोगमायाविशेषानुयुक्तो}
{विभासीश लीलानराकारवृत्तिः}
{त्वदानन्दलीलाकथापूर्णकर्णाः}
{सदानन्दरूपा भवन्तीह लोके} %13-28

\fourlineindentedshloka
{अहं मानपानाभिमत्तप्रमत्तो}
{न वेदाखिलेशाभिमानाभिमानः}
{इदानीं भवत्पादपद्मप्रसादात्}
{त्रिलोकाधिपत्याभिमानो विनष्टः} %13-29

\fourlineindentedshloka
{स्फुरद्रत्नकेयूरहाराभिरामम्}
{धराभारभूतासुरानीकदावम्}
{शरच्चन्द्रवक्त्रं लसत्पद्मनेत्रम्}
{दुरावारपारं भजे राघवेशम्} %13-30

\fourlineindentedshloka
{सुराधीशनीलाभ्रनीलाङ्गकान्तिम्}
{विराधादिरक्षोवधाल्लोकशान्तिम्}
{किरीटादिशोभं पुरारातिलाभम्}
{भजे रामचन्द्रं रघूणामधीशम्} %13-31

\fourlineindentedshloka
{लसच्चन्द्रकोटिप्रकाशादिपीठे}
{समासीनमङ्के समाधाय सीताम्}
{स्फुरद्धेमवर्णां तडित्पुञ्जभासाम्}
{भजे रामचन्द्रं निवृत्तार्तितन्द्रम्} %13-32

\twolineshloka
{ततः प्रोवाच भगवान् भवान्या सहितो भवः}
{रामं कमलपत्राक्षं विमानस्थो नभःस्थले} %13-33

\twolineshloka
{आगमिष्याम्ययोध्यायां द्रष्टुं त्वां राज्यसत्कृतम्}
{इदानीं पश्य पितरमस्य देहस्य राघव} %13-34


{॥इति श्रीमदध्यात्मरामायणे उमामहेश्वरसंवादे युद्धकाण्डे त्रयोदशे  सर्गे 
ब्रह्मेन्द्रदेवतादि  स्तुतिः  सम्पूर्णः॥}
