% !TeX program = XeLaTeX
% !TeX root = ../../shloka.tex

\sect{वेङ्कटेश अष्टकम्}

\twolineshloka
{वेङ्कटेशो वासुदेवः प्रद्युम्नोऽमितविक्रमः}
{सङ्कर्षणोऽनिरुद्धश्च शेषाद्रिपतिरेव च}

\twolineshloka
{जनार्दनः पद्मनाभो वेङ्कटाचलवासनः}
{सृष्टिकर्ता जगन्नाथो माधवो भक्तवत्सलः}

\twolineshloka
{गोविन्दो गोपतिः कृष्णः केशवो गरुडध्वजः}
{वराहो वामनश्चैव नारायण अधोक्षजः}

\twolineshloka
{श्रीधरः पुण्डरीकाक्षः सर्वदेवस्तुतो हरिः}
{श्रीनृसिंहो महासिंहः सूत्राकारः पुरातनः}

\twolineshloka
{रमानाथो महीभर्ता भूधरः पुरुषोत्तमः}
{चोळपुत्रप्रियः शान्तो ब्रह्मादीनां वरप्रदः}

\twolineshloka
{श्रीनिधिः सर्वभूतानां भयकृद्भयनाशनः}
{श्रीरामो रामभद्रश्च भवबन्धैकमोचकः}

\twolineshloka
{भूतावासो गिरिवासः श्रीनिवासः श्रियः पतिः}
{अच्युतानन्त गोविन्दो विष्णुर्वेङ्कटनायकः}

\twolineshloka
{सर्वदेवैकशरणं सर्वदेवैकदैवतम्}
{समस्तदेवकवचं सर्वदेवशिखामणिः}

\twolineshloka
{इतीदं कीर्तितं यस्य विष्णोरमिततेजसः}
{त्रिकाले यः पठेन्नित्यं पापं तस्य न विद्यते}

\twolineshloka
{राजद्वारे पठेद्-घोरे सङ्ग्रामे रिपुसङ्कटे}
{भूतसर्पपिशाचादिभयं नास्ति कदाचन}

\twolineshloka
{अपुत्रो लभते पुत्रान् निर्धनो धनवान् भवेत्}
{रोगार्तो मुच्यते रोगाद्बद्धो मुच्येत बन्धनात्}

\twolineshloka
{यद्यदिष्टतमं लोके तत्तत्प्राप्नोत्यसंशयः}
{ऐश्वर्यं राजसम्मानं भुक्तिमुक्तिफलप्रदम्}

\twolineshloka
{विष्णोर्लोकैकसोपानं सर्वदुःखैकनाशनम्}
{सर्वैश्वर्यप्रदं नॄणां सर्वमङ्गलकारकम्}

\twolineshloka
{मायावि परमानन्दं त्यक्त्वा वैकुण्ठमुत्तमम्}
{स्वामिपुष्करिणीतीरे रमया सह मोदते}

\twolineshloka
{कल्याणाद्भुतगात्राय कामितार्थप्रदायिने}
{श्रीमद्वेङ्कटनाथाय श्रीनिवासाय मङ्गलम्}

॥इति~श्री~ब्रह्माण्डपुराणे~ब्रह्मनारदसंवादे वेङ्कटगिरिमाहात्म्ये श्री~वेङ्कटेश अष्टकं सम्पूर्णम्॥