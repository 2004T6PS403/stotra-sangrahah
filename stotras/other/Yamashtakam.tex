% !TeX program = XeLaTeX
% !TeX root = ../../shloka.tex

\sect{यमाष्टकम्}
श्रीनारायण उवाच

\twolineshloka
{हरेरुत्कीर्तनं श्रुत्वा सावित्री यमवक्त्रतः}
{साश्रुनेत्रा सपुलका यमं पुनरुवाच सा}% ॥ १ ॥

सावित्र्युवाच

\twolineshloka
{हरेरुत्कीर्तनं धर्म स्वकुलोद्धारकारणम्}
{श्रोतॄणां चैव वक्तॄणां जन्ममृत्युजराहरम्}% ॥ २ ॥

\twolineshloka
{दानानां च व्रतानां च सिद्धीनां तपसां परम्}
{योगानां चैव वेदानां सेवनं कीर्तनं हरेः}% ॥ ३ ॥

\twolineshloka
{मुक्तत्वममरत्वं च सर्वसिद्धित्वमेव वा}
{श्रीकृष्णसेवनस्यैव कलां नार्हन्ति षोडशीम्}% ॥ ४ ॥

\twolineshloka
{भजामि केन विधिना श्रीकृष्णं प्रकृतेः परम्}
{मूढां मामबलां तात वद वेदविदां वर}% ॥ ५ ॥

\twolineshloka
{शुभकर्मविपाकं च श्रुतं नॄणां मनोहरम्}
{कर्माशुभविपाकं च तन्मे व्याख्यातुमर्हसि}% ॥ ६ ॥

\twolineshloka
{इत्युक्त्वा सा सती ब्रह्मन् भक्तिनम्रात्मकन्धरा}
{तुष्टाव धर्मराजं च वेदोक्तेन स्तवेन च}% ॥ ७ ॥

सावित्र्युवाच
\twolineshloka
{तपसा धर्ममाराध्य पुष्करे भास्करः पुरा}
{धर्मांशं यं सुतं प्राप धर्मराजं नमाम्यहम्}% ॥ ८ ॥

\twolineshloka
{समता सर्वभूतेषु यस्य सर्वस्य साक्षिणः}
{अतो यन्नाम शमनमिति तं प्रणमाम्यहम्}% ॥ ९ ॥

\twolineshloka
{येनान्तश्च कृतो विश्वे सर्वेषां जीविनां परम्}
{कामानुरूपकालेन तं कृतान्तं नमाम्यहम्}% ॥ १० ॥

\twolineshloka
{बिभर्ति दण्डं दण्ड्याय पापिनां शुद्धिहेतवे}
{नमामि तं दण्डधरं यः शास्ता सर्वकर्मणाम्}% ॥ ११ ॥

\twolineshloka
{विश्वे च कलयत्येव सर्वायुश्चापि सन्ततम्}
{अतीव दुर्निवार्यं च तं कालं प्रणमाम्यहम्}% ॥ १२ ॥

\twolineshloka
{तपस्वी वैष्णवो धर्मी यः संयमी विजितेन्द्रियः}
{जीविनां कर्मफलदं तं यमं प्रणमाम्यहम्}% ॥ १३ ॥

\twolineshloka
{स्वात्मारामश्च सर्वज्ञो मित्रं पुण्यकृतां भवेत्}
{पापिनां क्लेशदो यस्य पुण्यं मित्रं नमाम्यहम्}% ॥ १४ ॥

\twolineshloka
{यज्जन्म ब्रह्मणो वंशे ज्वलन्तं ब्रह्मतेजसा}
{यो ध्यायति परं ब्रह्म ब्रह्मवंशं नमाम्यहम्}% ॥ १५ ॥

\twolineshloka
{इत्युक्त्वा सा च सावित्री प्रणनाम यमं मुने}
{यमस्तां विष्णुभजनं कर्मपाकमुवाच ह}% ॥ १६ ॥

\twolineshloka
{इदं यमाष्टकं नित्यं प्रातरुत्थाय यः पठेत्}
{यमात्तस्य भयं नास्ति सर्वपापात्प्रमुच्यते}% ॥ १७ ॥

\twolineshloka
{महापापी यदि पठेन्नित्यं भक्त्या च नारद}
{यमः करोति संशुद्धं कायव्यूहेन निश्चितम्}% ॥ १८॥

{॥इति~श्रीब्रह्मवैवर्तमहापुराणे~प्रकृतिखण्डे श्री~नारद-नारायण-संवादे श्री~तुलस्योपाख्याने श्री~सावित्रीकृतयमस्तोत्रं सम्पूर्णम्॥}