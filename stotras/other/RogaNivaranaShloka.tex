% !TeX program = XeLaTeX
% !TeX root = ../../shloka.tex

\sect{रोगनिवारणश्लोकाः}

ममोपात्त- + श्रीपरमेश्वरप्रीत्यर्थं ज्वर-औपसर्गिकादि-नानाविध-साङ्क्रामिक-रोगाणाम् उन्मूलनार्थम् आरोग्य-प्राप्त्यर्थम् अस्मद्देशीयानां विदेशीयानां चापि सर्वेषां व्याधि-भय-निवृत्त्यर्थं सर्वलोकक्षेमार्थं रोगनिवारक-भगवन्नाम-स्तोत्र-पारायणं करिष्ये ।


\twolineshloka*
{अच्युतानन्तगोविन्द-नामोच्चारणभेषजात्}
{नश्यन्ति सकला रोगाः सत्यं सत्यं वदाम्यहम्}

\twolineshloka*
{शरीरे जर्जरीभूते व्याधिग्रस्ते कलेवरे}
{औषधं जाह्नवीतोयं वैद्यो नारायणो हरिः}

\twolineshloka*
{सुमीनाक्षिपते शम्भो सोमसुन्दरनायक}
{इमाम् आपदमुत्पन्नां मदीयां नाशय प्रभो}

\fourlineindentedshloka*
{आर्ता विषण्णाः शिथिलाश्च भीताः}
{घोरेषु च व्याधिषु वर्तमानाः}
{सङ्कीर्त्य नारायणशब्दमात्रं}
{विमुक्तदुःखाः सुखिनो भवन्ति}

\twolineshloka*
{बालाम्बिकेश वैद्येश भवरोगहरेति च}
{जपेन्नामत्रयं नित्यं महारोगनिवारणम्}

\twolineshloka*
{पञ्चापगेश जप्येश प्रणतार्तिहरेति च}
{जपेन्नामत्रयं नित्यं पुनर्जन्म न विद्यते}

\fourlineindentedshloka*
{अपस्मारकुष्ठक्षयार्शः-प्रमेह-}
{ज्वरोन्मादगुल्मादिरोगा महान्तः}
{पिशाचाश्च सर्वे भवत्पत्रभूतिं}
{विलोक्य क्षणात् तारकारे द्रवन्ते}

\fourlineindentedshloka*
{किरन्तीमङ्गेभ्यः किरणनिकुरुम्बामृतरसं}
{हृदि त्वामाधत्ते हिमकरशिलामूर्तिमिव यः}
{स सर्पाणां दर्पं शमयति शकुन्ताधिप इव}
{ज्वरप्लुष्टान् दृष्ट्या सुखयति सुधाधारसिरया}%20

\fourlineindentedshloka*
{मृगाः पक्षिणो दंशका ये च दुष्टाः}
{तथा व्याधयो बाधका ये मदङ्गे}
{भवच्छक्तितीक्ष्णाग्रभिन्नाः सुदूरे}
{विनश्यन्तु ते चूर्णितक्रौञ्चशैल}

