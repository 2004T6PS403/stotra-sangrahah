% !TeX program = XeLaTeX
% !TeX root = ../../shloka.tex

\sect{कामाक्षी सुप्रभातम्}

\fourlineindentedshloka*
{जगदवन विधौ त्वं जागरूका भवानि}
{तव तु जननि निद्रामात्मवत्कल्पयित्वा}
{प्रतिदिवसमहं त्वां बोधयामि प्रभाते}
{त्वयि कृतमपराधं सर्वमेतं क्षमस्व}

\fourlineindentedshloka*
{यदि प्रभातं तव सुप्रभातम्}
{तदा प्रभातं मम सुप्रभातम्}
{तस्मात् प्रभाते तव सुप्रभातम्}
{वक्ष्यामि मातः कुरु सुप्रभातम्}

\dnsub{गुरु ध्यानम्}
\fourlineindentedshloka*
{यस्याङ्घ्रिपद्म-मकरन्दनिषेवणात् त्वम्}
{जिह्वां गताऽसि वरदे मम मन्दबद्धः}
{यस्याम्ब नित्यमनघे हृदये विभासि}
{तं चन्द्रशेखरगुरुं प्रणमामि नित्यम्}

\fourlineindentedshloka*
{जये जयेन्द्रो गुरुणा ग्रहीतो}
{मठाधिपत्ये शशिशेखरेण}
{यथा गुरुः सर्वगुणोपपन्नो}
{जयत्यसौ मङ्गलमातनोतु}

\fourlineindentedshloka*
{शुभं दिशतु नो देवी कामाक्षी सर्वमङ्गला}
{शुभं दिशतु नो देवी कामकोटी-मठेशः}
{शुभं दिशतु तच्छिष्य-सद्गुरुर्नो जयेन्द्रो}
{सर्वंमङ्गलमेवास्तु मङ्गलानि भवन्तु नः}

\dnsub{सुप्रभातम्}
\fourlineindentedshloka
{कामाक्षि देव्यम्ब तवार्द्रदृष्ट्या}
{मूकः स्वयं मूककविर्यथाऽसीत्}
{तथा कुरु त्वं परमेशजाये}
{त्वत्पादमूले प्रणतं दयार्द्रे}

\twolineshloka
{उत्तिष्ठोत्तिष्ठ वरदे उत्तिष्ठ जगदीश्वरि}
{उत्तिष्ठ जगदाधारे त्रैलोक्यं मङ्गलं कुरु}

\fourlineindentedshloka
{शृणोषि कच्चिद्-ध्वनिरुत्थितोऽयम्}
{मृदङ्गभेरीपटहानकानाम्}
{वेदध्वनिं शिक्षितभूसुराणाम्}
{शृणोषि भद्रे कुरु सुप्रभातम्}

\fourlineindentedshloka
{शृणोषि भद्रे ननु शङ्खघोषम्}
{वैतालिकानां मधुरं च गानम्}
{शृणोषि मातः पिककुक्कुटानाम्}
{ध्वनिं प्रभाते कुरु सुप्रभातम्}

\fourlineindentedshloka
{मातर्निरीक्ष्य वदनं भगवान् शशाङ्को -}
{लज्जान्वितः स्वयमहो निलयं प्रविष्टः}
{द्रष्टुं त्वदीय वदनं भगवान् दिनेशो -}
{ह्यायाति देवि सदनं कुरु सुप्रभातम्}

\fourlineindentedshloka
{पश्याम्ब केचिद्-धृतपूर्णकुम्भाः}
{केचिद्-दयार्द्रे धृतपुष्पमालाः}
{काचित् शुभाङ्ग्यो ननुवाद्यहस्ताः}
{तिष्ठन्ति तेषां कुरु सुप्रभातम्}

\fourlineindentedshloka
{भेरीमृदङ्गपणवानकवाद्यहस्ताः}
{स्तोतुं महेशदयिते स्तुतिपाठकास्त्वाम्}
{तिष्ठन्ति देवि समयं तव काङ्क्षमाणाः}
{ह्युत्तिष्ठ दिव्यशयनात् कुरु सुप्रभातम्}

\fourlineindentedshloka
{मातर्निरीक्ष वदनं भगवान् त्वदीयम्}
{नैवोत्थितः शशिधिया शयितस्तवाङ्के}
{सम्बोधयाशु गिरिजे विमलं प्रभातम्}
{जातं महेशदयिते कुरु सुप्रभातम्}

\fourlineindentedshloka
{अन्तश्चरन्त्यास्तव भूषणानाम्}
{झल्झल्ध्वनिं नूपुरकङ्कणानाम्}
{श्रुत्वा प्रभाते तव दर्शनार्थी}
{द्वारि स्थितोऽहं कुरु सुप्रभातम्}

\fourlineindentedshloka
{वाणी पुस्तकमम्बिके गिरिसुते पद्मानि पद्मासना}
{रम्भा त्वम्बरडम्बरं गिरिसुता गङ्गा च गङ्गाजलम्}
{काली तालयुगं मृदङ्गयुगलं बृन्दा च नन्दा तथा}
{नीला निर्मलदर्पण-धृतवती तासां प्रभातं शुभम्}

\fourlineindentedshloka
{उत्थाय देवि शयनाद्भगवान् पुरारिः}
{स्नातुं प्रयाति गिरिजे सुरलोकनद्याम्}
{नैको हि गन्तुमनघे रमते दयार्द्रे}
{ह्युत्तिष्ठ देवि शयनात् कुरु सुप्रभातम्}

\fourlineindentedshloka
{पश्याम्ब केचित्फलपुष्पहस्ताः}
{केचित् पुराणानि पठन्ति मातः}
{पठन्ति वेदान् बहवस्तवारे}
{तेषां जनानां कुरु सुप्रभातम्}

\fourlineindentedshloka
{लावण्यशेवधिमवेक्ष्य चिरं त्वदीयम्}
{कन्दर्पदर्पदलनोऽपि वशं गतस्ते}
{कामारि-चुम्बित-कपोलयुगं त्वदीयम्}
{द्रष्टुं स्थिता वयमये कुरु सुप्रभातम्}

\fourlineindentedshloka
{गाङ्गेयतोयममवाह्य मुनीश्वरास्त्वाम्}
{गङ्गाजलैः स्नपयितुं बहवो घटांश्च}
{धृत्वा शिरःसु भवतीमभिकाङ्क्षमाणाः}
{द्वारि स्थिता हि वरदे कुरु सुप्रभातम्}

\fourlineindentedshloka
{मन्दार-कुन्द-कुसुमैरपि जातिपुष्पैः}
{मालाकृता विरचितानि मनोहराणि}
{माल्यानि दिव्यपदयोरपि दातुमम्ब}
{तिष्ठन्ति देवि मुनयः कुरु सुप्रभातम्}

\fourlineindentedshloka
{काञ्ची-कलाप-परिरम्भनितम्बबिम्बम्}
{काश्मीर-चन्दन-विलेपित-कण्ठदेशम्}
{कामेश-चुम्बित-कपोलमुदारनासाम्}
{द्रष्टुं स्थिता वयमये कुरु सुप्रभातम्}

\fourlineindentedshloka
{मन्दस्मितं विमलचारुविशालनेत्रम्}
{कण्ठस्थलं कमलकोमलगर्भगौरम्}
{चक्राङ्कितं च युगलं पदयोर्मृगाक्षि}
{द्रष्टुं स्थिता वयमये कुरु सुप्रभातम्}

\fourlineindentedshloka
{मन्दस्मितं त्रिपुरनाशकरं पुरारेः}
{कामेश्वरप्रणयकोपहरं स्मितं ते}
{मन्दस्मितं विपुलहासमवेक्षितुं ते}
{मातः स्थिता वयमये कुरु सुप्रभातम्}

\fourlineindentedshloka
{माता शिशूनां परिरक्षणार्थम्}
{न चैव निद्रावशमेति लोके}
{माता त्रयाणां जगतां गतिस्त्वम्}
{सदा विनिद्रा कुरु सुप्रभातम्}

\fourlineindentedshloka
{मातर्मुरारिकमलासनवन्दिताङ्घ्र्याः}
{हृद्यानि दिव्यमधुराणि मनोहराणि}
{श्रोतुं तवाम्ब वचनानि शुभप्रदानि}
{द्वारि स्थिता वयमये कुरु सुप्रभातम्}

\fourlineindentedshloka
{दिगम्बरो ब्रह्मकपालपाणिः}
{विकीर्णकेशः फणिवेष्टिताङ्गः}
{तथाऽपि मातस्तव देविसङ्गात्}
{महेश्वरोऽभूत् कुरु सुप्रभातम्}

\fourlineindentedshloka
{अयि तु जननि दत्तस्तन्यपानेन देवि}
{द्रविडशिशुरभूद्वै ज्ञानसम्पन्नमूर्तिः}
{द्रविडतनयभुक्तक्षीरशेषं भवानि}
{वितरसि यदि मातः सुप्रभातं भवेन्मे}

\fourlineindentedshloka
{जननि तव कुमारः स्तन्यपानप्रभावात्}
{शिशुरपि तव भर्तुः कर्णमूले भवानि}
{प्रणवपदविशेषं बोधयामास देवि}
{यदि मयि च कृपा ते सुप्रभातं भवेन्मे}

\fourlineindentedshloka
{त्वं विश्वनाथस्य विशालनेत्रा}
{हालास्यनाथस्य नु मीननेत्रा}
{एकाम्रनाथस्य नु कामनेत्रा}
{कामेशजाये कुरु सुप्रभातम्}

\fourlineindentedshloka
{श्रीचन्द्रशेखरगुरुर्भगवान् शरण्ये}
{त्वत्पादभक्तिभरितः फलपुष्पपाणिः}
{एकाम्रनाथदयिते तव दर्शनार्थी}
{तिष्ठत्ययं यतिवरो मम सुप्रभातम्}

\fourlineindentedshloka
{एकाम्रनाथदयिते ननु कामपीठे}
{सम्पूजिताऽसि वरदे गुरुशङ्करेण}
{श्रीशङ्करादिगुरुवर्य-समर्चिताङ्घ्रिम्}
{द्रष्टुं स्थिता वयमये कुरु सुप्रभातम्}

\fourlineindentedshloka
{दुरितशमनदक्षौ मृत्युसन्तासदक्षौ}
{चरणमुपगतानां मुक्तिदौ ज्ञानदौ तौ}
{अभयवरदहस्तौ द्रष्टुमम्ब स्थितोऽहम्}
{त्रिपुरदलनजाये सुप्रभातं ममार्ये}

\fourlineindentedshloka
{मातस्त्वदीयचरणं हरिपद्मजाद्यैः}
{वन्द्यं रथाङ्ग-सरसीरुह-शङ्खचिह्नम्}
{द्रष्टुं च योगिजनमानसराजहंसम्}
{द्वारि स्थितोऽस्मि वरदे कुरु सुप्रभातम्}

\fourlineindentedshloka
{पश्यन्तु केचिद्वदनं त्वदीयम्}
{स्तुवन्तु कल्याणगुणांस्तवान्ये}
{नमन्तु पादाब्जयुगं त्वदीयाः}
{द्वारि स्थितानां कुरु सुप्रभातम्}

\fourlineindentedshloka
{केचित् सुमेरोः शिखरेऽतितुङ्गे}
{केचिन्मणिद्वीपवरे विशाले}
{पश्यन्तु केचित् त्वमृताब्धिमध्ये}
{पश्याम्यहं त्वामिह सुप्रभातम्}
\setlength{\shlokaspaceskip}{16pt}
\fourlineindentedshloka
{शम्भोर्वामाङ्कसंस्थां शशिनिभवदनां नीलपद्मायताक्षीम्}
{श्यामाङ्गां चारुहासां निबिडतरकुचां पक्वबिम्बाधरोष्ठीम्}
{कामाक्षीं कामदात्रीं कुटिलकचभरां भूषणैर्भूषिताङ्गीम्}
{पश्यामः सुप्रभाते प्रणतजनिमतामद्य नः सुप्रभातम्}
\setlength{\shlokaspaceskip}{24pt}
\fourlineindentedshloka
{कामप्रदाकल्पतरुर्विभासि}
{नान्या गतिर्मे ननु चातकोऽहम्}
{वर्षस्यमोघः कनकाम्बुधाराः}
{काश्चित्तु धारा मयि कल्पयाशु}

\twolineshloka
{त्रिलोचनप्रियां वन्दे वन्दे त्रिपुरसुन्दरीम्}
{त्रिलोकनायिकां वन्दे सुप्रभातं ममाम्बिके}

\fourlineindentedshloka
{कामाक्षि देव्यम्ब तवार्द्रदृष्ट्या}
{कृतं मयेदं खलु सुप्रभातम्}
{सद्यः फलं मे सुखमम्ब लब्धम्}
{तथा च मे दुःखदशा गता हि}

\fourlineindentedshloka
{ये वा प्रभाते पुरतस्तवार्ये}
{पठन्ति भक्त्या ननु सुप्रभातम्}
{शृण्वन्ति ये वा त्वयि बद्धचित्ताः}
{तेषां प्रभातं कुरु सुप्रभातम्}

॥इति~श्री~लक्ष्मीकान्त-शर्मा-विरचितं श्री~कामाक्षीसुप्रभातं सम्पूर्णम्॥