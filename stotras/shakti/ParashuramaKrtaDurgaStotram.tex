% !TeX program = XeLaTeX
% !TeX root = ../../shloka.tex

\sect{परशुरामकृत-दुर्गास्तोत्रम्}
\uvacha{श्री परशुराम उवाच}

\twolineshloka
{श्रीकृष्णस्य च गोलोके परिपूर्णतमस्य च}
{आविर्भूता विग्रहतः परा सृष्ट्युन्मुखस्य च}

\twolineshloka
{सूर्यकोटिप्रभायुक्ता वस्त्रालङ्कारभूषिता}
{वह्निशुद्धांशुकाधाना सुस्मिता सुमनोहरा}

\twolineshloka
{नवयौवनसम्पन्ना सिन्दूरारुण्यशोभिता}
{ललितं कबरीभारं मालतीमाल्यमण्डितम्}

\twolineshloka
{अहोऽनिर्वचनीया त्वं चारुमूर्तिं च बिभ्रती}
{मोक्षप्रदा मुमुक्षूणां महाविष्णोर्विधिः स्वयम्}

\twolineshloka
{मुमोह क्षणमात्रेण दृष्ट्वा त्वां सर्वमोहिनीम्}
{बालैः सम्भूय सहसा सस्मिता धाविता पुरा}

\twolineshloka
{सद्भिः ख्याता तेन राधा मूलप्रकृतिरीश्वरी}
{कृष्णस्त्वां सहसा भीतो वीर्याधानं चकार ह}

\twolineshloka
{ततो डिम्भं महज्जज्ञे ततो जातो महान्विराट्}
{यस्यैव लोमकूपेषु ब्रह्माण्डान्यखिलानि च}

\twolineshloka
{राधारतिक्रमेणैव तन्निःश्वासो बभूव ह}
{स निःश्वासो महावायुः स विराड्\mbox{}विश्वधारकः}

\twolineshloka
{भयघर्मजलेनैव पुप्लुवे विश्वगोलकम्}
{स विराड् विश्वनिलयो जलराशिर्बभूव ह}

\twolineshloka
{ततस्त्वं पञ्चधा भूय पञ्चमूर्तीश्च बिभ्रती}
{प्राणाधिष्ठातृमूर्त्तिर्या कृष्णस्य परमात्मनः}

\twolineshloka
{कृष्णप्राणाधिकां राधां तां वदन्ति पुराविदः}
{वेदाधिष्ठात्रीमूर्तिर्या वेदशास्त्रप्रसूरपि}

\twolineshloka
{तं सावित्रीं शुद्धरूपां प्रवदन्ति मनीषिणः}
{ऐश्वर्याधिष्ठातृमूर्तिः शान्तिस्त्वं शान्तरूपिणी}

\twolineshloka
{लक्ष्मीं वदन्ति संतस्तां शुद्धां सत्त्‍‌वस्वरूपिणीम्}
{रागाधिष्ठात्री या देवी शुक्लमूर्तिः सतां प्रसूः}

\twolineshloka
{सरस्वतीं तां शास्त्रज्ञां शास्त्रज्ञाः प्रवदन्त्यहो}
{बुद्धिर्विद्या सर्वशक्तेर्या मूर्तिरधिदेवता}

\twolineshloka
{सर्वमङ्गलदा सन्तो वदन्ति सर्वमङ्गलाम्}
{सर्वमङ्गलमङ्गल्या सर्वमङ्गलरूपिणी}

\twolineshloka
{सर्वमङ्गलबीजस्य शिवस्य निलयेऽधुना}
{शिवे शिवास्वरूपा त्वं लक्ष्मीर्नारायणान्तिके}

\twolineshloka
{सरस्वती च सावित्री वेदसू‌र्ब्रह्मणः प्रिया}
{राधा रासेश्वरस्यैव परिपूर्णतमस्य च}

\twolineshloka
{परमानन्दरूपस्य परमानन्दरूपिणी}
{त्वत्कलांशांशकलया देवानामपि योषितः}

\twolineshloka
{त्वं विद्या योषितः सर्वाः सर्वेषां बीजरूपिणी}
{छाया सूर्यस्य चन्द्रस्य रोहिणी सर्वमोहिनी}

\twolineshloka
{शची शक्रस्य कामस्य कामिनी रतिरीश्वरी}
{वरुणानी जलेशस्य वायोः स्त्रीः प्राणवल्लभा}

\twolineshloka
{वह्नेः प्रिया हि स्वाहा च कुबेरस्य च सुन्दरी}
{यमस्य तु सुशीला च नैर्ऋतस्य च कैटभी}

\twolineshloka
{ऐशानी स्याच्छशिकला शतरूपा मनोः प्रिया}
{देवहूतिः कर्दमस्य वसिष्ठस्याप्यरुन्धती}

\twolineshloka
{लोपामुद्राऽप्यगस्त्यस्य देवमाताऽदितिस्तथा}
{अहल्या गौतमस्यापि सर्वाधारा वसुन्धरा}

\twolineshloka
{गङ्गा च तुलसी चापि पृथिव्यां या सरिद्वरा}
{एताः सर्वाश्च या ह्यन्या सर्वास्त्वत्कलयाऽम्बिके}

\twolineshloka
{गृहलक्ष्मीर्गृहे नॄणां राजलक्ष्मीश्च राजसु}
{तपस्विनां तपस्या त्वं गायत्री ब्राह्मणस्य च}

\twolineshloka
{सतां सत्त्‍‌वस्वरूपा त्वमसतां कलहाङ्कुरा}
{ज्योतीरूपा निर्गुणस्य शक्तिस्त्वं सगुणस्य च}

\twolineshloka
{सूर्ये प्रभास्वरूपा त्वं दाहिका च हुताशने}
{जले शैत्यस्वरूपा च शोभारूपा निशाकरे}

\twolineshloka
{त्वं भूमौ गन्धरूपा च आकाशे शब्दरूपिणी}
{क्षुत्पिपासादयस्त्वं च जीविनां सर्वशक्तयः}

\twolineshloka
{सर्वबीजस्वरूपा त्वं संसारे साररूपिणी}
{स्मृतिर्मेधा च बुद्धिर्वा ज्ञानशक्तिर्विपश्चिताम्}

\twolineshloka
{कृष्णेन विद्या या दत्ता सर्वज्ञानप्रसूः शुभा}
{शूलिने कृपया सा त्वं यया मृत्युञ्जयः शिवः}

\twolineshloka
{सृष्टिपालनसंहारशक्तयस्त्रिविधाश्च याः}
{ब्रह्मविष्णुमहेशानां सा त्वमेव नमोऽस्तु ते}

\twolineshloka
{मधुकैटभभीत्या च त्रस्तो धाता प्रकम्पितः}
{स्तुत्वा मुक्तश्च यां देवीं तां मूर्ध्ना प्रणमाम्यहम्}

\twolineshloka
{मधुकैटभयोर्युद्धे त्रातासौ विष्णुरीश्वरीम्}
{बभूव शक्तिमान् स्तुत्वा तां दुर्गां प्रणमाम्यहम्}

\twolineshloka
{त्रिपुरस्य महायुद्धे सरथे पतिते शिवे}
{यां तुष्टुवुः सुराः सर्वे तां दुर्गां प्रणमाम्यहम्}

\twolineshloka
{विष्णुना वृषरूपेण स्वयं शम्भुः समुत्थितः}
{जघान त्रिपुरं स्तुत्वा तां दुर्गां प्रणमाम्यहम्}

\twolineshloka
{यदाज्ञया वाति वातः सूर्यस्तपति सन्ततम्}
{वर्षतीन्द्रो दहत्यग्निस्तां दुर्गां प्रणमाम्यहम्}

\twolineshloka
{यदाज्ञया हि कालश्च शश्वद्-भ्रमति वेगतः}
{मृत्युश्चरति जन्तूनां तां दुर्गां प्रणमाम्यहम्}

\twolineshloka
{स्त्रष्टा सृजति सृष्टिं च पाता पाति यदाज्ञया}
{संहर्ता संहरेत् काले तां दुर्गां प्रणमाम्यहम्}

\twolineshloka
{ज्योतिःस्वरूपो भगवाञ्छ्रीकृष्णो निर्गुणः स्वयम्}
{यया विना न शक्तश्च सृष्टिं कर्तुं नमामि ताम्}

\twolineshloka
{रक्ष रक्ष जगन्मातरपराधं क्षमस्व मे}
{शिशूनामपराधेन कुतो माता हि कुप्यति}

\twolineshloka
{इत्युक्त्वा परशुरामश्च नत्वा तां च रुरोद ह}
{तुष्टा दुर्गा सम्भ्रमेण चाभयं च वरं ददौ}

\twolineshloka
{अमरो भव हे पुत्र वत्स सुस्थिरतां व्रज}
{शर्वप्रसादात् सर्वत्र जयोऽस्तु तव सन्ततम्}


\twolineshloka
{सर्वान्तरात्मा भगवांस्तुष्टः स्यात्सन्ततं हरिः}
{भक्तिर्भवतु ते कृष्णे शिवदे च शिवे गुरौ}

\twolineshloka
{इष्टदेवे गुरौ यस्य भक्तिर्भवति शाश्वती}
{तं हन्तुं न हि शक्ताश्च रुष्टा वा सर्वदेवताः}

\twolineshloka
{श्रीकृष्णस्य च भक्तस्त्वं शिष्यो वै शङ्करस्य च}
{गुरुपत्‍‌नीं स्तौषि यस्मात् कस्त्वां हन्तुमिहेश्वरः}

\twolineshloka
{अहो न कृष्णभक्तानामशुभं विद्यते क्वचित्}
{अन्यदेवेषु ये भक्ता न भक्ता वा निरङ्कुशाः}

\twolineshloka
{चन्द्रमा बलवांस्तुष्टो येषां भाग्यवतां भृगो}
{तेषां तारागणा रुष्टाः किं कुर्वन्ति च दुर्बलाः}

\twolineshloka
{यस्मै तुष्टः पालयति नरदेवो महान् सुखी}
{तस्य किं वा करिष्यन्ति रुष्टा भृत्याश्च दुर्बलाः}

\twolineshloka
{इत्युक्त्वा पार्वती तुष्टा दत्त्‍‌वा रामाय चाऽऽशिषम्}
{जगामान्तःपुरं तूर्णं हर्षशब्दो बभूव ह}

\dnsub{फलश्रुतिः}

\twolineshloka
{स्तोत्रं वै काण्वशाखोक्तं पूजाकाले च यः पठेत्}
{यात्राकाले च प्रातर्वा वाञ्छितार्थं लभेद्ध्रुवम्}

\twolineshloka
{पुत्रार्थी लभते पुत्रं कन्यार्थी कन्यकां लभेत्}
{विद्यार्थी लभते विद्यां प्रजार्थी चाऽऽप्नुयात् प्रजाम्}

\twolineshloka
{भ्रष्टराज्यो लभेद्राज्यं नष्टवित्तो धनं लभेत्}
{यस्य रुष्टो गुरुर्देवो राजा वा बान्धवोऽथवा}

\twolineshloka
{तस्मै तुष्टश्च वरदः स्तोत्रराजप्रसादतः}
{दस्युग्रस्तो फणिग्रस्तः शत्रुग्रस्तो भयानकः}

\twolineshloka
{व्याधिग्रस्तो भवेन्मुक्तः स्तोत्रस्मरणमात्रतः}
{राजद्वारे श्मशाने च कारागारे च बन्धने}

\twolineshloka
{जलराशौ निमग्नश्च मुक्तस्तत्स्मृतिमात्रतः}
{स्वामिभेदे पुत्रभेदे मित्रभेदे च दारुणे}

\twolineshloka
{स्तोत्रस्मरणमात्रेण वाञ्छितार्थं लभेद्ध्रुवम्}
{कृत्वा हविष्यं वर्षं च स्तोत्रराजं श्रृणोति या}

\threelineshloka
{भक्तया दुर्गां च सम्पूज्य महावन्ध्या प्रसूयते}
{लभते सा दिव्यपुत्रं ज्ञानिनं चिरजीविनम्}
{असौभाग्या च सौभाग्यं षण्मासश्रवणाल्लभेत्}

\twolineshloka
{नवमासं काकवन्ध्या मृतवत्सा च भक्तितः}
{स्तोत्रराजं या श्रृणोति सा पुत्रं लभते ध्रुवम्}

\twolineshloka
{कन्यामाता पुत्रहीना पञ्चमासं श्रृणोति या}
{घटे सम्पूज्य दुर्गां च सा पुत्रं लभते ध्रुवम्}


{॥इति~श्रीब्रह्मवैवर्तमहापुराणे~गणपतिखण्डे पञ्चचत्वारिंशेऽध्याये श्री~नारद-नारायण-संवादे श्री~परशुरामकृतं श्री~दुर्गास्तोत्रं सम्पूर्णम्॥}