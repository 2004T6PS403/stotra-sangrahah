% !TeX program = XeLaTeX
% !TeX root = ../../shloka.tex

\sect{षष्ठीदेवी स्तोत्रम्}
\centerline{प्रियव्रत उवाच}
\twolineshloka
{नमो देव्यै महादेव्यै सिद्‍ध्यै शान्त्यै नमो नमः}
{शुभायै देवसेनायै षष्ठीदेव्यै नमो नमः}%१॥

\twolineshloka
{वरदायै पुत्रदायै धनदायै नमो नमः}
{सुखदायै मोक्षदायै षष्ठीदेव्यै नमो नमः}%२॥

\twolineshloka
{शक्तेः षष्ठांशरूपायै सिद्धायै च नमो नमः}
{मायायै सिद्धयोगिन्यै षष्ठीदेव्यै नमो नमः}%३॥

\twolineshloka
{पारायै पारदायै च षष्ठीदेव्यै नमो नमः}
{सारायै सारदायै च पारायै सर्वकर्मणाम्}%४॥

\threelineshloka
{बालाधिष्टातृदेव्यै च षष्ठीदेव्यै नमो नमः}
{कल्याणदायै कल्याण्यै फलदायै च कर्मणाम्}
{प्रत्यक्षायै च भक्तानां षष्ठीदेव्यै नमो नमः}%५॥

\twolineshloka
{पूज्यायै स्कन्दकान्तायै सर्वेषां सर्वकर्मसु}
{देवरक्षणकारिण्यै षष्ठीदेव्यै नमो नमः}%६॥

\twolineshloka
{शुद्धसत्त्वस्वरूपायै वन्दितायै नृणां सदा}
{हिंसाक्रोधैर्वर्जितायै षष्ठीदेव्यै नमो नमः}%७॥

\twolineshloka
{धनं देहि प्रियां देहि पुत्रं देहि सुरेश्वरि}
{धर्मं देहि यशो देहि षष्ठीदेव्यै नमो नमः}%८॥

\twolineshloka
{भूमिं देहि प्रजां देहि देहि विद्यां सुपूजिते}
{कल्याणं च जयं देहि षष्ठीदेव्यै नमो नमः}%९॥

\twolineshloka
{इति देवीं च संस्तूय लेभे पुत्रं प्रियव्रतः}
{यशस्विनं च राजेन्द्रं षष्ठीदेवीप्रसादतः}%

\twolineshloka
{षष्ठीस्तोत्रमिदं ब्रह्मण् यः शृणोति च वत्सरम्}
{अपुत्रो लभते पुत्रं वरं सुचिरजीवनम्}

\twolineshloka
{वर्षमेकं च या भक्त्या संयतेदं शृणोति च}
{सर्वपापाद्विनिर्मुक्ता महावन्ध्या प्रसूयते}

\twolineshloka
{वीरपुत्रं च गुणिनं विद्यावन्तं यशस्विनम्}
{सुचिरायुष्मन्तमेव षष्ठीमातृप्रसादतः}

\twolineshloka
{काकवन्ध्या च या नारी मृतापत्या च या भवेत्}
{वर्षं श्रुत्वा लभेत्पुत्रं षष्ठीदेवीप्रसादतः}

\twolineshloka
{रोगयुक्ते च बाले च पिता माता शृणोति च}
{मासं  च मुच्यते बालः षष्ठीदेवीप्रसादतः}

{॥इति~श्रीब्रह्मवैवर्तमहापुराणे~प्रकृतिखण्डे श्री~नारद-नारायण-संवादे षष्ठ्युपाख्याने श्री~प्रियव्रतविरचितं श्री~षष्ठीदेवीस्तोत्रं सम्पूर्णम्॥}