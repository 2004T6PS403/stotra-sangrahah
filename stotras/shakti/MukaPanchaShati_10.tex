% !TeX program = XeLaTeX
% !TeX root = ../../shloka.tex

\sect{मूकपञ्चशती}

\dnsub{आर्याशतकम्}

\twolineshloka
{कारण-पर-चिद्रूपा काञ्चीपुर-सीम्नि कामपीठगता}
{काचन विहरति करुणा काश्मीर-स्तबक-कोमलाङ्गलता}% ॥ १॥

\twolineshloka
{कञ्चन काञ्ची-निलयं करधृत-कोदण्ड-बाण-सृणि-पाशम्}
{कठिन-स्तनभर-नम्रं कैवल्यानन्द-कन्दम् अवलम्बे}% ॥ २॥

\twolineshloka
{चिन्तित-फल-परिपोषण-चिन्तामणिरेव काञ्चिनिलया मे}
{चिरतर-सुचरित-सुलभा चित्तं शिशिरयतु चित्सुखाधारा}% ॥ ३॥

\twolineshloka
{कुटिलकचं कठिनकुचं कुन्दस्मित-कान्ति कुङ्कुमच्छायम्}
{कुरुते विहृतिं काञ्च्यां कुलपर्वत-सार्वभौम-सर्वस्वम्}% ॥ ४॥

\twolineshloka
{पञ्चशर-शास्त्रबोधन-परमाचार्येण दृष्टिपातेन}
{काञ्चीसीम्नि कुमारी काचन मोहयति कामजेतारम्}% ॥ ५॥

\twolineshloka
{परया काञ्चीपुरया पर्वत-पर्याय-पीनकुच-भरया}
{परतन्त्रा वयमनया पङ्कज-सब्रह्मचारि-लोचनया}% ॥ ६॥

\twolineshloka
{ऐश्वर्यमिन्दुमौलेः ऐकात्म्य-प्रकृति काञ्चिमध्यगतम्}
{ऐन्दव-किशोर-शेखरम् ऐदम्पर्यं चकास्ति निगमानाम्}% ॥ ७॥

\twolineshloka
{श्रितकम्प-सीमानं शिथिलित-परमशिव-धैर्य-महिमानम्}
{कलये पाटलिमानं कञ्चन कञ्चुकित-भुवन-भूमानम्}% ॥ ८॥

\twolineshloka
{आदृत-काञ्चीनिलयाम् आद्याम् आरूढ-यौवनाटोपाम्}
{आगम-वतंस-कलिकाम् आनन्दाद्वैत-कन्दलीं वन्दे}% ॥ ९॥

\twolineshloka
{तुङ्गाभिराम-कुचभर-शृङ्गारितम् आश्रयामि काञ्चिगतम्}
{गङ्गाधर-परतन्त्रं शृङ्गाराद्वैत-तन्त्र-सिद्धान्तम्}% ॥ १०॥

