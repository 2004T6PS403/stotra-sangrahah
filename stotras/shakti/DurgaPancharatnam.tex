% !TeX program = XeLaTeX
% !TeX root = ../../shloka.tex

\sect{दुर्गापञ्चरत्नम्}

\fourlineindentedshloka
{ते ध्यान-योगानुगता अपश्यन्}
{त्वामेव देवीं स्वगुणैर्निगूढाम्}
{त्वमेव शक्तिः परमेश्वरस्य}
{मां पाहि सर्वेश्वरि मोक्षदात्रि}%॥ १॥

\fourlineindentedshloka
{देवात्मशक्तिः श्रुतिवाक्यगीता}
{महर्षि लोकस्य पुरः प्रसन्ना}
{गुहा परं व्योम सतः प्रतिष्ठा}
{मां पाहि सर्वेश्वरि मोक्षदात्रि}%॥ २॥

\fourlineindentedshloka
{परास्य शक्तिर्विविधैव श्रूयसे}
{श्वेताश्व-वाक्योदित-देवि दुर्गे}
{स्वाभाविकी ज्ञानबलक्रिया ते}
{मां पाहि सर्वेश्वरि मोक्षदात्रि}%॥ ३॥

\fourlineindentedshloka
{देवात्मशब्देन शिवात्मभूता}
{यत्कूर्मवायव्यवचो विवृत्या}
{त्वं पाशविच्छेदकरी प्रसिद्धा}
{मां पाहि सर्वेश्वरि मोक्षदात्रि}%॥ ४॥

\fourlineindentedshloka
{त्वं ब्रह्मपुच्छा विविधा मयूरी}
{ब्रह्म-प्रतिष्ठाऽस्युपदिष्ट-गीता}
{ज्ञानस्वरूपात्मतयाऽखिलानाम्}
{मां पाहि सर्वेश्वरि मोक्षदात्रि}%॥ ५॥

{॥इति श्री काञ्चीपुरजगद्गुरुणा श्रीमच्चन्द्रशेखरेन्द्र-सरस्वती-स्वामिना विरचितं श्री दुर्गापञ्चरत्नं सम्पूर्णम्॥}

\closesection

\sect{रुक्मिणीकृत गौरीस्तोत्रम्}
\twolineshloka*
{नमस्ये त्वामम्बिकेऽभीक्ष्णं स्वसन्तानयुतां शिवाम्}
{भूयात्पतिर्मे भगवान् कृष्णस्तदनुमोदताम्}
