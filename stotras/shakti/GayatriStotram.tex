% !TeX program = XeLaTeX
% !TeX root = ../../shloka.tex

\sect{गायत्रीस्तोत्रम्}
नारद उवाच
\twolineshloka
{भक्तानुकम्पिन् सर्वज्ञ हृदयं पापनाशनम्}
{गायत्र्याः कथितं तस्माद्गायत्र्याः स्तोत्रमीरय}

\twolineshloka
{आदिशक्ते जगन्मातर्भक्तानुग्रहकारिणि}
{सर्वत्र व्यापिकेऽनन्ते श्रीसन्ध्ये ते नमोऽस्तु ते}

\twolineshloka
{त्वमेव सन्ध्या गायत्री सावित्री च सरस्वती}
{ब्रह्माणी वैष्णवी रौद्री रक्तश्वेता सितेतरा}

\twolineshloka
{प्रातर्बाला च मध्याह्ने यौवनस्था भवेत्पुनः}
{वृद्धा सायं भगवती चिन्त्यते मुनिभिः सदा}

\twolineshloka
{हंसस्था गरुडारूढा तथा वृषभवाहिनी}
{ऋग्वेदाध्यायिनी भूमौ दृश्यते या तपस्विभिः}

\twolineshloka
{यजुर्वेदं पठन्ती च अन्तरिक्षे विराजते}
{या सामगाऽपि सर्वेषु भ्राम्यमाणा तथा भुवि}

\twolineshloka
{रुद्रलोकं गता त्वं हि विष्णुलोकनिवासिनी}
{त्वमेव ब्रह्मणो लोकेऽमर्त्यानुग्रहकारिणी}

\twolineshloka
{सप्तर्षिप्रीतिजननी माया बहुवरप्रदा}
{शिवयोः करनेत्रोत्था ह्यश्रुस्वेदसमुद्भवा}

\twolineshloka
{आनन्दजननी दुर्गा दशधा परिपठ्यते}
{वरेण्या वरदा चैव वरिष्ठा वरवर्णिनी}

\twolineshloka
{गरिष्ठा च वराही च वरारोहा च सप्तमी}
{नीलगङ्गा तथा सन्ध्या सर्वदा भोगमोक्षदा}

\twolineshloka
{भागीरथी मर्त्यलोके पाताले भोगवत्यपि}
{त्रिलोकवाहिनी देवी स्थानत्रयनिवासिनी}

\twolineshloka
{भूर्लोकस्था त्वमेवासि धरित्री शोकधारिणी}
{भुवो लोके वायुशक्तिः स्वर्लोके तेजसां निधिः}

\twolineshloka
{महर्लोके महासिद्धिर्जनलोकेऽजनेत्यपि}
{तपस्विनी तपोलोके सत्यलोके तु सत्यवाक्}

\twolineshloka
{कमला विष्णुलोके च गायत्री ब्रह्मलोकगा}
{रुद्रलोके स्थिता गौरी हरार्धाङ्गनिवासिनी}

\twolineshloka
{अहमो महतश्चैव प्रकृतिस्त्वं हि गीयसे}
{साम्यावस्थात्मिका त्वं हि शबलब्रह्मरूपिणी}

\twolineshloka
{ततः परा पराशक्तिः परमा त्वं हि गीयसे}
{इच्छाशक्तिः क्रियाशक्तिर्ज्ञानशक्तिस्त्रिशक्तिदा}

\twolineshloka
{गङ्गा च यमुना चैव विपाशा च सरस्वती}
{सरयू रेविका सिन्धुर्नर्मदैरावती तथा}

\twolineshloka
{गोदावरी शतद्रुश्च कावेरी देवलोकगा}
{कौशिकी चन्द्रभागा च वितस्ता च सरस्वती}

\twolineshloka
{गण्डकी तापिनी तोया गोमती वेत्रवत्यपि}
{इडा च पिङ्गला चैव सुषुम्णा च तृतीयका}

\twolineshloka
{गान्धारी हस्तजिह्वा च पूषाऽपूषा तथैव च}
{अलम्बुषा कुहूश्चैव शङ्खिनी प्राणवाहिनी}

\twolineshloka
{नाडी च त्वं शरीरस्था गीयसे प्राक्तनैर्बुधैः}
{हृत्पद्मस्था प्राणशक्तिः कण्ठस्था स्वप्ननायिका}

\twolineshloka
{तालुस्था त्वं सदाधारा बिन्दुस्था बिन्दुमालिनी}
{मूले तु कुण्डलीशक्तिर्व्यापिनी केशमूलगा}

\twolineshloka
{शिखामध्यासना त्वं हि शिखाग्रे तु मनोन्मनी}
{किमन्यद्बहुनोक्तेन यत्किञ्चिज्जगतीत्रये}

\twolineshloka
{तत्सर्वं त्वं महादेवि श्रिये सन्ध्ये नमोऽस्तु ते}
{इतीदं कीर्तितं स्तोत्रं सन्ध्यायां बहुपुण्यदम्}

\twolineshloka
{महापापप्रशमनं महासिद्धिविदायकम्}
{य इदं कीर्तयेत् स्तोत्रं सन्ध्याकाले समाहितः}

\twolineshloka
{अपुत्रः प्राप्नुयात् पुत्रं धनार्थी धनमाप्नुयात्}
{सर्वतीर्थतपोदानयज्ञयोगफलं लभेत्}

\twolineshloka
{भोगान् भुक्त्वा चिरं कालमन्ते मोक्षमवाप्नुयात्}
{तपस्विभिः कृतं स्तोत्रं स्नानकाले तु यः पठेत्}

\twolineshloka
{यत्र कुत्र जले मग्नः सन्ध्यामज्जनजं फलम्}
{लभते नात्र सन्देहः सत्यं सत्यं तु नारद}

\twolineshloka
{शृणुयाद्योऽपि तद्भक्त्या स तु पापात् प्रमुच्यते}
{पीयूषसदृशं वाक्यं सन्ध्योक्तं नारदेरितम्}

॥इति श्री गायत्री स्तोत्रं सम्पूर्णम्॥