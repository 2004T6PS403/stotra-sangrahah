% !TeX program = XeLaTeX
% !TeX root = ../../shloka.tex

\sect{कामाक्षी माहात्म्यम्}

\twolineshloka
{स्वामिपुष्करिणीतीर्थं पूर्वसिन्धुः पिनाकिनी}
{शिलाह्रदश्चतुर्मध्यं यावत् तुण्डीरमण्डलम्}

\twolineshloka
{मध्ये तुण्डीरभूवृत्तं कम्पा-वेगवती-द्वयोः}
{तयोर्मध्यं कामकोष्ठं कामाक्षी तत्र वर्तते}

\twolineshloka
{स एव विग्रहो देव्या मूलभूतोऽद्रिराड्भुवः}
{नान्योऽस्ति विग्रहो देव्याः काञ्च्यां तन्मूलविग्रहः}

\twolineshloka
{जगत्कामकलाकारं नाभिस्थानं भुवः परम्}
{पदपद्मस्य कामाक्ष्याः महापीठमुपास्महे}

\threelineshloka
{कामकोटिः स्मृतः सोऽयं कारणादेव चिन्नभः}
{यत्र कामकृतो धर्मो जन्तुना येन केन वा}
{सकृद्वाऽपि सुधर्माणां फलं फलति कोटिशः}

\twolineshloka
{यो जपेत् कामकोष्ठेऽस्मिन् मन्त्रमिष्टार्थदैवतम्}
{कोटिवर्णफलेनैव मुक्तिलोकं स गच्छति}

\twolineshloka
{यो वसेत् कामकोष्ठेऽस्मिन् क्षणार्धं वा तदर्धकम्}
{मुच्यते सर्वपापेभ्यः साक्षाद्देवी नराकृतिः}

\twolineshloka
{गायत्रीमण्डपाधारं भूनाभिस्थानमुत्तमम्}
{पुरुषार्थप्रदं शम्भोर्बिलाभ्रं तं नमाम्यहम्}

\twolineshloka
{यः कुर्यात् कामकोष्ठस्य बिलाभ्रस्य प्रदक्षिणम्}
{पदसङ्ख्याक्रमेणैव गोगर्भजननं लभेत्}

\twolineshloka
{विश्वकारणनेत्राढ्यां श्रीमत्त्रिपुरसुन्दरीम्}
{बन्धकासुरसंहन्त्रीं कामाक्षीं तामहं भजे}

\threelineshloka
{पराजन्मदिने काञ्च्यां महाभ्यन्तरमार्गतः}
{योऽर्चयेत् तत्र कामाक्षीं कोटिपूजाफलं लभेत्}
{तत्फलोत्पन्नकैवल्यं सकृत् कामाक्षिसेवया}

\twolineshloka
{त्रिस्थाननिलयं देवं त्रिविधाकारमच्युतम्}
{प्रतिलिङ्गाग्रसंयुक्तं भूतबन्धं तमाश्रये}

\twolineshloka*
{य इदं प्रातरुत्थाय स्नानकाले पठेन्नरः}
{द्वादशश्लोकमात्रेण श्लोकोक्तफलमाप्नुयात्}

॥इति श्री कामाक्षी-विलासे त्रयोविंशे अध्याये श्री कामाक्षी माहात्म्यं सम्पूर्णम्॥