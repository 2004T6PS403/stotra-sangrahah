% !TeX program = XeLaTeX
% !TeX root = ../../shloka.tex

\sect{शीतलाष्टकम्}
अस्य श्रीशीतलास्तोत्रस्य महादेव ऋषिः।\\
अनुष्टुप् छन्दः। शीतला देवता। लक्ष्मीर्बीजम्।\\
भवानी शक्तिः। सर्वविस्फोटकनिवृत्यर्थे जपे विनियोगः॥\\

\uvacha{ईश्वर उवाच}
\twolineshloka
{वन्देऽहं शीतलां देवीं रासभस्थां दिगम्बराम्}
{मार्जनीकलशोपेतां शूर्पालङ्कृतमस्तकाम्}

\twolineshloka
{वन्देऽहं शीतलां देवीं सर्वरोगभयापहाम्}
{यामासाद्य निवर्तेत विस्फोटकभयं महत्}

\twolineshloka
{शीतले शीतले चेति यो ब्रूयाद्दाहपीडितः}
{विस्फोटकभयं घोरं क्षिप्रं तस्य प्रणश्यति}

\twolineshloka
{यस्त्वामुदकमध्ये तु ध्यात्वा सम्पूजयेन्नरः}
{विस्फोटकभयं घोरं गृहे तस्य न जायते}

\twolineshloka
{शीतले ज्वरदग्धस्य पूतिगन्धयुतस्य च}
{प्रणष्टचक्षुषः पुंसस्त्वामाहुर्जीवनौषधम्}

\twolineshloka
{शीतले तनुजान् रोगान् नृणां हरसि दुस्त्यजान्}
{विस्फोटकविदीर्णानां त्वमेकाऽमृतवर्षिणी}

\twolineshloka
{गलगण्डग्रहा रोगा ये चान्ये दारुणा नृणाम्}
{त्वदनुध्यानमात्रेण शीतले यान्ति सङ्क्षयम्}

\twolineshloka
{न मन्त्रो नौषधं तस्य पापरोगस्य विद्यते}
{त्वामेकां शीतले धात्रीं नान्यां पश्यामि देवताम्}

\twolineshloka
{मृणालतन्तुसदृशीं नाभिहृन्मध्यसंस्थिताम्}
{यस्त्वां सञ्चिन्तयेद्देवि तस्य मृत्युर्न जायते}

\twolineshloka
{अष्टकं शीतलादेव्या यो नरः प्रपठेत्सदा}
{विस्फोटकभयं घोरं गृहे तस्य न जायते}

\twolineshloka
{श्रोतव्यं पठितव्यं च श्रद्धाभक्तिसमन्वितैः}
{उपसर्गविनाशाय परं स्वस्त्ययनं महत्}

\twolineshloka
{शीतले त्वं जगन्माता शीतले त्वं जगत्पिता}
{शीतले त्वं जगद्धात्री शीतलायै नमो नमः}

\twolineshloka
{रासभो गर्दभश्चैव खरो वैशाखनन्दनः}
{शीतलावाहनश्चैव दूर्वाकन्दनिकृन्तनः}

\twolineshloka
{एतानि खरनामानि शीतलाग्रे तु यः पठेत्}
{तस्य गेहे शिशूनां च शीतलारुङ् न जायते}

\twolineshloka
{शीतलाष्टकमेवेदं न देयं यस्यकस्यचित्}
{दातव्यं च सदा तस्मै श्रद्धाभक्तियुताय वै}
॥इति श्री स्कान्दपुराणे श्री~शीतलाष्टकं सम्पूर्णम्॥