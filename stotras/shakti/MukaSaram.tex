% !TeX program = XeLaTeX
% !TeX root = ../../shloka.tex
\sect{मूकसारम्}
\twolineshloka*
{श्रुतिस्मृतिपुराणानाम् आलयं करुणालयम्}
{नमामि भगवत्पादं शङ्करं लोकशङ्करम्}

\twolineshloka*
{सदाशिवसमारम्भां शङ्कराचार्यमध्यमाम्}
{अस्मदाचार्यपर्यन्तां वन्दे गुरुपरम्पराम्}



\annofourlineindentedshloka
{राकाचन्द्रसमानकान्तिवदना नाकाधिराजस्तुता}
{मूकानामपि कुर्वती सुरधुनीनीकाशवाग्वैभवम्}
{श्रीकाञ्चीनगरीविहाररसिका शोकापहन्त्री सताम्}
{एका पुण्यपरम्परा पशुपतेराकारिणी राजते}
{१-११}

\annofourlineindentedshloka
{जाता शीतलशैलतः सुकृतिनां दृश्या परं देहिनाम्}
{लोकानां क्षणमात्रसंस्मरणतः सन्तापविच्छेदिनी}
{आश्चर्यं बहु खेलनं वितनुते नैश्चल्यमाबिभ्रती}
{कम्पायास्तटसीम्नि काऽपि तटिनी कारुण्यपाथोमयी}
{१-१२}

\annofourlineindentedshloka
{परामृतझरीप्लुता जयति नित्यमन्तश्चरी}
{भुवामपि बहिश्चरी परमसंविदेकात्मिका}
{महद्भिरपरोक्षिता सततमेव काञ्चीपुरे}
{ममान्वहमहम्मतिर्मनसि भातु माहेश्वरी}
{१-९०}

\annofourlineindentedshloka
{चराचरजगन्मयीं सकलहृन्मयीं चिन्मयीम्}
{गुणत्रयमयीं जगत्त्रयमयीं त्रिधामामयीम्}
{परापरमयीं सदा दशदिशां निशाहर्मयीम्}
{परां सततसन्मयीं मनसि चिन्मयीं शीलये}
{१-९७}

\annofourlineindentedshloka
{भवाम्भोधौ नौकां जडिमविपिने पावकशिखाम्}
{अमर्त्येन्द्रादीनामधिमुकुटमुत्तंसकलिकाम्}
{जगत्तापे ज्योत्स्नामकृतकवचःपञ्जरपुटे}
{शुकस्त्रीं कामाक्ष्या मनसि कलये पादयुगलीम्}
{२-४९}

\annofourlineindentedshloka
{परा विद्या हृद्याश्रितमदनविद्या मरकत-}
{प्रभानीला लीलापरवशितशूलायुधमनाः}
{तमःपूरं दूरं चरणनतपौरन्दरपुरी-}
{मृगाक्षी कामाक्षी कमलतरलाक्षी नयतु मे}
{३-५६}

\annofourlineindentedshloka
{समरविजयकोटी साधकानन्दधाटी}
{मृदुगुणपरिपेटी मुख्यकादम्बवाटी}
{मुनिनुतपरिपाटी मोहिताजाण्डकोटी}
{परमशिववधूटी पातु मां कामकोटी}
{३-१०१}

\annofourlineindentedshloka
{यस्या वाटी हृदयकमलं कौसुमी योगभाजाम्}
{यस्याः पीठी सततशिशिरा शीकरैर्माकरन्दैः}
{यस्याः पेटी श्रुतिपरिचलन्मौलिरत्नस्य काञ्ची}
{सा मे सोमाभरणमहिषी साधयेत्काङ्क्षितानि}
{३-७७}

\annotwolineshloka
{कुण्डलि कुमारि कुटिले चण्डि चराचरसवित्रि चामुण्डे}
{गुणिनि गुहारिणि गुह्ये गुरुमूर्ते त्वां नमामि कामाक्षि}
{१-४६}

\annotwolineshloka
{अभिदाकृतिर्भिदाकृतिरचिदाकृतिरपि चिदाकृतिर्मातः}
{अनहन्ता त्वमहन्ता भ्रमयसि कामाक्षि शाश्वती विश्वम्}
{१-४७}

\annotwolineshloka
{अन्तरपि बहिरपि त्वं जन्तुततेरन्तकान्तकृदहन्ते}
{चिन्तितसन्तानवतां सन्ततमपि तन्तनीषि महिमानम्}
{१-९८}

\annofourlineindentedshloka
{गिरां दूरौ चोरौ जडिमतिमिराणां कृतजगत्}
{परित्राणौ शोणौ मुनिहृदयलीलैकनिपुणौ}
{नखैः स्मेरौ सारौ निगमवचसां खण्डितभव-}
{ग्रहोन्मादौ पादौ तव जननि कामाक्षि कलये}
{२-४४}

\annofourlineindentedshloka
{जपालक्ष्मीशोणो जनितपरमज्ञाननलिनी-}
{विकासव्यासङ्गो विफलितजगज्जाड्यगरिमा}
{मनःपूर्वाद्रिं मे तिलकयतु कामाक्षि तरसा}
{तमस्काण्डद्रोही तव चरणपाथोजरमणः}
{२-१७}

\annofourlineindentedshloka
{वरीवर्तु स्थेमा त्वयि मम गिरां देवि मनसो}
{नरीनर्तु प्रौढा वदनकमले वाक्यलहरी}
{चरीचर्तु प्रज्ञाजननि जडिमानः परजने}
{सरीसर्तु स्वैरं जननि मयि कामाक्षि करुणा}
{२-४८}

\annofourlineindentedshloka
{नीलोऽपि रागमधिकं जनयन्पुरारेः}
{लोलोऽपि भक्तिमधिकां दृढयन्नराणाम्}
{वक्रोऽपि देवि नमतां समतां वितन्वन्}
{कामाक्षि नृत्यतु मयि त्वदपाङ्गपातः}
{४-१६}

\annofourlineindentedshloka
{अत्यन्तशीतलमतन्द्रयतु क्षणार्धम्}
{अस्तोकविभ्रममनङ्गविलासकन्दम्}
{अल्पस्मितादृतमपारकृपाप्रवाहम्}
{अक्षिप्ररोहमचिरान्मयि कामकोटि}
{४-२४}

\annofourlineindentedshloka
{कैवल्यदाय करुणारसकिङ्कराय}
{कामाक्षि कन्दलितविभ्रमशङ्कराय}
{आलोकनाय तव भक्तशिवङ्कराय}
{मातर्नमोऽस्तु परतन्त्रितशङ्कराय}
{४-४७}

\annofourlineindentedshloka
{संसारघर्मपरितापजुषां नराणाम्}
{कामाक्षि शीतलतराणि तवेक्षितानि}
{चन्द्रातपन्ति घनचन्दनकर्दमन्ति}
{मुक्तागुणन्ति हिमवारिनिषेचनन्ति}
{४-७७}

\annofourlineindentedshloka
{बाणेन पुष्पधनुषः परिकल्प्यमान-}
{त्राणेन भक्तमनसां करुणाकरेण}
{कोणेन कोमलदृशस्तव कामकोटि}
{शोणेन शोषय शिवे मम शोकसिन्धुम्}
{४-९४}

\annofourlineindentedshloka
{अज्ञातभक्तिरसमप्रसरद्विवेकम्}
{अत्यन्तगर्वमनधीतसमस्तशास्त्रम्}
{अप्राप्तसत्यमसमीपगतं च मुक्तेः}
{कामाक्षि नैव तव स्पृहयति दृष्टिपातः}
{४-१००}%९९?

\annofourlineindentedshloka
{इन्धाने भववीतिहोत्रनिवहे कर्मौघचण्डानिल-}
{प्रौढिम्ना बहुलीकृते निपतितं सन्तापचिन्ताकुलम्}
{मातर्मां परिषिञ्च किञ्चिदमलैः पीयूषवर्षैरिव}
{श्रीकामाक्षि तव स्मितद्युतिकणैः शैशिर्यलीलाकरैः}
{५-९४}

\annofourlineindentedshloka
{कर्पूरैरमृतैर्जगज्जननि ते कामाक्षि चन्द्रातपैः}
{मुक्ताहारगुणैर्मृणालवलयैर्मुग्धस्मितश्रीरियम्}
{श्रीकाञ्चीपुरनायिके समतया संस्तूयते सज्जनैः}
{तत्तादृङ्मम तापशान्तिविधये किं देवि मन्दायते}
{५-२४}

\annofourlineindentedshloka
{चेतः शीतलयन्तु नः पशुपतेरानन्दजीवातवो}
{नम्राणां नयनाध्वसीमसु शरच्चन्द्रातपोपक्रमाः}
{संसाराख्यसरोरुहाकरखलीकारे तुषारोत्कराः}
{कामाक्षि स्मरकीर्तिबीजनिकरास्त्वन्मन्दहासाङ्कुराः}
{५-३१}

\annofourlineindentedshloka
{सूतिः श्वेतिमकन्दलस्य वसतिः शृङ्गारसारश्रियः}
{पूर्तिः सूक्तिझरीरसस्य लहरी कारुण्यपाथोनिधेः}
{वाटी काचन कौसुमी मधुरिमस्वाराज्यलक्ष्म्यास्तव}
{श्रीकामाक्षि ममास्तु मङ्गलकरी हासप्रभाचातुरी}
{५-८५}

\annofourlineindentedshloka
{क्रीडालोलकृपासरोरुहमुखीसौधाङ्गणेभ्यः कवि-}
{श्रेणीवाक्परिपाटिकामृतझरीसूतीगृहेभ्यः शिवे}
{निर्वाणाङ्कुरसार्वभौमपदवीसिंहासनेभ्यस्तव}
{श्रीकामाक्षि मनोज्ञमन्दहसितज्योतिष्कणेभ्यो नमः}
{५-१००}

\annofourlineindentedshloka
{कवित्वश्रीमिश्रीकरणनिपुणौ रक्षणचणौ}
{विपन्नानां श्रीमन्नलिनमसृणौ शोणकिरणौ}
{मुनीन्द्राणामन्तःकरणशरणौ मन्दसरणौ}
{मनोज्ञौ कामाक्ष्या दुरितहरणौ नौमि चरणौ}
{२-७३}

\annofourlineindentedshloka
{परस्मात्सर्वस्मादपि च परयोर्मुक्तिकरयोः}
{नखश्रीभिर्ज्योत्स्नाकलिततुलयोस्ताम्रतलयोः}
{निलीये कामाक्ष्या निगमनुतयोर्नाकिनतयोः}
{निरस्तप्रोन्मीलन्नलिनमदयोरेव पदयोः}
{२-७४}

\annofourlineindentedshloka
{रणन्मञ्जीराभ्यां ललितगमनाभ्यां सुकृतिनाम्}
{मनोवास्तव्याभ्यां मथिततिमिराभ्यां नखरुचा}
{निधेयाभ्यां पत्या निजशिरसि कामाक्षि सततम्}
{नमस्ते पादाभ्यां नलिनमृदुलाभ्यां गिरिसुते}
{२-९६}

\annofourlineindentedshloka
{यशः सूते मातर्मधुरकवितां पक्ष्मलयते}
{श्रियं दत्ते चित्ते कमपि परिपाकं प्रथयते}
{सतां पाशग्रन्थिं शिथिलयति किं किं न कुरुते}
{प्रपन्ने कामाक्ष्याः प्रणतिपरिपाटी चरणयोः}
{२-९९}

\annofourlineindentedshloka
{मनीषां माहेन्द्रीं ककुभमिव ते कामपि दशाम्}
{प्रधत्ते कामाक्ष्याश्चरणतरुणादित्यकिरणः}
{यदीये सम्पर्के धृतरसमरन्दा कवयताम्}
{परीपाकं धत्ते परिमलवती सूक्तिनलिनी}
{२-१००}

\annofourlineindentedshloka
{भुवनजननि भूषाभूतचन्द्रे नमस्ते}
{कलुषशमनि कम्पातीरगेहे नमस्ते}
{निखिलनिगमवेद्ये नित्यरूपे नमस्ते}
{परशिवमयि पाशच्छेदहस्ते नमस्ते}
{३-९९}

॥इति श्री काञ्चीजगद्गुरुणा श्री चन्द्रशेखरेन्द्रसरस्वतीस्वामिना श्री मूकमहाकविप्रणीतायाः मूकपञ्चशत्याः सङ्ग्रहीतं मूकसारं सम्पूर्णम्॥

\footnotesize{१-आर्याशतकम् \quad २-पादारविन्दशतकम् \quad ३-स्तुतिशतकम् \quad ४-कटाक्षशतकम्\quad ५-मन्दस्मितशतकम्}