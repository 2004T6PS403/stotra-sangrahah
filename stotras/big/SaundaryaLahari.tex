% !TeX program = XeLaTeX
% !TeX root = ../../shloka.tex

\sect{सौन्दर्यलहरी}
\dnsub{आनन्दलहरी}

\fourlineindentedshloka
{शिवः शक्त्या युक्तो यदि भवति शक्तः प्रभवितुम्}
{न चेदेवं देवो न खलु कुशलः स्पन्दितुमपि}
{अतस्त्वामाराध्यां हरिहरविरिञ्चादिभिरपि}
{प्रणन्तुं स्तोतुं वा कथमकृतपुण्यः प्रभवति}%1

\fourlineindentedshloka
{तनीयांसं पांसुं तव चरणपङ्केरुहभवम्}
{विरिञ्चिः सञ्चिन्वन् विरचयति लोकानविकलम्}
{वहत्येनं शौरिः कथमपि सहस्रेण शिरसाम्}
{हरः सङ्क्षुद्यैनं भजति भसितोद्धूलनविधिम्}%2

\fourlineindentedshloka
{अविद्यानामन्तस्तिमिर-मिहिरद्वीपनगरी}
{जडानां चैतन्य-स्तबक-मकरन्द-स्रुतिझरी}
{दरिद्राणां चिन्तामणिगुणनिका जन्मजलधौ}
{निमग्नानां दंष्ट्रा मुररिपु-वराहस्य भवती}%3

\fourlineindentedshloka
{त्वदन्यः पाणिभ्यामभयवरदो दैवतगणः}
{त्वमेका नैवासि प्रकटितवराभीत्यभिनया}
{भयात् त्रातुं दातुं फलमपि च वाञ्छासमधिकम्}
{शरण्ये लोकानां तव हि चरणावेव निपुणौ}%4

\fourlineindentedshloka
{हरिस्त्वामाराध्य प्रणतजनसौभाग्यजननीम्}
{पुरा नारी भूत्वा पुररिपुमपि क्षोभमनयत्}
{स्मरोऽपि त्वां नत्वा रतिनयनलेह्येन वपुषा}
{मुनीनामप्यन्तः प्रभवति हि मोहाय महताम्}%5

\fourlineindentedshloka
{धनुः पौष्पं मौर्वी मधुकरमयी पञ्च विशिखाः}
{वसन्तः सामन्तो मलयमरुदायोधनरथः}
{तथाऽप्येकः सर्वं हिमगिरिसुते कामपि कृपाम्}
{अपाङ्गात्ते लब्ध्वा जगदिदमनङ्गो विजयते}%6

\fourlineindentedshloka
{क्वणत्काञ्चीदामा करिकलभकुम्भस्तननता}
{परिक्षीणा मध्ये परिणतशरच्चन्द्रवदना}
{धनुर्बाणान् पाशं सृणिमपि दधाना करतलैः}
{पुरस्तादास्तां नः पुरमथितुराहोपुरुषिका}%7

\fourlineindentedshloka
{सुधासिन्धोर्मध्ये सुरविटपिवाटीपरिवृते}
{मणिद्वीपे नीपोपवनवति चिन्तामणिगृहे}
{शिवाकारे मञ्चे परमशिवपर्यङ्कनिलयाम्}
{भजन्ति त्वां धन्याः कतिचन चिदानन्दलहरीम्}%8

\fourlineindentedshloka
{महीं मूलाधारे कमपि मणिपूरे हुतवहम्}
{स्थितं स्वाधिष्ठाने हृदि मरुतमाकाशमुपरि}
{मनोऽपि भ्रूमध्ये सकलमपि भित्वा कुलपथम्}
{सहस्रारे पद्मे सह रहसि पत्या विहरसे}%9

\fourlineindentedshloka
{सुधाधारासारैश्चरणयुगलान्तर्विगलितैः}
{प्रपञ्चं सिञ्चन्ती पुनरपि रसाम्नायमहसः}
{अवाप्य स्वां भूमिं भुजगनिभमध्युष्टवलयम्}
{स्वमात्मानं कृत्वा स्वपिषि कुलकुण्डे कुहरिणि}%10

\fourlineindentedshloka
{चतुर्भिः श्रीकण्ठैः शिवयुवतिभिः पञ्चभिरपि}
{प्रभिन्नाभिः शम्भोर्नवभिरपि मूलप्रकृतिभिः}
{चतुश्चत्वारिंशद्वसुदलकलाश्रत्रिवलय}
{त्रिरेखाभिः सार्धं तव शरणकोणाः परिणताः}%11

\fourlineindentedshloka
{त्वदीयं सौन्दर्यं तुहिनगिरिकन्ये तुलयितुम्}
{कवीन्द्राः कल्पन्ते कथमपि विरिञ्चिप्रभृतयः}
{यदालोकौत्सुक्यादमरललना यान्ति मनसा}
{तपोभिर्दुष्प्रापामपि गिरिशसायुज्यपदवीम्}%12

\fourlineindentedshloka
{नरं वर्षीयांसं नयनविरसं नर्मसु जडम्}
{तवापाङ्गालोके पतितमनुधावन्ति शतशः}
{गलद्वेणीबन्धाः कुचकलशविस्रस्तसिचया}
{हठात् त्रुट्यत्काञ्च्यो विगलितदुकूला युवतयः}%13

\fourlineindentedshloka
{क्षितौ षट्पञ्चाशद्-द्विसमधिकपञ्चाशदुदके}
{हुताशे द्वाषष्टिश्चतुरधिकपञ्चाशदनिले}
{दिवि द्विष्षट्त्रिंशन्मनसि च चतुष्षष्टिरिति ये}
{मयूखास्तेषामप्युपरि तव पादाम्बुजयुगम्}%14

\fourlineindentedshloka
{शरज्ज्योत्स्नाशुद्धां शशियुतजटाजूटमुकुटाम्}
{वरत्रासत्राणस्फटिकघुटिकापुस्तककराम्}
{सकृन्न त्वा नत्वा कथमिव सतां सन्निदधते}
{मधुक्षीरद्राक्षामधुरिमधुरीणाः कणितयः}%15

\fourlineindentedshloka
{कवीन्द्राणां चेतःकमलवनबालातपरुचिम्}
{भजन्ते ये सन्तः कतिचिदरुणामेव भवतीम्}
{विरिञ्चिप्रेयस्यास्तरुणतरशृङ्गारलहरी}
{गभीराभिर्वाग्भिर्विदधति सतां रञ्जनममी}%16

\fourlineindentedshloka
{सवित्रीभिर्वाचां शशिमणिशिलाभङ्गरुचिभिः}
{वशिन्याद्याभिस्त्वां सह जननि सञ्चिन्तयति यः}
{स कर्ता काव्यानां भवति महतां भङ्गिरुचिभिः}
{वचोभिर्वाग्देवीवदनकमलामोदमधुरैः}%17

\fourlineindentedshloka
{तनुच्छायाभिस्ते तरुणतरणिश्रीसरणिभिः}
{दिवं सर्वामुर्वीमरुणिमनिमग्नां स्मरति यः}
{भवन्त्यस्य त्रस्यद्वनहरिणशालीननयनाः}
{सहोर्वश्या वश्याः कति कति न गीर्वाणगणिकाः}%18

\fourlineindentedshloka
{मुखं बिन्दुं कृत्वा कुचयुगमधस्तस्य तदधो}
{हरार्धं ध्यायेद्योहरमहिषि ते मन्मथकलाम्}
{स सद्यः सङ्क्षोभं नयति वनिता इत्यतिलघु}
{त्रिलोकीमप्याशु भ्रमयति रवीन्दुस्तनयुगाम्}%19

\fourlineindentedshloka
{किरन्तीमङ्गेभ्यः किरणनिकुरम्बामृतरसम्}
{हृदि त्वामाधत्ते हिमकरशिलामूर्तिमिव यः}
{स सर्पाणां दर्पं शमयति शकुन्ताधिप इव}
{ज्वरप्लुष्टान् दृष्ट्या सुखयति सुधाऽऽसारसिरया}%20

\fourlineindentedshloka
{तटिल्लेखातन्वीं तपनशशिवैश्वानरमयीम्}
{निषण्णां षण्णामप्युपरि कमलानां तव कलाम्}
{महापद्माटव्यां मृदितमलमायेन मनसा}
{महान्तः पश्यन्तो दधति परमाह्लादलहरीम्}%21

\fourlineindentedshloka
{भवानि त्वं दासे मयि वितर दृष्टिं सकरुणाम्}
{इति स्तोतुं वाञ्छन् कथयति भवानि त्वमिति यः}
{तदैव त्वं तस्मै दिशसि निजसायुज्यपदवीं}
{मुकुन्दब्रह्मेन्द्रस्फुटमुकुटनीराजितपदाम्}%22

\fourlineindentedshloka
{त्वया हृत्वा वामं वपुरपरितृप्तेन मनसा}
{शरीरार्धं शम्भोरपरमपि शङ्के हृतमभूत्}
{यदेतत्त्वद्रूपं सकलमरुणाभं त्रिनयनम्}
{कुचाभ्यामानम्रं कुटिलशशिचूडालमुकुटम्}%23

\fourlineindentedshloka
{जगत्सूते धाता हरिरवति रुद्रः क्षपयते}
{तिरस्कुर्वन्नेतत्स्वमपि वपुरीशस्तिरयति}
{सदापूर्वः सर्वं तदिदमनुगृह्णाति च शिवः}
{तवऽऽज्ञामालम्ब्य क्षणचलितयोर्भ्रूलतिकयोः}%24

\fourlineindentedshloka
{त्रयाणां देवानां त्रिगुणजनितानां तव शिवे}
{भवेत् पूजा पूजा तव चरणयोर्या विरचिता}
{तथा हि त्वत्पादोद्वहनमणिपीठस्य निकटे}
{स्थिता ह्येते शश्वन् मुकुलितकरोत्तंसमकुटाः}%25

\fourlineindentedshloka
{विरिञ्चिः पञ्चत्वं व्रजति हरिराप्नोति विरतिम्}
{विनाशं कीनाशो भजति धनदो याति निधनम्}
{वितन्द्री माहेन्द्री विततिरपि सम्मीलित-दृशां}
{महासंहारेऽस्मिन् विहरति सति त्वत्पतिरसौ}%26

\fourlineindentedshloka
{जपो जल्पः शिल्पं सकलमपि मुद्राविरचना}
{गतिः प्रादक्षिण्यक्रमणमशनाद्याहुतिविधिः}
{प्रणामः संवेशः सुखमखिलमात्मार्पणदृशा}
{सपर्यापर्यायस्तव भवतु यन्मे विलसितम्}%27

\fourlineindentedshloka
{सुधामप्यास्वाद्य प्रतिभयजरामृत्युहरिणीम्}
{विपद्यन्ते विश्वे विधिशतमखाद्या दिविषदः}
{कराळं यत्क्ष्वेळं कबलितवतः कालकलना}
{न शम्भोस्तन्मूलं तव जननि ताटङ्कमहिमा}%28

\fourlineindentedshloka
{किरीटं वैरिञ्चं परिहर पुरः कैटभभिदः}
{कठोरे कोटीरे स्खलसि जहि जम्भारिमकुटम्}
{प्रणम्रेष्वेतेषु प्रसभमुपयातस्य भवनम्}
{भवस्याभ्युत्थाने तव परिजनोक्तिर्विजयते}%29

\fourlineindentedshloka
{स्वदेहोद्भूताभिर्घृणिभिरणिमाऽऽद्याभिरभितो}
{निषेव्ये नित्ये त्वामहमिति सदा भावयति यः}
{किमाश्चर्यं तस्य त्रिनयनसमृद्धिं तृणयतो}
{महासंवर्ताग्निर्विरचयति नीराजनविधिम्}%30

\fourlineindentedshloka
{चतुष्षष्ट्या तन्त्रैः सकलमतिसन्धाय भुवनम्}
{स्थितस्तत् तत्  सिद्धिप्रसवपरतन्त्रैः पशुपतिः}
{पुनस्त्वन्निर्बन्धादखिलपुरुषार्थैकघटना}
{स्वतन्त्रं ते तन्त्रं क्षितितलमवातीतरदिदम्}%31

\fourlineindentedshloka
{शिवः शक्तिः कामः क्षितिरथ रविः शीतकिरणः}
{स्मरो हंसः शक्रस्तदनु च परामारहरयः}
{अमी हृल्लेखाभिस्तिसृभिरवसानेषु घटिता}
{भजन्ते वर्णास्ते तव जननि नामावयवताम्}%32

\fourlineindentedshloka
{स्मरं योनिं लक्ष्मीं त्रितयमिदमादौ तव मनोः}
{निधायैके नित्ये निरवधिमहाभोगरसिकाः}
{भजन्ति त्वां चिन्तामणिगुणनिबद्धाक्षवलयाः}
{शिवाऽग्नौ जुह्वन्तः सुरभिघृतधाराऽऽहुतिशतैः}%33

\fourlineindentedshloka
{शरीरं त्वं शम्भोः शशिमिहिरवक्षोरुहयुगम्}
{तवऽऽत्मानं मन्ये भगवति नवात्मानमनघम्}
{अतः शेषः शेषीत्ययमुभयसाधारणतया}
{स्थितः सम्बन्धो वां समरसपरानन्दपरयोः}%34

\fourlineindentedshloka
{मनस्त्वं व्योम त्वं मरुदसि मरुत्सारथिरसि}
{त्वमापस्त्वं भूमिस्त्वयि परिणतायां न हि परम्}
{त्वमेव स्वात्मानं परिणमयितुं विश्ववपुषा}
{चिदानन्दाकारं शिवयुवति भावेन बिभृषे}%35

\fourlineindentedshloka
{तवऽऽज्ञाचक्रस्थं तपनशशिकोटिद्युतिधरम्}
{परं शम्भुं वन्दे परिमिलितपार्श्वं परचिता}
{यमाराध्यन् भक्त्या रविशशिशुचीनामविषये}
{निरालोकेऽलोके निवसति हि भालोकभवने}%36

\fourlineindentedshloka
{विशुद्धौ ते शुद्धस्फटिकविशदं व्योमजनकम्}
{शिवं सेवे देवीमपि शिवसमानव्यवसिताम्}
{ययोः कान्त्या यान्त्या शशिकिरणसारूप्यसरणिम्}
{विधूतान्तर्ध्वान्ता विलसति चकोरीव जगती}%37

\fourlineindentedshloka
{समुन्मीलत् संवित् कमलमकरन्दैकरसिकम्}
{भजे हंसद्वन्द्वं किमपि महतां मानसचरम्}
{यदालापादष्टादशगुणितविद्यापरिणतिः}
{यदादत्ते दोषाद्-गुणमखिलमद्भ्यः पय इव}%38

\fourlineindentedshloka
{तव स्वाधिष्ठाने हुतवहमधिष्ठाय निरतम्}
{तमीडे संवर्तं जननि महतीं तां च समयाम्}
{यदालोके लोकान् दहति महति क्रोधकलिते}
{दयार्द्रा यद्दृष्टिः शिशिरमुपचारं रचयति}%39

\fourlineindentedshloka
{तटित्त्वन्तं शक्त्या तिमिरपरिपन्थिस्फुरणया}
{स्फुरन्नानारत्नाभरणपरिणद्धेन्द्रधनुषम्}
{तव श्यामं मेघं कमपि मणिपूरैकशरणं}
{निषेवे वर्षन्तं हरमिहिरतप्तं त्रिभुवनम्}%40

\fourlineindentedshloka
{तवऽऽधारे मूले सह समयया लास्यपरया}
{नवात्मानं मन्ये नवरसमहाताण्डवनटम्}
{उभाभ्यामेताभ्यामुदयविधिमुद्दिश्य दयया}
{सनाथाभ्यां जज्ञे जनकजननीमज्जगदिदम्}%41

\mbox{}\\
\dnsub{सौन्दर्यलहरी}

\fourlineindentedshloka
{गतैर्माणिक्यत्वं गगनमणिभिः सान्द्रघटितम्}
{किरीटं ते हैमं हिमगिरिसुते कीर्तयति यः}
{स नीडेयच्छायाच्छुरणशबलं चन्द्रशकलम्}
{धनुः शौनासीरं किमिति न निबध्नाति धिषणाम्}%42

\fourlineindentedshloka
{धुनोतु ध्वान्तं नस्तुलितदलितेन्दीवरवनम्}
{घनस्निग्धश्लक्ष्णं चिकुरनिकुरम्बं तव शिवे}
{यदीयं सौरभ्यं सहजमुपलब्धुं सुमनसो}
{वसन्त्यस्मिन् मन्ये वलमथनवाटीविटपिनाम्}%43

\fourlineindentedshloka
{तनोतु क्षेमं नस्तव वदनसौन्दर्यलहरी}
{परीवाहस्रोतःसरणिरिव सीमन्तसरणिः}
{वहन्ती सिन्दूरं प्रबलकबरीभारतिमिर}
{द्विषां बृन्दैर्बन्दीकृतमिव नवीनार्ककिरणम्}%44

\fourlineindentedshloka
{अरालैः स्वाभाव्यादलिकलभसश्रीभिरलकैः}
{परीतं ते वक्त्रं परिहसति पङ्केरुहरुचिम्}
{दरस्मेरे यस्मिन् दशनरुचिकिञ्जल्करुचिरे}
{सुगन्धौ माद्यन्ति स्मरदहनचक्षुर्मधुलिहः}%45

\fourlineindentedshloka
{ललाटं लावण्यद्युतिविमलमाभाति तव यत्}
{द्वितीयं तन्मन्ये मकुटघटितं चन्द्रशकलम्}
{विपर्यासन्यासादुभयमपि सम्भूय च मिथः}
{सुधालेपस्यूतिः परिणमति राकाहिमकरः}%46

\fourlineindentedshloka
{भ्रुवौ भुग्ने  किञ्चिद्भुवनभयभङ्गव्यसनिनि}
{त्वदीये नेत्राभ्यां मधुकररुचिभ्यां धृतगुणम्}
{धनुर्मन्ये सव्येतरकरगृहीतं रतिपतेः}
{प्रकोष्ठे मुष्टौ च स्थगयति निगूढान्तरमुमे}%47

\fourlineindentedshloka
{अहः सूते सव्यं तव नयनमर्कात्मकतया}
{त्रियामां वामं ते सृजति रजनीनायकतया}
{तृतीया ते दृष्टिर्दरदलितहेमाम्बुजरुचिः}
{समाधत्ते सन्ध्यां दिवसनिशयोरन्तरचरीम्}%48

\fourlineindentedshloka
{विशाला कल्याणी स्फुटरुचिरयोध्या कुवलयैः}
{कृपाधाराधारा किमपि मधुरा भोगवतिका}
{अवन्ती दृष्टिस्ते बहुनगरविस्तारविजया}
{ध्रुवं तत्तन्नामव्यवहरणयोग्या विजयते}%49

\fourlineindentedshloka
{कवीनां सन्दर्भस्तबकमकरन्दैकरसिकम्}
{कटाक्षव्याक्षेपभ्रमरकलभौ कर्णयुगलम्}
{अमुञ्चन्तौ दृष्ट्वा तव नवरसास्वादतरलौ}
{असूयासंसर्गादलिकनयनं किञ्चिदरुणम्}%50

\fourlineindentedshloka
{शिवे शृङ्गारार्द्रा तदितरजने कुत्सनपरा}
{सरोषा गङ्गायां गिरिशचरिते विस्मयवती}
{हराहिभ्यो भीता सरसिरुहसौभाग्यजननी}
{सखीषु स्मेरा ते मयि जननि दृष्टिः सकरुणा}%51

\fourlineindentedshloka
{गते कर्णाभ्यर्णं गरुत इव पक्ष्माणि दधती}
{पुरां भेत्तुश्चित्तप्रशमरसविद्रावणफले}
{इमे नेत्रे गोत्राधरपतिकुलोत्तंसकलिके}
{तवाकर्णाकृष्टस्मरशरविलासं कलयतः}%52

\fourlineindentedshloka
{विभक्तत्रैवर्ण्यं व्यतिकरितलीलाञ्जनतया}
{विभाति त्वन्नेत्रत्रितयमिदमीशानदयिते}
{पुनः स्रष्टुं देवान् द्रुहिणहरिरुद्रानुपरतान्}
{रजः सत्त्वं बिभ्रत् तम इति गुणानां त्रयमिव}%53

\fourlineindentedshloka
{पवित्रीकर्तुं नः पशुपतिपराधीनहृदये}
{दयामित्रैर्नेत्रैररुणधवलश्यामरुचिभिः}
{नदः शोणो गङ्गा तपनतनयेति ध्रुवममुम्}
{त्रयाणां तीर्थानामुपनयसि सम्भेदमनघम्}%54

\fourlineindentedshloka
{निमेषोन्मेषाभ्यां प्रलयमुदयं याति जगती}
{तवेत्याहुः सन्तो धरणिधरराजन्यतनये}
{त्वदुन्मेषाज्जातं जगदिदमशेषं प्रलयतः}
{परित्रातुं शङ्के परिहृतनिमेषास्तव दृशः}%55

\fourlineindentedshloka
{तवापर्णे कर्णेजपनयनपैशुन्यचकिता}
{निलीयन्ते तोये नियतमनिमेषाः शफरिकाः}
{इयं च श्रीर्बद्धच्छदपुटकवाटं कुवलयम्}
{जहाति प्रत्यूषे निशि च विघटय्य प्रविशति}%56

\fourlineindentedshloka
{दृशा द्राघीयस्या दरदलितनीलोत्पलरुचा}
{दवीयांसं दीनं स्नपय कृपया मामपि शिवे}
{अनेनायं धन्यो भवति न च ते हानिरियता}
{वने वा हर्म्ये वा समकरनिपातो हिमकरः}%57

\fourlineindentedshloka
{अरालं ते पालीयुगलमगराजन्यतनये}
{न केषामाधत्ते कुसुमशरकोदण्डकुतुकम्}
{तिरश्चीनो यत्र श्रवणपथमुल्लङ्घ्य विलसन्}
{अपाङ्गव्यासङ्गो दिशति शरसन्धानधिषणाम्}%58

\fourlineindentedshloka
{स्फुरद्गण्डाभोगप्रतिफलितताटङ्कयुगलम्}
{चतुश्चक्रं मन्ये तव मुखमिदं मन्मथरथम्}
{यमारुह्य द्रुह्यत्यवनिरथम् अर्केन्दुचरणं}
{महावीरो मारः प्रमथपतये सज्जितवते}%59

\fourlineindentedshloka
{सरस्वत्याः सूक्तीरमृतलहरीकौशलहरीः}
{पिबन्त्याः शर्वाणि श्रवणचुलुकाभ्यामविरलम्}
{चमत्कारश्लाघाचलितशिरसः कुण्डलगणो}
{झणत्कारैस्तारैः प्रतिवचनमाचष्ट इव ते}%60

\fourlineindentedshloka
{असौ नासावंशस्तुहिनगिरिवंशध्वजपटि}
{त्वदीयो नेदीयः फलतु फलमस्माकमुचितम्}
{वहत्यन्तर्मुक्ताः शिशिरकरनिश्वासगलितम्}
{समृद्ध्या यस्तासां बहिरपि च मुक्तामणिधरः}%61

\fourlineindentedshloka
{प्रकृत्याऽऽरक्तायास्तव सुदति दन्तच्छदरुचेः}
{प्रवक्ष्ये सादृश्यं जनयतु फलं विद्रुमलता}
{न बिम्बं त्वद्बिम्बप्रतिफलनरागादरुणितम्}
{तुलामध्यारोढुं कथमिव विलज्जेत कलया}%62

\fourlineindentedshloka
{स्मितज्योत्स्नाजालं तव वदनचन्द्रस्य पिबताम्}
{चकोराणामासीदतिरसतया चञ्चुजडिमा}
{अतस्ते शीतांशोरमृतलहरीमाम्लरुचयः}
{पिबन्ति स्वच्छन्दं निशि निशि भृशं काञ्जिकधिया}%63

\fourlineindentedshloka
{अविश्रान्तं पत्युर्गुणगणकथाऽऽम्रेडनजपा}
{जपापुष्पच्छाया तव जननि जिह्वा जयति सा}
{यदग्रासीनायाः स्फटिकदृषदच्छच्छविमयी}
{सरस्वत्या मूर्तिः परिणमति माणिक्यवपुषा}%64

\fourlineindentedshloka
{रणे जित्वा दैत्यानपहृतशिरस्त्रैः कवचिभिः}
{निवृत्तैश्चण्डांशत्रिपुरहरनिर्माल्यविमुखैः}
{विशाखेन्द्रोपेन्द्रैः शशिविशदकर्पूरशकला}
{विलीयन्ते मातस्तव वदनताम्बूलकबलाः}%65

\fourlineindentedshloka
{विपञ्च्या गायन्ती विविधमपदानं पशुपतेः}
{त्वयाऽऽरब्धे वक्तुं चलितशिरसा साधुवचने}
{तदीयैर्माधुर्यैरपलपिततन्त्रीकलरवाम्}
{निजां वीणां वाणी निचुलयति चोलेन निभृतम्}%66

\fourlineindentedshloka
{कराग्रेण स्पृष्टं तुहिनगिरिणा वत्सलतया}
{गिरीशेनोदस्तं मुहुरधरपानाकुलतया}
{करग्राह्यं शम्भोर्मुखमुकुरवृन्तं गिरिसुते}
{कथङ्कारं ब्रूमस्तव चुबुकमौपम्यरहितम्}%67

\fourlineindentedshloka
{भुजाश्लेषान् नित्यं पुरदमयितुः कण्टकवती}
{तव ग्रीवा धत्ते मुखकमलनालश्रियमियम्}
{स्वतः श्वेता कालागरुबहुलजम्बालमलिना}
{मृणालीलालित्यम् वहति यदधो हारलतिका}%68

\fourlineindentedshloka
{गले रेखास्तिस्रो गतिगमकगीतैकनिपुणे}
{विवाहव्यानद्धप्रगुणगुणसङ्ख्याप्रतिभुवः}
{विराजन्ते नानाविधमधुररागाकरभुवाम्}
{त्रयाणां ग्रामाणां स्थितिनियमसीमान इव ते}%69

\fourlineindentedshloka
{मृणालीमृद्वीनां तव भुजलतानां चतसृणाम्}
{चतुर्भिः सौन्दर्यं सरसिजभवः स्तौति वदनैः}
{नखेभ्यः सन्त्रस्यन् प्रथममथनादन्धकरिपोः}
{चतुर्णां शीर्षाणां सममभयहस्तार्पणधिया}%70

\fourlineindentedshloka
{नखानामुद्द्योतैर्नवनलिनरागं विहसताम्}
{कराणां ते कान्तिं कथय कथयामः कथमुमे}
{कयाचिद्वा साम्यं भजतु कलया हन्त कमलम्}
{यदि क्रीडल्लक्ष्मीचरणतललाक्षारसचणम्}%71

\fourlineindentedshloka
{समं देवि स्कन्दद्विपवदनपीतं स्तनयुगम्}
{तवेदं नः खेदं हरतु सततं प्रस्नुतमुखम्}
{यदालोक्याशङ्काऽऽकुलितहृदयो हासजनकः}
{स्वकुम्भौ हेरम्बः परिमृशति हस्तेन झटिति}%72

\fourlineindentedshloka
{अमू ते वक्षोजावमृतरसमाणिक्यकुतुपौ}
{न सन्देहस्पन्दो नगपतिपताके मनसि नः}
{पिबन्तौ तौ यस्मादविदितवधूसङ्गरसिकौ}
{कुमारावद्यापि द्विरदवदनक्रौञ्चदलनौ}%73

\fourlineindentedshloka
{वहत्यम्ब स्तम्बेरमदनुजकुम्भप्रकृतिभिः}
{समारब्धां मुक्तामणिभिरमलां हारलतिकाम्}
{कुचाभोगो बिम्बाधररुचिभिरन्तः शबलिताम्}
{प्रतापव्यामिश्रां पुरदमयितुः कीर्तिमिव ते}%74

\fourlineindentedshloka
{तव स्तन्यं मन्ये धरणिधरकन्ये हृदयतः}
{पयःपारावारः परिवहति सारस्वतमिव}
{दयावत्या दत्तं द्रविडशिशुरास्वाद्य तव यत्}
{कवीनां प्रौढानामजनि कमनीयः कवयिता}%75

\fourlineindentedshloka
{हरक्रोधज्वालाऽऽवलिभिरवलीढेन वपुषा}
{गभीरे ते नाभीसरसि कृतसङ्गो मनसिजः}
{समुत्तस्थौ तस्मादचलतनये धूमलतिका}
{जनस्तां जानीते तव जननि रोमावलिरिति}%76

\fourlineindentedshloka
{यदेतत् कालिन्दीतनुतरतरङ्गाकृति शिवे}
{कृशे मध्ये किञ्चिज्जननि तव तद्भाति सुधियाम्}
{विमर्दादन्योऽन्यं कुचकलशयोरन्तरगतम्}
{तनूभूतं व्योम प्रविशदिव नाभिं कुहरिणीम्}%77

\fourlineindentedshloka
{स्थिरो गङ्गावर्तः स्तनमुकुलरोमावलिलता}
{निजावालं कुण्डं कुसुमशरतेजोहुतभुजः}
{रतेर्लीलागारं किमपि तव नाभिर्गिरिसुते}
{बिलद्वारं सिद्धेर्गिरिशनयनानां विजयते}%78

\fourlineindentedshloka
{निसर्गक्षीणस्य स्तनतटभरेण क्लमजुषो}
{नमन्मूर्तेर्नारीतिलक शनकैस्त्रुट्यत इव}
{चिरं ते मध्यस्य त्रुटिततटिनीतीरतरुणा}
{समावस्थास्थेम्नो भवतु कुशलं शैलतनये}%79

\fourlineindentedshloka
{कुचौ सद्यःस्विद्यत्तटघटितकूर्पासभिदुरौ}
{कषन्तौ दोर्मूले कनककलशाभौ कलयता}
{तव त्रातुं भङ्गादलमिति वलग्नं तनुभुवा}
{त्रिधा नद्धं देवि त्रिवलि लवलीवल्लिभिरिव}%80

\fourlineindentedshloka
{गुरुत्वं विस्तारं क्षितिधरपतिः पार्वति निजात्}
{नितम्बादाच्छिद्य त्वयि हरणरूपेण निदधे}
{अतस्ते विस्तीर्णो गुरुरयमशेषां वसुमतीम्}
{नितम्बप्राग्भारः स्थगयति लघुत्वं नयति च}%81

\fourlineindentedshloka
{करीन्द्राणां शुण्डान् कनककदलीकाण्डपटलीम्}
{उभाभ्यामूरुभ्यामुभयमपि निर्जित्य भवति}
{सुवृत्ताभ्यां पत्युः प्रणतिकठिनाभ्यां गिरिसुते}
{विधिज्ञे जानुभ्यां विबुधकरिकुम्भद्वयमसि}%82

\fourlineindentedshloka
{पराजेतुं रुद्रं द्विगुणशरगर्भौ गिरिसुते}
{निषङ्गौ जङ्घे ते विषमविशिखो बाढमकृत}
{यदग्रे दृश्यन्ते दश शरफलाः पादयुगली}
{नखाग्रच्छद्मानः सुरमकुटशाणैकनिशिताः}%83

\fourlineindentedshloka
{श्रुतीनां मूर्धानो दधति तव यौ शेखरतया}
{ममाप्येतौ मातः शिरसि दयया धेहि चरणौ}
{ययोः पाद्यं पाथः पशुपतिजटाजूटतटिनी}
{ययोर्लाक्षालक्ष्मीररुणहरिचूडामणिरुचिः}%84

\fourlineindentedshloka
{नमोवाकं ब्रूमो नयनरमणीयाय पदयोः}
{तवास्मै द्वन्द्वाय स्फुटरुचिरसालक्तकवते}
{असूयत्यत्यन्तं यदभिहननाय स्पृहयते}
{पशूनामीशानः प्रमदवनकङ्केलितरवे}%85

\fourlineindentedshloka
{मृषा कृत्वा गोत्रस्खलनमथ वैलक्ष्यनमितम्}
{ललाटे भर्तारं चरणकमले ताडयति ते}
{चिरादन्तःशल्यं दहनकृतमुन्मूलितवता}
{तुलाकोटिक्वाणैः किलिकिलितमीशानरिपुणा}%86

\fourlineindentedshloka
{हिमानीहन्तव्यं हिमगिरिनिवासैकचतुरौ}
{निशायां निद्राणां निशि चरमभागे च विशदौ}
{वरं लक्ष्मीपात्रं श्रियमतिसृजन्तौ समयिनाम्}
{सरोजं त्वत्पादौ जननि जयतश्चित्रमिह किम्}%87

\fourlineindentedshloka
{पदं ते कीर्तीनां प्रपदमपदं देवि विपदाम्}
{कथं नीतं सद्भिः कठिनकमठीकर्परतुलाम्}
{कथं वा बाहुभ्यामुपयमनकाले पुरभिदा}
{यदादाय न्यस्तं दृषदि दयमानेन मनसा}%88

\fourlineindentedshloka
{नखैर्नाकस्त्रीणां करकमलसङ्कोचशशिभिः}
{तरूणां दिव्यानां हसत इव ते चण्डि चरणौ}
{फलानि स्वःस्थेभ्यः किसलयकराग्रेण ददतां}
{दरिद्रेभ्यो भद्रां श्रियमनिशमह्नाय ददतौ}%89

\fourlineindentedshloka
{ददाने दीनेभ्यः श्रियमनिशमाशानुसदृशीम्}
{अमन्दं सौन्दर्यप्रकरमकरन्दं विकिरति}
{तवास्मिन् मन्दारस्तबकसुभगे यातु चरणे}
{निमज्जन् मज्जीवः करणचरणः षट्चरणताम्}%90

\fourlineindentedshloka
{पदन्यासक्रीडापरिचयमिवारब्धुमनसः}
{स्खलन्तस्ते खेलं भवनकलहंसा न जहति}
{अतस्तेषां शिक्षां सुभगमणिमञ्जीररणित-}
{च्छलादाचक्षाणं चरणकमलं चारुचरिते}%91

\fourlineindentedshloka
{गतास्ते मञ्चत्वं द्रुहिणहरिरुद्रेश्वरभृतः}
{शिवः स्वच्छच्छायाघटितकपटप्रच्छदपटः}
{त्वदीयानां भासां प्रतिफलनरागारुणतया}
{शरीरी शृङ्गारो रस इव दृशां दोग्धि कुतुकम्}%92

\fourlineindentedshloka
{अराला केशेषु प्रकृतिसरला मन्दहसिते}
{शिरीषाभा चित्ते दृषदुपलशोभा कुचतटे}
{भृशं तन्वी मध्ये पृथुरुरसिजारोहविषये}
{जगत्त्रातुं शम्भोर्जयति करुणा काचिदरुणा}%93

\fourlineindentedshloka
{कलङ्कः कस्तूरी रजनिकरबिम्बं जलमयम्}
{कलाभिः कर्पूरैर्मरकतकरण्डं निबिडितम्}
{अतस्त्वद्भोगेन प्रतिदिनमिदं रिक्तकुहरं}
{विधिर्भूयो भूयो निबिडयति नूनं तव कृते}%94

\fourlineindentedshloka
{पुरारातेरन्तःपुरमसि ततस्त्वच्चरणयोः}
{सपर्यामर्यादा तरलकरणानामसुलभा}
{तथा ह्येते नीताः शतमखमुखाः सिद्धिमतुलाम्}
{तव द्वारोपान्तस्थितिभिरणिमाद्याभिरमराः}%95

\fourlineindentedshloka
{कलत्रं वैधात्रं कति कति भजन्ते न कवयः}
{श्रियो देव्याः को वा न भवति पतिः कैरपि धनैः}
{महादेवं हित्वा तव सति सतीनामचरमे}
{कुचाभ्यामासङ्गः कुरवकतरोरप्यसुलभः}%96

\fourlineindentedshloka
{गिरामाहुर्देवीं द्रुहिणगृहिणीमागमविदो}
{हरेः पत्नीं पद्मां हरसहचरीमद्रितनयाम्}
{तुरीया काऽपि त्वं दुरधिगमनिःसीममहिमा}
{महामाया विश्वं भ्रमयसि परब्रह्ममहिषि}%97

\fourlineindentedshloka
{कदा काले मातः कथय कलितालक्तकरसम्}
{पिबेयं विद्यार्थी तव चरणनिर्णेजनजलम्}
{प्रकृत्या मूकानामपि च कविताकारणतया}
{कदा धत्ते वाणीमुखकमलताम्बूलरसताम्}%98

\fourlineindentedshloka
{सरस्वत्या लक्ष्म्या विधिहरिसपत्नो विहरते}
{रतेः पातिव्रत्यं शिथिलयति रम्येण वपुषा}
{चिरं जीवन्नेव क्षपितपशुपाशव्यतिकरः}
{परानन्दाभिख्यं रसयति रसं त्वद्भजनवान्}%99

\fourlineindentedshloka
{समानीतः पद्‌भ्यां मणिमुकुरतामम्बरमणिः}
{भयादास्यादन्तःस्तिमितकिरणश्रेणिमसृणः}
{दधाति त्वद्वक्त्रप्रतिफलनमश्रान्तविकचम्}
{निरातङ्कं चन्द्रान्निजहृदयपङ्केरुहमिव}%100

\fourlineindentedshloka
{समुद्भूतस्थूलस्तनभरमुरश्चारु हसितम्}
{कटाक्षे कन्दर्पः कतिचन कदम्बद्युति वपुः}
{हरस्य त्वद्भ्रान्तिं मनसि जनयाम् स्म विमला}
{भवत्या ये भक्ताः परिणतिरमीषामियमुमे}%101

\fourlineindentedshloka
{निधे नित्यस्मेरे निरवधिगुणे नीतिनिपुणे}
{निराघाटज्ञाने नियमपरचित्तैकनिलये}
{नियत्या निर्मुक्ते निखिलनिगमान्तस्तुतपदे}
{निरातङ्के नित्ये निगमय ममापि स्तुतिमिमाम्}%102

\fourlineindentedshloka
{प्रदीपज्वालाभिर्दिवसकरनीराजनविधिः}
{सुधासूतेश्चन्द्रोपलजललवैरर्घ्यरचना}
{स्वकीयैरम्भोभिः सलिलनिधिसौहित्यकरणम्}
{त्वदीयाभिर्वाग्भिस्तव जननि वाचां स्तुतिरियम्}%103

{॥ इति श्रीमच्छङ्कराचार्यविरचिता सौन्दर्यलहरी सम्पूर्णा॥}