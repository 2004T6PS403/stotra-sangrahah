% !TeX program = XeLaTeX
% !TeX root = ../../shloka.tex

\sect{सङ्क्षेपरामायणम्}

\twolineshloka*
{शुक्लाम्बरधरं विष्णुं शशिवर्णं चतुर्भुजम्}
{प्रसन्नवदनं ध्यायेत् सर्वविघ्नोपशान्तये}

\twolineshloka*
{वागीशाद्याः सुमनसः सर्वार्थानामुपक्रमे}
{यं नत्वा कृतकृत्याः स्युस्तं नमामि गजाननम्}

\dnsub{श्री~गुरु प्रार्थना}

\twolineshloka*
{गुरुर्ब्रह्मा गुरुर्विष्णुर्गुरुर्देवो महेश्वरः}
{गुरुः साक्षात् परं ब्रह्म तस्मै श्री गुरवे नमः}

\twolineshloka*
{सदाशिवसमारम्भां शङ्कराचार्यमध्यमाम्}
{अस्मदाचार्यपर्यन्तां वन्दे गुरुपरम्पराम्}

\twolineshloka*
{अखण्डमण्डलाकारं व्याप्तं येन चराचरम्}
{तत्पदं दर्शितं येन तस्मै श्री गुरवे नमः}

\dnsub{श्री~सरस्वती प्रार्थना}
\fourlineindentedshloka*
{दोर्भिर्युक्ता चतुर्भिः स्फटिकमणिनिभैरक्षमालां दधाना}
{हस्तेनैकेन पद्मं सितमपि च शुकं पुस्तकं चापरेण}
{भासा कुन्देन्दुशङ्खस्फटिकमणिनिभा भासमानाऽसमाना}
{सा मे वाग्देवतेयं निवसतु वदने सर्वदा सुप्रसन्ना}

\mbox{}\\
\dnsub{श्री~वाल्मीकि नमस्क्रिया}

\twolineshloka
{कूजन्तं राम रामेति मधुरं मधुराक्षरम्}
{आरुह्य कविताशाखां वन्दे वाल्मीकिकोकिलम्}

\twolineshloka
{वाल्मीकेर्मुनिसिंहस्य कवितावनचारिणः}
{शृण्वन् रामकथानादं को न याति परां गतिम्}

\twolineshloka
{यः पिबन् सततं रामचरितामृतसागरम्}
{अतृप्तस्तं मुनिं वन्दे प्राचेतसमकल्मषम्}

\resetShloka
\dnsub{श्री~हनुमन्नमस्क्रिया}

\twolineshloka
{गोष्पदीकृत-वाराशिं मशकीकृत-राक्षसम्}
{रामायण-महामाला-रत्नं वन्देऽनिलात्मजम्}

\twolineshloka
{अञ्जनानन्दनं वीरं जानकीशोकनाशनम्}
{कपीशमक्षहन्तारं वन्दे लङ्काभयङ्करम्}

\twolineshloka
{उल्लङ्घ्य सिन्धोः सलिलं सलीलं यः शोकवह्निं जनकात्मजायाः}
{आदाय तेनैव ददाह लङ्कां नमामि तं प्राञ्जलिराञ्जनेयम्}

\twolineshloka
{आञ्जनेयमतिपाटलाननं काञ्चनाद्रि-कमनीय-विग्रहम्}
{पारिजात-तरुमूल-वासिनं भावयामि पवमान-नन्दनम्}

\twolineshloka
{यत्र यत्र रघुनाथकीर्तनं तत्र तत्र कृतमस्तकाञ्जलिम्}
{बाष्पवारिपरिपूर्णलोचनं मारुतिं नमत राक्षसान्तकम्}

\twolineshloka
{मनोजवं मारुततुल्यवेगं जितेन्द्रियं बुद्धिमतां वरिष्ठम्}
{वातात्मजं वानरयूथमुख्यं श्रीरामदूतं शिरसा नमामि}

\mbox{}\\
\resetShloka
\dnsub{श्री~रामायणप्रार्थना}

\fourlineindentedshloka
{यः कर्णाञ्जलिसम्पुटैरहरहः सम्यक् पिबत्यादरात्}
{वाल्मीकेर्वदनारविन्दगलितं रामायणाख्यं मधु}
{जन्म-व्याधि-जरा-विपत्ति-मरणैरत्यन्त-सोपद्रवम्}
{संसारं स विहाय गच्छति पुमान् विष्णोः पदं शाश्वतम्}

\twolineshloka
{तदुपगत-समास-सन्धियोगं सममधुरोपनतार्थ-वाक्यबद्धम्}
{रघुवरचरितं मुनिप्रणीतं दशशिरसश्च वधं निशामयध्वम्}

\twolineshloka
{वाल्मीकि-गिरिसम्भूता रामसागरगामिनी}
{पुनातु भुवनं पुण्या रामायणमहानदी}

\twolineshloka
{श्लोकसारजलाकीर्णं सर्गकल्लोलसङ्कुलम्}
{काण्डग्राहमहामीनं वन्दे रामायणार्णवम्}

\twolineshloka
{वेदवेद्ये परे पुंसि जाते दशरथात्मजे}
{वेदः प्राचेतसादासीत् साक्षाद्रामायणात्मना}

\mbox{}\\
\resetShloka
\dnsub{श्री~रामध्यानम्}

\fourlineindentedshloka
{वैदेहीसहितं सुरद्रुमतले हैमे महामण्डपे}
{मध्ये पुष्पकमासने मणिमये वीरासने सुस्थितम्}
{अग्रे वाचयति प्रभञ्जनसुते तत्त्वं मुनिभ्यः परम्}
{व्याख्यान्तं भरतादिभिः परिवृतं रामं भजे श्यामलम्}

\fourlineindentedshloka
{वामे भूमिसुता पुरश्च हनुमान् पश्चात् सुमित्रासुतः}
{शत्रुघ्नो भरतश्च पार्श्वदलयोर्वाय्वादिकोणेषु च}
{सुग्रीवश्च विभीषणश्च युवराट् तारासुतो जाम्बवान्}
{मध्ये नीलसरोजकोमलरुचिं रामं भजे श्यामलम्}

\twolineshloka
{रामं रामानुजं सीतां भरतं भरतानुजम्}
{सुग्रीवं वायुसूनुं च प्रणमामि पुनः पुनः}

\twolineshloka
{नमोऽस्तु रामाय सलक्ष्मणाय देव्यै च तस्यै जनकात्मजायै}
{नमोऽस्तु रुद्रेन्द्रयमानिलेभ्यो नमोऽस्तु चन्द्रार्कमरुद्गणेभ्यः}

\centerline{\textbf{ॐ श्री गुरुभ्यो नमः।}}


\clearpage
\resetShloka
\dnsub{श्रीमद्रामायणम्}
\dnsub{बालकाण्डः}
\dnsub{अथ प्रथमोऽध्यायः}

\twolineshloka
{तपः स्वाध्यायनिरतं तपस्वी वाग्विदां वरम्}
{नारदं परिपप्रच्छ वाल्मीकिर्मुनिपुङ्गवम्}%1

\twolineshloka
{को न्वस्मिन् साम्प्रतं लोके गुणवान् कश्च वीर्यवान्}
{धर्मज्ञश्च कृतज्ञश्च सत्यवाक्यो दृढव्रतः}%2

\twolineshloka
{चारित्रेण च को युक्तः सर्वभूतेषु को हितः}
{विद्वान् कः कः समर्थश्च कश्चैकप्रियदर्शनः}%3

\twolineshloka
{आत्मवान् को जितक्रोधो मतिमान् कोऽनसूयकः}
{कस्य बिभ्यति देवाश्च जातरोषस्य संयुगे}%4

\twolineshloka
{एतदिच्छाम्यहं श्रोतुं परं कौतूहलं हि मे}
{महर्षे त्वं समर्थोऽसि ज्ञातुमेवंविधं नरम्}%5

\twolineshloka
{श्रुत्वा चैतत्त्रिलोकज्ञो वाल्मीकेर्नारदो वचः}
{श्रूयतामिति चाऽऽमन्त्र्य प्रहृष्टो वाक्यमब्रवीत्}%6

\twolineshloka
{बहवो दुर्लभाश्चैव ये त्वया कीर्तिता गुणाः}
{मुने वक्ष्याम्यहं बुद्‌ध्वा तैर्युक्तः श्रूयतां नरः}%7

\twolineshloka
{इक्ष्वाकुवंशप्रभवो रामो नाम जनैः श्रुतः}
{नियतात्मा महावीर्यो द्युतिमान् धृतिमान् वशी}%8

\twolineshloka
{बुद्धिमान् नीतिमान् वाग्मी श्रीमाञ्छत्रुनिबर्हणः}
{विपुलांसो महाबाहुः कम्बुग्रीवो महाहनुः}%9

\twolineshloka
{महोरस्को महेष्वासो गूढजत्रुररिन्दमः}
{आजानुबाहुः सुशिराः सुललाटः सुविक्रमः}%10

\twolineshloka
{समः समविभक्ताङ्गः स्निग्धवर्णः प्रतापवान्}
{पीनवक्षा विशालाक्षो लक्ष्मीवाञ्छुभलक्षणः}%11

\twolineshloka
{धर्मज्ञः सत्यसन्धश्च प्रजानां च हिते रतः}
{यशस्वी ज्ञानसम्पन्नः शुचिर्वश्यः समाधिमान्}%12

\twolineshloka
{प्रजापतिसमः श्रीमान् धाता रिपुनिषूदनः}
{रक्षिता जीवलोकस्य धर्मस्य परिरक्षिता}%13

\twolineshloka
{रक्षिता स्वस्य धर्मस्य स्वजनस्य च रक्षिता}
{वेदवेदाङ्गतत्त्वज्ञो धनुर्वेदे च निष्ठितः}%14

\twolineshloka
{सर्वशास्त्रार्थतत्त्वज्ञो स्मृतिमान् प्रतिभानवान्}
{सर्वलोकप्रियः साधुरदीनात्मा विचक्षणः}%15

\twolineshloka
{सर्वदाऽभिगतः सद्भिः समुद्र इव सिन्धुभिः}
{आर्यः सर्वसमश्चैव सदैव प्रियदर्शनः}%16

\twolineshloka
{स च सर्वगुणोपेतः कौसल्यानन्दवर्धनः}
{समुद्र इव गाम्भीर्ये धैर्येण हिमवानिव}%17

\twolineshloka
{विष्णुना सदृशो वीर्ये सोमवत् प्रियदर्शनः}
{कालाग्निसदृशः क्रोधे क्षमया पृथिवीसमः}%18

\twolineshloka
{धनदेन समस्त्यागे सत्ये धर्म इवापरः}
{तमेवङ्गुणसम्पन्नं रामं सत्यपराक्रमम्}%19

\twolineshloka
{ज्येष्ठं श्रेष्ठगुणैर्युक्तं प्रियं दशरथः सुतम्}
{प्रकृतीनां हितैर्युक्तं प्रकृतिप्रियकाम्यया}%20

\twolineshloka
{यौवराज्येन संयोक्तुम् ऐच्छत् प्रीत्या महीपतिः}
{तस्याभिषेकसम्भारान् दृष्ट्वा भार्याऽथ कैकयी}%21

\twolineshloka
{पूर्वं दत्तवरा देवी वरमेनमयाचत}
{विवासनं च रामस्य भरतस्याभिषेचनम्}%22

\twolineshloka
{स सत्यवचनाद्राजा धर्मपाशेन संयतः}
{विवासयामास सुतं रामं दशरथः प्रियम्}%23

\twolineshloka
{स जगाम वनं वीरः प्रतिज्ञामनुपालयन्}
{पितुर्वचननिर्देशात् कैकेय्याः प्रियकारणात्}%24

\twolineshloka
{तं व्रजन्तं प्रियो भ्राता लक्ष्मणोऽनुजगाम ह}
{स्नेहाद्विनयसम्पन्नः सुमित्रानन्दवर्धनः}%25

\twolineshloka
{भ्रातरं दयितो भ्रातुः सौभ्रात्रमनुदर्शयन्}
{रामस्य दयिता भार्या नित्यं प्राणसमा हिता}%26

\twolineshloka
{जनकस्य कुले जाता देवमायेव निर्मिता}
{सर्वलक्षणसम्पन्ना नारीणामुत्तमा वधूः}%27

\twolineshloka
{सीताऽप्यनुगता रामं शशिनं रोहिणी यथा}
{पौरैरनुगतो दूरं पित्रा दशरथेन च}%28

\twolineshloka
{शृङ्गवेरपुरे सूतं गङ्गाकूले व्यसर्जयत्}
{गुहमासाद्य धर्मात्मा निषादाधिपतिं प्रियम्}%29

\twolineshloka
{गुहेन सहितो रामो लक्ष्मणेन च सीतया}
{ते वनेन वनं गत्वा नदीस्तीर्त्वा बहूदकाः}%30

\twolineshloka
{चित्रकूटमनुप्राप्य भरद्वाजस्य शासनात्}
{रम्यमावसथं कृत्वा रममाणा वने त्रयः}%31

\twolineshloka
{देवगन्धर्वसङ्काशास्तत्र ते न्यवसन् सुखम्}
{चित्रकूटं गते रामे पुत्रशोकातुरस्तथा}%32

\twolineshloka
{राजा दशरथः स्वर्गं जगाम विलपन् सुतम्}
{मृते तु तस्मिन् भरतो वसिष्ठप्रमुखैर्द्विजैः}%33

\twolineshloka
{नियुज्यमानो राज्याय नैच्छद्राज्यं महाबलः}
{स जगाम वनं वीरो रामपादप्रसादकः}%34

\twolineshloka
{गत्वा तु स महात्मानं रामं सत्यपराक्रमम्}
{अयाचद्भ्रातरं रामम् आर्यभावपुरस्कृतः}%35

\twolineshloka
{त्वमेव राजा धर्मज्ञ इति रामं वचोऽब्रवीत्}
{रामोऽपि परमोदारः सुमुखः सुमहायशाः}%36

\twolineshloka
{न चेच्छत् पितुरादेशाद्राज्यं रामो महाबलः}
{पादुके चास्य राज्याय न्यासं दत्त्वा पुनः पुनः}%37

\twolineshloka
{निवर्तयामास ततो भरतं भरताग्रजः}
{स काममनवाप्यैव रामपादावुपस्पृशन्}%38

\twolineshloka
{नन्दिग्रामेऽकरोद्राज्यं रामागमनकाङ्क्षया}
{गते तु भरते श्रीमान् सत्यसन्धो जितेन्द्रियः}%39

\twolineshloka
{रामस्तु पुनरालक्ष्य नागरस्य जनस्य च}
{तत्राऽऽगमनमेकाग्रो दण्डकान् प्रविवेश ह}%40

\twolineshloka
{प्रविश्य तु महारण्यं रामो राजीवलोचनः}
{विराधं राक्षसं हत्वा शरभङ्गं ददर्श ह}%41

\twolineshloka
{सुतीक्ष्णं चाप्यगस्त्यं च अगस्त्यभ्रातरं तथा}
{अगस्त्यवचनाच्चैव जग्राहैन्द्रं शरासनम्}%42

\twolineshloka
{खड्गं च परमप्रीतस्तूणी चाक्षयसायकौ}
{वसतस्तस्य रामस्य वने वनचरैः सह}%43

\twolineshloka
{ऋषयोऽभ्यागमन् सर्वे वधायासुररक्षसाम्}
{स तेषां प्रति शुश्राव राक्षसानां तदा वने}%44

\twolineshloka
{प्रतिज्ञातश्च रामेण वधः संयति रक्षसाम्}
{ऋषीणामग्निकल्पानां दण्डकारण्यवासिनाम्}%45

\twolineshloka
{तेन तत्रैव वसता जनस्थाननिवासिनी}
{विरूपिता शूर्पणखा राक्षसी कामरूपिणी}%46

\twolineshloka
{ततः शूर्पणखावाक्यादुद्युक्तान् सर्वराक्षसान्}
{खरं त्रिशिरसं चैव दूषणं चैव राक्षसम्}%47

\twolineshloka
{निजघान रणे रामस्तेषां चैव पदानुगान्}
{वने तस्मिन् निवसता जनस्थाननिवासिनाम्}%48

\twolineshloka
{रक्षसां निहतान्यासन् सहस्राणि चतुर्दश}
{ततो ज्ञातिवधं श्रुत्वा रावणः क्रोधमूर्च्छितः}%49

\twolineshloka
{सहायं वरयामास मारीचं नाम राक्षसम्}
{वार्यमाणः सुबहुशो मारीचेन स रावणः}%50

\twolineshloka
{न विरोधो बलवता क्षमो रावण तेन ते}
{अनादृत्य तु तद्वाक्यं रावणः कालचोदितः}%51

\twolineshloka
{जगाम सहमारीचस्तस्याऽऽश्रमपदं तदा}
{तेन मायाविना दूरमपवाह्य नृपात्मजौ}%52

\twolineshloka
{जहार भार्यां रामस्य गृध्रं हत्वा जटायुषम्}
{गृध्रं च निहतं दृष्ट्वा हृतां श्रुत्वा च मैथिलीम्}%53

\twolineshloka
{राघवः शोकसन्तप्तो विललापाऽऽकुलेन्द्रियः}
{ततस्तेनैव शोकेन गृध्रं दग्ध्वा जटायुषम्}%54

\twolineshloka
{मार्गमाणो वने सीतां राक्षसं सन्ददर्श ह}
{कबन्धं नाम रूपेण विकृतं घोरदर्शनम्}%55

\twolineshloka
{तं निहत्य महाबाहुर्ददाह स्वर्गतश्च सः}
{स चास्य कथयामास शबरीं धर्मचारिणीम्}%56

\twolineshloka
{श्रमणीं धर्मनिपुणाम् अभिगच्छेति राघव}
{सोऽभ्यगच्छन् महातेजाः शबरीं शत्रुसूदनः}%57

\twolineshloka
{शबर्या पूजितः सम्यग्रामो दशरथात्मजः}
{पम्पातीरे हनुमता सङ्गतो वानरेण ह}%58

\twolineshloka
{हनुमद्वचनाच्चैव सुग्रीवेण समागतः}
{सुग्रीवाय च तत्सर्वं शंसद्रामो महाबलः}%59

\twolineshloka
{आदितस्तद्यथा वृत्तं सीतायाश्च विशेषतः}
{सुग्रीवश्चापि तत्सर्वं श्रुत्वा रामस्य वानरः}%60

\twolineshloka
{चकार सख्यं रामेण प्रीतश्चैवाग्निसाक्षिकम्}
{ततो वानरराजेन वैरानुकथनं प्रति}%61

\twolineshloka
{रामायाऽऽवेदितं सर्वं प्रणयाद्दुःखितेन च}
{प्रतिज्ञातं च रामेण तदा वालिवधं प्रति}%62

\twolineshloka
{वालिनश्च बलं तत्र कथयामास वानरः}
{सुग्रीवः शङ्कितश्चासीन्नित्यं वीर्येण राघवे}%63

\twolineshloka
{राघवः प्रत्ययार्थं तु दुन्दुभेः कायमुत्तमम्}
{दर्शयामास सुग्रीवो महापर्वतसन्निभम्}%64

\twolineshloka
{उत्स्मयित्वा महाबाहुः प्रेक्ष्य चास्थि महाबलः}
{पादाङ्गुष्ठेन चिक्षेप सम्पूर्णं दशयोजनम्}%65

\twolineshloka
{बिभेद च पुनः सालान् सप्तैकेन महेषुणा}
{गिरिं रसातलं चैव जनयन् प्रत्ययं तदा}%66

\twolineshloka
{ततः प्रीतमनास्तेन विश्वस्तः स महाकपिः}
{किष्किन्धां रामसहितो जगाम च गुहां तदा}%67

\twolineshloka
{ततोऽगर्जद्धरिवरः सुग्रीवो हेमपिङ्गलः}
{तेन नादेन महता निर्जगाम हरीश्वरः}%68

\twolineshloka
{अनुमान्य तदा तारां सुग्रीवेण समागतः}
{निजघान च तत्रैनं शरेणैकेन राघवः}%69

\twolineshloka
{ततः सुग्रीववचनाद्धत्वा वालिनमाहवे}
{सुग्रीवमेव तद्राज्ये राघवः प्रत्यपादयत्}%70

\twolineshloka
{स च सर्वान् समानीय वानरान् वानरर्षभः}
{दिशः प्रस्थापयामास दिदृक्षुर्जनकात्मजाम्}%71

\twolineshloka
{ततो गृध्रस्य वचनात्सम्पातेर्हनुमान् बली}
{शतयोजनविस्तीर्णं पुप्लुवे लवणार्णवम्}%72

\twolineshloka
{तत्र लङ्कां समासाद्य पुरीं रावणपालिताम्}
{ददर्श सीतां ध्यायन्तीमशोकवनिकां गताम्}%73

\twolineshloka
{निवेदयित्वाऽभिज्ञानं प्रवृत्तिं विनिवेद्य च}
{समाश्वास्य च वैदेहीं मर्दयामास तोरणम्}%74

\twolineshloka
{पञ्च सेनाग्रगान् हत्वा सप्त मन्त्रिसुतानपि}
{शूरमक्षं च निष्पिष्य ग्रहणं समुपागमत्}%75

\twolineshloka
{अस्त्रेणोन्मुक्तमात्मानं ज्ञात्वा पैतामहाद्वरात्}
{मर्षयन् राक्षसान् वीरो यन्त्रिणस्तान् यदृच्छया}%76

\twolineshloka
{ततो दग्ध्वा पुरीं लङ्काम् ऋते सीतां च मैथिलीम्}
{रामाय प्रियमाख्यातुं पुनरायान् महाकपिः}%77

\twolineshloka
{सोऽभिगम्य महात्मानं कृत्वा रामं प्रदक्षिणम्}
{न्यवेदयदमेयात्मा दृष्टा सीतेति तत्त्वतः}%78

\twolineshloka
{ततः सुग्रीवसहितो गत्वा तीरं महोदधेः}
{समुद्रं क्षोभयामास शरैरादित्यसन्निभैः}%79

\twolineshloka
{दर्शयामास चाऽऽत्मानं समुद्रः सरितां पतिः}
{समुद्रवचनाच्चैव नलं सेतुमकारयत्}%80

\twolineshloka
{तेन गत्वा पुरीं लङ्कां हत्वा रावणमाहवे}
{रामः सीतामनुप्राप्य परां व्रीडामुपागमत्}%81

\twolineshloka
{तामुवाच ततो रामः परुषं जनसंसदि}
{अमृष्यमाणा सा सीता विवेश ज्वलनं सती}%82

\twolineshloka
{ततोऽग्निवचनात् सीतां ज्ञात्वा विगतकल्मषाम्}
{कर्मणा तेन महता त्रैलोक्यं सचराचरम्}%83

\twolineshloka
{सदेवर्षिगणं तुष्टं राघवस्य महात्मनः}
{बभौ रामः सम्प्रहृष्टः पूजितः सर्वदैवतैः}%84

\twolineshloka
{अभिषिच्य च लङ्कायां राक्षसेन्द्रं विभीषणम्}
{कृतकृत्यस्तदा रामो विज्वरः प्रमुमोद ह}%85

\twolineshloka
{देवताभ्यो वरान् प्राप्य समुत्थाप्य च वानरान्}
{अयोध्यां प्रस्थितो रामः पुष्पकेण सुहृद्-वृतः}%86

\twolineshloka
{भरद्वाजाश्रमं गत्वा रामः सत्यपराक्रमः}
{भरतस्यान्तिके रामो हनूमन्तं व्यसर्जयत्}%87

\twolineshloka
{पुनराख्यायिकां जल्पन् सुग्रीवसहितस्तदा}
{पुष्पकं तत् समारुह्य नन्दिग्रामं ययौ तदा}%88

\twolineshloka
{नन्दिग्रामे जटां हित्वा भ्रातृभिः सहितोऽनघः}
{रामः सीतामनुप्राप्य राज्यं पुनरवाप्तवान्}%89

\twolineshloka
{प्रहृष्टमुदितो लोकस्तुष्टः पुष्टः सुधार्मिकः}
{निरामयो ह्यरोगश्च दुर्भिक्षभयवर्जितः}%90

\twolineshloka
{न पुत्रमरणं केचिद्-द्रक्ष्यन्ति पुरुषाः क्वचित्}
{नार्यश्चाविधवा नित्यं भविष्यन्ति पतिव्रताः}%91

\twolineshloka
{न चाग्निजं भयं किञ्चिन्नाप्सु मज्जन्ति जन्तवः}
{न वातजं भयं किञ्चिन्नापि ज्वरकृतं तथा}%92

\twolineshloka
{न चापि क्षुद्भयं तत्र न तस्करभयं तथा}
{नगराणि च राष्ट्राणि धनधान्ययुतानि च}%93

\twolineshloka
{नित्यं प्रमुदिताः सर्वे यथा कृतयुगे तथा}
{अश्वमेधशतैरिष्ट्वा तथा बहुसुवर्णकैः}%94

\twolineshloka
{गवां कोट्ययुतं दत्त्वा विद्वद्‌भ्यो विधिपूर्वकम्}
{असङ्ख्येयं धनं दत्त्वा ब्राह्मणेभ्यो महायशाः}%95

\twolineshloka
{राजवंशाञ्छतगुणान् स्थापयिष्यति राघवः}
{चातुर्वर्ण्यं च लोकेऽस्मिन् स्वे स्वे धर्मे नियोक्ष्यति}%96

\twolineshloka
{दशवर्षसहस्राणि दशवर्षशतानि च}
{रामो राज्यमुपासित्वा ब्रह्मलोकं गमिष्यति}%97

\twolineshloka
{इदं पवित्रं पापघ्नं पुण्यं वेदैश्च सम्मितम्}
{यः पठेद्रामचरितं सर्वपापैः प्रमुच्यते}%98

\twolineshloka
{एतदाख्यानमायुष्यं पठन् रामायणं नरः}
{सपुत्रपौत्रः सगणः प्रेत्य स्वर्गे महीयते}%99

\fourlineindentedshloka
{पठन् द्विजो वागृषभत्वमीयात्}
{स्यात् क्षत्रियो भूमिपतित्वमीयात्}
{वणिग्जनः पण्यफलत्वमीयात्}
{जनश्च शूद्रोऽपि महत्त्वमीयात्}%100
{॥इत्यार्षे श्रीमद्रामायणे वाल्मीकीये आदिकाव्ये बालकाण्डे प्रथमः सर्गः॥}

\mbox{}\\
\resetShloka
\dnsub{मङ्गलश्लोकाः}
\fourlineindentedshloka
{स्वस्ति प्रजाभ्यः परिपालयन्ताम्}
{न्यायेन मार्गेण महीं महीशाः}
{गोब्राह्मणेभ्यः शुभमस्तु नित्यम्}
{लोकाः समस्ताः सुखिनो भवन्तु}

\twolineshloka
{काले वर्षतु पर्जन्यः पृथिवी सस्यशालिनी}
{देशोऽयं क्षोभरहितो ब्राह्मणाः सन्तु निर्भयाः}

\twolineshloka
{अपुत्राः पुत्रिणः सन्तु पुत्रिणः सन्तु पौत्रिणः}
{अधनाः सधनाः सन्तु जीवन्तु शरदां शतम्}

\twolineshloka
{चरितं रघुनाथस्य शतकोटि-प्रविस्तरम्}
{एकैकमक्षरं पुंसां महापातकनाशनम्}

\twolineshloka
{शृण्वन् रामायणं भक्त्या यः पादं पदमेव वा}
{स याति ब्रह्मणः स्थानं ब्रह्मणा पूज्यते सदा}

\twolineshloka
{रामाय रामभद्राय रामचन्द्राय वेधसे}
{रघुनाथाय नाथाय सीतायाः पतये नमः}

\twolineshloka
{यन्मङ्गलं सहस्राक्षे सर्वदेवनमस्कृते}
{वृत्रनाशे समभवत् तत्ते भवतु मङ्गलम्}

\twolineshloka
{यन्मङ्गलं सुपर्णस्य विनताऽकल्पयत् पुरा}
{अमृतं प्रार्थयानस्य तत्ते भवतु मङ्गलम्}

\twolineshloka
{अमृतोत्पादने दैत्यान् घ्नतो वज्रधरस्य यत्}
{अदितिर्मङ्गलं प्रादात् तत्ते भवतु मङ्गलम्}

\twolineshloka
{त्रीन् विक्रमान् प्रक्रमतो विष्णोरमिततेजसः}
{यदासीन्मङ्गलं राम तत्ते भवतु मङ्गलम्}

\twolineshloka
{ऋषयः सागरा द्वीपा वेदा लोका दिशश्च ते}
{मङ्गलानि महाबाहो दिशन्तु तव सर्वदा}

\twolineshloka
{मङ्गलं कोसलेन्द्राय महनीयगुणाब्धये}
{चक्रवर्तितनूजाय सार्वभौमाय मङ्गलम्}

\fourlineindentedshloka*
{कायेन वाचा मनसेन्द्रियैर्वा}
{बुद्‌ध्याऽऽत्मना वा प्रकृतेः स्वभावात्}
{करोमि यद्यत् सकलं परस्मै}
{नारायणायेति समर्पयामि}\nopagebreak[4]