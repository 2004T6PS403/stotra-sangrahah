% !TeX program = XeLaTeX
% !TeX root = ../../shloka.tex
\sect{शिवसहस्रनामस्तोत्रम्}

\twolineshloka*
{शुक्लाम्बरधरं विष्णुं शशिवर्णं चतुर्भुजम्}
{प्रसन्नवदनं ध्यायेत् सर्वविघ्नोपशान्तये}

\twolineshloka*
{नमोऽस्तु ते व्यास विशालबुद्धे फुल्लारविन्दायतपत्रनेत्र}
{येन त्वया भारततैलपूर्णः प्रज्वालितो ज्ञानमयः प्रदीपः}% .. 2..


\twolineshloka*
{नारायणं नमस्कृत्य नरं चैव नरोत्तमम्}
{देवीं सरस्वतीं व्यासं ततो जयमुदीरयेत्}


\fourlineindentedshloka*
{वन्दे शम्भुमुमापतिं सुरगुरुं वन्दे जगत्कारणम्}
{वन्दे पन्नगभूषणं मृगधरं वन्दे पशूनां पतिम्}
{वन्दे सूर्यशशाङ्कवह्निनयनं वन्दे मुकुन्दप्रियम्}
{वन्दे भक्तजनाश्रयं च वरदं वन्दे शिवं शङ्करम्}

\dnsub{पूर्वभागः}

युधिष्ठिर उवाच
\twolineshloka
{त्वयाऽऽपगेय नामानि श्रुतानीह जगत्पतेः}%।
{पितामहेशाय विभोर्नामान्याचक्ष्व शम्भवे}%॥१}%॥

\twolineshloka
{बभ्रवे विश्वरूपाय महाभाग्यं च तत्त्वतः}%।
{सुरासुरगुरौ देवे शङ्करेऽव्यक्तयोनये}%॥२}%॥

भीष्म उवाच
\twolineshloka
{अशक्तोऽहं गुणान् वक्तुं महादेवस्य धीमतः}%।
{यो हि सर्वगतो देवो न च सर्वत्र दृश्यते}%॥३}%॥

\twolineshloka
{ब्रह्मविष्णुसुरेशानां स्रष्टा च प्रभुरेव च}%।
{ब्रह्मादयः पिशाचान्ता यं हि देवा उपासते}%॥४}%॥

\twolineshloka
{प्रकृतीनां परत्वेन पुरुषस्य च यः परः}%।
{चिन्त्यते यो योगविद्भिरृषिभिस्तत्त्वदर्शिभिः}%॥५}%॥

\twolineshloka
{प्रकृतिं पुरुषं चैव क्षोभयित्वा स्वतेजसा}%।
{ब्रह्माणमसृजत् तस्माद्देवदेवः प्रजापतिः}%॥६}%॥


\twolineshloka
{को हि शक्तो गुणान् वक्तुं देवदेवस्य धीमतः}%।
{गर्भजन्मजरायुक्तो मर्त्यो मृत्युसमन्वितः}%॥७}%॥

\twolineshloka
{को हि शक्तो भवं ज्ञातुं मद्विधः परमेश्वरम्}%।
{ऋते नारायणात् पुत्र शङ्खचक्रगदाधरात्}%॥८}%॥

\twolineshloka
{एष विद्वान् गुणश्रेष्ठो विष्णुः परमदुर्जयः}%।
{दिव्यचक्षुर्महातेजा वीक्ष्यते योगचक्षुषा}%॥९}%॥

\twolineshloka
{रुद्रभक्त्या तु कृष्णेन जगद्‌व्याप्तं महात्मना}%।
{तं प्रसाद्य तदा देवं बदर्यां किल भारत}%॥१०}%॥

\twolineshloka
{अर्थात् प्रियतरत्वं च सर्वलोकेषु वै तदा}%।
{प्राप्तवानेव राजेन्द्र सुवर्णाक्षान्महेश्वरात्}%॥११}%॥

\twolineshloka
{पूर्णं वर्षसहस्रं तु तप्तवानेष माधवः}%।
{प्रसाद्य वरदं देवं चराचरगुरुं शिवम्}%॥१२}%॥

\twolineshloka
{युगे युगे तु कृष्णेन तोषितो वै महेश्वरः}%।
{भक्त्या परमया चैव प्रीतश्चैव महात्मनः}%॥१३}%॥

\twolineshloka
{ऐश्वर्यं यादृशं तस्य जगद्योनेर्महात्मनः}%।
{तदयं दृष्टवान् साक्षात् पुत्रार्थे हरिरच्युतः}%॥१४}%॥

\twolineshloka
{यस्मात् परतरं चैव नान्यं पश्यामि भारत}%।
{व्याख्यातुं देवदेवस्य शक्तो नामान्यशेषतः}%॥१५}%॥

\twolineshloka
{एष शक्तो महाबाहुर्वक्तुं भगवतो गुणान्}%।
{विभूतिं चैव कार्त्स्न्येन सत्यां माहेश्वरीं नृप}%॥१६}%॥

\twolineshloka
{सुरासुरगुरो देव विष्णो त्वं वक्तुम् अर्हसि}%।
{शिवाय शिवरूपाय यन्माऽपृच्छद्युधिष्ठिरः}%॥१७}%॥

\twolineshloka
{नाम्नां सहस्रं देवस्य तण्डिना ब्रह्मवादिना}%।
{निवेदितं ब्रह्मलोके ब्रह्मणो यत् पुराऽभवत्}%॥१८}%॥

\twolineshloka
{द्वैपायनप्रभृतयस्तथा चेमे तपोधनाः}%।
{ऋषयः सुव्रता दान्ताः शृण्वन्तु गदतस्तव}%॥१९}%॥

%ध्रुवाय नन्दिने होत्रे गोप्त्रे विश्वसृजेऽग्नये}%।
%महाभाग्यं विभो ब्रूहि मुण्डिनेऽथ कपर्दिने}%॥

वासुदेव उवाच
\twolineshloka
{न गतिः कर्मणां शक्या वेत्तुमीशस्य तत्त्वतः}%।
{हिरण्यगर्भप्रमुखा देवाः सेन्द्रा महर्षयः}%॥२०}%॥

\twolineshloka
{न विदुर्यस्य निधनम् आदिं वा सूक्ष्मदर्शिनः}%।
{स कथं नाममात्रेण शक्यो ज्ञातुं सतां गतिः}%॥२१}%॥

\twolineshloka
{तस्याहम् असुरघ्नस्य कांश्चिद्भगवतो गुणान्}%।
{भवतां कीर्तयिष्यामि व्रतेशाय यथातथम्}%॥२२}%॥

वैशम्पायन उवाच
\twolineshloka
{एवमुक्त्वा तु भगवान् गुणांस्तस्य महात्मनः}%।
{उपस्पृश्य शुचिर्भूत्वा कथयामास धीमतः}%॥२३}%॥

\begin{minipage}{\linewidth}
\centering
वासुदेव उवाच
\twolineshloka
{ततः स प्रयतो भूत्वा मम तात युधिष्ठिर}%।
{प्राञ्जलिः प्राह विप्रर्षिर्नामसङ्ग्रहामादितः}%॥२४}%॥
\end{minipage}

\begin{minipage}{\linewidth}
\centering
उपमन्युरुवाच
\twolineshloka
{ब्रह्मप्रोक्तैरृषिप्रोक्तैर्वेदवेदाङ्गसम्भवैः}%।
{सर्वलोकेषु विख्यातं स्तुत्यं स्तोष्यामि नामभिः}%॥२५}%॥
\end{minipage}

\twolineshloka
{महद्भिर्विहितैः सत्यैः सिद्धैः सर्वार्थसाधकैः}%।
{ऋषिणा तण्डिना भक्त्या कृतैर्वेदकृतात्मना}%॥२६}%॥

\twolineshloka
{यथोक्तैः साधुभिः ख्यातैर्मुनिभिस्तत्त्वदर्शिभिः}%।
{प्रवरं प्रथमं स्वर्ग्यं सर्वभूतहितं शुभम्}%॥२७}%॥

\threelineshloka
{श्रुतैः सर्वत्र जगति ब्रह्मलोकावतारितैः}
{सत्यैस्तत् परमं ब्रह्म ब्रह्मप्रोक्तं सनातनम्}%।
{वक्ष्ये यदुकुलश्रेष्ठ शृणुष्वावहितो मम}%॥२८}%॥

\twolineshloka
{वरयैनं भवं देवं भक्तस्त्वं परमेश्वरम्}%।
{तेन ते श्रावयिष्यामि यत् तद्‌ब्रह्म सनातनम्}%॥२९}%॥

\twolineshloka
{न शक्यं विस्तरात् कृत्स्नं वक्तुं शर्वस्य केनचित्}%।
{युक्तेनापि विभूतीनामपि वर्षशतैरपि}%॥३०}%॥ 

\twolineshloka
{यस्यादिर्मध्यमन्तं च सुरैरपि न गम्यते}%।
{कस्तस्य शक्नुयाद्वक्तुं गुणान् कार्त्स्न्येन माधव}%॥३१}%॥

\twolineshloka
{किं तु देवस्य महतः सङ्क्षिप्तार्थपदाक्षरम्}%।
{शक्तितश्चरितं वक्ष्ये प्रसादात् तस्य धीमतः}%॥३२}%॥

\twolineshloka
{अप्राप्य तु ततोऽनुज्ञां न शक्यः स्तोतुमीश्वरः}%।
{यदा तेनाभ्यनुज्ञातः स्तुतो वै स तदा मया}%॥३३}%॥

\twolineshloka
{अनादिनिधनस्याहं जगद्योनेर्महात्मनः}%।
{नाम्नां कञ्चित् समुद्देश्यं वक्ष्याम्यव्यक्तयोनिनः}%॥३४}%॥

\twolineshloka
{वरदस्य वरेण्यस्य विश्वरूपस्य धीमतः}%।
{शृणु नाम्नां चयं कृष्ण यदुक्तं पद्मयोनिना}%॥३५}%॥

\twolineshloka
{दशनामसहस्राणि यान्याह प्रपितामहः}%।
{तानि निर्मथ्य मनसा दध्नो घृतमिवोद्धृतम्}%॥३६}%॥

\twolineshloka
{गिरेः सारं यथा हेम पुष्पसारं यथा मधु}%।
{घृतात् सारं यथा मण्डस्तथैतत् सारमुद्धृतम्}%॥३७}%॥

\twolineshloka
{सर्वपापापहमिदं चतुर्वेदसमन्वितम्}%।
{प्रयत्नेनाधिगन्तव्यं धार्यं च प्रयतात्मना}%॥३८}%॥

%शान्तिकं पौष्टिकं चैव रक्षोघ्नं पावनं महत्।
%इदं भक्ताय दातव्यं श्रद्दधानास्तिकाय च।
%नाश्रद्दधानरूपाय नास्तिकायाजितात्मने।
%यश् चाभ्यसूयते देवं भूतात्मानं पिनाकिनम्।
%स कृष्ण नरकं याति सह पूर्वैः सहानुगैः।
%इदं ध्यानम् इदं योगम् इदं ध्येयम् अनुत्तमम्।
%इदं जप्यम् इदं ज्ञानं रहस्यम् इदम् उत्तमम्।
%इदं ज्ञात्वान्तकाले ऽपि गच्छेद् धि परमां गतिम्।
%पवित्रं मङ्गलं पुण्यं कल्याणम् इदम् उत्तमम्।
%निगदिष्ये महाबाहो स्तवानाम् उत्तमं स्तवम्।
%इदं ब्रह्मा पुरा कृत्वा सर्वलोकपितामहः।
%सर्वस्तवानां दिव्यानां राजत्वे समकल्पयत्।
%तदाप्रभृति चैवायम् ईश्वरस्य महात्मनः।
%स्तवराजेति विख्यातो जगत्य् अमरपूजितः।
%ब्रह्मलोकाद् अयं चैव स्तवराजो ऽवतारितः।
%यस्मात् तण्डिः पुरा प्राह तेन तण्डिकृतो ऽभवत्।
%स्वर्गाच् चैवात्र भूलोकं तण्डिना ह्य् अवतारितः।
%सर्वमङ्गलमङ्गल्यं सर्वपापप्रणाशनम्।
%निगदिष्ये महाबाहो स्तवानाम् उत्तमं स्तवम्।
%ब्रह्मणाम् अपि यद् ब्रह्म पराणाम् अपि यत् परम्।
%तेजसाम् अपि यत् तेजस् तपसाम् अपि यत् तपः।
%शान्तीनाम् अपि या शान्तिर् द्युतीनाम् अपि या द्युतिः।
%दान्तानाम् अपि यो दान्तो धीमताम् अपि या च धीः।
%देवानाम् अपि यो देवो मुनीनाम् अपि यो मुनिः।
%यज्ञानाम् अपि यो यज्ञः शिवानाम् अपि यः शिवः।
%रुद्राणाम् अपि यो रुद्रः प्रभुः प्रभवताम् अपि।
%योगिनाम् अपि यो योगी कारणानां च कारणम्।
%यतो लोकाः संभवन्ति न भवन्ति यतः पुनः।

\threelineshloka
{सर्वभूतात्मभूतस्य हरस्यामिततेजसः}
{अष्टोत्तरसहस्रं तु नाम्नां शर्वस्य मे शृणु}
{यच्छ्रुत्वा मनुजव्याघ्र सर्वान् कामानवाप्स्यसि}%॥३९॥

\dnsub{ध्यानम्}

\fourlineindentedshloka*
{शान्तं पद्मानस्थं शशिधरमुकुटं पञ्चवक्त्रं त्रिनेत्रम्}
{शूलं वज्रं च खड्गं परशुमभयदं दक्षभागे वहन्तम्}
{नागं पाशं घण्टां प्रलयहुतवहं साङ्कुशं वामभागे}
{नानालङ्कारयुक्तं स्फटिकमणिनिभं पार्वतीशं नमामि}

\resetShloka
\dnsub{स्तोत्रम्}
\twolineshloka
{ॐ स्थिरः स्थाणुः प्रभुर्भीमः प्रवरो वरदो वरः}%
{सर्वात्मा सर्वविख्यातः सर्वः सर्वकरो भवः}% १॥

\twolineshloka
{जटी चर्मी शिखण्डी च सर्वाङ्गः सर्वभावनः}%
{हरश्च हरिणाक्षश्च सर्वभूतहरः प्रभुः}% २॥

\twolineshloka
{प्रवृत्तिश्च निवृत्तिश्च नियतः शाश्वतो ध्रुवः}%
{श्मशानवासी भगवान् खचरो गोचरोऽर्दनः}% ३॥

\twolineshloka
{अभिवाद्यो महाकर्मा तपस्वी भूतभावनः}%
{उन्मत्तवेषप्रच्छन्नः सर्वलोकप्रजापतिः}% ४॥

\twolineshloka
{महारूपो महाकायो वृषरूपो महायशाः}%
{महात्मा सर्वभूतात्मा विश्वरूपो महाहनुः}% ५॥

\twolineshloka
{लोकपालोऽन्तर्हितात्मा प्रसादो हयगर्दभिः}%
{पवित्रं च महांश्चैव नियमो नियमाश्रितः}% ६॥

\twolineshloka
{सर्वकर्मा स्वयम्भूत आदिरादिकरो निधिः}%
{सहस्राक्षो विशालाक्षः सोमो नक्षत्रसाधकः}% ७॥

\twolineshloka
{चन्द्रः सूर्यः शनिः केतुर्ग्रहो ग्रहपतिर्वरः}%
{अत्रिरत्र्यानमस्कर्ता मृगबाणार्पणोऽनघः}% ८॥

\twolineshloka
{महातपा घोरतपा अदीनो दीनसाधकः}%
{संवत्सरकरो मन्त्रः प्रमाणं परमं तपः}% ९॥

\twolineshloka
{योगी योज्यो महाबीजो महारेता महाबलः}%
{सुवर्णरेताः सर्वज्ञः सुबीजो बीजवाहनः}% १०॥

\twolineshloka
{दशबाहुस्त्वनिमिषो नीलकण्ठ उमापतिः}%
{विश्वरूपः स्वयंश्रेष्ठो बलवीरोऽबलो गणः}% ११॥

\twolineshloka
{गणकर्ता गणपतिर्दिग्वासाः काम एव च}%
{मन्त्रवित् परमो\hspace{.3ex}मन्त्रः सर्वभावकरो हरः}% १२॥

\twolineshloka
{कमण्डलुधरो धन्वी बाणहस्तः कपालवान्}%
{अशनी शतघ्नी खड्गी पट्टिशी चऽऽयुधी महान्}% १३॥

\twolineshloka
{स्रुवहस्तः सुरूपश्च तेजस्तेजस्करो निधिः}%
{उष्णिषी च सुवक्त्रश्च उदग्रो विनतस्तथा}% १४॥

\twolineshloka
{दीर्घश्च हरिकेशश्च सुतीर्थः कृष्ण एव च}%
{सृगालरूपः सिद्धार्थो मुण्डः सर्वशुभङ्करः}% १५॥

\twolineshloka
{अजश्च बहुरूपश्च गन्धधारी कपर्द्यपि}%
{ऊर्ध्वरेता ऊर्ध्वलिङ्ग ऊर्ध्वशायी नभःस्थलः}% १६॥

\twolineshloka
{त्रिजटी चीरवासाश्च रुद्रः सेनापतिर्विभुः}%
{अहश्चरो नक्तञ्चरस्तिग्ममन्युः सुवर्चसः}% १७॥

\twolineshloka
{गजहा दैत्यहा कालो लोकधाता गुणाकरः}%
{सिंहशार्दूलरूपश्च आर्द्रचर्माम्बरावृतः}% १८॥

\twolineshloka
{कालयोगी महानादः सर्वकामश्चतुष्पथः}%
{निशाचरः प्रेतचारी भूतचारी महेश्वरः}% १९॥

\twolineshloka
{बहुभूतो बहुधरः स्वर्भानुरमितो गतिः}%
{नृत्यप्रियो नित्यनर्तो नर्तकः सर्वलालसः}% २०॥

\twolineshloka
{घोरो महातपाः पाशो नित्यो गिरिरुहो नभः}%
{सहस्रहस्तो विजयो व्यवसायो ह्यतन्द्रितः}% २१॥

\twolineshloka
{अधर्षणो धर्षणात्मा यज्ञहा कामनाशकः}%
{दक्षयागापहारी च सुसहो मध्यमस्तथा}% २२॥

\twolineshloka
{तेजोपहारी बलहा मुदितोऽर्थोऽजितो वरः}%
{गम्भीरघोषो गम्भीरो गम्भीरबलवाहनः}% २३॥

\twolineshloka
{न्यग्रोधरूपो न्यग्रोधो वृक्षकर्णस्थितिर्विभुः}%
{सुतीक्ष्णदशनश्चैव महाकायो महाननः}% २४॥

\twolineshloka
{विष्वक्सेनो हरिर्यज्ञः संयुगापीडवाहनः}%
{तीक्ष्णतापश्च हर्यश्वः सहायः कर्मकालवित्}% २५॥

\twolineshloka
{विष्णुप्रसादितो यज्ञः समुद्रो बडवामुखः}%
{हुताशनसहायश्च प्रशान्तात्मा हुताशनः}% २६॥

\twolineshloka
{उग्रतेजा महातेजा जन्यो विजयकालवित्}%
{ज्योतिषामयनं सिद्धिः सर्वविग्रह एव च}% २७॥

\twolineshloka
{शिखी मुण्डी जटी ज्वाली मूर्तिजो मूर्धजो बली}%
{वैणवी पणवी ताली खली कालकटङ्कटः}% २८॥

\twolineshloka
{नक्षत्रविग्रहमतिर्गुणबुद्धिर्लयोऽगमः}%
{प्रजापतिर्विश्वबाहुर्विभागः सर्वगोऽमुखः}% २९॥

\twolineshloka
{विमोचनः सुसरणो हिरण्यकवचोद्भवः}%
{मेढ्रजो बलचारी च महीचारी स्रुतस्तथा}% ३०॥

\twolineshloka
{सर्वतूर्यविनोदी च सर्वातोद्यपरिग्रहः}%
{व्यालरूपो गुहावासी गुहो माली तरङ्गवित्}% ३१॥

\twolineshloka
{त्रिदशस्त्रिकालधृक् कर्मसर्वबन्धविमोचनः}%
{बन्धनस्त्वसुरेन्द्राणां युधि शत्रुविनाशनः}% ३२॥

\twolineshloka
{साङ्ख्यप्रसादो दुर्वासाः सर्वसाधुनिषेवितः}%
{प्रस्कन्दनो विभागज्ञोऽतुल्यो यज्ञविभागवित्}% ३३॥

\twolineshloka
{सर्ववासः सर्वचारी दुर्वासा वासवोऽमरः}%
{हैमो हेमकरो यज्ञः सर्वधारी धरोत्तमः}% ३४॥

\twolineshloka
{लोहिताक्षो महाक्षश्च विजयाक्षो विशारदः}%
{सङ्ग्रहो निग्रहः कर्ता सर्पचीरनिवासनः}% ३५॥

\twolineshloka
{मुख्योऽमुख्यश्च देहश्च काहलिः सर्वकामदः}%
{सर्वकालप्रसादश्च सुबलो बलरूपधृक्}% ३६॥

\twolineshloka
{सर्वकामवरश्चैव सर्वदः सर्वतोमुखः}%
{आकाशनिर्विरूपश्च निपाती ह्यवशः खगः}% ३७॥

\twolineshloka
{रौद्ररूपोंऽशुरादित्यो बहुरश्मिः सुवर्चसी}%
{वसुवेगो महावेगो मनोवेगो निशाचरः}% ३८॥

\twolineshloka
{सर्ववासी श्रियावासी उपदेशकरोऽकरः}%
{मुनिरात्मनिरालोकः सम्भग्नश्च सहस्रदः}% ३९॥

\twolineshloka
{पक्षी च पक्षरूपश्च अतिदीप्तो विशाम्पतिः}%
{उन्मादो मदनः कामो ह्यश्वत्थोऽर्थकरो यशः}% ४०॥

\twolineshloka
{वामदेवश्च वामश्च प्राग्दक्षिणश्च वामनः}%
{सिद्धयोगी महर्षिश्च सिद्धार्थः सिद्धसाधकः}% ४१॥

\twolineshloka
{भिक्षुश्च भिक्षुरूपश्च विपणो मृदुरव्ययः}%
{महासेनो विशाखश्च षष्ठिभागो गवां पतिः}% ४२॥

\twolineshloka
{वज्रहस्तश्च विष्कम्भी चमूस्तम्भन एव च}%
{वृत्तावृत्तकरस्तालो मधुर्मधुकलोचनः}% ४३॥

\twolineshloka
{वाचस्पत्यो वाजसनो नित्यमाश्रितपूजितः}%
{ब्रह्मचारी लोकचारी सर्वचारी विचारवित्}% ४४॥

\twolineshloka
{ईशान ईश्वरः कालो निशाचारी पिनाकवान्}%
{निमित्तस्थो निमित्तं च नन्दिर्नन्दिकरो हरिः}% ४५॥

\twolineshloka
{नन्दीश्वरश्च नन्दी च नन्दनो नन्दिवर्धनः}%
{भगहारी निहन्ता च कालो ब्रह्मा पितामहः}% ४६॥

\twolineshloka
{चतुर्मुखो महालिङ्गश्चारुलिङ्गस्तथैव च}%
{लिङ्गाध्यक्षः सुराध्यक्षो योगाध्यक्षो युगावहः}% ४७॥

\twolineshloka
{बीजाध्यक्षो बीजकर्ता अध्यात्माऽनुगतो बलः}%
{इतिहासः सकल्पश्च गौतमोऽथ निशाकरः}% ४८॥

\twolineshloka
{दम्भो ह्यदम्भो वैदम्भो वश्यो वशकरः कलिः}%
{लोककर्ता पशुपतिर्महाकर्ता ह्यनौषधः}% ४९॥

\twolineshloka
{अक्षरं परमं ब्रह्म बलवच्चक्र एव च}%
{नीतिर्ह्यनीतिः शुद्धात्मा शुद्धो मान्यो गतागतः}% ५०॥

\twolineshloka
{बहुप्रसादः सुस्वप्नो दर्पणोऽथ त्वमित्रजित्}%
{वेदकारो मन्त्रकारो विद्वान् समरमर्दनः}% ५१॥

\twolineshloka
{महामेघनिवासी च महाघोरो वशीकरः}%
{अग्निज्वालो महाज्वालो अतिधूम्रो हुतो हविः}% ५२॥

\twolineshloka
{वृषणः शङ्करो नित्यं वर्चस्वी धूमकेतनः}%
{नीलस्तथाऽङ्गलुब्धश्च शोभनो निरवग्रहः}% ५३॥

\twolineshloka
{स्वस्तिदः स्वस्तिभावश्च भागी भागकरो लघुः}%
{उत्सङ्गश्च महाङ्गश्च महागर्भपरायणः}% ५४॥

\twolineshloka
{कृष्णवर्णः सुवर्णश्च इन्द्रियं सर्वदेहिनाम्}%
{महापादो महाहस्तो महाकायो महायशाः}% ५५॥

\twolineshloka
{महामूर्धा महामात्रो महानेत्रो निशालयः}%
{महान्तको महाकर्णो महोष्ठश्च महाहनुः}% ५६॥

\twolineshloka
{महानासो महाकम्बुर्महाग्रीवः श्मशानभाक्}%
{महावक्षा महोरस्को ह्यन्तरात्मा मृगालयः}% ५७॥

\twolineshloka
{लम्बनो लम्बितोष्ठश्च महामायः पयोनिधिः}%
{महादन्तो महादंष्ट्रो महाजिह्वो महामुखः}% ५८॥

\twolineshloka
{महानखो महारोमो महाकोशो महाजटः}%
{प्रसन्नश्च प्रसादश्च प्रत्ययो गिरिसाधनः}% ५९॥

\twolineshloka
{स्नेहनोऽस्नेहनश्चैव अजितश्च महामुनिः}%
{वृक्षाकारो वृक्षकेतुरनलो वायुवाहनः}% ६०॥

\twolineshloka
{गण्डली मेरुधामा च देवाधिपतिरेव च}%
{अथर्वशीर्षः सामास्य ऋक्सहस्रामितेक्षणः}% ६१॥

\twolineshloka
{यजुः पादभुजो गुह्यः प्रकाशो जङ्गमस्तथा}%
{अमोघार्थः प्रसादश्च अभिगम्यः सुदर्शनः}% ६२॥

\twolineshloka
{उपकारः प्रियः सर्वः कनकः काञ्चनच्छविः}%
{नाभिर्नन्दिकरो भावः पुष्करः स्थपतिः स्थिरः}% ६३॥

\twolineshloka
{द्वादशस्त्रासनश्चाद्यो यज्ञो यज्ञसमाहितः}%
{नक्तं कलिश्च कालश्च मकरः कालपूजितः}% ६४॥

\twolineshloka
{सगणो गणकारश्च भूतवाहनसारथिः}%
{भस्मशयो भस्मगोप्ता भस्मभूतस्तरुर्गणः}% ६५॥

\twolineshloka
{लोकपालस्तथाऽलोको महात्मा सर्वपूजितः}%
{शुक्लस्त्रिशुक्लः सम्पन्नः शुचिर्भूतनिषेवितः}% ६६॥

\twolineshloka
{आश्रमस्थः क्रियावस्थो विश्वकर्ममतिर्वरः}%
{विशालशाखस्ताम्रोष्ठो ह्यम्बुजालः सुनिश्चलः}% ६७॥

\twolineshloka
{कपिलः कपिशः शुक्ल आयुश्चैव परोऽपरः}%
{गन्धर्वो ह्यदितिस्तार्क्ष्यः सुविज्ञेयः सुशारदः}% ६८॥

\twolineshloka
{परश्वधायुधो देव अनुकारी सुबान्धवः}%
{तुम्बवीणो महाक्रोध ऊर्ध्वरेता जलेशयः}% ६९॥

\twolineshloka
{उग्रो वंशकरो वंशो वंशनादो ह्यनिन्दितः}%
{सर्वाङ्गरूपो मायावी सुहृदो ह्यनिलोऽनलः}% ७०॥

\twolineshloka
{बन्धनो बन्धकर्ता च सुबन्धनविमोचनः}%
{सयज्ञारिः सकामारिर्महादंष्ट्रो महायुधः}% ७१॥

\twolineshloka
{बहुधा निन्दितः शर्वः शङ्करः शङ्करोऽधनः}%
{अमरेशो महादेवो विश्वदेवः सुरारिहा}% ७२॥

\twolineshloka
{अहिर्बुध्न्योऽनिलाभश्च चेकितानो हविस्तथा}%
{अजैकपाच्च कापाली त्रिशङ्कुरजितः शिवः}% ७३॥

\twolineshloka
{धन्वन्तरिर्धूमकेतुः स्कन्दो वैश्रवणस्तथा}%
{धाता शक्रश्च विष्णुश्च मित्रस्त्वष्टा ध्रुवो धरः}% ७४॥

\twolineshloka
{प्रभावः सर्वगो वायुरर्यमा सविता रविः}%
{उषङ्गुश्च विधाता च मान्धाता भूतभावनः}% ७५॥

\twolineshloka
{विभुर्वर्णविभावी च सर्वकामगुणावहः}%
{पद्मनाभो महागर्भश्चन्द्रवक्त्रोऽनिलोऽनलः}% ७६॥

\twolineshloka
{बलवांश्चोपशान्तश्च पुराणः पुण्यचञ्चुरी}%
{कुरुकर्ता कुरुवासी कुरुभूतो गुणौषधः}% ७७॥

\twolineshloka
{सर्वाशयो दर्भचारी सर्वेषां प्राणिनां पतिः}%
{देवदेवः सुखासक्तः सदसत् सर्वरत्नवित्}% ७८॥

\twolineshloka
{कैलासगिरिवासी च हिमवद्गिरिसंश्रयः}%
{कूलहारी कूलकर्ता बहुविद्यो बहुप्रदः}% ७९॥

\twolineshloka
{वणिजो वर्धकी वृक्षो वकुलश्चन्दनश्छदः}%
{सारग्रीवो महाजत्रुरलोलश्च महौषधः}% ८०॥

\twolineshloka
{सिद्धार्थकारी सिद्धार्थश्छन्दोव्याकरणोत्तरः}%
{सिंहनादः सिंहदंष्ट्रः सिंहगः सिंहवाहनः}% ८१॥

\twolineshloka
{प्रभावात्मा जगत्कालस्थालो लोकहितस्तरुः}%
{सारङ्गो नवचक्राङ्गः केतुमाली सभावनः}% ८२॥

%\twolineshloka
भूतालयो भूतपतिरहोरात्रमनिन्दितः॥८३॥
\refstepcounter{shlokacount}
\twolineshloka
{वाहिता सर्वभूतानां निलयश्च विभुर्भवः}%
{अमोघः संयतो ह्यश्वो भोजनः प्राणधारणः}% ८४॥

\twolineshloka
{धृतिमान् मतिमान् दक्षः सत्कृतश्च युगाधिपः}%
{गोपालिर्गोपतिर्ग्रामो गोचर्मवसनो हरिः}% ८५॥

\twolineshloka
{हिरण्यबाहुश्च तथा गुहापालः प्रवेशिनाम्}%
{प्रकृष्टारिर्महाहर्षो जितकामो जितेन्द्रियः}% ८६॥

\twolineshloka
{गान्धारश्च सुवासश्च तपःसक्तो रतिर्नरः}%
{महागीतो महानृत्यो ह्यप्सरोगणसेवितः}% ८७॥

\twolineshloka
{महाकेतुर्महाधातुर्नैकसानुचरश्चलः}%
{आवेदनीय आदेशः सर्वगन्धसुखावहः}% ८८॥

\twolineshloka
{तोरणस्तारणो वातः परिधीः पतिखेचरः}%
{संयोगो वर्धनो वृद्धो अतिवृद्धो गुणाधिकः}% ८९॥

\twolineshloka
{नित्यमात्मसहायश्च देवासुरपतिः पतिः}%
{युक्तश्च युक्तबाहुश्च देवो दिवि सुपर्वणः}% ९०॥

\twolineshloka
{आषाढश्च सुषाढश्च ध्रुवोऽथ हरिणो हरः}%
{वपुरावर्तमानेभ्यो वसुश्रेष्ठो महापथः}% ९१॥

\twolineshloka
{शिरोहारी विमर्शश्च सर्वलक्षणलक्षितः}%
{अक्षश्च रथयोगी च सर्वयोगी महाबलः}% ९२॥

\twolineshloka
{समाम्नायोऽसमाम्नायस्तीर्थदेवो महारथः}%
{निर्जीवो जीवनो मन्त्रः शुभाक्षो बहुकर्कशः}% ९३॥

\twolineshloka
{रत्नप्रभूतो रक्ताङ्गो महार्णवनिपानवित्}%
{मूलं विशालो ह्यमृतो व्यक्ताव्यक्तस्तपोनिधिः}% ९४॥

\twolineshloka
{आरोहणोऽधिरोहश्च शीलधारी महायशाः}%
{सेनाकल्पो महाकल्पो योगो युगकरो हरिः}% ९५॥

\twolineshloka
{युगरूपो महारूपो महानागहनो वधः}%
{न्यायनिर्वपणः पादः पण्डितो ह्यचलोपमः}% ९६॥

\twolineshloka
{बहुमालो महामालः शशी हरसुलोचनः}%
{विस्तारो लवणः कूपस्त्रियुगः सफलोदयः}% ९७॥

\twolineshloka
{त्रिलोचनो विषण्णाङ्गो मणिविद्धो जटाधरः}%
{बिन्दुर्विसर्गः सुमुखः शरः सर्वायुधः सहः}% ९८॥

\twolineshloka
{निवेदनः सुखाजातः सुगन्धारो महाधनुः}%
{गन्धपाली च भगवानुत्थानः सर्वकर्मणाम्}% ९९॥

\twolineshloka
{मन्थानो बहुलो वायुः सकलः सर्वलोचनः}%
{तलस्तालः करस्थाली ऊर्ध्वसंहननो महान्}% १००॥

\twolineshloka
{छत्रं सुच्छत्रो विख्यातो लोकः सर्वाश्रयः क्रमः}%
{मुण्डो विरूपो विकृतो दण्डी कुण्डी विकुर्वणः}% १०१॥

\twolineshloka
{हर्यक्षः ककुभो वज्री शतजिह्वः सहस्रपात्}%
{सहस्रमूर्धा देवेन्द्रः सर्वदेवमयो गुरुः}% १०२॥

\twolineshloka
{सहस्रबाहुः सर्वाङ्गः शरण्यः सर्वलोककृत्}%
{पवित्रं त्रिककुन्मन्त्रः कनिष्ठः कृष्णपिङ्गलः}% १०३॥

\twolineshloka
{ब्रह्मदण्डविनिर्माता शतघ्नीपाशशक्तिमान्}%
{पद्मगर्भो महागर्भो ब्रह्मगर्भो जलोद्भवः}% १०४॥

\twolineshloka
{गभस्तिर्ब्रह्मकृद्-ब्रह्मी ब्रह्मविद्-ब्राह्मणो गतिः}%
{अनन्तरूपो नैकात्मा तिग्मतेजाः स्वयम्भुवः}% १०५॥

\twolineshloka
{ऊर्ध्वगात्मा पशुपतिर्वातरंहा मनोजवः}%
{चन्दनी पद्मनालाग्रः सुरभ्युत्तरणो नरः}% १०६॥

\twolineshloka
{कर्णिकारमहास्रग्वी नीलमौलिः पिनाकधृक्}%
{उमापतिरुमाकान्तो जाह्नवीधृगुमाधवः}% १०७॥

\twolineshloka
{वरो वराहो वरदो वरेण्यः सुमहास्वनः}%
{महाप्रसादो दमनः शत्रुहा श्वेतपिङ्गलः}% १०८॥

\twolineshloka
{पीतात्मा परमात्मा च प्रयतात्मा प्रधानधृक्}%
{सर्वपार्श्वमुखस्त्र्यक्षो धर्मसाधारणो वरः}% १०९॥

\twolineshloka
{चराचरात्मा सूक्ष्मात्मा ह्यमृतो गोवृषेश्वरः}%
{साध्यर्षिर्वसुरादित्यो विवस्वान् सविताऽमृतः}% ११०॥

\twolineshloka
{व्यासः सर्गः सुसङ्क्षेपो विस्तरः पर्ययो नरः}%
{ऋतुः संवत्सरो मासः पक्षः सङ्ख्यासमापनः}% १११॥

\twolineshloka
{कलाः काष्ठा लवा मात्रा मुहूर्ताहः क्षपाः क्षणाः}%
{विश्वक्षेत्रं प्रजाबीजं लिङ्गमाद्यस्तु निर्गमः}% ११२॥

\twolineshloka
{सदसद्व्यक्तमव्यक्तं पिता माता पितामहः}%
{स्वर्गद्वारं प्रजाद्वारं मोक्षद्वारं त्रिविष्टपम्}% ११३॥

\twolineshloka
{निर्वाणं ह्लादनश्चैव ब्रह्मलोकः परा गतिः}%
{देवासुरविनिर्माता देवासुरपरायणः}% ११४॥

\twolineshloka
{देवासुरगुरुर्देवो देवासुरनमस्कृतः}%
{देवासुरमहामात्रो देवासुरगणाश्रयः}% ११५॥

\twolineshloka
{देवासुरगणाध्यक्षो देवासुरगणाग्रणीः}%
{देवातिदेवो देवर्षिर्देवासुरवरप्रदः}% ११६॥

\twolineshloka
{देवासुरेश्वरो विश्वो देवासुरमहेश्वरः}%
{सर्वदेवमयोऽचिन्त्यो देवतात्माऽऽत्मसम्भवः}% ११७॥

\twolineshloka
{उद्भित्त्रिविक्रमो वैद्यो विरजो नीरजोऽमरः}%
{ईड्यो हस्तीश्वरो व्याघ्रो देवसिंहो नरर्षभः}% ११८॥

\twolineshloka
{विबुधोऽग्रवरः सूक्ष्मः सर्वदेवस्तपोमयः}%
{सुयुक्तः शोभनो वज्री प्रासानां प्रभवोऽव्ययः}% ११९॥

\twolineshloka
{गुहः कान्तो निजः सर्गः पवित्रं सर्वपावनः}%
{शृङ्गी शृङ्गप्रियो बभ्रू राजराजो निरामयः}% १२०॥

\twolineshloka
{अभिरामः सुरगणो विरामः सर्वसाधनः}%
{ललाटाक्षो विश्वदेवो हरिणो ब्रह्मवर्चसः}% १२१॥

\twolineshloka
{स्थावराणां पतिश्चैव नियमेन्द्रियवर्धनः}%
{सिद्धार्थः सिद्धभूतार्थोऽचिन्त्यः सत्यव्रतः शुचिः}% १२२॥

\twolineshloka
{व्रताधिपः परं ब्रह्म भक्तानां परमा गतिः}%
{विमुक्तो मुक्ततेजाश्च श्रीमान् श्रीवर्धनो जगत्}% १२३॥

श्रीमान् श्रीवर्धनो जगत् ॐ नम इति।

\resetShloka
\dnsub{उत्तरभागः}

\twolineshloka
{यथा प्रधानं भगवान् इति भक्त्या स्तुतो मया}
{यं न ब्रह्मादयो देवा विदुस्तत्त्वेन नर्षयः}%॥१॥

\twolineshloka
{स्तोतव्यमर्च्यं वन्द्यं च कः स्तोष्यति जगत्पतिम्}%।
{भक्तिं त्वेवं पुरस्कृत्य मया यज्ञपतिर्विभुः}%।


\twolineshloka
{ततोऽभ्यनुज्ञां सम्प्राप्य स्तुतो मतिमतां वरः}%॥२}%॥
{शिवमेभिः स्तुवन् देवं नामभिः पुष्टिवर्धनैः}%।


\twolineshloka
{नित्ययुक्तः शुचिर्भक्तः प्राप्नोत्यात्मानमात्मना}%॥३}%॥
{ऋषयश्चैव देवाश्च स्तुवन्त्येतेन तत्परम्}%।


\twolineshloka
{स्तूयमानो महादेवस्तुष्यते नियमात्मभिः}%॥४}%॥
{भक्तानुकम्पी भगवान् आत्मसंस्थाकरो विभुः}%।


\twolineshloka
{तथैव च मनुष्येषु ये मनुष्याः प्रधानतः}%॥५}%॥
{आस्तिकाः श्रद्दधानाश्च बहुभिर्जन्मभिः स्तवैः}%।


\twolineshloka
{भक्त्या ह्यनन्यमीशानं परं देवं सनातनम्}%॥६}%॥
{कर्मणा मनसा वाचा भावेनामिततेजसः}%।


\twolineshloka
{शयाना जाग्रमाणाश्च व्रजन्नुपविशंस्तथा}%॥७}%॥
{उन्मिषन्निमिषंश्चैव चिन्तयन्तः पुनः पुनः}%।

\twolineshloka
{शृण्वन्तः श्रावयन्तश्च कथयन्तश्च ते भवम्}%॥८}%॥
{स्तुवन्तः स्तूयमानाश्च तुष्यन्ति च रमन्ति च}%।


\twolineshloka
{जन्मकोटिसहस्रेषु नानासंसारयोनिषु}%॥९}%॥
{जन्तोर्विगतपापस्य भवे भक्तिः प्रजायते}%।


\twolineshloka
{उत्पन्ना च भवे भक्तिरनन्या सर्वभावतः}%॥१०}%॥
{भाविनः कारणे चास्य सर्वयुक्तस्य सर्वथा}%।


\twolineshloka
{एतद्देवेषु दुष्प्रापं मनुष्येषु न लभ्यते}%॥११}%॥
{निर्विघ्ना निश्चला रुद्रे भक्तिरव्यभिचारिणी}

\twolineshloka
{तस्यैव च प्रसादेन भक्तिरुत्पद्यते नृणाम्}%।
{येन यान्ति परां सिद्धिं तद्भावगतचेतसः}%॥१२}%॥

\twolineshloka
{ये सर्वभावानुगताः प्रपद्यन्ते महेश्वरम्}%।
{प्रपन्नवत्सलो देवः संसारात् तान् समुद्धरेत्}%॥१३}%॥

\twolineshloka
{एवम् अन्ये विकुर्वन्ति देवाः संसारमोचनम्}%।
{मनुष्याणामृते देवं नान्या शक्तिस्तपोबलम्}%॥१४}%॥

\twolineshloka
{इति तेनेन्द्रकल्पेन भगवान् सदसत्पतिः}%।
{कृत्तिवासाः स्तुतः कृष्ण तण्डिना शुद्धबुद्धिना}%॥१५}%॥

\twolineshloka
{स्तवमेतं  भगवतो ब्रह्मा स्वयमधारयत्}%।
{गीयते च स बुद्‌ध्येत ब्रह्मा शङ्करसन्निधौ}%॥१६}%॥

\twolineshloka
{इदं पुण्यं पवित्रं च सर्वदा पापनाशनम्}%।
{योगदं मोक्षदं चैव स्वर्गदं तोषदं तथा}%॥१७}%॥

\twolineshloka
{एवमेतत् पठन्ते य एकभक्त्या तु शङ्करम्}%।
{या गतिः साङ्ख्ययोगानां व्रजन्त्येतां गतिं तदा}%॥१८}%॥

\twolineshloka
{स्तवमेतं प्रयत्नेन सदा रुद्रस्य सन्निधौ}%।
{अब्दमेकं चरेद्भक्तः प्राप्नुयादीप्सितं फलम्}%॥१९}%॥

\twolineshloka
{एतद्रहस्यं परमं ब्रह्मणो हृदि संस्थितम्}%।
{ब्रह्मा प्रोवाच शक्राय शक्रः प्रोवाच मृत्यवे}%॥२०}%॥

\twolineshloka
{मृत्युः प्रोवाच रुद्रेभ्यो रुद्रेभ्यस्तण्डिमागमत्}%।
{महता तपसा प्राप्तस्तण्डिना ब्रह्मसद्मनि}%॥२१}%॥

\twolineshloka
{तण्डिः प्रोवाच शुक्राय गौतमाय च भार्गवः}%।
{वैवस्वताय मनवे गौतमः प्राह माधव}%॥२२}%॥

\twolineshloka
{नारायणाय साध्याय समाधिष्ठाय धीमते}%।
{यमाय प्राह भगवान् साध्यो नारायणोऽच्युतः}%॥२३}%॥

\twolineshloka
{नाचिकेताय भगवान् आह वैवस्वतो यमः}%।
{मार्कण्डेयाय वार्ष्णेय नाचिकेतोऽभ्यभाषत}%॥२४}%॥

\twolineshloka
{मार्कण्डेयान्मया प्राप्तं नियमेन जनार्दन}%।
{तवाप्यहम् अमित्रघ्न स्तवं दद्यां ह्यविश्रुतम्}%॥२५}%॥



\threelineshloka
{स्वर्ग्यमारोग्यमायुष्यं धन्यं वेदेन सम्मितम्}
{नास्य विघ्नं विकुर्वन्ति दानवा यक्षराक्षसाः}%।
{पिशाचा यातुधानाश्च गुह्यका भुजगा अपि}%॥२६}%॥

\twolineshloka
{यः पठेत शुचिर्भूत्वा ब्रह्मचारी जितेन्द्रियः}%।
{अभग्नयोगो वर्षं तु सोऽश्वमेधफलं लभेत्}%॥२७}%॥

जैगीषव्य उवाच
\twolineshloka
{ममाष्टगुणमैश्वर्यं दत्तं भगवता पुरा}%।
{यत्नेनान्येन बलिना वाराणस्यां युधिष्ठिर}%॥२८}%॥

वाराणस्यां युधिष्ठिर ॐ नम इति।\\

\newpage
गर्ग उवाच
\twolineshloka
{चतुःषष्ट्यङ्गमददत् कलाज्ञानं ममाद्भुतम्}%।
{सरस्वत्यास्तटे तुष्टो मनोयज्ञेन पाण्डव}\mbox{}\\[-1.6em]%॥२९}%॥
मनोयज्ञेन पाण्डव ॐ नम इति।

वैशम्पायन उवाच
\threelineshloka
{ततः कृष्णोऽब्रवीद्वाक्यं पुनर्मतिमतां वरः}
{युधिष्ठिरं धर्मनिधिं पुरुहूतमिवेश्वरः}%।
{उपमन्युर्मयि प्राह तपन्निव दिवाकरः}%\mbox{}\\[-4ex]%॥३०}%॥

\threelineshloka
{अशुभैः पापकर्माणो ये नराः कलुषीकृताः}
{ईशानं न प्रपद्यन्ते तमोराजसवृत्तयः}%।
{ईश्वरं सम्प्रपद्यन्ते द्विजा भावितभावनाः}%॥३१}%॥


\twolineshloka
{एवमेव महादेव भक्ता ये मानवा भुवि}%।
{न ते संसारवशगा इति मे निश्चिता मतिः}\mbox{}\\[-1.6em]%॥३२}%॥
इति मे निश्चिता मतिः ॐ नम इति।

॥इति श्रीमन्महाभारते शतसाहस्र्यां संहितायां वैयासिक्याम् आनुशासनिकपर्वणि अष्टादशोऽध्यायः॥

\fourlineindentedshloka*
{दुःस्वप्न-दुःशकुन-दुर्गति-दौर्मनस्य}
{दुर्भिक्ष-दुर्व्यसन-दुःसह-दुर्यशांसि}
{उत्पात-ताप-विषभीतिम् असद्‌-ग्रहार्तिम्}
{व्याधींश्च नाशयतु मे जगतामधीशः}

॥इति श्री~शिवसहस्रनामस्तोत्रं सम्पूर्णम्॥