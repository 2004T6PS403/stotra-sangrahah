% !TeX program = XeLaTeX
% !TeX root = ../../shloka.tex
\sect{सूर्यसहस्रनामस्तोत्रम्}
शतानीक उवाच

\twolineshloka
{नाम्नां सहस्रं सवितुः श्रोतुमिच्छामि हे द्विज}
{येन ते दर्शनं यातः साक्षाद्देवो दिवाकरः}

\twolineshloka
{सर्वमङ्गलमङ्गल्यं सर्वपापप्रणाशनम्}
{स्तोत्रमेतन्महापुण्यं सर्वोपद्रवनाशनम्}

\twolineshloka
{न तदस्ति भयं किञ्चिद्यदनेन न नश्यति}
{ज्वराद्यैर्मुच्यते राजन् स्तोत्रेऽस्मिन् पठिते नरः}

\twolineshloka
{अन्ये च रोगाः शाम्यन्ति पठतः शृण्वतस्तथा}
{सम्पद्यन्ते यथा कामाः सर्व एव यथेप्सिताः}

\twolineshloka
{य एतदादितः श्रुत्वा सङ्ग्रामं प्रविशेन्नरः}
{स जित्वा समरे शत्रूनभ्येति गृहमक्षतः}

\twolineshloka
{वन्ध्यानां पुत्रजननं भीतानां भयनाशनम्}
{भूतिकारि दरिद्राणां कुष्ठिनां परमौषधम्}

\twolineshloka
{बालानां चैव सर्वेषां ग्रहरक्षोनिवारणम्}
{पठते संयतो राजन् स श्रेयः परमाप्नुयात्}

\twolineshloka
{स सिद्धः सर्वसङ्कल्पः सुखमत्यन्तमश्नुते}
{धर्मार्थिभिर्धर्मलुब्धैः सुखाय च सुखार्थिभिः}

\twolineshloka
{राज्याय राज्यकामैश्च पठितव्यमिदं नरैः}
{विद्यावहं तु विप्राणां क्षत्रियाणां जयावहम्}

\twolineshloka
{पश्वाहं तु वैश्यानां शूद्राणां धर्मवर्धनम्}
{पठतां शृण्वतामेतद्भवतीति न संशयः}

\twolineshloka
{तच्छृणुष्व नृपश्रेष्ठ प्रयतात्मा ब्रवीमि ते}
{नाम्नां सहस्रं विख्यातं देवदेवस्य धीमतः}

%\dnsub{न्यासः}
%अस्य श्रीसूर्यसहस्रनामस्तोत्रस्य भगवान् वेदव्यास ऋषिः।\\
%अनुष्टुप् छन्दः। सविता देवता। सकलाभीष्टसिद्ध्यर्थे जपे विनियोगः।\\

\dnsub{ध्यानम्}
\fourlineindentedshloka*
{ध्येयः सदा सवितृमण्डलमध्यवर्ती}
{नारायणः सरसिजासनसन्निविष्टः}
{केयूरवान् मकरकुण्डलवान् किरीटी}
{हारी हिरण्मयवपुर्धृतशङ्खचक्रः}

\dnsub{स्तोत्रम्}
\resetShloka
\twolineshloka
{ॐ विश्वविद्विश्वजित्कर्ता विश्वात्मा विश्वतोमुखः}
{विश्वेश्वरो विश्वयोनिर्नियतात्मा जितेन्द्रियः}% १

\twolineshloka
{कालाश्रयः कालकर्ता कालहा कालनाशनः}
{महायोगी महासिद्धिर्महात्मा सुमहाबलः}% २

\twolineshloka
{प्रभुर्विभुर्भूतनाथो भूतात्मा भुवनेश्वरः}
{भूतभव्यो भावितात्मा भूतान्तःकरणं शिवः}% ३

\twolineshloka
{शरण्यः कमलानन्दो नन्दनो नन्दवर्धनः}
{वरेण्यो वरदो योगी सुसंयुक्तः प्रकाशकः} %४

\twolineshloka
{प्राप्तयानः परप्राणः पूतात्मा प्रयतः प्रियः}
{नयः सहस्रपात् साधुर्दिव्यकुण्डलमण्डितः}% ५

\twolineshloka
{अव्यङ्गधारी धीरात्मा सविता वायुवाहनः}
{समाहितमतिर्दाता विधाता कृतमङ्गलः}% ६

\twolineshloka
{कपर्दी कल्पपाद्रुद्रः सुमना धर्मवत्सलः}
{समायुक्तो विमुक्तात्मा कृतात्मा कृतिनां वरः}% ७

\twolineshloka
{अविचिन्त्यवपुः श्रेष्ठो महायोगी महेश्वरः}
{कान्तः कामारिरादित्यो नियतात्मा निराकुलः}% ८

\twolineshloka
{कामः कारुणिकः कर्ता कमलाकरबोधनः}
{सप्तसप्तिरचिन्त्यात्मा महाकारुणिकोत्तमः}% ९

\twolineshloka
{सञ्जीवनो जीवनाथो जयो जीवो जगत्पतिः}
{अयुक्तो विश्वनिलयः संविभागी वृषध्वजः}% १०

\twolineshloka
{वृषाकपिः कल्पकर्ता कल्पान्तकरणो रविः}
{एकचक्ररथो मौनी सुरथो रथिनां वरः}% ११

\twolineshloka
{सक्रोधनो रश्मिमाली तेजोराशिर्विभावसुः}
{दिव्यकृद्दिनकृद्देवो देवदेवो दिवस्पतिः}% १२

\twolineshloka
{दीननाथो हरो  होता दिव्यबाहुर्दिवाकरः}
{यज्ञो यज्ञपतिः पूषा स्वर्णरेताः परावरः}% १३

\twolineshloka
{परापरज्ञस्तरणिरंशुमाली मनोहरः}
{प्राज्ञः प्राज्ञपतिः सूर्यः सविता विष्णुरंशुमान्}% १४

\twolineshloka
{सदागतिर्गन्धवहो विहितो विधिराशुगः}
{पतङ्गः पतगः स्थाणुर्विहङ्गो विहगो वरः}% १५

\twolineshloka
{हर्यश्वो हरिताश्वश्च हरिदश्वो जगत्प्रियः}
{त्र्यम्बकः सर्वदमनो भावितात्मा भिषग्वरः}% १६

\twolineshloka
{आलोककृल्लोकनाथो लोकालोकनमस्कृतः}
{कालः कल्पान्तको वह्निस्तपनः सम्प्रतापनः}% १७

\twolineshloka
{विरोचनो विरूपाक्षः सहस्राक्षः पुरन्दरः}
{सहस्ररश्मिर्मिहिरो विविधाम्बरभूषणः}% १८

\twolineshloka
{खगः प्रतर्दनो धन्यो हयगो वाग्विशारदः}
{श्रीमानशिशिरो वाग्मी श्रीपतिः श्रीनिकेतनः}% १९

\twolineshloka
{श्रीकण्ठः श्रीधरः श्रीमान्  श्रीनिवासो वसुप्रदः}
{कामचारी महामायो महोग्रोऽविदितामयः}% २०

\twolineshloka
{तीर्थक्रियावान् सुनयो विभक्तो भक्तवत्सलः}
{कीर्तिः कीर्तिकरो नित्यः कुण्डली कवची रथी}% २१

\twolineshloka
{हिरण्यरेताः सप्ताश्वः प्रयतात्मा परन्तपः}
{बुद्धिमानमरश्रेष्ठो रोचिष्णुः पाकशासनः}% २२

\twolineshloka
{समुद्रो धनदो धाता मान्धाता कश्मलापहः}
{तमोघ्नो ध्वान्तहा वह्निर्होताऽन्तःकरणो गुहः}% २३

\twolineshloka
{पशुमान् प्रयतानन्दो भूतेशः श्रीमतां वरः}
{नित्योऽदितो नित्यरथः सुरेशः सुरपूजितः}% २४

\twolineshloka
{अजितो विजितो जेता जङ्गमस्थावरात्मकः}
{जीवानन्दो नित्यगामी विजेता विजयप्रदः}% २५

\twolineshloka
{पर्जन्योऽग्निः स्थितिः स्थेयः स्थविरोऽथ निरञ्जनः}
{प्रद्योतनो रथारूढः सर्वलोकप्रकाशकः}% २६

\twolineshloka
{ध्रुवो मेषी महावीर्यो हंसः संसारतारकः}
{सृष्टिकर्ता क्रियाहेतुर्मार्तण्डो मरुतां पतिः}% २७

\twolineshloka
{मरुत्वान् दहनस्त्वष्टा भगो भर्गोऽर्यमा कपिः}
{वरुणेशो जगन्नाथः कृतकृत्यः सुलोचनः}% २८

\twolineshloka
{विवस्वान् भानुमान् कार्यः कारणस्तेजसां निधिः}
{असङ्गगामी तिग्मांशुर्घर्मांशुर्दीप्तदीधितिः}% २९

\twolineshloka
{सहस्रदीधितिर्ब्रध्नः सहस्रांशुर्दिवाकरः}
{गभस्तिमान् दीधितिमान् स्रग्वी मणिकुलद्युतिः}% ३०

\twolineshloka
{भास्करः सुरकार्यज्ञः सर्वज्ञस्तीक्ष्णदीधितिः}
{सुरज्येष्ठः सुरपतिर्बहुज्ञो वचसां पतिः}% ३१

\twolineshloka
{तेजोनिधिर्बृहत्तेजा बृहत्कीर्तिर्बृहस्पतिः}
{अहिमानूर्जितो धीमानामुक्तः कीर्तिवर्धनः}% ३२

\twolineshloka
{महावैद्यो गणपतिर्धनेशो गणनायकः}
{तीव्रप्रतापनस्तापी तापनो विश्वतापनः}% ३३

\twolineshloka
{कार्तस्वरो हृषीकेशः पद्मानन्दोऽतिनन्दितः}
{पद्मनाभोऽमृताहारः स्थितिमान् केतुमान् नभः}% ३४

\twolineshloka
{अनाद्यन्तोऽच्युतो विश्वो विश्वामित्रो घृणिर्विराट्}
{आमुक्तकवचो वाग्मी कञ्चुकी विश्वभावनः}% ३५

\twolineshloka
{अनिमित्तगतिः श्रेष्ठः शरण्यः सर्वतोमुखः}
{विगाही वेणुरसहः समायुक्तः समाक्रतुः}% ३६

\twolineshloka
{धर्मकेतुर्धर्मरतिः संहर्ता संयमो यमः}
{प्रणतार्तिहरो वायुः सिद्धकार्यो जनेश्वरः}% ३७

\twolineshloka
{नभो विगाहनः सत्यः सवितात्मा मनोहरः}
{हारी हरिर्हरो वायुरृतुः कालानलद्युतिः}% ३८

\twolineshloka
{सुखसेव्यो महातेजा जगतामेककारणम्}
{महेन्द्रो विष्टुतः स्तोत्रं स्तुतिहेतुः प्रभाकरः}% ३९

\twolineshloka
{सहस्रकर आयुष्मान् अरोषः सुखदः सुखी}
{व्याधिहा सुखदः सौख्यं कल्याणः कलतां वरः}% ४०

\twolineshloka
{आरोग्यकारणं सिद्धिरृद्धिर्वृद्धिर्बृहस्पतिः}
{हिरण्यरेता आरोग्यं विद्वान् ब्रध्नो बुधो महान्}% ४१

\twolineshloka
{प्राणवान् धृतिमान् घर्मो घर्मकर्ता रुचिप्रदः}
{सर्वप्रियः सर्वसहः सर्वशत्रुविनाशनः}% ४२

\twolineshloka
{प्रांशुर्विद्योतनो द्योतः सहस्रकिरणः कृती}
{केयूरी भूषणोद्भासी भासितो भासनोऽनलः}% ४३

\twolineshloka
{शरण्यार्तिहरो होता खद्योतः खगसत्तमः}
{सर्वद्योतो भवद्योतः सर्वद्युतिकरो मतः}% ४४

\twolineshloka
{कल्याणः कल्याणकरः कल्यः कल्यकरः कविः}
{कल्याणकृत् कल्यवपुः सर्वकल्याणभाजनम्}% ४५

\twolineshloka
{शान्तिप्रियः प्रसन्नात्मा प्रशान्तः प्रशमप्रियः}
{उदारकर्मा सुनयः सुवर्चा वर्चसोज्ज्वलः}% ४६

\twolineshloka
{वर्चस्वी वर्चसामीशस्त्रैलोक्येशो वशानुगः}
{तेजस्वी सुयशा वर्ष्मी वर्णाध्यक्षो बलिप्रियः}% ४७

\twolineshloka
{यशस्वी तेजोनिलयस्तेजस्वी प्रकृतिस्थितः}
{आकाशगः शीघ्रगतिराशुगो गतिमान् खगः}% ४८

\twolineshloka
{गोपतिर्ग्रहदेवेशो गोमानेकः प्रभञ्जनः}
{जनिता प्रजनो जीवो दीपः सर्वप्रकाशकः}% ४९

\twolineshloka
{सर्वसाक्षी योगनित्यो नभस्वानसुरान्तकः}
{रक्षोघ्नो विघ्नशमनः किरीटी सुमनःप्रियः}% ५०

\twolineshloka
{मरीचिमाली सुमतिः कृताभिख्यविशेषकः}
{शिष्टाचारः शुभाकारः स्वचाराचारतत्परः}% ५१

\twolineshloka
{मन्दारो माठरो वेणुः क्षुधापः क्ष्मापतिर्गुरुः}
{सुविशिष्टो विशिष्टात्मा विधेयो ज्ञानशोभनः}% ५२

\twolineshloka
{महाश्वेतः प्रियो ज्ञेयः सामगो मोक्षदायकः}
{सर्ववेदप्रगीतात्मा सर्ववेदलयो महान्}% ५३

\twolineshloka
{वेदमूर्तिश्चतुर्वेदो वेदभृद्वेदपारगः}
{क्रियावानसितो जिष्णुर्वरीयांशुर्वरप्रदः}% ५४

\twolineshloka
{व्रतचारी व्रतधरो लोकबन्धुरलङ्कृतः}
{अलङ्काराक्षरो वेद्यो विद्यावान् विदिताशयः}% ५५

\twolineshloka
{आकारो भूषणो भूष्यो भूष्णुर्भुवनपूजितः}
{चक्रपाणिर्ध्वजधरः सुरेशो लोकवत्सलः}% ५६

\twolineshloka
{वाग्मिपतिर्महाबाहुः प्रकृतिर्विकृतिर्गुणः}
{अन्धकारापहः श्रेष्ठो युगावर्तो युगादिकृत्}% ५७

\twolineshloka
{अप्रमेयः सदायोगी निरहङ्कार ईश्वरः}
{शुभप्रदः शुभः शास्ता शुभकर्मा शुभप्रदः}% ५८

\twolineshloka
{सत्यवान् श्रुतिमानुच्चैर्नकारो वृद्धिदोऽनलः}
{बलभृद्बलदो बन्धुर्मतिमान् बलिनां वरः}% ५९

\twolineshloka
{अनङ्गो नागराजेन्द्रः पद्मयोनिर्गणेश्वरः}
{संवत्सर ऋतुर्नेता कालचक्रप्रवर्तकः}% ६०

\twolineshloka
{पद्मेक्षणः पद्मयोनिः प्रभावानमरः प्रभुः}
{सुमूर्तिः सुमतिः सोमो गोविन्दो जगदादिजः}% ६१

\twolineshloka
{पीतवासाः कृष्णवासा दिग्वासास्त्विन्द्रियातिगः}
{अतीन्द्रियोऽनेकरूपः स्कन्दः परपुरञ्जयः}% ६२

\twolineshloka
{शक्तिमाञ्जलधृग्भास्वान् मोक्षहेतुरयोनिजः}
{सर्वदर्शी  जितादर्शो दुःस्वप्नाशुभनाशनः}% ६३

\twolineshloka
{माङ्गल्यकर्ता तरणिर्वेगवान् कश्मलापहः}
{स्पष्टाक्षरो महामन्त्रो विशाखो यजनप्रियः}% ६४

\twolineshloka
{विश्वकर्मा महाशक्तिर्द्युतिरीशो विहङ्गमः}
{विचक्षणो दक्ष इन्द्रः प्रत्यूषः प्रियदर्शनः}% ६५

\twolineshloka
{अखिन्नो वेदनिलयो वेदविद्विदिताशयः}
{प्रभाकरो जितरिपुः सुजनोऽरुणसारथिः}% ६६

\twolineshloka
{कुनाशी सुरतः स्कन्दो महितोऽभिमतो गुरुः}
{ग्रहराजो ग्रहपतिर्ग्रहनक्षत्रमण्डलः}% ६७

\twolineshloka
{भास्करः सततानन्दो नन्दनो नरवाहनः}
{मङ्गलोऽथ मङ्गलवान् माङ्गल्यो मङ्गलावहः}% ६८

\twolineshloka
{मङ्गल्यचारुचरितः शीर्णः सर्वव्रतो व्रती}
{चतुर्मुखः पद्ममाली पूतात्मा प्रणतार्तिहा}% ६९

\twolineshloka
{अकिञ्चनः सतामीशो निर्गुणो गुणवाञ्छुचिः}
{सम्पूर्णः पुण्डरीकाक्षो विधेयो योगतत्परः}% ७०

\twolineshloka
{सहस्रांशुः क्रतुमतिः सर्वज्ञः सुमतिः सुवाक्}
{सुवाहनो माल्यदामा कृताहारो हरिप्रियः}% ७१

\twolineshloka
{ब्रह्मा प्रचेताः प्रथितः प्रयतात्मा स्थिरात्मकः}
{शतविन्दुः शतमुखो गरीयाननलप्रभः}% ७२

\twolineshloka
{धीरो महत्तरो विप्रः पुराणपुरुषोत्तमः}
{विद्याराजाधिराजो हि विद्यावान् भूतिदः स्थितः}%७३

\twolineshloka
{अनिर्देश्यवपुः श्रीमान् विपाप्मा बहुमङ्गलः}
{स्वःस्थितः सुरथः स्वर्णो मोक्षदो बलिकेतनः}% ७४

\twolineshloka
{निर्द्वन्द्वो द्वन्द्वहा सर्गः सर्वगः सम्प्रकाशकः}
{दयालुः सूक्ष्मधीः क्षान्तिः क्षेमाक्षेमस्थितिप्रियः}% ७५

\twolineshloka
{भूधरो भूपतिर्वक्ता पवित्रात्मा त्रिलोचनः}
{महावराहः प्रियकृद्दाता भोक्ताऽभयप्रदः}% ७६

\twolineshloka
{चक्रवर्ती धृतिकरः सम्पूर्णोऽथ महेश्वरः}
{चतुर्वेदधरोऽचिन्त्यो विनिन्द्यो विविधाशनः}% ७७

\twolineshloka
{विचित्ररथ एकाकी सप्तसप्तिः परात्परः}
{सर्वोदधिस्थितिकरः स्थितिस्थेयः स्थितिप्रियः}% ७८

\twolineshloka
{निष्कलः पुष्कलो विभुर्वसुमान् वासवप्रियः}
{पशुमान् वासवस्वामी वसुधामा वसुप्रदः}% ७९

\twolineshloka
{बलवान् ज्ञानवांस्तत्त्वमोङ्कारस्त्रिषु संस्थितः}
{सङ्कल्पयोनिर्दिनकृद्भगवान् कारणापहः}% ८०

\twolineshloka
{नीलकण्ठो धनाध्यक्षश्चतुर्वेदप्रियंवदः}
{वषट्कारोद्गाता होता स्वाहाकारो हुताहुतिः}% ८१

\twolineshloka
{जनार्दनो जनानन्दो नरो नारायणोऽम्बुदः}
{सन्देहनाशनो वायुर्धन्वी सुरनमस्कृतः}% ८२

\twolineshloka
{विग्रही विमलो विन्दुर्विशोको विमलद्युतिः}
{द्युतिमान् द्योतनो विद्युद्विद्यावान् विदितो बली}% ८३

\twolineshloka
{घर्मदो हिमदो हासः कृष्णवर्त्मा सुताजितः}
{सावित्रीभावितो राजा विश्वामित्रो घृणिर्विराट्}% ८४

\twolineshloka
{सप्तार्चिः सप्ततुरगः सप्तलोकनमस्कृतः}
{सम्पूर्णोऽथ जगन्नाथः सुमनाः शोभनप्रियः}% ८५

\twolineshloka
{सर्वात्मा सर्वकृत् सृष्टिः सप्तिमान् सप्तमीप्रियः}
{सुमेधा मेधिको मेध्यो मेधावी मधुसूदनः}% ८६

\twolineshloka
{अङ्गिरःपतिः कालज्ञो धूमकेतुः सुकेतनः}
{सुखी सुखप्रदः सौख्यं कामी कान्तिप्रियो मुनिः}% ८७

\twolineshloka
{सन्तापनः सन्तपन आतपः तपसां पतिः}
{उमापतिः सहस्रांशुः प्रियकारी प्रियङ्करः}% ८८

\twolineshloka
{प्रीतिर्विमन्युरम्भोत्थः खञ्जनो जगतां पतिः}
{जगत्पिता प्रीतमनाः सर्वः खर्वो गुहोऽचलः}% ८९

\twolineshloka
{सर्वगो जगदानन्दो जगन्नेता सुरारिहा}
{श्रेयः श्रेयस्करो ज्यायान् महानुत्तम उद्भवः}% ९०

\twolineshloka
{उत्तमो मेरुमेयोऽथ धरणो धरणीधरः}
{धराध्यक्षो धर्मराजो धर्माधर्मप्रवर्तकः}% ९१

\twolineshloka
{रथाध्यक्षो रथगतिस्तरुणस्तनितोऽनलः}
{उत्तरोऽनुत्तरस्तापी अवाक्पतिरपां पतिः}% ९२

\twolineshloka
{पुण्यसङ्कीर्तनः पुण्यो हेतुर्लोकत्रयाश्रयः}
{स्वर्भानुर्विगतानन्दो विशिष्टोत्कृष्टकर्मकृत्}% ९३

\twolineshloka
{व्याधिप्रणाशनः क्षेमः शूरः सर्वजितां वरः}
{एकरथो रथाधीशः पिता शनैश्चरस्य हि}% ९४

\twolineshloka
{वैवस्वतगुरुर्मृत्युर्धर्मनित्यो महाव्रतः}
{प्रलम्बहारसञ्चारी प्रद्योतो द्योतितानलः}% ९५

\twolineshloka
{सन्तापहृत् परो मन्त्रो मन्त्रमूर्तिर्महाबलः}
{श्रेष्ठात्मा सुप्रियः शम्भुर्मरुतामीश्वरेश्वरः}% ९६

\twolineshloka
{संसारगतिविच्छेता संसारार्णवतारकः}
{सप्तजिह्वः सहस्रार्ची रत्नगर्भोऽपराजितः}% ९७

\twolineshloka
{धर्मकेतुरमेयात्मा धर्माधर्मवरप्रदः}
{लोकसाक्षी लोकगुरुर्लोकेशश्चण्डवाहनः}% ९८

\twolineshloka
{धर्मयूपो यूपवृक्षो धनुष्पाणिर्धनुर्धरः}
{पिनाकधृङ्महोत्साहो महामायो महाशनः}% ९९

\twolineshloka
{वीरः शक्तिमतां श्रेष्ठः सर्वशस्त्रभृतां वरः}
{ज्ञानगम्यो दुराराध्यो लोहिताङ्गो विवर्धनः}% १००

\twolineshloka
{खगोऽन्धो धर्मदो नित्यो धर्मकृच्चित्रविक्रमः}
{भगवानात्मवान् मन्त्रस्त्र्यक्षरो नीललोहितः}% १०१

\twolineshloka
{एकोऽनेकस्त्रयी कालः सविता समितिञ्जयः}
{शार्ङ्गधन्वाऽनलो भीमः सर्वप्रहरणायुधः}% १०२

\twolineshloka
{सुकर्मा परमेष्ठी च नाकपाली दिविस्थितः}
{वदान्यो वासुकिर्वैद्य आत्रेयोऽथ पराक्रमः}% १०३

\twolineshloka
{द्वापरः परमोदारः परमो ब्रह्मचर्यवान्}
{उदीच्यवेषो मुकुटी पद्महस्तो हिमांशुभृत्}% १०४

\twolineshloka
{सितः प्रसन्नवदनः पद्मोदरनिभाननः}
{सायं दिवा दिव्यवपुरनिर्देश्यो महालयः}% १०५

\twolineshloka
{महारथो महानीशः शेषः सत्त्वरजस्तमः}
{धृतातपत्रप्रतिमो विमर्षी निर्णयः स्थितः}% १०६

\twolineshloka
{अहिंसकः शुद्धमतिरद्वितीयो विवर्धनः}
{सर्वदो धनदो मोक्षो विहारी बहुदायकः}% १०७

\twolineshloka
{चारुरात्रिहरो नाथो भगवान् सर्वगोऽव्ययः}
{मनोहरवपुः शुभ्रः शोभनः सुप्रभावनः}% १०८

\twolineshloka
{सुप्रभावः सुप्रतापः सुनेत्रो दिग्विदिक्पतिः}
{राज्ञीप्रियः शब्दकरो ग्रहेशस्तिमिरापहः}% १०९

\twolineshloka
{सैंहिकेयरिपुर्देवो वरदो वरनायकः}
{चतुर्भुजो महायोगी योगीश्वरपतिस्तथा}% ११०

\twolineshloka
{अनादिरूपोऽदितिजो रत्नकान्तिः प्रभामयः}
{जगत्प्रदीपो विस्तीर्णो महाविस्तीर्णमण्डलः}% १११

\twolineshloka
{एकचक्ररथः स्वर्णरथः स्वर्णशरीरधृक्}
{निरालम्बो गगनगो धर्मकर्मप्रभावकृत्}% ११२

\twolineshloka
{धर्मात्मा कर्मणां  साक्षी प्रत्यक्षः परमेश्वरः}
{मेरुसेवी सुमेधावी मेरुरक्षाकरो महान्}% ११३

\twolineshloka
{आधारभूतो रतिमांस्तथा च धनधान्यकृत्}
{पापसन्तापहर्ता मनोवाञ्छितदायकः}% ११४

\twolineshloka
{रोगहर्ता राज्यदायी रमणीयगुणोऽनृणी}
{कालत्रयानन्तरूपो मुनिवृन्दनमस्कृतः}% ११५

\twolineshloka
{सन्ध्यारागकरः सिद्धः सन्ध्यावन्दनवन्दितः}
{साम्राज्यदाननिरतः समाराधनतोषवान्}% ११६

\threelineshloka
{भक्तदुःखक्षयकरो भवसागरतारकः}
{भयापहर्ता भगवानप्रमेयपराक्रमः}
{मनुस्वामी मनुपतिर्मान्यो मन्वन्तराधिपः}% ११७

\dnsub{फलश्रुतिः}
\resetShloka
\twolineshloka
{एतत्ते सर्वमाख्यातं यन्मां त्वं परिपृच्छसि}
{नाम्नां सहस्रं सवितुः पाराशर्यो यदाह मे}

\twolineshloka
{धन्यं यशस्यमायुष्यं दुःखदुःस्वप्ननाशनम्}
{बन्धमोक्षकरं चैव भानोर्नामानुकीर्तनात्}

\twolineshloka
{यस्त्विदं शृणुयान्नित्यं पठेद्वा प्रयतो नरः}
{अक्षयं सुखमन्नाद्यं भवेत्तस्योपसाधितम्}

\twolineshloka
{नृपाग्नितस्करभयं व्याधितो न भयं भवेत्}
{विजयी च भवेन्नित्यमाश्रयं परमाप्नुयात्}

\twolineshloka
{कीर्तिमान् सुभगो विद्वान् स सुखी प्रियदर्शनः}
{जीवेद्वर्षशतायुश्च सर्वव्याधिविवर्जितः}

\fourlineindentedshloka
{नाम्नां सहस्रमिदमंशुमतः पठेद्यः}
{प्रातः शुचिर्नियमवान् सुसमृद्धियुक्तः}
{दूरेण तं परिहरन्ति सदैव रोगाः}
{भूताः सुपर्णमिव सर्वमहोरगेन्द्राः}

॥इति श्री भविष्यपुराणे सप्तमकल्पे श्रीभगवत्सूर्यस्य सहस्रनामस्तोत्रं सम्पूर्णम्॥