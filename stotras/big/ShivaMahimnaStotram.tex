% !TeX program = XeLaTeX
% !TeX root = ../../shloka.tex

\sect{शिवमहिम्नः स्तोत्रम्}
\fourlineindentedshloka
{महिम्नः पारं ते परमविदुषो यद्यसदृशी}
{स्तुतिर्ब्रह्मादीनामपि तदवसन्नास्त्वयि गिरः}
{अथाऽवाच्यः सर्वः स्वमतिपरिणामावधि गृणन्}
{ममाप्येष स्तोत्रे हर निरपवादः परिकरः}%1

\fourlineindentedshloka
{अतीतः पन्थानं तव च महिमा वाङ्मनसयोः}
{अतद्-व्यावृत्त्या यं चकितमभिधत्ते श्रुतिरपि}
{स कस्य स्तोतव्यः कतिविधगुणः कस्य विषयः}
{पदे त्वर्वाचीने पतति न मनः कस्य न वचः}%2

\fourlineindentedshloka
{मधुस्फीता वाचः परमममृतं निर्मितवतः}
{तव ब्रह्मन् किं वागपि सुरगुरोर्विस्मयपदम्}
{मम त्वेतां वाणीं गुणकथनपुण्येन भवतः}
{पुनामीत्यर्थेऽस्मिन् पुरमथन बुद्धिर्व्यवसिता}%3

\fourlineindentedshloka
{तवैश्वर्यं यत्तज्जगदुदयरक्षाप्रलयकृत्}
{त्रयीवस्तु व्यस्तं तिस्रुषु गुणभिन्नासु तनुषु}
{अभव्यानामस्मिन् वरद रमणीयामरमणीम्}
{विहन्तुं व्याक्रोशीं विदधत इहैके जडधियः}%4

\fourlineindentedshloka
{किमीहः किं कायः स खलु किमुपायस्त्रिभुवनम्}
{किमाधारो धाता सृजति किमुपादान इति च}
{अतर्क्यैश्वर्ये त्वय्यनवसर दुःस्थो हतधियः}
{कुतर्कोऽयं कांश्चित् मुखरयति मोहाय जगतः}%5

\fourlineindentedshloka
{अजन्मानो लोकाः किमवयववन्तोऽपि जगताम्}
{अधिष्ठातारं किं भवविधिरनादृत्य भवति}
{अनीशो वा कुर्याद्भुवनजनने कः परिकरो}
{यतो मन्दास्त्वां प्रत्यमरवर संशेरत इमे}%6

\fourlineindentedshloka
{त्रयी साङ्ख्यं योगः पशुपतिमतं वैष्णवमिति}
{प्रभिन्ने प्रस्थाने परमिदमदः पथ्यमिति च}
{रुचीनां वैचित्र्यादृजुकुटिल नानापथजुषां}
{नृणामेको गम्यस्त्वमसि पयसामर्णव इव}%7

\fourlineindentedshloka
{महोक्षः खट्वाङ्गं परशुरजिनं भस्मफणिनः}
{कपालं चेतीयत्तव वरद तन्त्रोपकरणम्}
{सुरास्तां तामृद्धिं दधति तु भवद्भूप्रणिहितां}
{न हि स्वात्मारामं विषयमृगतृष्णा भ्रमयति}%8

\fourlineindentedshloka
{ध्रुवं कश्चित् सर्वं सकलमपरस्त्वध्रुवमिदं}
{परो ध्रौव्याऽध्रौव्ये जगति गदति व्यस्तविषये}
{समस्तेऽप्येतस्मिन् पुरमथन तैर्विस्मित इव}
{स्तुवन् जिह्रेमि त्वां न खलु ननु धृष्टा मुखरता}%9

\fourlineindentedshloka
{तवैश्वर्यं यत्नाद्-यदुपरि विरिञ्चिर्हरिरधः}
{परिच्छेतुं यातावनिलमनलस्कन्धवपुषः}
{ततो भक्तिश्रद्धा-भरगुरु-गृणद्भ्यां गिरिश यत्}
{स्वयं तस्थे ताभ्यां तव किमनुवृत्तिर्न फलति}%10

\fourlineindentedshloka
{अयत्नादासाद्य त्रिभुवनमवैरव्यतिकरं}
{दशास्यो यद्बाहूनभृत-रणकण्डू-परवशान्}
{शिरःपद्मश्रेणी-रचितचरणाम्भोरुहबलेः}
{स्थिरायास्त्वद्भक्तेस्त्रिपुरहर विस्फूर्जितमिदम्}%11

\fourlineindentedshloka
{अमुष्य त्वत्सेवा-समधिगतसारं भुजवनम्}
{बलात् कैलासेऽपि त्वदधिवसतौ विक्रमयतः}
{अलभ्यापातालेऽप्यलसचलिताङ्गुष्ठशिरसि}
{प्रतिष्ठा त्वय्यासीद्-ध्रुवमुपचितो मुह्यति खलः}%12

\fourlineindentedshloka
{यदृद्धिं सुत्राम्णो वरद परमोच्चैरपि सतीं}
{अधश्चक्रे बाणः परिजनविधेयत्रिभुवनः}
{न तच्चित्रं तस्मिन् वरिवसितरि त्वच्चरणयोः}
{न कस्याप्युन्नत्यै भवति शिरसस्त्वय्यवनतिः}%13

\fourlineindentedshloka
{अकाण्ड-ब्रह्माण्ड-क्षयचकित-देवासुरकृपा}
{विधेयस्याऽऽसीद्-यस्त्रिनयन विषं संहृतवतः}
{स कल्माषः कण्ठे तव न कुरुते न श्रियमहो}
{विकारोऽपि श्लाघ्यो भुवन-भयभङ्ग-व्यसनिनः}%14

\fourlineindentedshloka
{असिद्धार्था नैव क्वचिदपि सदेवासुरनरे}
{निवर्तन्ते नित्यं जगति जयिनो यस्य विशिखाः}
{स पश्यन्नीश त्वामितरसुरसाधारणमभूत्}
{स्मरः स्मर्तव्यात्मा न हि वशिषु पथ्यः परिभवः}%15

\fourlineindentedshloka
{मही पादाघाताद्-व्रजति सहसा संशयपदम्}
{पदं विष्णोर्भ्राम्यद्भुज-परिघ-रुग्ण-ग्रहगणम्}
{मुहुर्द्यौर्दौस्थ्यं यात्यनिभृत-जटा-ताडित-तटा}
{जगद्रक्षायै त्वं नटसि ननु वामैव विभुता}%16

\fourlineindentedshloka
{वियद्‍व्यापी तारागण-गुणित-फेनोद्गम-रुचिः}
{प्रवाहो वारां यः पृषतलघुदृष्टः शिरसि ते}
{जगद्द्वीपाकारं जलधिवलयं तेन कृतमिति}
{अनेनैवोन्नेयं धृतमहिम दिव्यं तव वपुः}%17

\fourlineindentedshloka
{रथः क्षोणी यन्ता शतधृतिरगेन्द्रो धनुरथो}
{रथाङ्गे चन्द्रार्कौ रथ-चरण-पाणिः शर इति}
{दिधक्षोस्ते कोऽयं त्रिपुरतृणमाडम्बर-विधिः}
{विधेयैः क्रीडन्त्यो न खलु परतन्त्राः प्रभुधियः}%18

\fourlineindentedshloka
{हरिस्ते साहस्रं कमल-बलिमाधाय पदयोः}
{यदेकोने तस्मिन् निजमुदहरन्नेत्रकमलम्}
{गतो भक्त्युद्रेकः परिणतिमसौ चक्रवपुषः}
{त्रयाणां रक्षायै त्रिपुरहर  जागर्ति जगताम्}%19

\fourlineindentedshloka
{क्रतौ सुप्ते जाग्रत् त्वमसि फलयोगे क्रतुमताम्}
{क्व कर्म प्रध्वस्तं फलति पुरुषाराधनमृते}
{अतस्त्वां सम्प्रेक्ष्य क्रतुषु फलदान-प्रतिभुवम्}
{श्रुतौ श्रद्धां बध्वा दृढपरिकरः कर्मसु जनः}%20

\fourlineindentedshloka
{क्रियादक्षो दक्षः क्रतुपतिरधीशस्तनुभृताम्}
{ऋषीणामार्त्विज्यं शरणद सदस्याः सुरगणाः}
{क्रतुभ्रंशस्त्वत्तः क्रतुफल-विधान-व्यसनिनः}
{ध्रुवं कर्तुं श्रद्धा विधुरमभिचाराय हि मखाः}%21

\fourlineindentedshloka
{प्रजानाथं नाथ प्रसभमभिकं स्वां दुहितरम्}
{गतं रोहिद्भूतां रिरमयिषुमृष्यस्य वपुषा}
{धनुष्पाणेर्यातं दिवमपि सपत्राकृतममुम्}
{त्रसन्तं तेऽद्यापि त्यजति न मृगव्याधरभसः}%22

\fourlineindentedshloka
{स्वलावण्याशंसा धृतधनुषमह्नाय तृणवत्}
{पुरः प्लुष्टं दृष्ट्वा पुरमथन पुष्पायुधमपि}
{यदि स्त्रैणं देवी यमनिरत-देहार्ध-घटनात्}
{अवैति त्वामद्धा बत वरद मुग्धा युवतयः}%23

\fourlineindentedshloka
{श्मशानेष्वाक्रीडा स्मरहर पिशाचाः सहचराः}
{चिता-भस्मालेपः स्रगपि नृकरोटी-परिकरः}
{अमङ्गल्यं शीलं तव भवतु नामैवमखिलं}
{तथाऽपि स्मर्तॄणां वरद परमं मङ्गलमसि}%24

\fourlineindentedshloka
{मनः प्रत्यक् चित्ते सविधमविधायात्त-मरुतः}
{प्रहृष्यद्रोमाणः प्रमद-सलिलोत्सङ्गति-दृशः}
{यदालोक्याऽऽह्लादं ह्रद इव निमज्यामृतमये}
{दधत्यन्तस्तत्त्वं किमपि यमिनस्तत् किल भवान्}%25

\fourlineindentedshloka
{त्वमर्कस्त्वं सोमस्त्वमसि पवनस्त्वं हुतवहः}
{त्वमापस्त्वं व्योम त्वमु धरणिरात्मा त्वमिति च}
{परिच्छिन्नामेवं त्वयि परिणता बिभ्रति गिरम्}
{न विद्मस्तत्तत्त्वं वयमिह तु यत् त्वं न भवसि}%26

\fourlineindentedshloka
{त्रयीं तिस्रो वृत्तिस्त्रिभुवनमथो त्रीनपि सुरान्}
{अकाराद्यैर्वर्णैस्त्रिभिरभिदधत् तीर्णविकृति}
{तुरीयं ते धाम ध्वनिभिरवरुन्धानमणुभिः}
{समस्तव्यस्तं त्वां शरणद गृणात्योमिति पदम्}%27

\fourlineindentedshloka
{भवः शर्वो रुद्रः पशुपतिरथोग्रः सहमहान्}
{तथा भीमेशानाविति यदभिधानाष्टकमिदम्}
{अमुष्मिन् प्रत्येकं प्रविचरति देव श्रुतिरपि}
{प्रियायास्मै धाम्ने प्रणिहित-नमस्योऽस्मि भवते}%28

\fourlineindentedshloka
{नमो नेदिष्ठाय प्रियदव दविष्ठाय च नमः}
{नमः क्षोदिष्ठाय स्मरहर महिष्ठाय च नमः}
{नमो वर्षिष्ठाय त्रिनयन यविष्ठाय च नमः}
{नमः सर्वस्मै ते तदिदमतिसर्वाय च नमः}%29

\fourlineindentedshloka
{बहुलरजसे विश्वोत्पत्तौ भवाय नमो नमः}
{प्रबलतमसे तत् संहारे हराय नमो नमः}
{जनसुखकृते सत्त्वोद्रिक्तौ मृडाय नमो नमः}
{प्रमहसि पदे निस्त्रैगुण्ये शिवाय नमो नमः}%30

\fourlineindentedshloka
{कृश-परिणति-चेतः क्लेशवश्यं क्व चेदं}
{क्व च तव गुण-सीमोल्लङ्घिनी शश्वदृद्धिः}
{इति चकितममन्दीकृत्य मां भक्तिराधाद्-}
{वरद चरणयोस्ते वाक्य-पुष्पोपहारम्}%31

\fourlineindentedshloka
{असितगिरिसमं स्यात् कज्जलं सिन्धुपात्रे}
{सुरतरुवर-शाखा लेखनी पत्रमुर्वी}
{लिखति यदि गृहीत्वा शारदा सर्वकालं}
{तदपि तव गुणानामीश पारं न याति}%32

\fourlineindentedshloka
{असुर-सुर-मुनीन्द्रैरर्चितस्येन्दुमौलेः}
{ग्रथित-गुणमहिम्नो निर्गुणस्येश्वरस्य}
{सकल-गण-वरिष्ठः पुष्पदन्ताभिधानः}
{रुचिरमलघुवृत्तैः स्तोत्रमेतच्चकार}%33

\fourlineindentedshloka
{अहरहरनवद्यं धूर्जटेः स्तोत्रमेतत्}
{पठति परमभक्त्या शुद्धचित्तः पुमान् यः}
{स भवति शिवलोके रुद्रतुल्यस्तथाऽत्र}
{प्रचुरतर-धनायुः पुत्रवान् कीर्तिमांश्च}%34

\twolineshloka
{महेशान्नापरो देवो महिम्नो नापरा स्तुतिः}
{अघोरान्नापरो मन्त्रो नास्ति तत्त्वं गुरोः परम्}%35

\twolineshloka
{दीक्षा दानं तपस्तीर्थं ज्ञानं यागादिकाः क्रियाः}
{महिम्नस्तव पाठस्य कलां नार्हन्ति षोडशीम्}%36

\fourlineindentedshloka
{कुसुमदशन-नामा सर्वगन्धर्वराजः}
{शशिधरवर-मौलेर्देवदेवस्य दासः}
{स खलु निजमहिम्नो भ्रष्ट एवास्य रोषात्}
{स्तवनमिदमकार्षीद्-दिव्य-दिव्यं महिम्नः}%37

\fourlineindentedshloka
{सुरगुरुमभिपूज्य स्वर्गमोक्षैकहेतुं}
{पठति यदि मनुष्यः प्राञ्जलिर्नान्यचेताः}
{व्रजति शिवसमीपं किन्नरैः स्तूयमानः}
{स्तवनमिदममोघं पुष्पदन्तप्रणीतम्}%38

\twolineshloka
{आसमाप्तमिदं स्तोत्रं पुण्यं गन्धर्वभाषितम्}
{अनौपम्यं मनोहारि सर्वमीश्वरवर्णनम्}%39

\twolineshloka
{इत्येषा वाङ्मयी पूजा श्रीमच्छङ्करपादयोः}
{अर्पिता तेन देवेशः प्रीयतां मे सदाशिवः}%40

\twolineshloka
{तव तत्त्वं न जानामि कीदृशोऽसि महेश्वर}
{यादृशोऽसि महादेव तादृशाय नमो नमः}%41

\twolineshloka
{एककालं द्विकालं वा त्रिकालं यः पठेन्नरः}
{सर्वपापविनिर्मुक्तः शिवलोके महीयते}%42

\fourlineindentedshloka
{श्री पुष्पदन्त-मुखपङ्कज-निर्गतेन}
{स्तोत्रेण किल्बिष-हरेण हरप्रियेण}
{कण्ठस्थितेन पठितेन समाहितेन}
{सुप्रीणितो भवति भूतपतिर्महेशः}%43

॥इति श्री~पुष्पदन्तविरचितं श्री~शिवमहिम्नः स्तोत्रं सम्पूर्णम्॥