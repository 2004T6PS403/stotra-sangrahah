% !TeX program = XeLaTeX
% !TeX root = ../../shloka.tex

%\chapter[विष्णुसहस्रनामस्तोत्रम्]{॥विष्णुसहस्रनामस्तोत्रम्॥}
\sect{विष्णुसहस्रनामस्तोत्रम्}
\twolineshloka
{शुक्लाम्बरधरं विष्णुं शशिवर्णं चतुर्भुजम्}
{प्रसन्नवदनं ध्यायेत् सर्वविघ्नोपशान्तये}

\twolineshloka
{यस्य द्विरदवक्त्राद्याः पारिषद्याः परः शतम्}
{विघ्नं निघ्नन्ति सततं विष्वक्सेनं तमाश्रये}

\twolineshloka
{नारायणं नमस्कृत्य नरं चैव नरोत्तमम्}
{देवीं सरस्वतीं व्यासं ततो जयमुदीरयेत्}

\twolineshloka
{व्यासं वसिष्ठनप्तारं शक्तेः पौत्रमकल्मषम्}
{पराशरात्मजं वन्दे शुकतातं तपोनिधिम्}

\twolineshloka
{व्यासाय विष्णुरूपाय व्यासरूपाय विष्णवे}
{नमो वै ब्रह्मनिधये वासिष्ठाय नमो नमः}

\twolineshloka
{अविकाराय शुद्धाय नित्याय परमात्मने}
{सदैकरूपरूपाय विष्णवे सर्वजिष्णवे}

\twolineshloka
{यस्य स्मरणमात्रेण जन्मसंसारबन्धनात्}
{विमुच्यते नमस्तस्मै विष्णवे प्रभविष्णवे}

\centerline{ॐ नमो विष्णवे प्रभविष्णवे}
श्री वैशम्पायन उवाच\nopagebreak[4]
\twolineshloka
{श्रुत्वा धर्मानशेषेण पावनानि च सर्वशः}
{युधिष्ठिरः शान्तनवं पुनरेवाभ्यभाषत}
%\pagebreak
श्री युधिष्ठिर उवाच\nopagebreak[4]
\twolineshloka
{किमेकं दैवतं लोके किं वाऽप्येकं परायणम्}
{स्तुवन्तः कं कमर्चन्तः प्राप्नुयुर्मानवाः शुभम्}

\twolineshloka
{को धर्मः सर्वधर्माणां भवतः परमो मतः}
{किं जपन् मुच्यते जन्तुर्जन्मसंसारबन्धनात्}

श्री भीष्म उवाच\nopagebreak[4]
\twolineshloka
{जगत्प्रभुं देवदेवमनन्तं पुरुषोत्तमम्}
{स्तुवन् नामसहस्रेण पुरुषः सततोत्थितः}

\twolineshloka
{तमेव चार्चयन्नित्यं भक्त्या पुरुषमव्ययम्}
{ध्यायन् स्तुवन् नमस्यंश्च यजमानस्तमेव च}

\twolineshloka
{अनादिनिधनं विष्णुं सर्वलोकमहेश्वरम्}
{लोकाध्यक्षं स्तुवन्नित्यं सर्वदुःखातिगो भवेत्}

\twolineshloka
{ब्रह्मण्यं सर्वधर्मज्ञं लोकानां कीर्तिवर्धनम्}
{लोकनाथं महद्भूतं सर्वभूतभवोद्भवम्}

\twolineshloka
{एष मे सर्वधर्माणां धर्मोऽधिकतमो मतः}
{यद्भक्त्या पुण्डरीकाक्षं स्तवैरर्चेन्नरः सदा}

\twolineshloka
{परमं यो महत्तेजः परमं यो महत्तपः}
{परमं यो महद्ब्रह्म परमं यः परायणम्}

\twolineshloka
{पवित्राणां पवित्रं यो मङ्गलानां च मङ्गलम्}
{दैवतं दैवतानां च भूतानां योऽव्ययः पिता}

\twolineshloka
{यतः सर्वाणि भूतानि भवन्त्यादियुगागमे}
{यस्मिंश्च प्रलयं यान्ति पुनरेव युगक्षये}

\twolineshloka
{तस्य लोकप्रधानस्य जगन्नाथस्य भूपते}
{विष्णोर्नामसहस्रं मे शृणु पापभयापहम्}

\twolineshloka
{यानि नामानि गौणानि विख्यातानि महात्मनः}
{ऋषिभिः परिगीतानि तानि वक्ष्यामि भूतये}

\twolineshloka
{ऋषिर्नाम्नां सहस्रस्य वेदव्यासो महामुनिः}
{छन्दोऽनुष्टुप् तथा देवो भगवान् देवकीसुतः}

\twolineshloka
{अमृतांशूद्भवो बीजं शक्तिर्देवकिनन्दनः}
{त्रिसामा हृदयं तस्य शान्त्यर्थे विनियुज्यते}

\twolineshloka
{विष्णुं जिष्णुं महाविष्णुं प्रभविष्णुं महेश्वरम्}
{अनेकरूपदैत्यान्तं नमामि पुरुषोत्तमं}

\dnsub{पूर्वन्यासः}
अस्य श्रीविष्णोर्दिव्यसहस्रनामस्तोत्रमहामन्त्रस्य।\\
श्री वेदव्यासो भगवान् ऋषिः। अनुष्टुप् छन्दः।\\
श्रीमहाविष्णुः परमात्मा श्रीमन्नारायणो देवता।\\
अमृतांशूद्भवो भानुरिति बीजम्। देवकीनन्दनः स्रष्टेति शक्तिः।\\
उद्भवः क्षोभणो देव इति परमो मन्त्रः।\\
शङ्खभृन्नन्दकी चक्रीति कीलकम्।\\
शार्ङ्गधन्वा गदाधर इत्यस्त्रम्। \\
रथाङ्गपाणिरक्षोभ्य इति नेत्रम्। \\
त्रिसामा सामगः सामेति कवचम्।\\
आनन्दं परब्रह्मेति योनिः।\\
ऋतुः सुदर्शनः काल इति दिग्बन्धः। \\
श्रीविश्वरूप इति ध्यानम्।\\
श्रीमहाविष्णुप्रीत्यर्थे सहस्रनामजपे विनियोगः॥\\

\resetShloka
\setlength{\shlokaspaceskip}{12pt}
\dnsub{ध्यानम्}
\fourlineindentedshloka
{क्षीरोदन्वत्प्रदेशे शुचिमणिविलसत्सैकतेर्मौक्तिकानाम्}
{मालाकॢप्तासनस्थः स्फटिकमणिनिभैर्मौक्तिकैर्मण्डिताङ्गः}
{शुभ्रैरभ्रैरदभ्रैरुपरिविरचितैर्मुक्तपीयूषवर्षैः}
{आनन्दी नः पुनीयादरिनलिनगदाशङ्खपाणिर्मुकुन्दः}

\fourlineindentedshloka
{भूः पादौ यस्य नाभिर्वियदसुरनिलश्चन्द्रसूर्यौ च नेत्रे}
{कर्णावाशाः शिरो द्यौर्मुखमपि दहनो यस्य वास्तेयमब्धिः}
{अन्तःस्थं यस्य विश्वं सुरनरखगगोभोगिगन्धर्वदैत्यैः}
{चित्रं रंरम्यते तं त्रिभुवनवपुषं विष्णुमीशं नमामि}

\centerline{ॐ नमो भगवते वासुदेवाय}
\fourlineindentedshloka
{शान्ताकारं भुजगशयनं पद्मनाभं सुरेशम्}
{विश्वाधारं गगनसदृशं मेघवर्णं शुभाङ्गम्}
{लक्ष्मीकान्तं कमलनयनं योगिहृद्‌ध्यानगम्यम्}
{वन्दे विष्णुं भवभयहरं सर्वलोकैकनाथम्}

\fourlineindentedshloka
{मेघश्यामं पीतकौशेयवासम्}
{श्रीवत्साङ्कं कौस्तुभोद्भासिताङ्गम्}
{पुण्योपेतं पुण्डरीकायताक्षम्}
{विष्णुं वन्दे सर्वलोकैकनाथम्}

\twolineshloka
{नमः समस्तभूतानामादिभूताय भूभृते}
{अनेकरूपरूपाय विष्णवे प्रभविष्णवे}

\fourlineindentedshloka
{सशङ्खचक्रं सकिरीटकुण्डलम्}
{सपीतवस्त्रं सरसीरुहेक्षणम्}
{सहारवक्षःस्थलशोभिकौस्तुभम्}
{नमामि विष्णुं शिरसा चतुर्भुजम्}

\fourlineindentedshloka
{छायायां पारिजातस्य हेमसिंहासनोपरि}
{आसीनमम्बुदश्याममायताक्षमलङ्कृतम्}
{चन्द्राननं चतुर्बाहुं श्रीवत्साङ्कितवक्षसम्}
{रुक्मिणीसत्यभामाभ्यां सहितं कृष्णमाश्रये}

\dnsub{हरिः ॐ}\nopagebreak[4]
\centerline{॥विश्वस्मै नमः॥}\nopagebreak[4]
\resetShloka
\twolineshloka
{विश्वं विष्णुर्वषट्कारो भूतभव्यभवत्प्रभुः}
{भूतकृद्भूतभृद्भावो भूतात्मा भूतभावनः}

\twolineshloka
{पूतात्मा परमात्मा च मुक्तानां परमा गतिः}
{अव्ययः पुरुषः साक्षी क्षेत्रज्ञोऽक्षर एव च}

\twolineshloka
{योगो योगविदां नेता प्रधानपुरुषेश्वरः}
{नारसिंहवपुः श्रीमान् केशवः पुरुषोत्तमः}

\twolineshloka
{सर्वः शर्वः शिवः स्थाणुर्भूतादिर्निधिरव्ययः}
{सम्भवो भावनो भर्ता प्रभवः प्रभुरीश्वरः}

\twolineshloka
{स्वयम्भूः शम्भुरादित्यः पुष्कराक्षो महास्वनः}
{अनादिनिधनो धाता विधाता धातुरुत्तमः}

\twolineshloka
{अप्रमेयो हृषीकेशः पद्मनाभोऽमरप्रभुः}
{विश्वकर्मा मनुस्त्वष्टा स्थविष्ठः स्थविरो ध्रुवः}

\twolineshloka
{अग्राह्यः शाश्वतः कृष्णो लोहिताक्षः प्रतर्दनः}
{प्रभूतस्त्रिककुब्धाम पवित्रं मङ्गलं परम्}

\twolineshloka
{ईशानः प्राणदः प्राणो ज्येष्ठः श्रेष्ठः प्रजापतिः}
{हिरण्यगर्भो भूगर्भो माधवो मधुसूदनः}

\twolineshloka
{ईश्वरो विक्रमी धन्वी मेधावी विक्रमः क्रमः}
{अनुत्तमो दुराधर्षः कृतज्ञः कृतिरात्मवान्}

\twolineshloka
{सुरेशः शरणं शर्म विश्वरेता प्रजाभवः}
{अहः संवत्सरो व्यालः प्रत्ययः सर्वदर्शनः}

\twolineshloka
{अजः सर्वेश्वरः सिद्धः सिद्धिः सर्वादिरच्युतः}
{वृषाकपिरमेयात्मा सर्वयोगविनिःसृतः}

\twolineshloka
{वसुर्वसुमनाः सत्यः समात्माऽसम्मितः समः}
{अमोघः पुण्डरीकाक्षो वृषकर्मा वृषाकृतिः}

\twolineshloka
{रुद्रो बहुशिरा बभ्रुर्विश्वयोनिः शुचिश्रवाः}
{अमृतः शाश्वतः स्थाणुर्वरारोहो महातपाः}

\twolineshloka
{सर्वगः सर्वविद्भानुर्विष्वक्सेनो जनार्दनः}
{वेदो वेदविदव्यङ्गो वेदाङ्गो वेदवित् कविः}

\twolineshloka
{लोकाध्यक्षः सुराध्यक्षो धर्माध्यक्षः कृताकृतः}
{चतुरात्मा चतुर्व्यूहश्चतुर्दंष्ट्रश्चतुर्भुजः}

\twolineshloka
{भ्राजिष्णुर्भोजनं भोक्ता सहिष्णुर्जगदादिजः}
{अनघो विजयो जेता विश्वयोनिः पुनर्वसुः}

\twolineshloka
{उपेन्द्रो वामनः प्रांशुरमोघः शुचिरूर्जितः}
{अतीन्द्रः सङ्ग्रहः सर्गो धृतात्मा नियमो यमः}

\twolineshloka
{वेद्यो वैद्यः सदायोगी वीरहा माधवो मधुः}
{अतीन्द्रियो महामायो महोत्साहो महाबलः}

\twolineshloka
{महाबुद्धिर्महावीर्यो महाशक्तिर्महाद्युतिः}
{अनिर्देश्यवपुः श्रीमानमेयात्मा महाद्रिधृक्}

\twolineshloka
{महेष्वासो महीभर्ता श्रीनिवासः सतां गतिः}
{अनिरुद्धः सुरानन्दो गोविन्दो गोविदां पतिः}

\twolineshloka
{मरीचिर्दमनो हंसः सुपर्णो भुजगोत्तमः}
{हिरण्यनाभः सुतपा पद्मनाभः प्रजापतिः}

\twolineshloka
{अमृत्युः सर्वदृक् सिंहः सन्धाता सन्धिमान् स्थिरः}
{अजो दुर्मर्षणः शास्ता विश्रुतात्मा सुरारिहा}

\twolineshloka
{गुरुर्गुरुतमो धाम सत्यः सत्यपराक्रमः}
{निमिषोऽनिमिषः स्रग्वी वाचस्पतिरुदारधीः}

\twolineshloka
{अग्रणीर्ग्रामणीः श्रीमान् न्यायो नेता समीरणः}
{सहस्रमूर्धा विश्वात्मा सहस्राक्षः सहस्रपात्}

\twolineshloka
{आवर्तनो निवृत्तात्मा संवृतः सम्प्रमर्दनः}
{अहः संवर्तको वह्निरनिलो धरणीधरः}

\twolineshloka
{सुप्रसादः प्रसन्नात्मा विश्वधृग्विश्वभुग्विभुः}
{सत्कर्ता सत्कृतः साधुर्जह्नुर्नारायणो नरः}

\twolineshloka
{असङ्ख्येयोऽप्रमेयात्मा विशिष्टः शिष्टकृच्छुचिः}
{सिद्धार्थः सिद्धसङ्कल्पः सिद्धिदः सिद्धिसाधनः}

\twolineshloka
{वृषाही वृषभो विष्णुर्वृषपर्वा वृषोदरः}
{वर्धनो वर्धमानश्च विविक्तः श्रुतिसागरः}

\twolineshloka
{सुभुजो दुर्धरो वाग्मी महेन्द्रो वसुदो वसुः}
{नैकरूपो बृहद्रूपः शिपिविष्टः प्रकाशनः}

\twolineshloka
{ओजस्तेजोद्युतिधरः प्रकाशात्मा प्रतापनः}
{ऋद्धः स्पष्टाक्षरो मन्त्रश्चन्द्रांशुर्भास्करद्युतिः}

\twolineshloka
{अमृतांशूद्भवो भानुः शशबिन्दुः सुरेश्वरः}
{औषधं जगतः सेतुः सत्यधर्मपराक्रमः}

\twolineshloka
{भूतभव्यभवन्नाथः पवनः पावनोऽनलः}
{कामहा कामकृत्कान्तः कामः कामप्रदः प्रभुः}

\twolineshloka
{युगादिकृद्युगावर्तो नैकमायो महाशनः}
{अदृश्यो व्यक्तरूपश्च सहस्रजिदनन्तजित्}

\twolineshloka
{इष्टोऽविशिष्टः शिष्टेष्टः शिखण्डी नहुषो वृषः}
{क्रोधहा क्रोधकृत्कर्ता विश्वबाहुर्महीधरः}

\twolineshloka
{अच्युतः प्रथितः प्राणः प्राणदो वासवानुजः}
{अपान्निधिरधिष्ठानमप्रमत्तः प्रतिष्ठितः}

\twolineshloka
{स्कन्दः स्कन्दधरो धुर्यो वरदो वायुवाहनः}
{वासुदेवो बृहद्भानुरादिदेवः पुरन्दरः}

\twolineshloka
{अशोकस्तारणस्तारः शूरः शौरिर्जनेश्वरः}
{अनुकूलः शतावर्तः पद्मी पद्मनिभेक्षणः}

\twolineshloka
{पद्मनाभोऽरविन्दाक्षः पद्मगर्भः शरीरभृत्}
{महर्द्धिरृद्धो वृद्धात्मा महाक्षो गरुडध्वजः}

\twolineshloka
{अतुलः शरभो भीमः समयज्ञो हविर्हरिः}
{सर्वलक्षणलक्षण्यो लक्ष्मीवान् समितिञ्जयः}

\twolineshloka
{विक्षरो रोहितो मार्गो हेतुर्दामोदरः सहः}
{महीधरो महाभागो वेगवानमिताशनः}

\twolineshloka
{उद्भवः क्षोभणो देवः श्रीगर्भः परमेश्वरः}
{करणं कारणं कर्ता विकर्ता गहनो गुहः}

\twolineshloka
{व्यवसायो व्यवस्थानः संस्थानः स्थानदो ध्रुवः}
{परर्द्धिः परमस्पष्टस्तुष्टः पुष्टः शुभेक्षणः}

\twolineshloka
{रामो विरामो विरतो मार्गो नेयो नयोऽनयः}
{वीरः शक्तिमतां श्रेष्ठो धर्मो धर्मविदुत्तमः}

\twolineshloka
{वैकुण्ठः पुरुषः प्राणः प्राणदः प्रणवः पृथुः}
{हिरण्यगर्भः शत्रुघ्नो व्याप्तो वायुरधोक्षजः}

\twolineshloka
{ऋतुः सुदर्शनः कालः परमेष्ठी परिग्रहः}
{उग्रः संवत्सरो दक्षो विश्रामो विश्वदक्षिणः}

\twolineshloka
{विस्तारः स्थावरः स्थाणुः प्रमाणं बीजमव्ययम्}
{अर्थोऽनर्थो महाकोशो महाभोगो महाधनः}

\twolineshloka
{अनिर्विण्णः स्थविष्ठोऽभूर्धर्मयूपो महामखः}
{नक्षत्रनेमिर्नक्षत्री क्षमः क्षामः समीहनः}

\twolineshloka
{यज्ञ इज्यो महेज्यश्च क्रतुः सत्रं सतां गतिः}
{सर्वदर्शी विमुक्तात्मा सर्वज्ञो ज्ञानमुत्तमम्}

\twolineshloka
{सुव्रतः सुमुखः सूक्ष्मः सुघोषः सुखदः सुहृत्}
{मनोहरो जितक्रोधो वीरबाहुर्विदारणः}

\twolineshloka
{स्वापनः स्ववशो व्यापी नैकात्मा नैककर्मकृत्}
{वत्सरो वत्सलो वत्सी रत्नगर्भो धनेश्वरः}

\twolineshloka
{धर्मगुब्धर्मकृद्धर्मी सदसत्क्षरमक्षरम्}
{अविज्ञाता सहस्रांशुर्विधाता कृतलक्षणः}

\twolineshloka
{गभस्तिनेमिः सत्त्वस्थः सिंहो भूतमहेश्वरः}
{आदिदेवो महादेवो देवेशो देवभृद्गुरुः}

\twolineshloka
{उत्तरो गोपतिर्गोप्ता ज्ञानगम्यः पुरातनः}
{शरीरभूतभृद्भोक्ता कपीन्द्रो भूरिदक्षिणः}

\twolineshloka
{सोमपोऽमृतपः सोमः पुरुजित् पुरुसत्तमः}
{विनयो जयः सत्यसन्धो दाशार्हः सात्त्वतां पतिः}

\twolineshloka
{जीवो विनयितासाक्षी मुकुन्दोऽमितविक्रमः}
{अम्भोनिधिरनन्तात्मा महोदधिशयोऽन्तकः}

\twolineshloka
{अजो महार्हः स्वाभाव्यो जितामित्रः प्रमोदनः}
{आनन्दो नन्दनो नन्दः सत्यधर्मा त्रिविक्रमः}

\twolineshloka
{महर्षिः कपिलाचार्यः कृतज्ञो मेदिनीपतिः}
{त्रिपदस्त्रिदशाध्यक्षो महाशृङ्गः कृतान्तकृत्}

\twolineshloka
{महावराहो गोविन्दः सुषेणः कनकाङ्गदी}
{गुह्यो गभीरो गहनो गुप्तश्चक्रगदाधरः}

\twolineshloka
{वेधाः स्वाङ्गोऽजितः कृष्णो दृढः सङ्कर्षणोऽच्युतः}
{वरुणो वारुणो वृक्षः पुष्कराक्षो महामनाः}

\twolineshloka
{भगवान् भगहाऽऽनन्दी वनमाली हलायुधः}
{आदित्यो ज्योतिरादित्यः सहिष्णुर्गतिसत्तमः}

\twolineshloka
{सुधन्वा खण्डपरशुर्दारुणो द्रविणप्रदः}
{दिवःस्पृक् सर्वदृग्व्यासो वाचस्पतिरयोनिजः}

\twolineshloka
{त्रिसामा सामगः साम निर्वाणं भेषजं भिषक्}
{सन्न्यासकृच्छमः शान्तो निष्ठा शान्तिः परायणम्}

\twolineshloka
{शुभाङ्गः शान्तिदः स्रष्टा कुमुदः कुवलेशयः}
{गोहितो गोपतिर्गोप्ता वृषभाक्षो वृषप्रियः}

\twolineshloka
{अनिवर्ती निवृत्तात्मा सङ्क्षेप्ता क्षेमकृच्छिवः}
{श्रीवत्सवक्षाः श्रीवासः श्रीपतिः श्रीमतां वरः}

\twolineshloka
{श्रीदः श्रीशः श्रीनिवासः श्रीनिधिः श्रीविभावनः}
{श्रीधरः श्रीकरः श्रेयः श्रीमाँल्लोकत्रयाश्रयः}

\twolineshloka
{स्वक्षः स्वङ्गः शतानन्दो नन्दिर्ज्योतिर्गणेश्वरः}
{विजितात्माऽविधेयात्मा सत्कीर्तिश्छिन्नसंशयः}

\twolineshloka
{उदीर्णः सर्वतश्चक्षुरनीशः शाश्वतः स्थिरः}
{भूशयो भूषणो भूतिर्विशोकः शोकनाशनः}

\twolineshloka
{अर्चिष्मानर्चितः कुम्भो विशुद्धात्मा विशोधनः}
{अनिरुद्धोऽप्रतिरथः प्रद्युम्नोऽमितविक्रमः}

\twolineshloka
{कालनेमिनिहा वीरः शौरिः शूरजनेश्वरः}
{त्रिलोकात्मा त्रिलोकेशः केशवः केशिहा हरिः}

\twolineshloka
{कामदेवः कामपालः कामी कान्तः कृतागमः}
{अनिर्देश्यवपुर्विष्णुर्वीरोऽनन्तो धनञ्जयः}

\twolineshloka
{ब्रह्मण्यो ब्रह्मकृद्-ब्रह्मा ब्रह्म ब्रह्मविवर्धनः}
{ब्रह्मविद्-ब्राह्मणो ब्रह्मी ब्रह्मज्ञो ब्राह्मणप्रियः}

\twolineshloka
{महाक्रमो महाकर्मा महातेजा महोरगः}
{महाक्रतुर्महायज्वा महायज्ञो महाहविः}

\twolineshloka
{स्तव्यः स्तवप्रियः स्तोत्रं स्तुतिः स्तोता रणप्रियः}
{पूर्णः पूरयिता पुण्यः पुण्यकीर्तिरनामयः}

\twolineshloka
{मनोजवस्तीर्थकरो वसुरेता वसुप्रदः}
{वसुप्रदो वासुदेवो वसुर्वसुमना हविः}

\twolineshloka
{सद्गतिः सत्कृतिः सत्ता सद्भूतिः सत्परायणः}
{शूरसेनो यदुश्रेष्ठः सन्निवासः सुयामुनः}

\twolineshloka
{भूतावासो वासुदेवः सर्वासुनिलयोऽनलः}
{दर्पहा दर्पदो दृप्तो दुर्धरोऽथापराजितः}

\twolineshloka
{विश्वमूर्तिर्महामूर्तिर्दीप्तमूर्तिरमूर्तिमान्}
{अनेकमूर्तिरव्यक्तः शतमूर्तिः शताननः}

\twolineshloka
{एको नैकः सवः कः किं यत् तत्पदमनुत्तमम्}
{लोकबन्धुर्लोकनाथो माधवो भक्तवत्सलः}

\twolineshloka
{सुवर्णवर्णो हेमाङ्गो वराङ्गश्चन्दनाङ्गदी}
{वीरहा विषमः शून्यो घृताशीरचलश्चलः}

\twolineshloka
{अमानी मानदो मान्यो लोकस्वामी त्रिलोकधृक्}
{सुमेधा मेधजो धन्यः सत्यमेधा धराधरः}

\twolineshloka
{तेजोवृषो द्युतिधरः सर्वशस्त्रभृतां वरः}
{प्रग्रहो निग्रहो व्यग्रो नैकशृङ्गो गदाग्रजः}

\twolineshloka
{चतुर्मूर्तिश्चतुर्बाहुश्चतुर्व्यूहश्चतुर्गतिः}
{चतुरात्मा चतुर्भावश्चतुर्वेदविदेकपात्}

\twolineshloka
{समावर्तोऽनिवृत्तात्मा दुर्जयो दुरतिक्रमः}
{दुर्लभो दुर्गमो दुर्गो दुरावासो दुरारिहा}

\twolineshloka
{शुभाङ्गो लोकसारङ्गः सुतन्तुस्तन्तुवर्धनः}
{इन्द्रकर्मा महाकर्मा कृतकर्मा कृतागमः}

\twolineshloka
{उद्भवः सुन्दरः सुन्दो रत्ननाभः सुलोचनः}
{अर्को वाजसनः शृङ्गी जयन्तः सर्वविज्जयी}

\twolineshloka
{सुवर्णबिन्दुरक्षोभ्यः सर्ववागीश्वरेश्वरः}
{महाह्रदो महागर्तो महाभूतो महानिधिः}

\twolineshloka
{कुमुदः कुन्दरः कुन्दः पर्जन्यः पावनोऽनिलः}
{अमृताशोऽमृतवपुः सर्वज्ञः सर्वतोमुखः}

\twolineshloka
{सुलभः सुव्रतः सिद्धः शत्रुजिच्छत्रुतापनः}
{न्यग्रोधोऽदुम्बरोऽश्वत्थश्चाणूरान्ध्रनिषूदनः}

\twolineshloka
{सहस्रार्चिः सप्तजिह्वः सप्तैधाः सप्तवाहनः}
{अमूर्तिरनघोऽचिन्त्यो भयकृद्भयनाशनः}

\twolineshloka
{अणुर्बृहत् कृशः स्थूलो गुणभृन्निर्गुणो महान्}
{अधृतः स्वधृतः स्वास्यः प्राग्वंशो वंशवर्धनः}

\twolineshloka
{भारभृत् कथितो योगी योगीशः सर्वकामदः}
{आश्रमः श्रमणः क्षामः सुपर्णो वायुवाहनः}

\twolineshloka
{धनुर्धरो धनुर्वेदो दण्डो दमयिता दमः}
{अपराजितः सर्वसहो नियन्ताऽनियमोऽयमः}

\twolineshloka
{सत्त्ववान् सात्त्विकः सत्यः सत्यधर्मपरायणः}
{अभिप्रायः प्रियार्होऽर्हः प्रियकृत् प्रीतिवर्धनः}

\twolineshloka
{विहायसगतिर्ज्योतिः सुरुचिर्हुतभुग्विभुः}
{रविर्विरोचनः सूर्यः सविता रविलोचनः}

\twolineshloka
{अनन्तो हुतभुग्भोक्ता सुखदो नैकजोऽग्रजः}
{अनिर्विण्णः सदामर्षी लोकाधिष्ठानमद्भुतः}

\twolineshloka
{सनात् सनातनतमः कपिलः कपिरव्ययः}
{स्वस्तिदः स्वस्तिकृत् स्वस्ति स्वस्तिभुक् स्वस्तिदक्षिणः}

\twolineshloka
{अरौद्रः कुण्डली चक्री विक्रम्यूर्जितशासनः}
{शब्दातिगः शब्दसहः शिशिरः शर्वरीकरः}

\twolineshloka
{अक्रूरः पेशलो दक्षो दक्षिणः क्षमिणां वरः}
{विद्वत्तमो वीतभयः पुण्यश्रवणकीर्तनः}

\twolineshloka
{उत्तारणो दुष्कृतिहा पुण्यो दुःस्वप्ननाशनः}
{वीरहा रक्षणः सन्तो जीवनः पर्यवस्थितः}

\twolineshloka
{अनन्तरूपोऽनन्तश्रीर्जितमन्युर्भयापहः}
{चतुरश्रो गभीरात्मा विदिशो व्यादिशो दिशः}

\twolineshloka
{अनादिर्भूर्भुवो लक्ष्मीः सुवीरो रुचिराङ्गदः}
{जननो जनजन्मादिर्भीमो भीमपराक्रमः}

\twolineshloka
{आधारनिलयोऽधाता पुष्पहासः प्रजागरः}
{ऊर्ध्वगः सत्पथाचारः प्राणदः प्रणवः पणः}

\twolineshloka
{प्रमाणं प्राणनिलयः प्राणभृत् प्राणजीवनः}
{तत्त्वं तत्त्वविदेकात्मा जन्ममृत्युजरातिगः}

\twolineshloka
{भूर्भुवःस्वस्तरुस्तारः सविता प्रपितामहः}
{यज्ञो यज्ञपतिर्यज्वा यज्ञाङ्गो यज्ञवाहनः}

\twolineshloka
{यज्ञभृद्-यज्ञकृद्-यज्ञी यज्ञभुग्-यज्ञसाधनः}
{यज्ञान्तकृद्-यज्ञगुह्यमन्नमन्नाद एव च}

\twolineshloka
{आत्मयोनिः स्वयञ्जातो वैखानः सामगायनः}
{देवकीनन्दनः स्रष्टा क्षितीशः पापनाशनः}

\twolineshloka
{शङ्खभृन्नन्दकी चक्री शार्ङ्गधन्वा गदाधरः}
{रथाङ्गपाणिरक्षोभ्यः सर्वप्रहरणायुधः}
सर्वप्रहरणायुध ॐ नम इति।

\twolineshloka
{वनमाली गदी शार्ङ्गी शङ्खी चक्री च नन्दकी}
{श्रीमान् नारायणो विष्णुर्वासुदेवोऽभिरक्षतु}%{(एवं त्रिः)}
श्री वासुदेवोऽभिरक्षतु ॐ नम इति।

\dnsub{फलश्रुति श्लोकाः}
\resetShloka
\twolineshloka
{इतीदं कीर्तनीयस्य केशवस्य महात्मनः}
{नाम्नां सहस्रं दिव्यानामशेषेण प्रकीर्तितम्}

\twolineshloka
{य इदं शृणुयान्नित्यं यश्चापि परिकीर्तयेत्}
{नाशुभं प्राप्नुयात् किञ्चित् सोऽमुत्रेह च मानवः}

\twolineshloka
{वेदान्तगो ब्राह्मणः स्यात् क्षत्रियो विजयी भवेत्}
{वैश्यो धनसमृद्धः स्याच्छूद्रः सुखमवाप्नुयात्}

\twolineshloka
{धर्मार्थी प्राप्नुयाद्धर्ममर्थार्थी चार्थमाप्नुयात्}
{कामानवाप्नुयात् कामी प्रजार्थी चऽऽप्नुयात्प्रजाम्}

\twolineshloka
{भक्तिमान् यः सदोत्थाय शुचिस्तद्गतमानसः}
{सहस्रं वासुदेवस्य नाम्नामेतत् प्रकीर्तयेत्}

\twolineshloka
{यशः प्राप्नोति विपुलं याति प्राधान्यमेव च}
{अचलां श्रियमाप्नोति श्रेयः प्राप्नोत्यनुत्तमम्}

\twolineshloka
{न भयं क्वचिदाप्नोति वीर्यं तेजश्च विन्दति}
{भवत्यरोगो द्युतिमान् बलरूपगुणान्वितः}

\twolineshloka
{रोगार्तो मुच्यते रोगाद्बद्धो मुच्येत बन्धनात्}
{भयान्मुच्येत भीतस्तु मुच्येतऽऽपन्न आपदः}

\twolineshloka
{दुर्गाण्यतितरत्याशु पुरुषः पुरुषोत्तमम्}
{स्तुवन्नामसहस्रेण नित्यं भक्तिसमन्वितः}

\twolineshloka
{वासुदेवाश्रयो मर्त्यो वासुदेवपरायणः}
{सर्वपापविशुद्धात्मा याति ब्रह्म सनातनम्}

\twolineshloka
{न वासुदेवभक्तानामशुभं विद्यते क्वचित्}
{जन्ममृत्युजराव्याधिभयं नैवोपजायते}

\twolineshloka
{इमं स्तवमधीयानः श्रद्धाभक्तिसमन्वितः}
{युज्येतऽऽत्मसुखक्षान्तिश्रीधृतिस्मृतिकीर्तिभिः}

\twolineshloka
{न क्रोधो न च मात्सर्यं न लोभो नाशुभा मतिः}
{भवन्ति कृतपुण्यानां भक्तानां पुरुषोत्तमे}

\twolineshloka
{द्यौः सचन्द्रार्कनक्षत्रा खं दिशो भूर्महोदधिः}
{वासुदेवस्य वीर्येण विधृतानि महात्मनः}

\twolineshloka
{ससुरासुरगन्धर्वं सयक्षोरगराक्षसम्}
{जगद्वशे वर्ततेदं कृष्णस्य सचराचरम्}

\twolineshloka
{इन्द्रियाणि मनो बुद्धिः सत्त्वं तेजो बलं धृतिः}
{वासुदेवात्मकान्याहुः क्षेत्रं क्षेत्रज्ञ एव च}

\twolineshloka
{सर्वागमानामाचारः प्रथमं परिकल्पते}
{आचारप्रभवो धर्मो धर्मस्य प्रभुरच्युतः}

\twolineshloka
{ऋषयः पितरो देवा महाभूतानि धातवः}
{जङ्गमाजङ्गमं चेदं जगन्नारायणोद्भवम्}

\twolineshloka
{योगो ज्ञानं तथा साङ्ख्यं विद्याः शिल्पादि कर्म च}
{वेदाः शास्त्राणि विज्ञानमेतत्सर्वं जनार्दनात्}

\twolineshloka
{एको विष्णुर्महद्भूतं पृथग्भूतान्यनेकशः}
{त्रीँल्लोकान् व्याप्य भूतात्मा भुङ्क्ते विश्वभुगव्ययः}

\twolineshloka
{इमं स्तवं भगवतो विष्णोर्व्यासेन कीर्तितम्}
{पठेद्य इच्छेत् पुरुषः श्रेयः प्राप्तुं सुखानि च}

\twolineshloka
{विश्वेश्वरमजं देवं जगतः प्रभुमव्ययम्}
{भजन्ति ये पुष्कराक्षं न ते यान्ति पराभवम्}
न ते यान्ति पराभवम् ॐ नम इति।

अर्जुन उवाच\nopagebreak[4]
\twolineshloka
{पद्मपत्रविशालाक्ष पद्मनाभ सुरोत्तम}
{भक्तानामनुरक्तानां त्राता भव जनार्दन}

श्रीभगवानुवाच\nopagebreak[4]
\twolineshloka
{यो मां नामसहस्रेण स्तोतुमिच्छति पाण्डव}
{सोऽहमेकेन श्लोकेन स्तुत एव न संशयः}
स्तुत एव न संशय ॐ नम इति।

व्यास उवाच\nopagebreak[4]
\twolineshloka
{वासनाद्वासुदेवस्य वासितं भुवनत्रयम्}
{सर्वभूतनिवासोऽसि वासुदेव नमोऽस्तु ते}
श्री वासुदेव नमोऽस्तुत ॐ नम इति।

पार्वत्युवाच\nopagebreak[4]
\twolineshloka
{केनोपायेन लघुना विष्णोर्नामसहस्रकम्}
{पठ्यते पण्डितैर्नित्यं श्रोतुमिच्छाम्यहं प्रभो}

श्री ईश्वर उवाच\nopagebreak[4]
\twolineshloka
{श्रीराम राम रामेति रमे रामे मनोरमे}
{सहस्रनाम तत्तुल्यं राम नाम वरानने}%{(एवं त्रिः)}
श्रीरामनाम वरानन ॐ नम इति।

ब्रह्मोवाच\nopagebreak[4]
\twolineshloka
{नमोऽस्त्वनन्ताय सहस्रमूर्तये सहस्रपादाक्षिशिरोरुबाहवे}
{सहस्रनाम्ने पुरुषाय शाश्वते सहस्रकोटियुगधारिणे नमः}
सहस्रकोटियुगधारिणे नम ॐ नम इति।

सञ्जय उवाच\nopagebreak[4]
\twolineshloka
{यत्र योगेश्वरः कृष्णो यत्र पार्थो धनुर्धरः}
{तत्र श्रीर्विजयो भूतिर्ध्रुवा नीतिर्मतिर्मम}

श्रीभगवानुवाच\nopagebreak[4]
\twolineshloka
{अनन्याश्चिन्तयन्तो मां ये जनाः पर्युपासते}
{तेषां नित्याभियुक्तानां योगक्षेमं वहाम्यहम्}

\twolineshloka
{परित्राणाय साधूनां विनाशाय च दुष्कृताम्}
{धर्मसंस्थापनार्थाय सम्भवामि युगे युगे}

\fourlineindentedshloka
{आर्ता विषण्णाः शिथिलाश्च भीताः}
{घोरेषु च व्याधिषु वर्तमानाः}
{सङ्कीर्त्य नारायणशब्दमात्रम्}
{विमुक्तदुःखाः सुखिनो भवन्तु}

\twolineshloka*
{कायेन वाचा मनसेन्द्रियैर्वा बुद्‌ध्याऽऽत्मना वा प्रकृतेः स्वभावात्}
{करोमि यद्यत् सकलं परस्मै नारायणायेति समर्पयामि}
॥ॐ तत्सदिति श्रीमन्महाभारते शतसाहस्र्यां संहितायां वैयासिक्याम् आनुशासनिकपर्वणि श्री भीष्मयुधिष्ठिरसंवादे श्री विष्णोर्दिव्यसहस्रनामस्तोत्रं सम्पूर्णम्॥
\setlength{\shlokaspaceskip}{24pt}
