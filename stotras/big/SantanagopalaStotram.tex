% !TeX program = XeLaTeX
% !TeX root = ../../shloka.tex

\sect{सन्तानगोपाल स्तोत्रम्}

\twolineshloka
{श्रीशं कमलपत्राक्षं देवकीनन्दनं हरिम्}
{सुतसम्प्राप्तये कृष्णं नमामि मधुसूदनम्}%॥ १॥

\twolineshloka
{नमाम्यहं वासुदेवं सुतसम्प्राप्तये हरिम्}
{यशोदाङ्कगतं बालं गोपालं नन्दनन्दनम्}%॥ २॥

\twolineshloka
{अस्माकं पुत्रलाभाय गोविन्दं मुनिवन्दितम्}
{नमाम्यहं वासुदेवं देवकीनन्दनं सदा}%॥ ३॥

\twolineshloka
{गोपालं डिम्भकं वन्दे कमलापतिमच्युतम्}
{पुत्रसम्प्राप्तये कृष्णं नमामि यदुपुङ्गवम्}%॥ ४॥

\twolineshloka
{पुत्रकामेष्टिफलदं कञ्जाक्षं कमलापतिम्}
{देवकीनन्दनं वन्दे सुतसम्प्राप्तये मम}%॥ ५॥

\twolineshloka
{पद्मापते पद्मनेत्र पद्मनाभ जनार्दन}
{देहि मे तनयं श्रीश वासुदेव जगत्पते}%॥ ६॥

\twolineshloka
{यशोदाङ्कगतं बालं गोविन्दं मुनिवन्दितम्}
{अस्माकं पुत्रलाभाय नमामि श्रीशमच्युतम्}%॥ ७॥

\twolineshloka
{श्रीपते देवदेवेश दीनार्तिहरणाच्युत}
{गोविन्द मे सुतं देहि नमामि त्वां जनार्दन}%॥ ८॥

\twolineshloka
{भक्तकामद गोविन्द भक्तं रक्ष शुभप्रद}
{देहि मे तनयं कृष्ण रुक्मिणीवल्लभ प्रभो}%॥ ९॥

\twolineshloka
{रुक्मिणीनाथ सर्वेश देहि मे तनयं सदा}
{भक्तमन्दार पद्माक्ष त्वामहं शरणं गतः}%॥ १०॥

\twolineshloka
{देवकीसुत गोविन्द वासुदेव जगत्पते}
{देहि मे तनयं कृष्ण त्वामहं शरणं गतः}%॥ ११॥

\twolineshloka
{वासुदेव जगद्वन्द्य श्रीपते पुरुषोत्तम}
{देहि मे तनयं कृष्ण त्वामहं शरणं गतः}%॥ १२॥

\twolineshloka
{कञ्जाक्ष कमलानाथ परकारुणिकोत्तम}
{देहि मे तनयं कृष्ण त्वामहं शरणं गतः}%॥ १३॥

\twolineshloka
{लक्ष्मीपते पद्मनाभ मुकुन्द मुनिवन्दित}
{देहि मे तनयं कृष्ण त्वामहं शरणं गतः}%॥ १४॥

\twolineshloka
{कार्यकारणरूपाय वासुदेवाय ते सदा}
{नमामि पुत्रलाभार्थं सुखदाय बुधाय ते}%॥ १५॥

\twolineshloka
{राजीवनेत्र श्रीराम रावणारे हरे कवे}
{तुभ्यं नमामि देवेश तनयं देहि मे हरे}%॥ १६॥

\twolineshloka
{अस्माकं पुत्रलाभाय भजामि त्वां जगत्पते}
{देहि मे तनयं कृष्ण वासुदेव रमापते}%॥ १७॥

\twolineshloka
{श्रीमानिनीमानचोर गोपीवस्त्रापहारक}
{देहि मे तनयं कृष्ण वासुदेव जगत्पते}%॥ १८॥

\twolineshloka
{अस्माकं पुत्रसम्प्राप्तिं कुरुष्व यदुनन्दन}
{रमापते वासुदेव मुकुन्द मुनिवन्दित}%॥ १९॥

\twolineshloka
{वासुदेव सुतं देहि तनयं देहि माधव}
{पुत्रं मे देहि श्रीकृष्ण वत्सं देहि महाप्रभो}%॥ २०॥

\twolineshloka
{डिम्भकं देहि श्रीकृष्ण आत्मजं देहि राघव}
{भक्तमन्दार मे देहि तनयं नन्दनन्दन}%॥ २१॥

\twolineshloka
{नन्दनं देहि मे कृष्ण वासुदेव जगत्पते}
{कमलानाथ गोविन्द मुकुन्द मुनिवन्दित}%॥ २२॥

\twolineshloka
{अन्यथा शरणं नास्ति त्वमेव शरणं मम}
{सुतं देहि श्रियं देहि श्रियं पुत्रं प्रदेहि मे}%॥ २३॥

\twolineshloka
{यशोदा-स्तन्यपानज्ञं पिबन्तं यदुनन्दनम्}
{वन्देऽहं पुत्रलाभार्थं कपिलाक्षं हरिं सदा}%॥ २४॥

\twolineshloka
{नन्दनन्दन देवेश नन्दनं देहि मे प्रभो}
{रमापते वासुदेव श्रियं पुत्रं जगत्पते}%॥ २५॥

\twolineshloka
{पुत्रं श्रियं श्रियं पुत्रं पुत्रं मे देहि माधव}
{अस्माकं दीनवाक्यस्य अवधारय श्रीपते}%॥ २६॥

\twolineshloka
{गोपालडिम्भ गोविन्द वासुदेव रमापते}
{अस्माकं डिम्भकं देहि श्रियं देहि जगत्पते}%॥ २७॥

\twolineshloka
{मद्वाञ्छितफलं देहि देवकीनन्दनाच्युत}
{मम पुत्रार्थितं धन्यं कुरुष्व यदुनन्दन}%॥ २८॥

\twolineshloka
{याचेऽहं त्वां श्रियं पुत्रं देहि मे पुत्रसम्पदम्}
{भक्तचिन्तामणे राम कल्पवृक्ष महाप्रभो}%॥ २९॥  }

\twolineshloka
{आत्मजं नन्दनं पुत्रं कुमारं डिम्भकं सुतम्}
{अर्भकं तनयं देहि सदा मे रघुनन्दन}%॥ ३०॥

\twolineshloka
{वन्दे सन्तानगोपालं माधवं भक्तकामदम्}
{अस्माकं पुत्रसम्प्राप्त्यै सदा गोविन्दमच्युतम्}%॥ ३१॥

\twolineshloka
{ओङ्कारयुक्तं गोपालं श्रीयुक्तं यदुनन्दनम्}
{क्लींयुक्तं देवकीपुत्रं नमामि यदुनायकम्}%॥ ३२॥  }

\twolineshloka
{वासुदेव मुकुन्देश गोविन्द माधवाच्युत}
{देहि मे तनयं कृष्ण रमानाथ महाप्रभो}%॥ ३३॥

\twolineshloka
{राजीवनेत्र गोविन्द कपिलाक्ष हरे प्रभो}
{समस्तकाम्यवरद देहि मे तनयं सदा}%॥ ३४॥

\twolineshloka
{अब्जपद्मनिभं पद्मवृन्दरूप जगत्पते}
{देहि मे वरसत्पुत्रं रमानायक माधव}%॥ ३५॥

\twolineshloka
{नन्दपाल धरापाल गोविन्द यदुनन्दन}
{देहि मे तनयं कृष्ण रुक्मिणीवल्लभ प्रभो}%॥ ३६॥

\twolineshloka
{दासमन्दार गोविन्द मुकुन्द माधवाच्युत}
{गोपाल पुण्डरीकाक्ष देहि मे तनयं श्रियम्}%॥ ३७॥

\twolineshloka
{यदुनायक पद्मेश नन्दगोपवधूसुत}
{देहि मे तनयं कृष्ण श्रीधर प्राणनायक}%॥ ३८॥

\twolineshloka
{अस्माकं वाञ्छितं देहि देहि पुत्रं रमापते}
{भगवन् कृष्ण सर्वेश वासुदेव जगत्पते}%॥ ३९॥

\twolineshloka
{रमाहृदयसम्भार सत्यभामामनःप्रिय}
{देहि मे तनयं कृष्ण रुक्मिणीवल्लभ प्रभो}%॥ ४०॥

\twolineshloka
{चन्द्रसूर्याक्ष गोविन्द पुण्डरीकाक्ष माधव}
{अस्माकं भाग्यसत्पुत्रं देहि देव जगत्पते}%॥ ४१॥

\twolineshloka
{कारुण्यरूप पद्माक्ष पद्मनाभसमर्चित}
{देहि मे तनयं कृष्ण देवकी-नन्दनन्दन}%॥ ४२॥

\twolineshloka
{देवकीसुत श्रीनाथ वासुदेव जगत्पते}
{समस्तकामफलद देहि मे तनयं सदा}%॥

\twolineshloka
{भक्तमन्दार गम्भीर शङ्कराच्युत माधव}
{देहि मे तनयं गोपबालवत्सल श्रीपते}%॥ ४३॥

\twolineshloka
{श्रीपते वासुदेवेश देवकीप्रियनन्दन}
{भक्तमन्दार मे देहि तनयं जगतां प्रभो}%॥ ४४॥

\twolineshloka
{जगन्नाथ रमानाथ भूमिनाथ दयानिधे}
{वासुदेवेश सर्वेश देहि मे तनयं प्रभो}%॥ ४५॥

\twolineshloka
{श्रीनाथ कमलपत्राक्ष वासुदेव जगत्पते}
{देहि मे तनयं कृष्ण त्वामहं शरणं गतः}%॥ ४६॥

\twolineshloka
{दासमन्दार गोविन्द भक्तचिन्तामणे प्रभो}
{देहि मे तनयं कृष्ण त्वामहं शरणं गतः}%॥ ४७॥

\twolineshloka
{गोविन्द पुण्डरीकाक्ष रमानाथ महाप्रभो}
{देहि मे तनयं कृष्ण त्वामहं शरणं गतः}%॥ ४८॥

\twolineshloka
{श्रीनाथ कमलपत्राक्ष गोविन्द मधुसूदन}
{मत्पुत्रफलसिद्‌ध्यर्थं भजामि त्वां जनार्दन}%॥ ४९॥

\fourlineindentedshloka
{स्तन्यं पिबन्तं जननीमुखाम्बुजं}
{विलोक्य मन्दस्मितमुज्ज्वलाङ्गम्}
{स्पृशन्तमन्यस्तनमङ्गुलीभिः}
{वन्दे यशोदाङ्कगतं मुकुन्दम्}%॥ ५०॥

\twolineshloka
{याचेऽहं पुत्रसन्तानं भवन्तं पद्मलोचन}
{देहि मे तनयं कृष्ण त्वामहं शरणं गतः}%॥ ५१॥

\twolineshloka
{अस्माकं पुत्रसम्पत्तेश्चिन्तयामि जगत्पते}
{शीघ्रं मे देहि दातव्यं भवता मुनिवन्दित}%॥ ५२॥

\twolineshloka
{वासुदेव जगन्नाथ श्रीपते पुरुषोत्तम}
{कुरु मां पुत्रदत्तं च कृष्ण देवेन्द्रपूजित}%॥ ५३॥

\twolineshloka
{कुरु मां पुत्रदत्तं च यशोदा-प्रियनन्दन}
{मह्यं च पुत्र-सन्तानं दातव्यं भवता हरे}%॥ ५४॥

\twolineshloka
{वासुदेव जगन्नाथ गोविन्द देवकीसुत}
{देहि मे तनयं राम कौसल्याप्रियनन्दन}%॥ ५५॥

\twolineshloka
{पद्मपत्राक्ष गोविन्द विष्णो वामन माधव}
{देहि मे तनयं सीताप्राणनायक राघव}%॥ ५६॥

\twolineshloka
{कञ्जाक्ष कृष्ण देवेन्द्रमण्डित मुनिवन्दित}
{लक्ष्मणाग्रज श्रीराम देहि मे तनयं सदा}%॥ ५८॥

\twolineshloka
{देहि मे तनयं राम दशरथ-प्रियनन्दन}
{सीतानायक कञ्जाक्ष मुचुकुन्दवरप्रद}%॥ ५९॥

\twolineshloka
{विभीषणस्य या लङ्का प्रदत्ता भवता पुरा}
{अस्माकं तत्प्रकारेण तनयं देहि माधव}%॥ ६०॥

\twolineshloka
{भवदीयपदाम्भोजे चिन्तयामि निरन्तरम्}
{देहि मे तनयं सीताप्राणवल्लभ राघव}%॥ ६१॥

\twolineshloka
{राम मत्काम्यवरद पुत्रोत्पत्तिफलप्रद}
{देहि मे तनयं श्रीश कमलासनवन्दित}%॥ ६२॥

\twolineshloka
{राम राघव सीतेश लक्ष्मणाग्रज देहि मे}
{भाग्यवत् पुत्रसन्तानं दशरथात्मज श्रीपते}%॥ ६३॥

\twolineshloka
{देवकीगर्भसञ्जात यशोदाप्रियनन्दन}
{देहि मे तनयं राम कृष्ण गोपाल माधव}%॥ ६४॥

\twolineshloka
{कृष्ण माधव गोविन्द वामनाच्युत शङ्कर}
{देहि मे तनयं श्रीश गोपबालकनायक}%॥ ६५॥

\twolineshloka
{गोपबाल महाधन्य गोविन्दाच्युत माधव}
{देहि मे तनयं कृष्ण वासुदेव जगत्पते}%॥ ६६॥

\fourlineindentedshloka
{दिशतु दिशतु पुत्रं देवकीनन्दनोऽयं}
{दिशतु दिशतु शीघ्रं भाग्यवत्पुत्रलाभम्}
{दिशतु दिशतु श्रीशो राघवो रामचन्द्रो}
{दिशतु दिशतु पुत्रं वंशविस्तारहेतोः}%॥ ६७॥

\twolineshloka
{दीयतां वासुदेवेन तनयो सत्प्रियः सुतः}
{कुमारो नन्दनः सीतानायकेन सदा मम}%॥ ६८॥

\twolineshloka
{राम राघव गोविन्द देवकीसुत माधव}
{देहि मे तनयं श्रीश गोपबालकनायक}%॥ ६९॥

\twolineshloka
{वंशविस्तारकं पुत्रं देहि मे मधुसूदन}
{सुतं देहि सुतं देहि त्वामहं शरणं गतः}%॥ ७०॥

\twolineshloka
{ममाभीष्टसुतं देहि कंसारे माधवाच्युत}
{सुतं देहि सुतं देहि त्वामहं शरणं गतः}%॥ ७१॥

\twolineshloka
{चन्द्रार्ककल्पपर्यन्तं तनयं देहि माधव}
{सुतं देहि सुतं देहि त्वामहं शरणं गतः}%॥ ७२॥

\twolineshloka
{विद्यावन्तं बुद्धिमन्तं श्रीमन्तं तनयं सदा}
{देहि मे तनयं कृष्ण देवकीनन्दन प्रभो}%॥ ७३॥

\twolineshloka
{नमामि त्वां पद्मनेत्र सुतलाभाय कामदम्}
{मुकुन्दं पुण्डरीकाक्षं गोविन्दं मधुसूदनम्}%॥ ७४॥

\twolineshloka
{भगवन् कृष्ण गोविन्द सर्वकामफलप्रद}
{देहि मे तनयं स्वामिंस्त्वामहं शरणं गतः}%॥ ७५॥

\twolineshloka
{स्वामिंस्त्वं भगवन् राम कृष्ण माधव कामद}
{देहि मे तनयं नित्यं त्वामहं शरणं गतः}%॥ ७६॥

\twolineshloka
{तनयं देहि गोविन्द कञ्जाक्ष कमलापते}
{सुतं देहि सुतं देहि त्वामहं शरणं गतः}%॥ ७७॥

\twolineshloka
{पद्मापते पद्मनेत्र प्रद्युम्नजनक प्रभो}
{सुतं देहि सुतं देहि त्वामहं शरणं गतः}%॥ ७८॥

\twolineshloka
{शङ्खचक्रगदाखड्गशार्ङ्गपाणे रमापते}
{देहि मे तनयं कृष्ण त्वामहं शरणं गतः}%॥ ७९॥

\twolineshloka
{नारायण रमानाथ राजीवपत्रलोचन}
{सुतं मे देहि देवेश पद्मपद्मानुवन्दित}%॥ ८०॥

\twolineshloka
{राम राघव गोविन्द देवकीवरनन्दन}
{रुक्मिणीनाथ सर्वेश नारदादिसुरार्चित}%॥ ८१॥

\twolineshloka
{देवकीसुत गोविन्द वासुदेव जगत्पते}
{देहि मे तनयं श्रीश गोपबालकनायक}%॥ ८२॥

\twolineshloka
{मुनिवन्दित गोविन्द रुक्मिणीवल्लभ प्रभो}
{देहि मे तनयं कृष्ण त्वामहं शरणं गतः}%॥ ८३॥

\twolineshloka
{गोपिकार्जितपङ्केजमरन्दासक्तमानस}
{देहि मे तनयं कृष्ण त्वामहं शरणं गतः}%॥ ८४॥

\twolineshloka
{रमाहृदयपङ्केजलोल माधव कामद}
{ममाभीष्टसुतं देहि त्वामहं शरणं गतः}%॥ ८५॥

\twolineshloka
{वासुदेव रमानाथ दासानां मङ्गलप्रद}
{देहि मे तनयं कृष्ण त्वामहं शरणं गतः}%॥ ८६॥

\twolineshloka
{कल्याणप्रद गोविन्द मुरारे मुनिवन्दित}
{देहि मे तनयं कृष्ण त्वामहं शरणं गतः}%॥ ८७॥

\twolineshloka
{पुत्रप्रद मुकुन्देश रुक्मिणीवल्लभ प्रभो}
{देहि मे तनयं कृष्ण त्वामहं शरणं गतः}%॥ ८८॥

\twolineshloka
{पुण्डरीकाक्ष गोविन्द वासुदेव जगत्पते}
{देहि मे तनयं कृष्ण त्वामहं शरणं गतः}%॥ ८९॥

\twolineshloka
{दयानिधे वासुदेव मुकुन्द मुनिवन्दित}
{देहि मे तनयं कृष्ण त्वामहं शरणं गतः}%॥ ९०॥

\twolineshloka
{पुत्रसम्पत्प्रदातारं गोविन्दं देवपूजितम्}
{वन्दामहे सदा कृष्णं पुत्रलाभप्रदायिनम्}%॥ ९१॥

\twolineshloka
{कारुण्यनिधये गोपीवल्लभाय मुरारये}
{नमस्ते पुत्रलाभार्थं देहि मे तनयं विभो}%॥ ९२॥

\twolineshloka
{नमस्तस्मै रमेशाय रुक्मिणीवल्लभाय ते}
{देहि मे तनयं श्रीश गोपबालकनायक}%॥ ९३॥

\twolineshloka
{नमस्ते वासुदेवाय नित्यश्रीकामुकाय च}
{पुत्रदाय च सर्पेन्द्रशायिने रङ्गशायिने}%॥ ९४॥

\twolineshloka
{रङ्गशायिन् रमानाथ मङ्गलप्रद माधव}
{देहि मे तनयं श्रीश गोपबालकनायक}%॥ ९५॥

\twolineshloka
{दासस्य मे सुतं देहि दीनमन्दार राघव}
{सुतं देहि सुतं देहि पुत्रं देहि रमापते}%॥ ९६॥

\twolineshloka
{यशोदातनयाभीष्टपुत्रदानरतः सदा}
{देहि मे तनयं कृष्ण त्वामहं शरणं गतः}%॥ ९७॥

\twolineshloka
{मदिष्टदेव गोविन्द वासुदेव जनार्दन}
{देहि मे तनयं कृष्ण त्वामहं शरणं गतः}%॥ ९८॥

\twolineshloka
{नीतिमान् धनवान् पुत्रो विद्यावांश्च प्रजायते}
{भगवंस्त्वत्कृपायाश्च वासुदेवेन्द्रपूजित}%॥ ९९॥

\twolineshloka
{यः पठेत् पुत्रशतकं सोऽपि सत्पुत्रवान् भवेत्}
{श्रीवासुदेवकथितं स्तोत्ररत्नं सुखाय च}%॥ १००॥

\twolineshloka
{जपकाले पठेन्नित्यं पुत्रलाभं धनं श्रियम्}
{ऐश्वर्यं राजसम्मानं सद्यो याति न संशयः}%॥ १०१॥

{॥इति श्री~सन्तानगोपालस्तोत्रं सम्पूर्णम्॥}