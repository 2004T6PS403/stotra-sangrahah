% !TeX program = XeLaTeX
% !TeX root = ../../shloka.tex

\sect{रङ्गनाथ गद्यम्}
\begin{flushleft}
स्वाधीन-त्रिविध-चेतनाचेतन-स्वरूप-स्थिति-प्रवृत्ति-भेदम्'
क्लेश-कर्माद्यशेष-दोषासंस्पृष्टं' स्वाभाविकानवधिकातिशय-
ज्ञान'-बलैश्वर्य'-वीर्य'-शक्ति-तेजः सौशील्य'-वात्सल्य-
मार्दवार्जव'-सौहार्द'-साम्य'-कारुण्य-माधुर्य-गाम्भीर्यौदार्य'-
चातुर्य'-स्थैर्य'-धैर्य'-शौर्य-पराक्रम'-सत्यकाम'-सत्यसङ्कल्प'-
कृतित्व'-कृतज्ञताद्यसङ्ख्येय-कल्याण-गुणगणौघ-महार्णवम्'
\mbox{परब्रह्मभूतं' पुरुषोत्तमं' श्रीरङ्गशायिनम्' अस्मत्स्वामिनं' प्रबुद्ध'}\\
नित्य-नियाम्य' नित्य-दास्यैकरसात्मस्वभावोऽहम्'
तदेकानुभवः' तदेकप्रियः' परिपूर्णं भगवन्तं'
विशदतमानुभवेन निरन्तरमनुभूय' तदनुभव-जनितानवधिकातिशय-%
प्रीतिकारित-अशेषावस्थोचित-
अशेषशेषतैकरतिरूप' नित्य-किङ्करो भवानि॥१॥


स्वात्म-नित्य-नियाम्य'-नित्यदास्यैकरसात्म-स्वभावानुसन्धान-पूर्वक'
भगवदनवधिकातिशय-स्वाम्याद्यखिल-गुणगणानुभवजनित-
अनवधिकातिशय-प्रीतिकारित-अशेषावस्थोचित-अशेषशेषतैकरतिरूप'-%
नित्य-कैङ्कर्य-प्राप्त्युपाय-भूतभक्ति'\\
तदुपाय-सम्यग्ज्ञान' तदुपाय-समीचीनक्रिया'
तदनुगुण-सात्विकतास्तिक्यादि समस्तात्म-गुणविहीनः'
दुरुत्तरानन्त' तद्विपर्यय-ज्ञानक्रियानुगुण-%
अनादिपाप- वासना-महार्णवान्तर्निमग्नः'
तिलतैलवत्' दारुवह्निवत्' दुर्विवेच-त्रिगुणक्षणक्षरण-%
स्वभावाचेतन-प्रकृति-व्याप्तिरूप'-दुरत्यय'-भगवन्माया-तिरोहित-स्वप्रकाशः'
अनाद्यविद्या-सञ्चित-अनन्त-अशक्य-विस्रंसन'-कर्मपाश-प्रग्रथितः'
अनागत-अनन्तकाल-समीक्षयाऽपि' अदृष्ट-सन्तारोपायः'
निखिल-जन्तु-जात-शरण्य! श्रीमन्! नारायण!
तव चरणारविन्दयुगळं शरणमहं प्रपद्ये॥२॥

एवमवस्थितस्याऽपि' अर्थित्वमात्रेण' परमकारुणिको भगवान्'\\
स्वानुभव-प्रीत्या' उपनीतैकान्तिक-अत्यन्तिक-\\
नित्य-कैङ्कर्यैकरतिरूप-नित्य-दास्यं' दास्यतीति' \\
विश्वासपूर्वकं भगवन्तं नित्य-किङ्करतां प्रार्थये॥३॥

तवानुभूति-सम्भूत-प्रीतिकारित-दासताम्।
देहि मे कृपया नाथ! न जाने गतिमन्यथा॥४॥

सर्वावस्थोचित-अशेषशेषतैकरतिस्तव।
भवेयं पुण्डरीकाक्ष! त्वमेवैवं कुरुष्व माम्॥५॥

एवम्भूत-तत्त्वयाथात्म्यावबोध-तदिच्छारहितस्याऽपि'\\
एतदुच्चारण-मात्रावलम्बनेन उच्यमानार्थ-परमार्थ-निष्ठम्'\\
मे मनः त्वमेव अद्यैव कारय॥६॥

अपार-करुणाम्बुधे! अनालोचित-विशेषाशेष-लोक-शरण्य!
प्रणतार्तिहर! आश्रित-वात्सल्यैक-महोदधे! 
अनवरत-विदित-निखिल-भूत-जात-याथात्म्य!
सत्यकाम! सत्यसङ्कल्प! आपत्सख! काकुत्स्थ! श्रीमन्!
नारायण! पुरुषोत्तम! श्रीरङ्गनाथ! मम नाथ! नमोऽस्तु ते॥\\[6mm]
\centerline{॥इति श्रीमद्रामानुजविरचितं श्री~रङ्गनाथ गद्यं सम्पूर्णम्॥}
\end{flushleft}