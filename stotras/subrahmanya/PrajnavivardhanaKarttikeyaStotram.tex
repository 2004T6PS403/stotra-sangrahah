% !TeX program = XeLaTeX
% !TeX root = ../../shloka.tex

\sect{प्रज्ञाविवर्धन-कार्त्तिकेय-स्तोत्रम्}

\uvacha{स्कन्द उवाच}
\twolineshloka
{योगीश्वरो महासेनः कार्त्तिकेयोऽग्निनन्दनः}
{स्कन्दः कुमारः सेनानीः स्वामी शङ्करसम्भवः}

\twolineshloka
{गाङ्गेयस्ताम्रचूडश्च ब्रह्मचारी शिखिध्वजः}
{तारकारिरुमापुत्रः क्रौञ्चारिश्च षडाननः}

\twolineshloka
{शब्दब्रह्मसमुद्रश्च सिद्धः सारस्वतो गुहः}
{सनत्कुमारो भगवान् भोगमोक्षफलप्रदः}

\twolineshloka
{शरजन्मा गणाधीशपूर्वजो मुक्तिमार्गकृत्}
{सर्वागमप्रणेता च वाञ्छितार्थप्रदायकः}

\twolineshloka
{अष्टाविंशतिनामानि मदीयानीति यः पठेत्}
{प्रत्यूषं श्रद्धया युक्तो मूको वाचस्पतिर्भवेत्}

\twolineshloka
{महामन्त्रमयानीति मम नामानुकीर्तनम्}
{महाप्रज्ञामवाप्नोति नात्र कार्या विचारणा}

\threelineshloka
{पुष्यनक्षत्रम् आरभ्य दशवारं पठेन्नरः}
{पुष्यनक्षत्र-पर्यन्तेऽश्वत्थमूले दिने दिने}
{पुरश्चरण-मात्रेण सर्वपापैः प्रमुच्यते}

॥इति~श्री~रुद्रयामलरहस्ये~प्रज्ञाविवर्धन-कार्त्तिकेय-स्तोत्रं~सम्पूर्णम्॥
