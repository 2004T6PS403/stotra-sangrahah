% !TeX program = XeLaTeX
% !TeX root = ../../shloka.tex

\sect{बिल्वाष्टकम्}
\twolineshloka
{त्रिदलं त्रिगुणाकारं त्रिनेत्रं च त्रियायुधम्}
{त्रिजन्मपापसंहारं एकबिल्वं शिवार्पणम्}% १॥

\twolineshloka
{त्रिशाखैः बिल्वपत्रैश्च ह्यच्छिद्रैः कोमलैः शुभैः}
{शिवपूजां करिष्यामि ह्येकबिल्वं शिवार्पणम्}% २॥

\twolineshloka
{अखण्डबिल्वपत्रेण पूजिते नन्दिकेश्वरे}
{शुद्‌ध्यन्ति सर्वपापेभ्यो ह्येकबिल्वं शिवार्पणम्}% ३॥

\twolineshloka
{शालिग्राम-शिलामेकां विप्राणां जातु चार्पयेत्}
{सोमयज्ञ-महापुण्यं एकबिल्वं शिवार्पणम्}% ४॥

\twolineshloka
{दन्तिकोटि-सहस्राणि वाजपेय-शतानि च}
{कोटिकन्या-महादानं एकबिल्वं शिवार्पणम्}% ५॥

\twolineshloka
{लक्ष्म्यास्तनुत उत्पन्नं महादेवस्य च प्रियम्}
{बिल्ववृक्षं प्रयच्छामि ह्येकबिल्वं शिवार्पणम्}% ६॥

\twolineshloka
{दर्शनं बिल्ववृक्षस्य स्पर्शनं पापनाशनम्}
{अघोरपापसंहारं एकबिल्वं शिवार्पणम्}% ७॥

\twolineshloka
{काशीक्षेत्रनिवासं च कालभैरवदर्शनम्}
{प्रयागमाधवं दृष्ट्वा ह्येकबिल्वं शिवार्पणम्}%

\twolineshloka
{मूलतो ब्रह्मरूपाय मध्यतो विष्णुरूपिणे}
{अग्रतः शिवरूपाय ह्येकबिल्वं शिवार्पणम्}% ८॥

\twolineshloka*
{बिल्वाष्टकमिदं पुण्यं यः पठेत् शिवसन्निधौ}
{सर्वपापविनिर्मुक्तः शिवलोकमवाप्नुयात्}%

{॥इति बिल्वाष्टकं सम्पूर्णम्॥}