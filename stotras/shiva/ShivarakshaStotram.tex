% !TeX program = XeLaTeX
% !TeX root = ../../shloka.tex

\sect{शिवरक्षास्तोत्रम्}
अस्य श्रीशिवरक्षास्तोत्रमन्त्रस्य याज्ञवल्क्य ऋषिः।\\ 
श्रीसदाशिवो देवता।अनुष्टुप् छन्दः।\\ 
श्रीसदाशिवप्रीत्यर्थे शिवरक्षास्तोत्रजपे विनियोगः॥

\twolineshloka*
{चरितं देवदेवस्य महादेवस्य पावनम्}
{अपारं परमोदारं चतुर्वर्गस्य साधनम्}

\twolineshloka*
{गौरीविनायकोपेतं पञ्चवक्त्रं त्रिनेत्रकम्}
{शिवं ध्यात्वा दशभुजं शिवरक्षां पठेन्नरः}


\dnsub{कवचम्}\resetShloka
\twolineshloka
{गङ्गाधरः शिरः पातु फालमर्धेन्दुशेखरः}
{नयने मदनध्वंसी कर्णो सर्पविभूषणः} 

\twolineshloka
{घ्राणं पातु पुरारातिर्मुखं पातु जगत्पतिः}
{जिह्वां वागीश्वरः पातु कन्धरं शितिकन्धरः}

\twolineshloka
{श्रीकण्ठः पातु मे कण्ठं स्कन्धौ विश्वधुरन्धरः}
{भुजौ भूभारसंहर्ता करौ पातु पिनाकधृक्}

\twolineshloka
{हृदयं शङ्करः पातु जठरं गिरिजापतिः}
{नाभिं मृत्युञ्जयः पातु कटी व्याघ्राजिनाम्बरः}

\twolineshloka
{सक्थिनी पातु दीनार्तशरणागतवत्सलः}
{ऊरू महेश्वरः पातु जानुनी जगदीश्वरः}

\twolineshloka
{जङ्घे पातु जगत्कर्ता गुल्फौ पातु गणाधिपः}
{चरणौ करुणासिन्धुः सर्वाङ्गानि सदाशिवः}

\dnsub{फलश्रुतिः}
\twolineshloka*
{एतां शिवबलोपेतां रक्षां यः सुकृती पठेत्}
{स भुक्त्वा सकलान् कामान् शिवसायुज्यमाप्नुयात्}

\twolineshloka*
{ग्रहभूतपिशाचाद्यास्त्रैलोक्ये विचरन्ति ये}
{दूरादाशु पलायन्ते शिवनामाभिरक्षणाम्}

\twolineshloka*
{अभयङ्करनामेदं कवचं पार्वतीपतेः}
{भक्त्या बिभर्ति यः कण्ठे तस्य वश्यं जगत्त्रयम्}

\twolineshloka*
{इमां नारायणः स्वप्ने शिवरक्षां यथाऽदिशत्}
{प्रातरूत्थाय योगीन्द्रो याज्ञवल्क्यस्तथाऽलिखत्}

॥इति श्रीमद्याज्ञवल्क्यमुनिप्रोक्तं शिवरक्षास्तोत्रं सम्पूर्णम्॥