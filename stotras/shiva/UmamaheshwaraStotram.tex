% !TeX program = XeLaTeX
% !TeX root = ../../shloka.tex

\sect{उमामहेश्वरस्तोत्रम्}

\fourlineindentedshloka
{नमः शिवाभ्यां नवयौवनाभ्याम्‌}
{परस्पराश्लिष्टवपुर्धराभ्याम्‌}
{नगेन्द्रकन्यावृषकेतनाभ्याम्‌}
{नमो नमः शङ्करपार्वतीभ्याम्‌}%१

\fourlineindentedshloka
{नमः शिवाभ्यां सरसोत्सवाभ्याम्‌}
{नमस्कृताभीष्टवरप्रदाभ्याम्‌}
{नारायणेनार्चितपादुकाभ्याम्}
{नमो नमः शङ्करपार्वतीभ्याम्‌}%२

\fourlineindentedshloka
{नमः शिवाभ्यां वृषवाहनाभ्याम्‌}
{विरिञ्चिविष्ण्विन्द्रसुपूजिताभ्याम्‌}
{विभूतिपाटीरविलेपनाभ्याम्‌}
{नमो नमः शङ्करपार्वतीभ्याम्‌}%३

\fourlineindentedshloka
{नमः शिवाभ्यां जगदीश्वराभ्याम्}
{जगत्पतिभ्यां जयविग्रहाभ्याम्‌}
{जम्भारिमुख्यैरभिवन्दिताभ्याम्‌}
{नमो नमः शङ्करपार्वतीभ्याम्‌}%४

\fourlineindentedshloka
{नमः शिवाभ्यां परमौषधाभ्याम्‌}
{पञ्चाक्षरी-पञ्जररञ्जिताभ्याम्‌}
{प्रपञ्च-सृष्टि-स्थिति-संहृताभ्याम्‌}
{नमो नमः शङ्करपार्वतीभ्याम्‌}%५

\fourlineindentedshloka
{नमः शिवाभ्यामतिसुन्दराभ्याम्‌}
{अत्यन्तमासक्तहृदम्बुजाभ्याम्‌}
{अशेषलोकैकहितङ्कराभ्याम्‌}
{नमो नमः शङ्करपार्वतीभ्याम्‌}%६

\fourlineindentedshloka
{नमः शिवाभ्यां कलिनाशनाभ्याम्‌}
{कङ्कालकल्याणवपुर्धराभ्याम्‌}
{कैलासशैलस्थितदेवताभ्याम्‌}
{नमो नमः शङ्करपार्वतीभ्याम्‌}%७

\fourlineindentedshloka
{नमः शिवाभ्यामशुभापहाभ्याम्‌}
{अशेषलोकैकविशेषिताभ्याम्‌}
{अकुण्ठिताभ्यां स्मृतिसम्भृताभ्याम्‌}
{नमो नमः शङ्करपार्वतीभ्याम्‌}%८

\fourlineindentedshloka
{नमः शिवाभ्यां रथवाहनाभ्याम्‌}
{रवीन्दुवैश्वानरलोचनाभ्याम्‌}
{राका-शशाङ्काभ-मुखाम्बुजाभ्याम्‌}
{नमो नमः शङ्करपार्वतीभ्याम्‌}%९

\fourlineindentedshloka
{नमः शिवाभ्यां जटिलन्धराभ्याम्‌}
{जरामृतिभ्यां च विवर्जिताभ्याम्‌}
{जनार्दनाब्जोद्भवपूजिताभ्याम्‌}
{नमो नमः शङ्करपार्वतीभ्याम्‌}%१०

\fourlineindentedshloka
{नमः शिवाभ्यां विषमेक्षणाभ्याम्‌}
{बिल्वच्छदामल्लिकदामभृद्‌भ्याम्‌}
{शोभावती-शान्तवतीश्वराभ्याम्‌}
{नमो नमः शङ्करपार्वतीभ्याम्‌}%११

\fourlineindentedshloka
{नमः शिवाभ्यां पशुपालकाभ्याम्‌}
{जगत्त्रयीरक्षण-बद्धहृद्‌भ्याम्‌}
{समस्तदेवासुरपूजिताभ्याम्‌}
{नमो नमः शङ्करपार्वतीभ्याम्‌}%१२

\fourlineindentedshloka
{स्तोत्रं त्रिसन्ध्यं शिवपार्वतीभ्याम्‌}
{भक्त्या पठेद्-द्वादशकं नरो यः}
{स सर्वसौभाग्य-फलानि भुङ्क्ते}
{शतायुरन्ते शिवलोकमेति}%१३

{॥इति श्रीमच्छङ्कराचार्यविरचितं श्री उमामहेश्वरस्तोत्रं सम्पूर्णम्‌॥}