% !TeX program = XeLaTeX
% !TeX root = ../../shloka.tex

\sect{तोटकाष्टकम्}

\twolineshloka*
{शङ्करं शङ्कराचार्यं केशवं बादरायणम्}%
{सूत्रभाष्यकृतौ वन्दे भगवन्तौ पुनः पुनः}% 

\fourlineindentedshloka
{नारायणं पद्मभुवं वसिष्ठं शक्तिं च तत्पुत्रपराशरं च}%
{व्यासं शुकं गौडपदं महान्तं गोविन्दयोगीन्द्रमथास्य शिष्यम्}% 
{श्री शङ्कराचार्यमथास्य पद्मपादं च हस्तामलकं च शिष्यम्}%
{तं तोटकं वार्तिककारमन्यानस्मद्गुरून् सन्ततमानतोऽस्मि}% 

\twolineshloka
{विदिताखिलशास्त्रसुधाजलधे महितोपनिषत् कथितार्थनिधे}%
{हृदये कलये विमलं चरणं भव शङ्कर देशिक मे शरणम्}% १}% 


\twolineshloka
{करुणावरुणालय पालय मां भवसागरदुःखविदूनहृदम्}%
{रचयाखिलदर्शनतत्त्वविदं भव शङ्कर देशिक मे शरणम्}% २}% 


\twolineshloka
{भवता जनता सुहिता भविता निजबोधविचारण चारुमते}%
{कलयेश्वरजीवविवेकविदं भव शङ्कर देशिक मे शरणम्}% ३}% 


\twolineshloka
{भव एव भवानिति मे नितरां समजायत चेतसि कौतुकिता}%
{मम वारय मोहमहाजलधिं भव शङ्कर देशिक मे शरणं }% ४}% 


\twolineshloka
{सुकृतेऽधिकृते बहुधा भवतो भविता समदर्शनलालसता}%
{अतिदीनमिमं परिपालय मां भव शङ्कर देशिक मे शरणम्}% ५}% 


\twolineshloka
{जगतीमवितुं कलिताकृतयो विचरन्ति महामहसश्छलतः}%
{अहिमांशुरिवात्र विभासि गुरो भव शङ्कर देशिक मे शरणम्}% ६}% 


\twolineshloka
{गुरुपुङ्गव पुङ्गवकेतन ते समतामयतां न हि कोऽपि सुधीः}%
{शरणागतवत्सल तत्त्वनिधे भव शङ्कर देशिक मे शरणम्}% ७}% 


\twolineshloka
{विदिता न मया विशदैककला न च किञ्चन काञ्चनमस्ति गुरो}%
{द्रुतमेव विधेहि कृपां सहजां भव शङ्कर देशिक मे शरणम्}% ८}% 

॥इति श्री तोटकाचार्यविरचितं श्री तोटकाष्टकं सम्पूर्णम्॥